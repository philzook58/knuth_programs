\input cwebmac
\datethis

\N{1}{1}Intro. This program computes the BDD size of the hidden
weighted bit function, given a permutation of the input variables.
After I wrote the program {\mc HWB} a few days ago, and ran it
for an hour in the case $n=100$ with 8 gigabytes of memory,
I realized that the whole calculation can really be done
{\it much\/} faster---indeed, in polynomial time.

So now I'm doing it a better way. The new way is so efficient,
in fact, that I'm going to have fun and implement it by
simulating decimal arithmetic, using one byte per digit,
throwing all ordinary notions of efficiency out the window.

The previous method generated ``slot tables.'' Now I've renamed
them ``slate tables,'' and discussed the relevant theory in
Section 7.1.4 of TAOCP. With this theory I
don't need to ``work top down'' and effectively generate each
node of the BDD. Instead, I determine the number of
beads of height $m$ by a direct calculation.

\Y\B\4\D$\|n$ \5
\T{100}\C{ the number of variables }\par
\B\4\D$\\{memsize}$ \5
\T{1000000}\C{ the number of bytes for arithmetic; must exceed $3n$ }\par
\Y\B\8\#\&{include} \.{<stdio.h>}\6
\8\#\&{include} \.{<stdlib.h>}\6
\&{char} \\{mem}[\\{memsize}];\C{ the big storage area }\6
\&{int} \\{memptr};\C{ the number of bytes in use }\6
\&{int} \\{numptr};\C{ the number of numbers in use }\6
\&{int} ${}\\{start}[\|n*\|n]{}$;\C{ where remembered numbers begin in \PB{%
\\{mem}} }\6
\&{int} \\{bico}[\|n][\|n];\C{ table of binomial coefficients that I've
computed }\6
\&{int} \\{addedA}[\|n]${},{}$ \\{addedB}[\|n]${},{}$ \\{addedC}[\|n]${},{}$ %
\\{addedD}[\|n];\C{ constant memos }\6
\&{unsigned} \&{char} \\{rev}[\T{256}];\C{ bit-reversal table: $0^R$, $1^R$, %
\dots, $255^R$ }\6
\&{int} ${}\\{perm}[\|n+\T{1}]{}$;\C{ the permutation }\6
\&{int} \\{nonbeads};\C{ nonbeads found at the current height }\6
\&{int} \\{tnonbeads};\C{ total nonbeads so far }\7
\X3:Subroutines\X\7
\\{main}(\&{int} \\{argc})\1\1\2\2\6
${}\{{}$\1\6
\&{register} \&{int} \|i${},{}$ \|j${},{}$ \|k${},{}$ \|m${},{}$ \|p${},{}$ %
\|s${},{}$ \\{ss}${},{}$ \|t${},{}$ \\{tt}${},{}$ \|w${},{}$ \\{ww};\7
\X2:Set up the permutation, \PB{\\{perm}}\X;\6
\X8:Initialize \PB{\\{mem}}\X;\6
\&{for} ${}(\|k\K\T{0};{}$ ${}\|k<\|n;{}$ ${}\|k\PP){}$\5
${}\{{}$\1\6
\X10:Compute $b_k$\X;\6
\X11:Print $b_k$ and add it to the grand total\X;\6
\4${}\}{}$\2\6
\\{printf}(\.{"height\ 0:\ 2\\n"});\C{ \PB{$\|k\K\|n$} is a simple special case
}\6
\X4:Print the grand total\X;\6
\4${}\}{}$\2\par
\fi

\M{2}The purpose of this step is to set $\PB{\\{perm}}[j]=j\pi$ for $1\le j\le
n$, where
$\pi$ is the desired permutation of the input variables. And I set
\PB{$\\{perm}[\T{0}]\K\|n+\T{1}$}, because \PB{$\\{perm}[\T{0}]\K\T{0}$} would
make $x_0$ appear to be
a member of $\{x_1,\ldots,x_k\}$.

In this implementation I use (almost) the ``hybrid'' ordering of
Bollig, L\"obbing, Sauerhoff, and Wegener.
That means the first $n/5$ elements come alternately
from the top $n/10$ and the bottom $n/10$.
The remaining $4n/5$ elements are ordered according
to the bit reversal of the difference between them and $9n/10$.

\Y\B\4\X2:Set up the permutation, \PB{\\{perm}}\X${}\E{}$\6
\&{for} ${}(\|j\K\T{\^80},\39\|m\K\T{1};{}$ \|j; ${}\|j\MRL{{\GG}{\K}}\T{1},\39%
\|m\MRL{{\LL}{\K}}\T{1}){}$\1\6
\&{for} ${}(\|k\K\T{0};{}$ ${}\|k<\T{\^100};{}$ ${}\|k\MRL{+{\K}}\|j+\|j){}$\1\5
${}\\{rev}[\|k+\|j]\K\\{rev}[\|k]+\|m;{}$\2\2\6
\&{for} ${}(\|j\K\|m\K\T{1},\39\|k\K\|n;{}$ ${}\|j\Z\|n/\T{10};{}$ ${}\|j\PP,%
\39\|k\MM,\39\|m\MRL{+{\K}}\T{2}){}$\1\5
${}\\{perm}[\|k]\K\|m,\39\\{perm}[\|j]\K\|m+\T{1};{}$\2\6
\&{for} ${}(\|i\K\T{0};{}$ ${}\|m\Z\|n;{}$ ${}\|i\PP,\39\|m\PP){}$\5
${}\{{}$\1\6
\&{while} ${}(\\{rev}[\|i]>\|k-\|j){}$\1\5
${}\|i\PP;{}$\2\6
${}\\{perm}[\|k-\\{rev}[\|i]]\K\|m;{}$\6
\4${}\}{}$\2\6
\\{printf}(\.{"Starting\ from\ perm"});\6
\&{for} ${}(\|j\K\T{1};{}$ ${}\|j\Z\|n;{}$ ${}\|j\PP){}$\1\5
${}\\{printf}(\.{"\ \%d"},\39\\{perm}[\|j]);{}$\2\6
\\{printf}(\.{"\\n"});\6
${}\\{perm}[\T{0}]\K\|n+\T{1}{}$;\par
\U1.\fi

\N{1}{3}Decimal addition. The \PB{\|k}th number in my decimal memory starts
at location \PB{\\{start}[\|k]} in \PB{\\{mem}}, and ends just before location
\PB{$\\{start}[\|k+\T{1}]$}. Each byte of \PB{\\{mem}} contains a single digit,
and
the least significant digits come first.

Number 0 is the grand total, and number 1 is the
total-so-far at height~\PB{\|m}. The other numbers are binomial coefficients,
which I compute from scratch as needed.

To warm up, here's a routine to print out the \PB{\|k}th number:

\Y\B\4\X3:Subroutines\X${}\E{}$\6
\&{void} \\{printnum}(\&{int} \|k)\1\1\2\2\6
${}\{{}$\1\6
\&{register} \&{int} \|j;\7
\&{for} ${}(\|j\K\\{start}[\|k+\T{1}]-\T{1};{}$ ${}\|j>\\{start}[\|k];{}$ ${}%
\|j\MM){}$\1\6
\&{if} (\\{mem}[\|j])\1\5
\&{break};\2\2\6
\&{for} ( ; ${}\|j\G\\{start}[\|k];{}$ ${}\|j\MM){}$\1\5
${}\\{printf}(\.{"\%d"},\39\\{mem}[\|j]);{}$\2\6
\4${}\}{}$\2\par
\As5, 6, 7\ETs9.
\U1.\fi

\M{4}\B\X4:Print the grand total\X${}\E{}$\6
\\{printf}(\.{"Altogether\ "});\6
\\{printnum}(\T{0});\6
${}\\{printf}(\.{"-\%d\ nodes;\ I\ used\ \%}\)\.{d\ bytes\ of\ memory\ fo}\)%
\.{r\ \%d\ numbers.\\n"},\39\\{tnonbeads},\39\\{memptr},\39\\{numptr}){}$;\par
\U1.\fi

\M{5}\B\X3:Subroutines\X${}\mathrel+\E{}$\6
\&{void} \\{clearnum}(\&{int} \|k)\1\1\2\2\6
${}\{{}$\1\6
\&{register} \&{int} \|j;\7
\&{for} ${}(\|j\K\\{start}[\|k];{}$ ${}\|j<\\{start}[\|k+\T{1}];{}$ ${}\|j%
\PP){}$\1\5
${}\\{mem}[\|j]\K\T{0};{}$\2\6
\4${}\}{}$\2\par
\fi

\M{6}The \PB{\\{add}} routine adds number \PB{\|k} to number \PB{\|l} and
stores the result as a brand new number, whose index is returned.

We assume (conservatively) that all numbers have at most \PB{\|n} digits.

\Y\B\4\X3:Subroutines\X${}\mathrel+\E{}$\6
\&{int} \\{add}(\&{int} \|k${},\39{}$\&{int} \|l)\1\1\2\2\6
${}\{{}$\1\6
\&{register} \&{int} \|c${},{}$ \|i${},{}$ \|j;\7
\&{if} ${}(\\{memptr}+\|n\G\\{memsize}){}$\5
${}\{{}$\1\6
${}\\{fprintf}(\\{stderr},\39\.{"Out\ of\ memory\ (mems}\)\.{ize=\%d)!\\n"},\39%
\\{memsize});{}$\6
${}\\{exit}({-}\T{1});{}$\6
\4${}\}{}$\2\6
\&{for} ${}(\|c\K\T{0},\39\|i\K\\{start}[\|k],\39\|j\K\\{start}[\|l];{}$  ; ${}%
\|i\PP,\39\|j\PP,\39\\{memptr}\PP){}$\5
${}\{{}$\1\6
\&{if} ${}(\|i<\\{start}[\|k+\T{1}]){}$\5
${}\{{}$\1\6
\&{if} ${}(\|j<\\{start}[\|l+\T{1}]){}$\1\5
${}\\{mem}[\\{memptr}]\K\\{mem}[\|i]+\\{mem}[\|j]+\|c;{}$\2\6
\&{else}\1\5
${}\\{mem}[\\{memptr}]\K\\{mem}[\|i]+\|c;{}$\2\6
\4${}\}{}$\5
\2\&{else}\5
${}\{{}$\1\6
\&{if} ${}(\|j<\\{start}[\|l+\T{1}]){}$\1\5
${}\\{mem}[\\{memptr}]\K\\{mem}[\|j]+\|c;{}$\2\6
\&{else}\1\5
\&{break};\2\6
\4${}\}{}$\2\6
\&{if} ${}(\\{mem}[\\{memptr}]\G\T{10}){}$\1\5
${}\|c\K\T{1},\39\\{mem}[\\{memptr}]\MRL{-{\K}}\T{10};{}$\2\6
\&{else}\1\5
${}\|c\K\T{0};{}$\2\6
\4${}\}{}$\2\6
\&{if} (\|c)\1\5
${}\\{mem}[\\{memptr}\PP]\K\T{1};{}$\2\6
${}\\{numptr}\PP;{}$\6
${}\\{start}[\\{numptr}+\T{1}]\K\\{memptr};{}$\6
\&{return} \\{numptr};\6
\4${}\}{}$\2\par
\fi

\M{7}Another variant of addition adds number \PB{\|l} to number \PB{\|k},
and replaces number \PB{\|k} by the sum. This routine is used only
when \PB{$\\{start}[\|k+\T{1}]-\\{start}[\|k]$} is large enough to contain the
sum.

\Y\B\4\X3:Subroutines\X${}\mathrel+\E{}$\6
\&{void} \\{addto}(\&{int} \|k${},\39{}$\&{int} \|l)\1\1\2\2\6
${}\{{}$\1\6
\&{register} \|c${},{}$ \|i${},{}$ \|j;\7
\&{for} ${}(\|c\K\T{0},\39\|i\K\\{start}[\|k],\39\|j\K\\{start}[\|l];{}$ ${}%
\|i<\\{start}[\|k+\T{1}];{}$ ${}\|i\PP,\39\|j\PP){}$\5
${}\{{}$\1\6
${}\\{mem}[\|i]\MRL{+{\K}}(\|j<\\{start}[\|l+\T{1}]\?\\{mem}[\|j]:\T{0})+%
\|c;{}$\6
\&{if} ${}(\\{mem}[\|i]\G\T{10}){}$\1\5
${}\|c\K\T{1},\39\\{mem}[\|i]\MRL{-{\K}}\T{10};{}$\2\6
\&{else}\1\5
${}\|c\K\T{0};{}$\2\6
\4${}\}{}$\C{ here I could check to make sure that \PB{$\|c\K\T{0}$}, but I
won't bother }\2\6
\4${}\}{}$\2\par
\fi

\M{8}Number 2 in \PB{\\{mem}} is actually the constant `0', and number 3 is
`1'.

\Y\B\4\D$\\{grandtotal}$ \5
\T{0}\par
\B\4\D$\\{subtotal}$ \5
\T{1}\par
\B\4\D$\\{zero}$ \5
\T{2}\par
\B\4\D$\\{one}$ \5
\T{3}\par
\Y\B\4\X8:Initialize \PB{\\{mem}}\X${}\E{}$\6
$\\{start}[\\{grandtotal}]\K\T{0};{}$\6
${}\\{mem}[\T{0}]\K\T{2}{}$;\C{ the grand total is initially 2 }\6
${}\\{start}[\\{subtotal}]\K\\{start}[\\{grandtotal}]+\|n;{}$\6
${}\\{start}[\\{zero}]\K\\{start}[\\{subtotal}]+\|n,\39\\{start}[\\{zero}+%
\T{1}]\K\\{start}[\\{zero}]+\T{1};{}$\6
${}\\{mem}[\\{start}[\\{one}]]\K\T{1},\39\\{start}[\\{one}+\T{1}]\K\\{start}[%
\\{one}]+\T{1};{}$\6
${}\\{numptr}\K\\{one},\39\\{memptr}\K\\{start}[\\{numptr}+\T{1}]{}$;\par
\U1.\fi

\M{9}Here's how I compute binomial coefficients $m\choose k$,
without attempting to optimize.

\Y\B\4\X3:Subroutines\X${}\mathrel+\E{}$\6
\&{int} \\{binom}(\&{int} \|m${},\39{}$\&{int} \|k)\1\1\2\2\6
${}\{{}$\1\6
\&{if} ${}(\|k<\T{0}\V\|k>\|m){}$\1\5
\&{return} \\{zero};\2\6
\&{if} ${}(\|k\E\T{0}\V\|k\E\|m){}$\1\5
\&{return} \\{one};\2\6
\&{if} ${}(\R\\{bico}[\|m][\|k]){}$\1\5
${}\\{bico}[\|m][\|k]\K\\{add}(\\{binom}(\|m-\T{1},\39\|k),\39\\{binom}(\|m-%
\T{1},\39\|k-\T{1}));{}$\2\6
\&{return} \\{bico}[\|m][\|k];\6
\4${}\}{}$\2\par
\fi

\N{1}{10}The algorithm. So much for infrastructure; let's get to work.

\Y\B\4\X10:Compute $b_k$\X${}\E{}$\6
\\{clearnum}(\\{subtotal});\6
${}\\{nonbeads}\K\T{0};{}$\6
${}\|m\K\|n-\|k;{}$\6
\X14:Clear the four constant tables\X;\6
\&{for} ${}(\|s\K\T{0};{}$ ${}\|s\Z\|k;{}$ ${}\|s\PP){}$\5
${}\{{}$\1\6
\X12:Add contributions for slates $(s,k)$ to \PB{\\{subtotal}}\X;\6
\4${}\}{}$\2\6
\X16:Correct for constant nonbeads\X;\par
\U1.\fi

\M{11}\B\X11:Print $b_k$ and add it to the grand total\X${}\E{}$\6
$\\{printf}(\.{"height\ \%d:\ "},\39\|m);{}$\6
\\{printnum}(\\{subtotal});\6
\&{if} (\\{nonbeads})\1\5
${}\\{printf}(\.{"-\%d\\n"},\39\\{nonbeads});{}$\2\6
\&{else}\1\5
\\{printf}(\.{"\\n"});\2\6
${}\\{addto}(\\{grandtotal},\39\\{subtotal});{}$\6
${}\\{tnonbeads}\MRL{+{\K}}\\{nonbeads}{}$;\par
\U1.\fi

\M{12}Each slate for $(s,k)$ is $[r_0,\ldots,r_m]$, where $r_j$ is
0~or~1 when \PB{$\\{perm}[\|s+\|j]\Z\|k$}, otherwise $r_j$ is $x_l$ where \PB{$%
\\{perm}[\|s+\|j]\K\|l$}.
(The latter case represents one of the $m$ remaining variables.)
I~compute the quantity $w$, which is the number of times the
former case occurs; this is what Bollig et al.\ have called the
``window size.''

However, we set $r_0\gets0$ and $r_m\gets1$ if they aren't already constant;
$r_0$ and/or $r_m$ are then called ``false constants.''
With these conventions, there's exactly one slate table for each subfunction
at height~$m$.

Let $t=k-s$. The $w$ settings of the constant $r_j$'s run through all
combinations of \PB{\\{ss}} 1s and \PB{\\{tt}} 0s such that \PB{$\\{ss}+\\{tt}%
\K\|w$}, \PB{$\\{ss}\Z\|s$},
and \PB{$\\{tt}\Z\|t$}.

If at least one of the positions $\{r_1,\ldots,r_{m-1}\}$ is nonconstant,
a particular slate can occur only for one value of~$s$. Otherwise the
situation is more subtle, and I need to consider constant slates
of four types depending on the boundary conditions.

\smallskip
$\bullet$ Type A, $r_0=0$ and $r_m=0$: Here $r_0$ might be a false constant.

$\bullet$ Type B, $r_0=0$ and $r_m=1$: Here $r_0$ and/or $r_m$ might be false.

$\bullet$ Type C, $r_0=1$ and $r_m=0$: Both $r_0$ and $r_m$ are true.

$\bullet$ Type D, $r_0=1$ and $r_m=1$: Maybe $r_m$ is false.

\smallskip\noindent
A setting of \PB{\\{ss}} 1s and \PB{\\{tt}} 0s contributes to all four types if
$r_0$
and $r_m$ are true. It contributes only to type B if $r_0$ and $r_m$
are false. It contributes only to types A and~B if $r_0$ is false but
$r_m$ is true; only to B and~D if $r_0$ is true but $r_m$ is false.

\Y\B\4\X12:Add contributions for slates $(s,k)$ to \PB{\\{subtotal}}\X${}\E{}$\6
\&{for} ${}(\|w\K\\{ww}\K\|j\K\T{0};{}$ ${}\|j\Z\|m;{}$ ${}\|j\PP){}$\1\6
\&{if} ${}(\\{perm}[\|s+\|j]\Z\|k){}$\5
${}\{{}$\1\6
${}\|w\PP;{}$\6
\&{if} ${}(\|j>\T{0}\W\|j<\|m){}$\1\5
${}\\{ww}\PP;{}$\2\6
\4${}\}{}$\2\2\6
\&{if} ${}(\\{ww}\E\|m-\T{1}){}$\1\5
\X15:Add contributions for a constant case\X\2\6
\&{else}\5
${}\{{}$\1\6
\X13:Correct for nonconstant nonbeads\X;\6
\&{for} ${}(\|t\K\|k-\|s,\39\\{ss}\K\|s,\39\\{tt}\K\|w-\\{ss};{}$ ${}\\{tt}\Z%
\|t;{}$ ${}\\{ss}\MM,\39\\{tt}\PP){}$\5
${}\{{}$\1\6
${}\\{addto}(\\{subtotal},\39\\{binom}(\|w,\39\\{ss}));{}$\6
\&{if} ${}(\\{ss}\E\|p){}$\1\5
${}\\{nonbeads}\PP{}$;\C{ see below }\2\6
\4${}\}{}$\2\6
\4${}\}{}$\2\par
\U10.\fi

\M{13}Nonbeads $[r_0,\ldots,r_m]$ are of four kinds: (a)~$r_p=x_{k+1}$,
$r_j=1$ for $j<p$, and $r_j=0$ for $j>p$; (b)~$[0,x_n,1]$;
(c)~$[r_0,\ldots,r_m]=[0,\ldots,0]$, within Type~A;
(d)~$[r_0,\ldots,r_m]=[1,\ldots,1]$, within Type~D.
Here we look for (a) and (b).

\Y\B\4\X13:Correct for nonconstant nonbeads\X${}\E{}$\6
$\|p\K\|n+\T{1}{}$;\C{ \PB{$\|n+\T{1}$} is ``infinity'' }\6
\&{if} ${}(\\{ww}\E\|m-\T{2}){}$\5
${}\{{}$\1\6
\&{if} ${}(\|m\E\T{2}\W\\{perm}[\|s+\T{1}]\E\|n){}$\1\5
${}\|p\K(\\{perm}[\|s+\T{2}]\Z\|k\?\T{1}:\T{0});{}$\2\6
\&{else} \&{if} ${}(\|w\E\|m){}$\5
${}\{{}$\1\6
\&{for} ${}(\|p\K\T{1};{}$  ; ${}\|p\PP){}$\1\6
\&{if} ${}(\\{perm}[\|s+\|p]>\|k){}$\1\5
\&{break};\2\2\6
\&{if} ${}(\\{perm}[\|s+\|p]\I\|k+\T{1}){}$\1\5
${}\|p\K\|n+\T{1};{}$\2\6
\4${}\}{}$\2\6
\4${}\}{}$\2\par
\U12.\fi

\M{14}Each constant type is a symmetric function. I need to contribute
$m-1\choose r$ to the subtotal for each possible value of
$r=r_1+\cdots+r_{m-1}$. But I want to contribute exactly once for
every such~$r$; equal values of $r$ can arise from different values of~$s$.
So there are tables \PB{\\{addedA}}, \PB{\\{addedB}}, \PB{\\{addedC}}, \PB{%
\\{addedD}}, to remember
when a particular $r$ has been contributed already to the counts of
each type.

\Y\B\4\X14:Clear the four constant tables\X${}\E{}$\6
\&{for} ${}(\|j\K\T{0};{}$ ${}\|j<\|m;{}$ ${}\|j\PP){}$\1\5
${}\\{addedA}[\|j]\K\\{addedB}[\|j]\K\\{addedC}[\|j]\K\\{addedD}[\|j]\K%
\T{0}{}$;\2\par
\U10.\fi

\M{15}Here's where I hope logic hasn't failed me.

\Y\B\4\X15:Add contributions for a constant case\X${}\E{}$\6
${}\{{}$\1\6
\&{for} ${}(\|t\K\|k-\|s,\39\\{ss}\K\|s,\39\\{tt}\K\|w-\\{ss};{}$ ${}\\{tt}\Z%
\|t;{}$ ${}\\{ss}\MM,\39\\{tt}\PP){}$\1\6
\&{if} ${}(\\{ss}\G\T{0}\W\\{tt}\G\T{0}){}$\5
${}\{{}$\1\6
\&{if} ${}(\\{perm}[\|s+\|m]\Z\|k){}$\5
${}\{{}$\C{ true constant at right }\1\6
\&{if} ${}(\R\\{addedA}[\\{ss}]){}$\1\5
${}\\{addedA}[\\{ss}]\K\T{1},\39\\{addto}(\\{subtotal},\39\\{binom}(\|m-\T{1},%
\39\\{ss}));{}$\2\6
\&{if} ${}(\\{ss}>\T{0}\W\R\\{addedB}[\\{ss}-\T{1}]){}$\1\5
${}\\{addedB}[\\{ss}-\T{1}]\K\T{1},\39\\{addto}(\\{subtotal},\39\\{binom}(\|m-%
\T{1},\39\\{ss}-\T{1}));{}$\2\6
\4${}\}{}$\5
\2\&{else} \&{if} ${}(\R\\{addedB}[\\{ss}]){}$\1\5
${}\\{addedB}[\\{ss}]\K\T{1},\39\\{addto}(\\{subtotal},\39\\{binom}(\|m-\T{1},%
\39\\{ss}));{}$\2\6
\&{if} ${}(\\{ss}>\T{0}\W\\{perm}[\|s]\Z\|k){}$\5
${}\{{}$\C{ true constant at left }\1\6
\&{if} ${}(\\{perm}[\|s+\|m]\Z\|k){}$\5
${}\{{}$\C{ and also at right }\1\6
\&{if} ${}(\R\\{addedC}[\\{ss}-\T{1}]){}$\1\5
${}\\{addedC}[\\{ss}-\T{1}]\K\T{1},\39\\{addto}(\\{subtotal},\39\\{binom}(\|m-%
\T{1},\39\\{ss}-\T{1}));{}$\2\6
\&{if} ${}(\\{ss}>\T{1}\W\R\\{addedD}[\\{ss}-\T{2}]){}$\1\5
${}\\{addedD}[\\{ss}-\T{2}]\K\T{1},\39\\{addto}(\\{subtotal},\39\\{binom}(\|m-%
\T{1},\39\\{ss}-\T{2}));{}$\2\6
\4${}\}{}$\5
\2\&{else} \&{if} ${}(\R\\{addedD}[\\{ss}-\T{1}]){}$\1\5
${}\\{addedD}[\\{ss}-\T{1}]\K\T{1},\39\\{addto}(\\{subtotal},\39\\{binom}(\|m-%
\T{1},\39\\{ss}-\T{1}));{}$\2\6
\4${}\}{}$\2\6
\4${}\}{}$\2\2\6
\4${}\}{}$\2\par
\U12.\fi

\M{16}\B\X16:Correct for constant nonbeads\X${}\E{}$\6
\&{if} (\\{addedA}[\T{0}])\1\5
${}\\{nonbeads}\PP{}$;\C{ all 0s }\2\6
\&{if} ${}(\\{addedD}[\|m-\T{1}]){}$\1\5
${}\\{nonbeads}\PP{}$;\C{ all 1s }\2\par
\U10.\fi

\N{1}{17}Index.
\fi

\inx
\fin
\con
