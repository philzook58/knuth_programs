\input cwebmac
\datethis
\def\falling#1{^{\ff{#1}}}
\def\ff#1{\mkern1mu\underline{\mkern-1mu#1\mkern-2mu}\mkern2mu}


\N{1}{1}Intro. This program, inspired by {\mc HISTOSCAPE-COUNT}, calculates
the number of $m\times n$ ``whirlpool permutations,'' given $m$ and~$n$.

What's a whirlpool permutation, you ask? Good question.
An $m\times n$ matrix has $(m-1)(n-1)$ submatrices of size $2\times2$.
An $m\times n$ whirlpool permutation is a permutation of $(mn)!$
elements for which the relative order of the elements in each of those
submatrices is a ``vortex''---that is, it travels a cyclic path from
smallest to largest, either clockwise or counterclockwise.

Thus there are exactly eight $2\times2$ whirlpool permutations.
If the entries of the matrix are denoted $abcd$ from top to bottom
and left to right, they are 1243, 1423, 2134, 2314, 3241, 3421, 4132, 4312.
One simple test is to compare $a:b$, $b:d$, $d:c$, $c:a$; the number
of `$<$' must be odd. (Hence the number of `$>$' must also be odd.)

The enumeration is by a somewhat mind-boggling variant of dynamic programming
that I've not seen before. It needs to represent $n+1$ elements of
a permutation of~$t$ elements, where $t$ is at most $mn$,
and there are up to $(mn)\falling{n+1}$ such partial permutations;
so I can't expect to solve the problem unless $m$ and $n$ are
fairly small. Even so, when I {\it can\/} solve the problem it's kind of
thrilling, because this program makes use of a really interesting
way to represent $t\falling{n+1}$ counts in computer memory.

The same method could actually be used to enumerate matrices of permutations
whose $2\times2$ submatrices satisfy any arbitrary relations, when those
relations depend only the relative order of the four elements.
(Thus any of $2^{24}$ constraints might be prescribed for each of the
$(m-1)(n-1)$ submatrices. The whirlpool case, which accepts
only the eight relative orderings listed above, is just one of
zillions of possibilities.)

It's better to have $m\ge n$. But I'll try some cases with $m<n$ too,
for purposes of testing.

\Y\B\4\D$\\{maxn}$ \5
\T{8}\par
\B\4\D$\\{maxmn}$ \5
\T{36}\par
\B\4\D$\|o$ \5
$\\{mems}\PP{}$\par
\B\4\D$\\{oo}$ \5
$\\{mems}\MRL{+{\K}}{}$\T{2}\par
\B\4\D$\\{ooo}$ \5
$\\{mems}\MRL{+{\K}}{}$\T{3}\par
\Y\B\8\#\&{include} \.{<stdio.h>}\6
\8\#\&{include} \.{<stdlib.h>}\6
\&{int} \|m${},{}$ \|n;\C{ command-line parameters }\6
\&{unsigned} \&{long} \&{long} ${}{*}\\{count}{}$;\C{ the big array of counts }%
\6
\&{unsigned} \&{long} \&{long} \\{newcount}[\\{maxmn}];\C{ counts that will
replace old ones }\6
\&{unsigned} \&{long} \&{long} \\{mems};\C{ memory references to octabytes }\6
\&{int} ${}\|x[\\{maxn}+\T{1}]{}$;\C{ indices being looped over }\6
\&{int} ${}\\{ay}[\\{maxn}+\T{1}];{}$\6
\&{int} \|l[\\{maxmn}]${},{}$ \|u[\\{maxmn}];\6
\&{int} ${}\\{tpow}[\\{maxmn}+\T{1}]{}$;\C{ factorial powers $t\falling{n+1}$ }%
\7
\X4:Subroutines\X;\7
\\{main}(\&{int} \\{argc}${},\39{}$\&{char} ${}{*}\\{argv}[\,]){}$\1\1\2\2\6
${}\{{}$\1\6
\&{register} \&{int} \|a${},{}$ \|b${},{}$ \|c${},{}$ \|d${},{}$ \|i${},{}$ %
\|j${},{}$ \|k${},{}$ \|p${},{}$ \|q${},{}$ \|r${},{}$ \\{mn}${},{}$ \|t${},{}$
\\{tt}${},{}$ \\{kk}${},{}$ \\{bb}${},{}$ \\{cc}${},{}$ \\{pdel};\7
\X2:Process the command line\X;\6
\&{for} ${}(\|i\K\T{1};{}$ ${}\|i<\|m;{}$ ${}\|i\PP){}$\1\6
\&{for} ${}(\|j\K\T{0};{}$ ${}\|j<\|n;{}$ ${}\|j\PP){}$\1\5
\X8:Handle constraint $(i,j)$\X;\2\2\6
\X9:Print the grand total\X;\6
\4${}\}{}$\2\par
\fi

\M{2}\B\X2:Process the command line\X${}\E{}$\6
\&{if} ${}(\\{argc}\I\T{3}\V\\{sscanf}(\\{argv}[\T{1}],\39\.{"\%d"},\39{\AND}%
\|m)\I\T{1}\V\\{sscanf}(\\{argv}[\T{2}],\39\.{"\%d"},\39{\AND}\|n)\I\T{1}){}$\5
${}\{{}$\1\6
${}\\{fprintf}(\\{stderr},\39\.{"Usage:\ \%s\ m\ n\\n"},\39\\{argv}[\T{0}]);{}$%
\6
${}\\{exit}({-}\T{1});{}$\6
\4${}\}{}$\2\6
${}\\{mn}\K\|m*\|n;{}$\6
\&{if} ${}(\|m<\T{2}\V\|m>\\{maxn}\V\|n<\T{2}\V\|n>\\{maxn}\V\\{mn}>%
\\{maxmn}){}$\5
${}\{{}$\1\6
${}\\{fprintf}(\\{stderr},\39\.{"Sorry,\ m\ and\ n\ shou}\)\.{ld\ be\ between\
2\ and\ }\)\.{\%d,\ with\ mn<=\%d!\\n"},\39\\{maxn},\39\\{maxmn});{}$\6
${}\\{exit}({-}\T{2});{}$\6
\4${}\}{}$\2\6
\&{for} ${}(\|k\K\|n+\T{1};{}$ ${}\|k\Z\\{mn};{}$ ${}\|k\PP){}$\5
${}\{{}$\1\6
\&{register} \&{unsigned} \&{long} \&{long} \\{acc}${}\K\T{1};{}$\7
\&{for} ${}(\|j\K\T{0};{}$ ${}\|j\Z\|n;{}$ ${}\|j\PP){}$\1\5
${}\\{acc}\MRL{*{\K}}\|k-\|j;{}$\2\6
\&{if} ${}(\\{acc}\G\T{\^80000000}){}$\5
${}\{{}$\1\6
${}\\{fprintf}(\\{stderr},\39\.{"Sorry,\ mn\\\\falling(}\)\.{n+1)\ must\ be\
less\ th}\)\.{an\ 2\^31!\\n"});{}$\6
${}\\{exit}({-}\T{666});{}$\6
\4${}\}{}$\2\6
${}\\{tpow}[\|k]\K\\{acc};{}$\6
\4${}\}{}$\2\6
${}\\{count}\K{}$(\&{unsigned} \&{long} \&{long} ${}{*}){}$ \\{malloc}${}(%
\\{tpow}[\\{mn}]*{}$\&{sizeof}(\&{unsigned} \&{long} \&{long}));\6
\&{if} ${}(\R\\{count}){}$\5
${}\{{}$\1\6
${}\\{fprintf}(\\{stderr},\39\.{"I\ couldn't\ allocate}\)\.{\ \%d\ entries\ for%
\ the\ }\)\.{counts!\\n"},\39\\{tpow}[\\{mn}]);{}$\6
${}\\{exit}({-}\T{4});{}$\6
\4${}\}{}$\2\par
\U1.\fi

\M{3}Suppose I want to represent $n+1$ specified elements of a permutation
of $t+1$ elements. For example, we might have $n=3$ and $t=8$, and
the final four elements of a permutation $y_0\ldots y_8$ might be
$y_5y_6y_7y_8=3142$. There are $(t+1)\falling{n+1}$
such partial permutations, and I can represent them compactly with
$n+1$ integer variables $x_{t-n}$, \dots, $x_{t-1}$,~$x_t$, where
$0\le x_j\le j$. The rule is that $x_j$ is $y_j-w_j$,
where $w_j$ is the number of elements ``inverted'' by $y_j$
(the number of elements to the right of $y_j$ that are less than~$y_j$).
In the example, $w_0w_1w_2w_3=2010$, so $x_5x_6x_7x_8=1132$.
(Or going backward, if $x_5x_6x_7x_8=3141$ then $y_5y_6y_7y_8=6251$.)

This representation has a beautiful property that we shall exploit.
Every permutation $y_0\ldots y_t$ of $\{0,\ldots,t\}$ yields
$t+2$ permutations $y'_0\ldots y'_{t+1}$ of $\{0,\ldots,t+1\}$ if
we choose $y'_{t+1}$ arbitrarily,
and then set $y'_j=y_j+\hbox{[$y_j{\ge}y'_{t+1}$]}$.
For example, if $t=8$ and $y_5y_6y_7y_8=3142$, the ten permutations
obtained from $y_0\ldots y_8$ will have $y'_5y'_6y'_7y'_8y'_9=
42530$, 42531, 41532, 41523, 31524, 31425, 31426, 31427, 31428, or 31429.
And the representations $x'_5x'_6x'_7x'_8x'_9$ of those last five
elements will simply be respectively 31420, 31421, \dots, 31429!
In general, we'll have $x'_j=x_j$ for $0\le j\le t$, and $x'_{t+1}=y'_{t+1}$
will be arbitrary.

\fi

\M{4}Now comes the mind-boggling part.
I want to maintain a count $c(x_{t-n},\ldots,x_t)$ for each setting of
the indices $(x_{t-n},\ldots,x_t)$, but I want to put those counts
into memory in such a way that I won't clobber any of the existing
counts when I'm updating $t$ to $t+1$. In particular, if $x'_{t+1}\le t-n$,
I'll want $c(x'_{t+1-n},\ldots,x'_t,x'_{t+1})$ to be stored in exactly
the same place as $c(x'_{t+1},x_{t+1-n},\ldots,x_t)$ was stored
in the previous round. But if $x'_{t+1}>t-n$, I'll store
$c(x'_{t+1-n},\ldots,x'_t,x'_{t+1})$ in a brand-new, previously
unused location of memory.

Thus we shall use a memory mapping function $\mu_t$, different for each~$t$,
such that $c(x_{t-n},x_{t-n+1},\ldots,x_t)$ is stored in location
$\mu_t(x_{t-n},x_{t-n+1},\ldots,x_t)$ during round~$t$ of the computation.

Let's go back to the example in the previous section and apply it to
whirlpool permutations for $n=3$. Denote the permutation in the first
three rows by $y_0\ldots y_8$, where $y_6y_7y_8$ is the third row and
$y_5$ is the last element of the second row. (It's a permutation
of $\{0,\ldots,8\}$, representing the relative order of a final
permutation of $\{0,\ldots,3m-1\}$ that will fill the entire matrix.)
At this point we've
calculated counts $c(x_5,x_6,x_7,x_8)$ that tell us how many such
partial whirlpool permutations have any given setting of $y_5y_6y_7y_8$. In
particular, $c(1,1,3,2)$ counts those for which $y_5y_6y_7y_8=3142$.

To get to the next round, we essentially want to know how many partial
permutations $y'_0\ldots y'_9$ of $\{0,\ldots,9\}$ will have a given setting
of $y'_6y'_7y'_8y'_9$; the second row is now irrelevant to future computations.
It's the same as asking how many permutations have $y_6y_7y_8=142$.
Answer: $c(0,1,3,2)+c(1,1,3,2)+c(2,1,3,2)+c(3,1,3,2)+c(4,1,3,2)+c(5,1,3,2)$,
because these count the permutations with $y_5y_6y_7y_8=0142$, 3142,
5142, 6142, 7142, 8142.

Those six counts $c(k,1,3,2)$ appear in memory locations $\mu_8(k,1,3,2)$,
for $0\le k\le5$.
On the next round we'll want $c'(x'_6,x'_7,x'_8,x'_9)=c'(1,3,2,x'_9)$
to be set to their sum. These new counts will appear in memory locations
$\mu_9(1,3,2,x'_9)$. So we'd like $\mu_9(1,3,2,k)=\mu_8(k,1,3,2)$
when $0\le k\le5$.

Let
$\lambda_t(x_{t-n},\ldots,x_t)=
\bigl(\cdots((x_tt+x_{t-1})(t-1)+x_{t-2})\cdots\bigr)(t-n+1)+x_{t-n}=
x_t t\falling n+x_{t-1}(t-1)\falling{n-1}+\cdots x_{t-n}(t-n)\falling0$
be the standard mixed-radix representation of $(x_t\ldots x_{t-n})$
with radices $(t+1,t,\ldots,t-n+1)$. When each $x_j$ ranges from 0 to~$j$,
$\lambda_t(x_{t-n},\ldots,x_t)$ ranges from $\lambda_t(0,\ldots,0)=0$
to $\lambda_t(t-n,\ldots,t)=(t+1)\falling{n+1}-1$.
Therefore $\lambda_t$ would be the natural choice for $\mu_t$,
if we didn't want to avoid clobbering.

Instead, we use $\lambda_t$ only when $x_t$ is sufficiently large:
We define
$$\mu_t(x_{t-n},\ldots,x_t)=\cases{
\lambda_t(x_{t-n},\ldots,x_t),&if $x_t\ge t-n$;\cr
\mu_{t-1}(x_t,x_{t-n},\ldots,x_{t-1}),&if $x_t\le t-n-1$.\cr}$$
This recursion terminates with $\mu_n=\lambda_n$, because
$x_n$ is always $\ge0$. One can also show that
$\mu_{n+1}=\lambda_{n+1}$.

Back to our earlier example, what is $\mu_8(k,1,3,2)$? Since
$2\le4$, it's $\mu_7(2,k,1,3)$. And since $3\le3$, it's
$\mu_6(3,2,k,1)$. Which is $\mu_5(1,3,2,k)$. Finally, therefore,
if $k\le1$, the value is $\lambda_4(k,1,3,2)=68+k$;
but if $2\le k\le5$ it's $\lambda_5(1,3,2,k)=60k+34$.

In this program we will keep $x_j$ in location $x_{j\bmod(n+1)}$.
Consequently the arguments to $\mu_t$ and $\lambda_t$ will always be in
locations $(x_{(t+1)\bmod(n+1)},x_{(t+2)\bmod(n+1)},\ldots,
x_{t\bmod(n+1)})$.

\Y\B\4\X4:Subroutines\X${}\E{}$\6
\&{int} \\{mu}(\&{int} \|t)\1\1\2\2\6
${}\{{}$\1\6
\&{register} \&{int} \|r${},{}$ \|a${},{}$ \|p${},{}$ \\{tt};\7
\&{for} ${}(\|r\K\|t\MOD(\|n+\T{1}),\39\\{tt}\K\|t;{}$ \|o${},\39\|x[\|r]<%
\\{tt}-\|n;{}$ ${}\\{tt}\MM,\39\|r\K(\|r\?\|r-\T{1}:\|n)){}$\1\5
;\2\6
\&{for} ${}(\|o,\39\|p\K\|x[\|r],\39\|r\K(\|r\?\|r-\T{1}:\|n),\39\|a\K\T{0};{}$
${}\|a<\|n;{}$ ${}\|a\PP,\39\|r\K(\|r\?\|r-\T{1}:\|n)){}$\1\5
${}\|o,\39\|p\K\|p*(\\{tt}-\|a)+\|x[\|r];{}$\2\6
\&{return} \|p;\6
\4${}\}{}$\2\par
\U1.\fi

\M{5}A backtrack essentially like Algorithm 7.2.1.2X nicely runs
through all combinations of $x_{t-n+1}\ldots x_t$ and
$y_{t-n+1}\ldots y_t$ simultaneously, while also providing a linked list that
shows the possibilities for $y_{t-n}$ as $x_{t-n}$ varies
from 0 to~$t-n$.

The algorithm generates all of the ``$n$-variations'' of
$\{0,\ldots,t\}$, namely all $n$-tuples $a_0\ldots a_{n-1}$ of distinct
integers in that set, where $a_j$ corresponds to
$y_{t-j}$ in the discussion above.

\Y\B\4\X5:Generate the $x$'s and $y$'s\X${}\E{}$\6
\4\\{x1}:\5
\&{for} ${}(\|k\K\T{0};{}$ ${}\|k\Z\|t;{}$ ${}\|k\PP){}$\1\5
${}\|o,\39\|l[\|k]\K\|k+\T{1};{}$\2\6
${}\|o,\39\|l[\|t+\T{1}]\K\T{0}{}$;\C{ circularly linked list }\6
${}\|k\K\T{0},\39\\{kk}\K\|t\MOD(\|n+\T{1});{}$\6
\4\\{x2}:\5
\&{if} ${}(\|k\E\|n){}$\1\5
\X6:Visit $a_0\ldots a_{n-1}$ and \PB{\&{goto} \\{x6}}\X;\2\6
${}\\{oo},\39\|p\K\|t+\T{1},\39\|q\K\|l[\|p],\39\|x[\\{kk}]\K\T{0};{}$\6
\4\\{x3}:\5
${}\|o,\39\\{ay}[\|k]\K\|q;{}$\6
\4\\{x4}:\5
${}\\{ooo},\39\|u[\|k]\K\|p,\39\|l[\|p]\K\|l[\|q],\39\|k\PP,\39\\{kk}\K(\\{kk}%
\?\\{kk}-\T{1}:\|n);{}$\6
\&{goto} \\{x2};\6
\4\\{x5}:\5
${}\|o,\39\|p\K\|q,\39\|q\K\|l[\|p];{}$\6
\&{if} ${}(\|q\Z\|t){}$\5
${}\{{}$\1\6
${}\\{oo},\39\|x[\\{kk}]\PP;{}$\6
\&{goto} \\{x3};\6
\4${}\}{}$\2\6
\4\\{x6}:\5
\&{if} ${}(\MM\|k\G\T{0}){}$\5
${}\{{}$\1\6
${}\\{kk}\K(\\{kk}\E\|n\?\T{0}:\\{kk}+\T{1});{}$\6
${}\\{ooo},\39\|p\K\|u[\|k],\39\|q\K\\{ay}[\|k],\39\|l[\|p]\K\|q;{}$\6
\&{goto} \\{x5};\6
\4${}\}{}$\2\par
\U8.\fi

\M{6}At this point we're ready to do the ``inner loop'' calculation,
by using all counts $c(x_{t-n},\ldots,x_t)$ for $0\le x_{t-n}\le t-n$
to obtain updated counts that will allow us to increase~$t$. The array
$a_{n-1}\ldots a_0$ corresponds to $y_{t-n+1}\ldots y_t$ in the discussion
above; we want to loop over all choices for $y_{t-n}$, namely all
choices for $a_n$. Fortunately there's a linked list containing precisely
those choices, beginning at \PB{$\|l[\|t+\T{1}]$}.

\Y\B\4\X6:Visit $a_0\ldots a_{n-1}$ and \PB{\&{goto} \\{x6}}\X${}\E{}$\6
${}\{{}$\1\6
\X7:If possible, find \PB{\|p} and \PB{\\{pdel}} so that $c(x_{t-n},%
\ldots,x_t)$ is \PB{$\\{count}[\|p+\\{pdel}*\|x[\\{kk}]]$}\X;\6
\&{for} ${}(\|d\K\T{0};{}$ ${}\|d\Z\|t+\T{1};{}$ ${}\|d\PP){}$\1\5
${}\|o,\39\\{newcount}[\|d]\K\T{0};{}$\2\6
${}\\{oo},\39\|b\K\\{ay}[\|n-\T{1}],\39\|c\K\\{ay}[\T{0}];{}$\6
\&{if} ${}(\|b<\|c){}$\1\5
${}\\{bb}\K\|b,\39\\{cc}\K\|c;{}$\2\6
\&{else}\1\5
${}\\{bb}\K\|c,\39\\{cc}\K\|b{}$;\C{ min and max }\2\6
${}\{{}$\1\6
\&{register} \&{unsigned} \&{long} \&{long} \\{tmp};\7
\&{for} ${}(\\{oo},\39\|a\K\|l[\|t+\T{1}],\39\|x[\\{kk}]\K\T{0};{}$ ${}\|a\Z%
\|t;{}$ \\{oo}${},\39\|a\K\|l[\|a],\39\|x[\\{kk}]\PP){}$\5
${}\{{}$\1\6
\&{if} (\\{pdel})\1\5
${}\\{tmp}\K\\{count}[\|p+\|x[\\{kk}]*\\{pdel}];{}$\2\6
\&{else}\1\5
${}\\{tmp}\K\\{count}[\\{mu}(\|t-\|n)]{}$;\C{ if \PB{$\\{pdel}\K\T{0}$} then %
\PB{$\\{mu}(\|t)\K\\{mu}(\|t-\|n)$} }\2\6
\&{if} ${}(\|j\E\T{0}){}$\1\5
${}\\{newcount}[\T{0}]\MRL{+{\K}}\\{tmp}{}$;\C{ no constraint, beginning a new
row }\2\6
\&{else} \&{if} ${}(\|a<\\{bb}\V\|a>\\{cc}){}$\5
${}\{{}$\C{ whirlpool constraint when \PB{\|a} not middle }\1\6
\&{for} ${}(\|d\K\\{bb}+\T{1};{}$ ${}\|d\Z\\{cc};{}$ ${}\|d\PP){}$\1\5
${}\\{oo},\39\\{newcount}[\|d]\MRL{+{\K}}\\{tmp};{}$\2\6
\4${}\}{}$\5
\2\&{else}\5
${}\{{}$\C{ whirlpool constraint when \PB{\|d} not middle }\1\6
\&{for} ${}(\|d\K\T{0};{}$ ${}\|d\Z\\{bb};{}$ ${}\|d\PP){}$\1\5
${}\\{oo},\39\\{newcount}[\|d]\MRL{+{\K}}\\{tmp};{}$\2\6
\&{for} ${}(\|d\K\\{cc}+\T{1};{}$ ${}\|d\Z\|t+\T{1};{}$ ${}\|d\PP){}$\1\5
${}\\{oo},\39\\{newcount}[\|d]\MRL{+{\K}}\\{tmp};{}$\2\6
\4${}\}{}$\2\6
\4${}\}{}$\2\6
\&{if} (\\{pdel})\5
${}\{{}$\1\6
\&{for} ${}(\|d\K\T{0};{}$ ${}\|d\Z\|t-\|n;{}$ ${}\|d\PP){}$\1\5
${}\\{oo},\39\\{count}[\|p+\|d*\\{pdel}]\K\\{newcount}[\|j\?\|d:\T{0}];{}$\2\6
\&{for} ( ; ${}\|d\Z\|t+\T{1};{}$ ${}\|d\PP){}$\1\5
${}\\{ooo},\39\|x[\\{kk}]\K\|d,\39\\{count}[\\{mu}(\|t+\T{1})]\K\\{newcount}[%
\|j\?\|d:\T{0}];{}$\2\6
\4${}\}{}$\5
\2\&{else}\5
${}\{{}$\1\6
\&{for} ${}(\|d\K\T{0};{}$ ${}\|d\Z\|t+\T{1};{}$ ${}\|d\PP){}$\1\5
${}\\{ooo},\39\|x[\\{kk}]\K\|d,\39\\{count}[\\{mu}(\|t+\T{1})]\K\\{newcount}[%
\|j\?\|d:\T{0}];{}$\2\6
\4${}\}{}$\2\6
\4${}\}{}$\2\6
\&{goto} \\{x6};\6
\4${}\}{}$\2\par
\U5.\fi

\M{7}Our example of $\mu_8(k,1,3,2)$ shows that the mission of this
step is sometimes impossible. But the addressing scheme is usually simple,
so I decided to exploit that fact. (Being aware, of course, that
premature optimization is the root of all evil in programming.)

\Y\B\4\X7:If possible, find \PB{\|p} and \PB{\\{pdel}} so that $c(x_{t-n},%
\ldots,x_t)$ is \PB{$\\{count}[\|p+\\{pdel}*\|x[\\{kk}]]$}\X${}\E{}$\6
\&{for} ${}(\\{tt}\K\|t,\39\|a\K\T{0},\39\|r\K\|t\MOD(\|n+\T{1});{}$ ${}\|a<%
\|n;{}$ ${}\|a\PP,\39\\{tt}\MM,\39\|r\K(\|r\?\|r-\T{1}:\|n)){}$\1\6
\&{if} ${}(\|o,\39\|x[\|r]\G\\{tt}-\|n){}$\1\5
\&{break};\2\2\6
\&{if} ${}(\|a\E\|n){}$\1\5
${}\\{pdel}\K\T{0}{}$;\C{ a difficult case }\2\6
\&{else}\5
${}\{{}$\1\6
\&{for} ${}(\|p\K\\{pdel}\K\T{0},\39\|a\K\T{0};{}$ ${}\|a\Z\|n;{}$ ${}\|a\PP,%
\39\|r\K(\|r\?\|r-\T{1}:\|n)){}$\5
${}\{{}$\1\6
\&{if} ${}(\|r\I\\{kk}){}$\1\5
${}\|p\K\|p*(\\{tt}+\T{1}-\|a)+\|x[\|r],\39\\{pdel}\K\\{pdel}*(\\{tt}+\T{1}-%
\|a);{}$\2\6
\&{else}\1\5
${}\|p\K\|p*(\\{tt}+\T{1}-\|a),\39\\{pdel}\K\\{pdel}*(\\{tt}+\T{1}-\|a)+%
\T{1};{}$\2\6
\4${}\}{}$\2\6
\4${}\}{}$\2\par
\U6.\fi

\M{8}\B\X8:Handle constraint $(i,j)$\X${}\E{}$\6
${}\{{}$\1\6
${}\|t\K\|n*\|i+\|j-\T{1};{}$\6
\&{if} ${}(\|t<\|n){}$\5
${}\{{}$\1\6
\&{for} ${}(\|p\K\T{0};{}$ ${}\|p<\\{tpow}[\|n+\T{1}];{}$ ${}\|p\PP){}$\1\5
${}\|o,\39\\{count}[\|p]\K\T{1};{}$\2\6
\&{continue};\6
\4${}\}{}$\2\6
\X5:Generate the $x$'s and $y$'s\X;\6
${}\\{fprintf}(\\{stderr},\39\.{"\ done\ with\ \%d,\%d\ ..}\)\.{\%lld,\ \%lld\
mems\\n"},\39\|i,\39\|j,\39\\{count}[\T{0}],\39\\{mems});{}$\6
\4${}\}{}$\2\par
\U1.\fi

\M{9}\B\D$\\{thresh}$ \5
\T{1000000000000000000}\par
\Y\B\4\X9:Print the grand total\X${}\E{}$\6
\&{for} ${}(\\{newcount}[\T{0}]\K\\{newcount}[\T{1}]\K\\{newcount}[\T{2}]\K%
\T{0},\39\|p\K\\{tpow}[\\{mn}]-\T{1};{}$ ${}\|p\G\T{0};{}$ ${}\|p\MM){}$\5
${}\{{}$\1\6
\&{if} ${}(\\{count}[\|p]>\\{newcount}[\T{2}]){}$\1\5
${}\\{newcount}[\T{2}]\K\\{count}[\|p],\39\\{pdel}\K\|p;{}$\2\6
${}\|o,\39\\{newcount}[\T{0}]\MRL{+{\K}}\\{count}[\|p];{}$\6
\&{if} ${}(\\{newcount}[\T{0}]\G\\{thresh}){}$\1\5
${}\\{ooo},\39\\{newcount}[\T{0}]\MRL{-{\K}}\\{thresh},\39\\{newcount}[\T{1}]%
\PP;{}$\2\6
\4${}\}{}$\2\6
${}\\{fprintf}(\\{stderr},\39\.{"(Maximum\ count\ \%lld}\)\.{\ is\ obtained\
for\ par}\)\.{ams"},\39\\{newcount}[\T{2}])\hbox{)};{}$\6
\&{for} ${}(\|q\K\\{mn}-\|n-\T{1};{}$ ${}\|q<\\{mn};{}$ ${}\|q\PP){}$\5
${}\{{}$\1\6
${}\\{fprintf}(\\{stderr}\hbox{)},\39\.{"\ \%d"},\39\\{pdel}\MOD(\|q+%
\T{1}));{}$\6
${}\\{pdel}\MRL{{/}{\K}}\|q+\T{1};{}$\6
\4${}\}{}$\2\6
${}\\{fprintf}(\\{stderr},\39\.{")\\n"}\hbox{(});{}$\6
\&{if} ${}(\\{newcount}[\T{1}]\E\T{0}){}$\1\5
${}\\{printf}(\.{"Altogether\ \%lld\ \%dx}\)\.{\%d\ whirlpool\ perms\ (}\)\.{%
\%lld\ mems).\\n"},\39\\{newcount}[\T{0}],\39\|m,\39\|n,\39\\{mems});{}$\2\6
\&{else}\1\5
${}\\{printf}(\.{"Altogether\ \%lld\%018}\)\.{lld\ \%dx\%d\ whirlpool\ }\)%
\.{perms\ (\%lld\ mems).\\n}\)\.{"},\39\\{newcount}[\T{1}],\39\\{newcount}[%
\T{0}],\39\|m,\39\|n,\39\\{mems}){}$;\2\par
\U1.\fi

\N{1}{10}Index.
\fi

\inx
\fin
\con
