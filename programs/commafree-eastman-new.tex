\input cwebmac
\datethis
\everypar{\looseness=-1}


\N{1}{1}Intro. This program is an iterative implementation of
an interesting recursive algorithm due to
Willard~L. Eastman, {\sl IEEE Trans.\ \bf IT-11} (1965), 263--267:
Given a sequence of nonnegative integers $x=x_0x_1\ldots x_{n-1}$ of
odd length~$n$, where $x$ is not equal to any of its cyclic shifts
$x_k\ldots x_{n-1}x_0\ldots x_{k-1}$ for $1\le k<n$, we output a
cyclic shift $\sigma x$ such that the set of all such $\sigma x$
forms a commafree code of block length~$n$ (over an infinite alphabet).

The integers are given as command-line arguments.

The simplest nontrivial example occurs when $n=3$. If $x=abc$, where
$a$, $b$, and~$c$ aren't all equal, then exactly one of the cyclic shifts
$y_0y_1y_2=abc$, $bca$, $cab$ will satisfy $y_0\ge y_1<y_2$, and we
choose that one. It's easy to check that the triples chosen in
this way are commafree.

Similar constructions are possible when $n=5$ or $n=7$. But the case
$n=9$ already gets a bit dicey, and when $n$ is really large it's not
at all clear that commafreeness is possible. Eastman's paper resolved
a conjecture made by Golomb, Gordon, and Welch in their pioneering paper about
comma-free codes (1958).

(Of course, it's not at all clear
that we would want to actually {\it use\/} a commafree code when
$n$ is large; but that's another story, and beside the point. The point
is that Eastman discovered a really interesting algorithm.)

Note: This program was written after I presented a lecture about
Eastman's algorithm at Stanford on 3~December 2015. While preparing
the lecture I realized that some nice structure was present, and
a day later it occurred to me that the algorithm could therefore
be streamlined. This program significantly simplifies
the method of the previous one, which was called {\mc COMMAFREE-EASTMAN}.
It produces essentially the same outputs, but they are reflected left-to-right.
(More precisely, if the former program gave the codeword $y$ from the
the input sequence $x=x_0\ldots x_{n-1}$, this program gives
the reverse of~$y$ when given the reverse of~$x$.)

\Y\B\4\D$\\{maxn}$ \5
\T{105}\par
\Y\B\8\#\&{include} \.{<stdio.h>}\6
\8\#\&{include} \.{<stdlib.h>}\6
\&{int} ${}\|x[\\{maxn}+\\{maxn}];{}$\6
\&{int} ${}\|b[\\{maxn}+\\{maxn}];{}$\6
\&{int} \\{bb}[\\{maxn}];\7
\X5:Subroutines\X;\7
\\{main}(\&{int} \\{argc}${},\39{}$\&{char} ${}{*}\\{argv}[\,]){}$\1\1\2\2\6
${}\{{}$\1\6
\&{register} \&{int} \|i${},{}$ \\{i0}${},{}$ \|j${},{}$ \|k${},{}$ \|n${},{}$ %
\|p${},{}$ \|q${},{}$ \|t${},{}$ \\{tt};\7
\X2:Process the command line\X;\6
\X3:Do Eastman's algorithm\X;\6
\X8:Print the solution\X;\6
\4${}\}{}$\2\par
\fi

\M{2}\B\X2:Process the command line\X${}\E{}$\6
\&{if} ${}(\\{argc}<\T{4}){}$\5
${}\{{}$\1\6
${}\\{fprintf}(\\{stderr},\39\.{"Usage:\ \%s\ x1\ x2\ ...}\)\.{\ xn\\n"},\39%
\\{argv}[\T{0}]);{}$\6
${}\\{exit}({-}\T{1});{}$\6
\4${}\}{}$\2\6
${}\|n\K\\{argc}-\T{1};{}$\6
\&{if} ${}((\|n\AND\T{1})\E\T{0}){}$\5
${}\{{}$\1\6
${}\\{fprintf}(\\{stderr},\39\.{"The\ number\ of\ items}\)\.{,\ n,\ should\ be\
odd,\ }\)\.{not\ \%d!\\n"},\39\|n);{}$\6
${}\\{exit}({-}\T{2});{}$\6
\4${}\}{}$\2\6
\&{for} ${}(\|j\K\T{1};{}$ ${}\|j<\\{argc};{}$ ${}\|j\PP){}$\5
${}\{{}$\1\6
\&{if} ${}(\\{sscanf}(\\{argv}[\|j],\39\.{"\%d"},\39{\AND}\|x[\|j-\T{1}])\I%
\T{1}\V\|x[\|j-\T{1}]<\T{0}){}$\5
${}\{{}$\1\6
${}\\{fprintf}(\\{stderr},\39\.{"Argument\ \%d\ should\ }\)\.{be\ a\
nonnegative\ int}\)\.{eger,\ not\ `\%s'!\\n"},\39\|j,\39\\{argv}[\|j]);{}$\6
${}\\{exit}({-}\T{3});{}$\6
\4${}\}{}$\2\6
\4${}\}{}$\2\par
\U1.\fi

\N{1}{3}The algorithm.
We think of $x$ as written cyclically, with $x_{n+j}=x_j$ for all~$j\ge0$.
The basic idea in the algorithm below is to also think of $x$ as partitioned
into $t\le n$ subwords by boundary markers $b_j$ where $0\le b_0<b_1<\cdots
<b_{t-1}<n$; then subword $y_j$ is $x_{b_j}x_{b_j+1}\ldots x_{b_{j+1}-1}$,
for $0\le j<t$, where $b_t=b_0$. If $t=1$, there's just one subword,
and it's a cyclic shift of~$x$. The number $t$ of subwords during each phase
will be odd.

Eastman's algorithm essentially begins with $b_j=j$ for $0\le j<n$, so that
$x$ is partitioned into $n$ subwords of length~1.
It successively {\it removes\/}
boundary points until only one subword is left; that subword is the answer.
It operates in phases, so that all subwords during the $j$th phase have
length $3^{j-1}$ or more; thus at most $\lfloor\log_3n\rfloor$ phases
are needed. (For example, the case $n=9$ is ``dicey'' because it might
require two phases.)

The algorithm is based on comparison of adjacent subwords $y_{j-1}$ and~$y_j$.
If those subwords have the same length, we use lexicographic comparison;
otherwise we declare that the longer subword is bigger.

The algorithm is based on an interesting factorization of strings into
substrings that have the form $z=z_1\ldots z_k$ where $k\ge2$ and
$z_1\ge\cdots\ge z_{k-1}<z_k$. Let's call such a substring a ``dip.''
It is not difficult to see that any string $y=y_0y_1\ldots{}$ in
which the condition $y_i<y_{i+1}$ occurs infinitely often can be
factored {\it uniquely\/} as a sequence of dips,
$y=z^{(0)}z^{(1)}\ldots{}\,$. For example,
$3141592653589\ldots{}=
314\,15\,926\,535\,89\,\dots{}\,$.

Furthermore if $y$ is a periodic sequence, its factorization is also
ultimately periodic, although some of its initial factors may not occur in the
period. Consider, for example, the factorizations
$$\displaylines{
1234501234501234501\ldots{}=12\,34\,501\,23\,45\,01\,23\,45\,01\,\ldots{}\,;\cr
1234560123456012345601\ldots{}=
12\,34\,56\,01\,23\,45\,601\,23\,45\,601\,\ldots{}\,.\cr}$$
If the period length is~$t$, and if $i_0$ is the smallest~$i$ such that
$y_{i-3}\ge y_{i-2}<y_{i-1}$, then one of the factors ends at~$i_0$
and all factors are periodic after that point. The value of~$i_0$ is
at most~$t+2$.

Since $t$ is odd, the period contains an odd number of dips of odd length.
Each phase of Eastman's algorithm simply retains the boundary points that
precede those odd dips.

\Y\B\4\X3:Do Eastman's algorithm\X${}\E{}$\6
\X4:Initialize\X;\6
\&{for} ${}(\|p\K\T{1},\39\|t\K\|n;{}$ ${}\|t>\T{1};{}$ ${}\|t\K\\{tt},\39\|p%
\PP){}$\1\5
\X6:Do one phase of Eastman's algorithm, putting \PB{\\{tt}} boundary points
into \PB{\\{bb}}\X;\2\par
\U1.\fi

\M{4}We might need to refer to \PB{$\|b[\|n+\|n-\T{1}]$}, but not \PB{$\|b[\|n+%
\|n]$}.

\Y\B\4\X4:Initialize\X${}\E{}$\6
\&{for} ${}(\|j\K\|n;{}$ ${}\|j<\|n+\|n;{}$ ${}\|j\PP){}$\1\5
${}\|x[\|j]\K\|x[\|j-\|n];{}$\2\6
\&{for} ${}(\|j\K\T{0};{}$ ${}\|j<\|n+\|n;{}$ ${}\|j\PP){}$\1\5
${}\|b[\|j]\K\|j;{}$\2\6
${}\|t\K\|n{}$;\par
\U3.\fi

\M{5}Here's a basic subroutine that returns 1 if subword $y_{i-1}$ is less than
subword~$y_i$, otherwise it returns~0.

\Y\B\4\X5:Subroutines\X${}\E{}$\6
\&{int} \\{less}(\&{register} \&{int} \|i)\1\1\2\2\6
${}\{{}$\1\6
\&{register} \&{int} \|j;\7
\&{if} ${}(\|b[\|i]-\|b[\|i-\T{1}]\E\|b[\|i+\T{1}]-\|b[\|i]){}$\5
${}\{{}$\1\6
\&{for} ${}(\|j\K\T{0};{}$ ${}\|b[\|i]+\|j<\|b[\|i+\T{1}];{}$ ${}\|j\PP){}$\5
${}\{{}$\1\6
\&{if} ${}(\|x[\|b[\|i-\T{1}]+\|j]\E\|x[\|b[\|i]+\|j]){}$\1\5
\&{continue};\2\6
\&{return} ${}(\|x[\|b[\|i-\T{1}]+\|j]<\|x[\|b[\|i]+\|j]);{}$\6
\4${}\}{}$\2\6
\&{return} \T{0};\C{ $y_{i-1}=y_i$ }\6
\4${}\}{}$\2\6
\&{return} ${}(\|b[\|i]-\|b[\|i-\T{1}]<\|b[\|i+\T{1}]-\|b[\|i]);{}$\6
\4${}\}{}$\2\par
\U1.\fi

\M{6}\B\X6:Do one phase of Eastman's algorithm, putting \PB{\\{tt}} boundary
points into \PB{\\{bb}}\X${}\E{}$\6
${}\{{}$\1\6
\&{for} ${}(\|i\K\T{1};{}$  ; ${}\|i\PP){}$\1\6
\&{if} ${}(\R\\{less}(\|i)){}$\1\5
\&{break};\C{ now \PB{$\|i\Z\|t$} and \PB{$\|y[\|i-\T{1}]\G\|y[\|i]$} }\2\2\6
\&{for} ${}(\|i\MRL{+{\K}}\T{2};{}$ ${}\|i\Z\|t+\T{2};{}$ ${}\|i\PP){}$\1\6
\&{if} ${}(\\{less}(\|i-\T{1})){}$\1\5
\&{break};\2\2\6
\&{if} ${}(\|i>\|t+\T{2}){}$\5
${}\{{}$\1\6
${}\\{fprintf}(\\{stderr},\39\.{"The\ input\ is\ cyclic}\)\.{!\\n"});{}$\6
${}\\{exit}({-}\T{666});{}$\6
\4${}\}{}$\C{ now \PB{$\|y[\|i-\T{3}]\G\|y[\|i-\T{2}]<\|y[\|i-\T{1}]$} }\2\6
\&{if} ${}(\|i<\|t){}$\1\5
${}\\{i0}\K\|i{}$;\5
\2\&{else}\1\5
${}\|i\K\\{i0}\K\|i-\|t;{}$\2\6
\&{for} ${}(\\{tt}\K\T{0};{}$ ${}\|i<\\{i0}+\|t;{}$ ${}\|i\K\|j){}$\5
${}\{{}$\1\6
\&{for} ${}(\|j\K\|i+\T{2};{}$  ; ${}\|j\PP){}$\1\6
\&{if} ${}(\\{less}(\|j-\T{1})){}$\1\5
\&{break};\C{ advance past the next dip }\2\2\6
\&{if} ${}((\|j-\|i)\AND\T{1}){}$\1\5
\X7:Retain \PB{\|i} as a boundary point\X;\2\6
\4${}\}{}$\2\6
${}\\{printf}(\.{"Phase\ \%d\ leaves"},\39\|p);{}$\6
\&{for} ${}(\|k\K\T{0};{}$ ${}\|k<\\{tt};{}$ ${}\|k\PP){}$\1\5
${}\|b[\|k]\K\\{bb}[\|k],\39\\{printf}(\.{"\ \%d"},\39\\{bb}[\|k]);{}$\2\6
\\{printf}(\.{"\\n"});\6
\&{for} ( ; ${}\|b[\|k-\\{tt}]<\|n+\|n;{}$ ${}\|k\PP){}$\1\5
${}\|b[\|k]\K\|b[\|k-\\{tt}]+\|n;{}$\2\6
\4${}\}{}$\2\par
\U3.\fi

\M{7}If \PB{$\|i\G\|t$} at this point, we have ``wrapped around,'' so we
actually
want to retain the boundary point \PB{$\|i-\|t$}. (This case will arise at most
once per phase, because \PB{$\|j\G\|i+\T{3}$} and we must have \PB{$\|j\K%
\\{i0}+\|t$}. Therefore
\PB{$\|i-\|t$} will be smaller than all of the previously retained points, and
we want to shift them one space to the right.)

\Y\B\4\X7:Retain \PB{\|i} as a boundary point\X${}\E{}$\6
${}\{{}$\1\6
\&{if} ${}(\|i<\|t){}$\1\5
${}\\{bb}[\\{tt}\PP]\K\|b[\|i];{}$\2\6
\&{else}\5
${}\{{}$\1\6
\&{for} ${}(\|k\K\\{tt}\PP;{}$ ${}\|k>\T{0};{}$ ${}\|k\MM){}$\1\5
${}\\{bb}[\|k]\K\\{bb}[\|k-\T{1}];{}$\2\6
${}\\{bb}[\T{0}]\K\|b[\|i-\|t];{}$\6
\4${}\}{}$\2\6
\4${}\}{}$\2\par
\U6.\fi

\M{8}\B\X8:Print the solution\X${}\E{}$\6
\&{for} ${}(\|j\K\|b[\T{0}];{}$ ${}\|j<\|b[\T{0}]+\|n;{}$ ${}\|j\PP){}$\1\5
${}\\{printf}(\.{"\ \%d"},\39\|x[\|j]);{}$\2\6
\\{printf}(\.{"\\n"});\par
\U1.\fi

\N{1}{9}Index.
\fi

\inx
\fin
\con
