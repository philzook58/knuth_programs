\input cwebmac
\datethis

\N{1}{1}Intro. This program is the fourteenth in a series of exploratory
studies by
which I'm attempting to gain first-hand experience with OBDD structures, as I
prepare Section 7.1.4 of {\sl The Art of Computer Programming}.
It's basically the same as {\mc BDD11}, but extended to include some
rudimentary methods for changing and ``sifting'' the order of variables.

In this program I try to implement simplified versions of the basic routines
that are needed in a ``large'' BDD package.

\def\<#1>{\hbox{$\langle\,$#1$\,\rangle$}}
\chardef\ttv='174 % vertical line
\chardef\tta='046 % ampersand
\chardef\tth='136 % hat
\chardef\ttt='176 % tilde
The computation is governed by primitive commands in a language called BDDL;
these commands can either be
read from a file or typed online (or both).
BDDL commands have the following simple syntax, where \<number> denotes
a nonnegative decimal integer:
$$\eqalign{
&\<const>\gets\.{c0}\mid\.{c1}\cr
&\<var>\gets\.x\<number>\cr
&\<func>\gets\.f\<number>\cr
&\<atom>\gets\<const>\mid\<var>\mid\<func>\cr
&\<expr>\gets\<unop>\<atom>\mid
\<atom>\<binop>\<atom>\mid
\<atom>\.{[y]}\mid\cr
&\hskip15em \<atom>\<ternop>\<atom>\<ternop>\<atom>\cr
&\<command>\gets\<special>\mid\<func>\.=\<expr>\mid\<func>\.{=.}\mid
\.y\<number>\.=\<atom>\mid\.y\<number>\.{=.}\cr}$$
The special commands \<special>,
the unary operators \<unop>, the binary operators \<binop>, and the
ternary operators \<ternop> are explained below. One short example
will give the general flavor: After the commands
$$\halign{\qquad\tt#\hfil\cr
f1=x1{\tth}x2\cr
f2=x3{\ttv}x4\cr
f1=f1{\tta}f2\cr
f2={\ttt}f1\cr}$$
the function $f_1$ will be
$(x_1\oplus x_2)\land(x_3\lor x_4)$,
and $f_2$ will be $\lnot f_1$.
Then `\.{f1=.}' will undefine~$f_1$.

If the command line specifies an input file, all commands are taken
from that file and standard input is ignored. Otherwise the user is
prompted for commands.

For simplicity, I do my own memory allocation in a big array
called \PB{\\{mem}}. The bottom part of that array is devoted to
BDD nodes, which each occupy two octabytes. The upper part
is divided into dynamically allocated pages of a fixed size
(usually 4096 bytes). The cache of computed results, and
the hash tables for each variable, are kept in arrays whose elements
appear in the upper pages. These elements
need not be consecutive, because the $k$th byte of each dynamic array
is kept in location \PB{$\\{mem}[\|b[\|k\GG\T{12}]+(\|k\AND\T{\^fff})]$}, for
some array~\PB{\|b} of base
addresses.

Each node of the BDD base is responsible for roughly 28 bytes in \PB{\\{mem}},
assuming 16 bytes for the node itself, plus about 8 for its entry in
a hash table, plus about 4 for its entry in a cache. (I could reduce
the storage cost from 28 to 21 by choosing algorithms that run slower; but I
decided to give up some space in the interests of time. For example,
I'm devoting four bytes to each reference count, so that there's no
need to consider saturation. And this program uses linear probing for
its hash tables, at the expense of about 3 bytes per node, because
I like the sequential memory accesses of linear probing.)

Many compile-time parameters affect the sizes of various tables and the
heuristic strategies of various methods adopted here.
To browse through them all, see the entry ``Tweakable parameters''
in the index at the end.

\fi

\M{2}Here's the overall program structure:

\Y\B\8\#\&{include} \.{<stdio.h>}\6
\8\#\&{include} \.{<stdlib.h>}\6
\8\#\&{include} \.{<ctype.h>}\6
\8\#\&{include} \.{"gb\_flip.h"}\C{ random number generator }\6
\8\#\&{define} \\{verbose} \5\\{Verbose}\C{ because `\PB{\\{verbose}}' is \PB{%
\&{long}} in libgb }\6
\X10:Type definitions\X\6
\X4:Global variables\X\6
\X25:Templates for subroutines\X\6
\X7:Subroutines\X\7
\\{main}(\&{int} \\{argc}${},\39{}$\&{char} ${}{*}\\{argv}[\,]){}$\1\1\2\2\6
${}\{{}$\1\6
\X19:Local variables\X;\7
\X3:Check the command line\X;\6
\X5:Initialize everything\X;\6
\&{while} (\T{1})\1\5
\X110:Read a command and obey it; \PB{\&{goto} \\{alldone}} if done\X;\2\6
\4\\{alldone}:\5
\X6:Print statistics about this run\X;\6
\\{exit}(\T{0});\C{ normal termination }\6
\4${}\}{}$\2\par
\fi

\M{3}\B\D$\\{file\_given}$ \5
$(\\{argc}\E\T{2}{}$)\par
\Y\B\4\X3:Check the command line\X${}\E{}$\6
\&{if} ${}(\\{argc}>\T{2}\V(\\{file\_given}\W\R(\\{infile}\K\\{fopen}(\\{argv}[%
\T{1}],\39\.{"r"})))){}$\5
${}\{{}$\1\6
${}\\{fprintf}(\\{stderr},\39\.{"Usage:\ \%s\ [commandf}\)\.{ile]\\n"},\39%
\\{argv}[\T{0}]);{}$\6
${}\\{exit}({-}\T{1});{}$\6
\4${}\}{}$\2\par
\U2.\fi

\M{4}\B\X4:Global variables\X${}\E{}$\6
\&{FILE} ${}{*}\\{infile}{}$;\C{ input file containing commands }\6
\&{int} \\{verbose}${}\K{-}\T{1}{}$;\C{ master control for debugging output;
$-1$ gives all }\par
\As8, 13, 22, 30, 40, 42, 51, 59, 101, 112, 132, 136, 154, 158\ETs161.
\U2.\fi

\M{5}\B\X5:Initialize everything\X${}\E{}$\6
\\{gb\_init\_rand}(\T{0});\C{ initialize the random number generator }\par
\As9, 12, 23, 44\ETs62.
\U2.\fi

\M{6}One of the main things I hope to learn with this program is the total
number of \PB{\\{mems}} that the computation needs, namely the total number of
memory references to octabytes.

I'm not sure how many mems to charge for recursion overhead. A machine
like \.{MMIX} needs to use memory only when the depth gets sufficiently
deep that 256 registers aren't enough; then it needs two mems for
each saved item (one to push it and another to pop it). Most
of \.{MMIX}'s recursive activity takes place in the deepest levels, whose
parameters never need to descend to memory. So I'm making a separate
count of \PB{\\{rmems}}, the number of entries to recursive subroutines.

Some of the mems are classified as \PB{\\{zmems}}, because they arise only when
zeroing out pages of memory during initializations.

\Y\B\4\D$\|o$ \5
$\\{mems}\PP{}$\C{ a convenient macro for instrumenting a memory access }\par
\B\4\D$\\{oo}$ \5
$\\{mems}\MRL{+{\K}}{}$\T{2}\par
\B\4\D$\\{ooo}$ \5
$\\{mems}\MRL{+{\K}}{}$\T{3}\par
\B\4\D$\\{oooo}$ \5
$\\{mems}\MRL{+{\K}}{}$\T{4}\par
\Y\B\4\X6:Print statistics about this run\X${}\E{}$\6
\\{printf}(\.{"Job\ stats:\\n"});\6
${}\\{printf}(\.{"\ \ \%llu\ mems\ plus\ \%l}\)\.{lu\ rmems\ plus\ \%llu\ z}\)%
\.{mems\\n"},\39\\{mems},\39\\{rmems},\39\\{zmems});{}$\6
\X18:Print total memory usage\X;\par
\Us2\ET159.\fi

\M{7}\B\X7:Subroutines\X${}\E{}$\6
\&{void} \\{show\_stats}(\&{void})\1\1\2\2\6
${}\{{}$\1\6
${}\\{printf}(\.{"stats:\ \%d/\%d\ nodes,}\)\.{\ \%d\ dead,\ \%d\ pages,"},\39%
\\{totalnodes},\39\\{nodeptr}-\\{botsink},\39\\{deadnodes},\39\\{topofmem}-%
\\{pageptr});{}$\6
${}\\{printf}(\.{"\ \%llu\ mems,\ \%llu\ rm}\)\.{ems,\ \%llu\ zmems\\n"},\39%
\\{mems},\39\\{rmems},\39\\{zmems});{}$\6
\4${}\}{}$\2\par
\As14, 15, 16, 24, 26, 35, 41, 43, 45, 46, 49, 52, 53, 54, 55, 56, 58, 67, 71,
72, 73, 75, 77, 79, 82, 84, 86, 88, 91, 93, 95, 97, 104, 106, 108, 109, 131,
137, 138, 141, 145, 148, 151, 156, 159\ETs162.
\U2.\fi

\M{8}This program uses `\PB{\&{long} \&{long}}' to refer to 64-bit integers,
because a single `\PB{\&{long}}' isn't treated consistently by the
\CEE/~compilers available to me. (When I first learned~\CEE/,
`\PB{\&{int}}' was traditionally `\PB{\&{short}}', so I was obliged
to say `\PB{\&{long}}' when I wanted 32-bit integers. Consequently
the programs of the Stanford GraphBase, written in the 90s,
now get 64-bit integers---contrary to my original intent.
C'est tragique; c'est la vie.)

\Y\B\4\X4:Global variables\X${}\mathrel+\E{}$\6
\&{unsigned} \&{long} \&{long} \\{mems}${},{}$ \\{rmems}${},{}$ \\{zmems};\C{
mem counters }\par
\fi

\M{9}\B\X5:Initialize everything\X${}\mathrel+\E{}$\6
\&{if} (\&{sizeof}(\&{long} \&{long})${}\I\T{8}){}$\5
${}\{{}$\1\6
${}\\{fprintf}(\\{stderr},\39\.{"Sorry,\ I\ assume\ tha}\)\.{t\ sizeof(long\
long)\ }\)\.{is\ 8!\\n"});{}$\6
${}\\{exit}({-}\T{2});{}$\6
\4${}\}{}$\2\par
\fi

\M{10}Speaking of compilers, the one I use at present insists that
pointers occupy 64 bits. As a result, I need to pack and unpack
pointer data, in all the key data structures of this program;
otherwise I would basically be giving up half of my memory and half
of the hardware cache.

I could solve this problem by using arrays with integer subscripts.
Indeed, that approach would be simple and clean.

But I anticipate doing some fairly long calculations, and
speed is also important to me. So I've chosen a slightly more
complex (and slightly dirtier) approach, equivalent to using
short pointers; I wrap such pointers up with syntax that doesn't
offend my compiler. The use of this scheme allows me to use
the convenient syntax of~\CEE/ for fields within structures.

Namely, data is stored here with a type called \PB{\\{addr}}, which is simply
an unsigned 32-bit integer. An \PB{\\{addr}} contains
all the information of a pointer, since I'm not planning to use
this program with more than $2^{32}$ bytes of memory.
It has a special name only to indicate its pointerly nature.

With this approach the program goes fast, as with usual pointers,
because it doesn't have to shift left by 4~bits and add the base
address of~\PB{\\{mem}} whenever addressing the memory. But I do limit
myself to BDD bases of at most about 30 million nodes.

(At the cost of shift-left-four each time, I could extend this
scheme to handling a 35-bit address space, if I ever get a
computer with 32 gigabytes of RAM. I~still would want to keep
32-bit pointers in memory, in order to double the effective cache size.)

The \PB{\\{addr\_}} macro converts an arbitrary pointer to an \PB{\\{addr}}.

\Y\B\4\D$\\{addr\_}(\|p)$ \5
((\\{addr})(\&{size\_t})(\|p))\par
\Y\B\4\X10:Type definitions\X${}\E{}$\6
\&{typedef} \&{unsigned} \&{int} \&{addr};\par
\As11, 20\ETs39.
\U2.\fi

\N{1}{11}Dynamic arrays. Before I get into the BDD stuff, I might as well
give myself some infrastructure to work with.

The giant \PB{\\{mem}} array mentioned earlier has nodes at the bottom,
in locations \PB{\\{mem}} through \PB{$\\{nodeptr}-\T{1}$}. It has pages at the
top,
in locations \PB{\\{pageptr}} through \PB{$\\{mem}+\\{memsize}-\T{1}$}. We must
therefore keep
\PB{$\\{nodeptr}\Z\\{pageptr}$}.

A node has four fields, called \PB{\\{lo}}, \PB{\\{hi}}, \PB{\\{xref}}, and %
\PB{\\{index}}.
I shall explain their significance eventually,
when I {\it do\/} ``get into the BDD stuff.''

A page is basically unstructured, although we will eventually fill
it either with hash-table data or cache memos.

The \PB{\\{node\_}} and \PB{\\{page\_}} macros are provided to make pointers
from stored items of type \PB{\&{addr}}.

\Y\B\4\D$\\{logpagesize}$ \5
\T{12}\C{ must be at least 4 }\par
\B\4\D$\\{memsize}$ \5
$(\T{1}\LL\T{29}{}$)\C{ bytes in \PB{\\{mem}}, must be a multiple of \PB{%
\\{pagesize}} }\Y\par
\B\4\D$\\{pagesize}$ \5
$(\T{1}\LL\\{logpagesize}{}$)\C{ the number of bytes per page }\par
\B\4\D$\\{pagemask}$ \5
$(\\{pagesize}-\T{1}{}$)\par
\B\4\D$\\{pageints}$ \5
$(\\{pagesize}/\&{sizeof}(\&{int}){}$)\par
\B\4\D$\\{node\_}(\|a)$ \5
((\&{node} ${}{*})(\&{size\_t})(\|a){}$)\par
\B\4\D$\\{page\_}(\|a)$ \5
((\&{page} ${}{*})(\&{size\_t})(\|a){}$)\par
\Y\B\4\X10:Type definitions\X${}\mathrel+\E{}$\6
\&{typedef} \&{struct} \&{node\_struct} ${}\{{}$\1\6
\&{addr} \\{lo}${},{}$ \\{hi};\6
\&{int} \\{xref};\C{ reference count minus one }\6
\&{unsigned} \&{int} \\{index};\C{ variable ID followed by random bits }\2\6
${}\}{}$ \&{node};\6
\&{typedef} \&{struct} \&{page\_struct} ${}\{{}$\1\6
\&{addr} \\{dat}[\\{pageints}];\2\6
${}\}{}$ \&{page};\par
\fi

\M{12}Here's how we launch the dynamic memory setup.

Incidentally, I tried to initialize \PB{\\{mem}} by declaring it to be
a variable of type \PB{\&{void} ${}{*}$}, then saying `\PB{$\\{mem}\K%
\\{malloc}(\\{memsize})$}'.
But that failed spectacularly, because the geniuses who developed
the standard library for my 64-bit version of Linux decided in their
great wisdom to make \PB{\\{malloc}} return a huge pointer like
\PB{\T{\^2adaf3739010}}, even when the program could fit comfortably in
a 30-bit address space. D'oh.

\Y\B\4\D$\\{topofmem}$ \5
((\&{page} ${}{*}){}$ ${}{\AND}\\{mem}[\\{memsize}]{}$)\par
\Y\B\4\X5:Initialize everything\X${}\mathrel+\E{}$\6
$\\{botsink}\K{}$(\&{node} ${}{*}){}$ \\{mem};\C{ this is the sink node for the
all-zero function }\6
${}\\{topsink}\K\\{botsink}+\T{1}{}$;\C{ this is the sink node for the all-one
function }\6
${}\|o,\39\\{botsink}\MG\\{lo}\K\\{botsink}\MG\\{hi}\K\\{addr\_}(%
\\{botsink});{}$\6
${}\|o,\39\\{topsink}\MG\\{lo}\K\\{topsink}\MG\\{hi}\K\\{addr\_}(%
\\{topsink});{}$\6
${}\\{oo},\39\\{botsink}\MG\\{xref}\K\\{topsink}\MG\\{xref}\K\T{0};{}$\6
${}\\{oooo},\39\\{botsink}\MG\\{index}\K\\{gb\_next\_rand}(\,);{}$\6
${}\\{oooo},\39\\{topsink}\MG\\{index}\K\\{gb\_next\_rand}(\,);{}$\6
${}\\{totalnodes}\K\T{2};{}$\6
${}\\{nodeptr}\K\\{topsink}+\T{1};{}$\6
${}\\{pageptr}\K\\{topofmem}{}$;\par
\fi

\M{13}\B\X4:Global variables\X${}\mathrel+\E{}$\6
\&{char} \\{mem}[\\{memsize}];\C{ where we store most of the stuff }\6
\&{node} ${}{*}\\{nodeptr}{}$;\C{ the smallest unused node in \PB{\\{mem}} }\6
\&{page} ${}{*}\\{pageptr}{}$;\C{ the smallest used page in \PB{\\{mem}} }\6
\&{node} ${}{*}\\{nodeavail}{}$;\C{ stack of nodes available for reuse }\6
\&{page} ${}{*}\\{pageavail}{}$;\C{ stack of pages available for reuse }\6
\&{node} ${}{*}\\{botsink},{}$ ${}{*}\\{topsink}{}$;\C{ the sink nodes, which
never go away }\6
\&{int} \\{totalnodes};\C{ this many nodes are currently in use }\6
\&{int} \\{deadnodes};\C{ and this many of them currently have \PB{$\\{xref}<%
\T{0}$} }\6
\&{int} \\{leasesonlife}${}\K\T{10}{}$;\par
\fi

\M{14}Here's how we get a fresh (but uninitialized) node.
The \PB{\\{nodeavail}} stack is linked by its \PB{\\{xref}} fields.

If memory is completely full, \PB{$\NULL$} is returned. In such cases
we need not abandon all hope; a garbage collection may be able
to reclaim enough memory to continue. (I've tried to write this
entire program in such a way that such temporary failures are harmless.)

\Y\B\4\X7:Subroutines\X${}\mathrel+\E{}$\6
\&{node} ${}{*}\\{reserve\_node}(\&{void}){}$\1\1\2\2\6
${}\{{}$\1\6
\&{register} \&{node} ${}{*}\|r\K\\{nodeavail};{}$\7
\&{if} (\|r)\1\5
${}\|o,\39\\{nodeavail}\K\\{node\_}(\\{nodeavail}\MG\\{xref});{}$\2\6
\&{else}\5
${}\{{}$\1\6
${}\|r\K\\{nodeptr};{}$\6
\&{if} ${}(\|r<{}$(\&{node} ${}{*}){}$ \\{pageptr})\1\5
${}\\{nodeptr}\PP;{}$\2\6
\&{else}\5
${}\{{}$\1\6
${}\\{leasesonlife}\MM;{}$\6
${}\\{fprintf}(\\{stderr},\39\.{"NULL\ node\ forced\ (\%}\)\.{d\ pages,\ \%d\
nodes,\ \%}\)\.{d\ dead)\\n"},\39\\{topofmem}-\\{pageptr},\39\\{nodeptr}-%
\\{botsink},\39\\{deadnodes});{}$\6
${}\\{fprintf}(\\{stderr},\39\.{"(I\ will\ try\ \%d\ more}\)\.{\ times)\\n"},%
\39\\{leasesonlife});{}$\6
\&{if} ${}(\\{leasesonlife}\E\T{0}){}$\5
${}\{{}$\1\6
\\{show\_stats}(\,);\5
${}\\{exit}({-}\T{98}){}$;\C{ sigh }\6
\4${}\}{}$\2\6
\&{return} ${}\NULL;{}$\6
\4${}\}{}$\2\6
\4${}\}{}$\2\6
${}\\{totalnodes}\PP;{}$\6
\&{return} \|r;\6
\4${}\}{}$\2\par
\fi

\M{15}Conversely, nodes can always be recycled. In such cases, there
had better not be any other nodes pointing to them.

\Y\B\4\X7:Subroutines\X${}\mathrel+\E{}$\6
\&{void} \\{free\_node}(\&{register} \&{node} ${}{*}\|p){}$\1\1\2\2\6
${}\{{}$\1\6
${}\|o,\39\|p\MG\\{xref}\K\\{addr\_}(\\{nodeavail});{}$\6
${}\\{nodeavail}\K\|p;{}$\6
${}\\{totalnodes}\MM;{}$\6
\4${}\}{}$\2\par
\fi

\M{16}Occupation and liberation of pages is similar, but it takes place
at the top of \PB{\\{mem}}.

\Y\B\4\X7:Subroutines\X${}\mathrel+\E{}$\6
\&{page} ${}{*}\\{reserve\_page}(\&{void}){}$\1\1\2\2\6
${}\{{}$\1\6
\&{register} \&{page} ${}{*}\|r\K\\{pageavail};{}$\7
\&{if} (\|r)\1\5
${}\|o,\39\\{pageavail}\K\\{page\_}(\\{pageavail}\MG\\{dat}[\T{0}]);{}$\2\6
\&{else}\5
${}\{{}$\1\6
${}\|r\K\\{pageptr}-\T{1};{}$\6
\&{if} ((\&{node} ${}{*}){}$ \|r${}\G\\{nodeptr}){}$\1\5
${}\\{pageptr}\K\|r;{}$\2\6
\&{else}\5
${}\{{}$\1\6
${}\\{leasesonlife}\MM;{}$\6
${}\\{fprintf}(\\{stderr},\39\.{"NULL\ page\ forced\ (\%}\)\.{d\ pages,\ \%d\
nodes,\ \%}\)\.{d\ dead)\\n"},\39\\{topofmem}-\\{pageptr},\39\\{nodeptr}-%
\\{botsink},\39\\{deadnodes});{}$\6
${}\\{fprintf}(\\{stderr},\39\.{"(I\ will\ try\ \%d\ more}\)\.{\ times)\\n"},%
\39\\{leasesonlife});{}$\6
\&{if} ${}(\\{leasesonlife}\E\T{0}){}$\5
${}\{{}$\1\6
\\{show\_stats}(\,);\5
${}\\{exit}({-}\T{97}){}$;\C{ sigh }\6
\4${}\}{}$\2\6
\&{return} ${}\NULL;{}$\6
\4${}\}{}$\2\6
\4${}\}{}$\2\6
\&{return} \|r;\6
\4${}\}{}$\2\7
\&{void} \\{free\_page}(\&{register} \&{page} ${}{*}\|p){}$\1\1\2\2\6
${}\{{}$\1\6
${}\|o,\39\|p\MG\\{dat}[\T{0}]\K\\{addr\_}(\\{pageavail});{}$\6
${}\\{pageavail}\K\|p;{}$\6
\4${}\}{}$\2\par
\fi

\M{17}\B\X17:If there are at least three free pages and at least three free
nodes, \PB{\&{break}}\X${}\E{}$\6
$\|j\K{}$(\&{node} ${}{*})(\\{pageptr}-\T{3})-\\{nodeptr};{}$\6
\&{if} ${}(\|j\G\T{0}){}$\5
${}\{{}$\1\6
\&{for} ${}(\|p\K\\{nodeavail};{}$ ${}\|p\W\|j<\T{3};{}$ \|o${},\39\|p\K(%
\\{node\_}(\|p\MG\\{xref}))){}$\1\5
${}\|j\PP;{}$\2\6
\&{if} ${}(\|j\G\T{3}){}$\1\5
\&{break};\2\6
\4${}\}{}$\2\par
\U111.\fi

\M{18}\B\X18:Print total memory usage\X${}\E{}$\6
$\|j\K\\{nodeptr}-{}$(\&{node} ${}{*}){}$ \\{mem};\6
${}\|k\K\\{topofmem}-\\{pageptr};{}$\6
${}\\{printf}(\.{"\ \ \%llu\ bytes\ of\ mem}\)\.{ory\ (\%d\ nodes,\ \%d\ pa}\)%
\.{ges)\\n"},\39{}$((\&{long} \&{long}) \|j)${}*\&{sizeof}(\&{node})+{}$((%
\&{long} \&{long}) \|k)${}*\&{sizeof}(\&{page}),\39\|j,\39\|k){}$;\par
\U6.\fi

\M{19}\B\X19:Local variables\X${}\E{}$\6
\&{register} \&{int} \|j${},{}$ \|k;\par
\A115.
\U2.\fi

\N{1}{20}Variables and hash tables. Our BDD base represents functions
on the variables $x_v$ for $0\le v<\PB{\\{varsize}}$, where \PB{\\{varsize}}
is a power of~2.

When $x_v$ is first mentioned, we create a \PB{\\{var}} record for it,
from which it is possible to find all the nodes that branch on
this variable. The list of all such nodes is implicitly present
in a hash table, which contains a pointer to node $(v,l,h)$
near the hash address of the pair $(l,h)$. This hash table is
called the {\it unique table\/} for~$v$, because of the BDD property
that no two nodes have the same triple of values $(v,l,h)$.

When there are $n$ nodes that branch on $x_v$, the unique table
for~$v$ has size $m$, where $m$ is a power of~2 such that
$n$ lies between $m/8$ and $3m/4$, inclusive. Thus at least
one of every eight table slots is occupied, and
at least one of every four is unoccupied, on the average.
If $n=25$, for example, we might have $m=64$ or $m=128$; but $m=256$ would make
the table too sparse.

Each unique table has a maximum size, which must be small enough
that we don't need too many base addresses for its pages, yet large
enough that we can accommodate big BDDs. If, for example,
\PB{$\\{logmaxhashsize}\K\T{19}$} and \PB{$\\{logpagesize}\K\T{12}$}, a unique
table might contain as
many as $2^{19}$ \PB{\&{addr}}s, filling $2^9$ pages. Then we must make room
for
512 base addresses in each \PB{\\{var}} record, and we can handle up to
$2^{19}-2^{17}=393216$ nodes that branch on any particular variable.

\Y\B\4\D$\\{logmaxhashsize}$ \5
\T{21}\par
\B\4\D$\\{slotsperpage}$ \5
$(\\{pagesize}/\&{sizeof}(\&{addr}){}$)\par
\B\4\D$\\{maxhashpages}$ \5
$(((\T{1}\LL\\{logmaxhashsize})+\\{slotsperpage}-\T{1})/\\{slotsperpage}{}$)\par
\Y\B\4\X10:Type definitions\X${}\mathrel+\E{}$\6
\&{typedef} \&{struct} \&{var\_struct} ${}\{{}$\1\6
\&{addr} \\{proj};\C{ address of the projection function $x_v$ }\6
\&{addr} \\{repl};\C{ address of the replacement function $y_v$ }\6
\&{int} \\{free};\C{ the number of unused slots in the unique table for $v$ }\6
\&{int} \\{mask};\C{ the number of slots in that unique table, times 4, minus 1
}\6
\&{addr} \\{base}[\\{maxhashpages}];\C{ base addresses for its pages }\6
\&{int} \\{name};\C{ the user's name (subscript) for this variable }\6
\&{unsigned} \&{int} \\{timestamp};\C{ time stamp for composition }\6
\&{int} \\{aux};\C{ flag used by \PB{\\{math\_print}} or the sifting algorithm
}\6
\&{struct} \&{var\_struct} ${}{*}\\{up},{}$ ${}{*}\\{down}{}$;\C{ the
neighboring active variables }\2\6
${}\}{}$ \&{var};\par
\fi

\M{21}Every node \PB{\|p} that branches on $x_v$ in the BDD has a field \PB{$%
\|p\MG\\{index}$},
whose leftmost \PB{\\{logvarsize}} bits contain the index~$v$. The rightmost
\PB{$\T{32}-\\{logvarsize}$} bits of \PB{$\|p\MG\\{index}$} are chosen
randomly, in order to
provide convenient hash coding.

The SGB random-number generator used here makes four memory references
per number generated.

N.B.: The hashing scheme will fail dramatically unless
\PB{$\\{logvarsize}+\\{logmaxhashsize}\Z\T{32}$}.

\Y\B\4\D$\\{logvarsize}$ \5
\T{10}\par
\B\4\D$\\{varsize}$ \5
$(\T{1}\LL\\{logvarsize}{}$)\C{ the number of permissible variables }\par
\B\4\D$\\{varpart}(\|x)$ \5
$((\|x)\GG(\T{32}-\\{logvarsize}){}$)\par
\B\4\D$\\{initnewnode}(\|p,\|v,\|l,\|h)$ \5
$\\{oo},\39\|p\MG\\{lo}\K\\{addr\_}(\|l),\39\|p\MG\\{hi}\K\\{addr\_}(\|h),\39%
\|p\MG\\{xref}\K\T{0},\3{-1}\39\\{oooo},\39\|p\MG\\{index}\K((\|v)\LL(\T{32}-%
\\{logvarsize}))+(\\{gb\_next\_rand}(\,)\GG(\\{logvarsize}-\T{1}){}$)\par
\fi

\M{22}Variable $x_v$ in this documentation means the variable whose information
record is \PB{\\{varhead}[\|v]}. But the user's variable `\.{x5}' might not be
represented by \PB{\\{varhead}[\T{5}]}, because the ordering of variables can
change
as a program runs. If \.{x5} is really the variable in \PB{\\{varhead}[%
\T{13}]}, say, we
will have \PB{$\\{varmap}[\T{5}]\K\T{13}$} and \PB{$\\{varhead}[\T{13}].%
\\{name}\K\T{5}$}.

\Y\B\4\D$\\{topofvars}$ \5
${\AND}\\{varhead}{}$[\\{varsize}]\par
\Y\B\4\X4:Global variables\X${}\mathrel+\E{}$\6
\&{var} \\{varhead}[\\{varsize}];\C{ basic info about each variable }\6
\&{var} ${}{*}\\{tvar}\K\\{topofvars}{}$;\C{ threshold for verbose printouts }\6
\&{int} \\{varmap}[\\{varsize}];\C{ the variable that has a given name }\par
\fi

\M{23}\B\X5:Initialize everything\X${}\mathrel+\E{}$\6
\&{for} ${}(\|k\K\T{0};{}$ ${}\|k<\\{varsize};{}$ ${}\|k\PP){}$\1\5
${}\\{varmap}[\|k]\K\|k{}$;\2\par
\fi

\M{24}The simplest nonconstant Boolean expression is a projection function,
$x_v$.
We access it with the following subroutine, creating it from scratch
if necessary.

(The calling routine will have ensured that at least one free page
and at least one free node exist when \PB{\\{projection}} is invoked.)

Beware: Garbage collection might occur when \PB{\\{unique\_find}} is called
here.

\Y\B\4\X7:Subroutines\X${}\mathrel+\E{}$\6
\&{node} ${}{*}{}$\\{projection}(\&{int} \|v)\1\1\2\2\6
${}\{{}$\1\6
\&{register} \&{node} ${}{*}\|p;{}$\6
\&{register} \&{var} ${}{*}\\{hv}\K{\AND}\\{varhead}[\|v];{}$\7
${}\|o,\39\|p\K\\{node\_}(\\{hv}\MG\\{proj});{}$\6
\&{if} (\|p)\1\5
\&{return} \|p;\C{ the projection function has already been created }\2\6
${}\|o,\39\\{hv}\MG\\{base}[\T{0}]\K\\{addr\_}(\\{reserve\_page}(\,)){}$;\C{ it
won't be \PB{$\NULL$} }\6
\X28:Create a unique table for variable \PB{\\{hv}} with size 2\X;\6
${}\|p\K\\{unique\_find}({\AND}\\{varhead}[\|v],\39\\{botsink},\39%
\\{topsink}){}$;\C{ see below }\6
${}\\{oooo},\39\\{botsink}\MG\\{xref}\PP,\39\\{topsink}\MG\\{xref}\PP;{}$\6
${}\|o,\39\\{hv}\MG\\{proj}\K\\{addr\_}(\|p){}$;\C{ \PB{\|p} won't be \PB{$%
\NULL$} either }\6
\&{if} ${}(\\{verbose}\AND\T{2}){}$\1\5
${}\\{printf}(\.{"\ \%x=x\%d\\n"},\39\\{id}(\|p),\39\|v);{}$\2\6
${}\|o,\39\\{hv}\MG\\{name}\K\|v;{}$\6
\&{return} \|p;\6
\4${}\}{}$\2\par
\fi

\M{25}I sometimes like to use a subroutine before I'm in the mood to write
its innards. In such cases, a pre-specification
like the one given here allows me to procrastinate.

\Y\B\4\X25:Templates for subroutines\X${}\E{}$\6
\&{node} ${}{*}{}$\\{unique\_find}(\&{var} ${}{*}\|v,\39{}$\&{node} ${}{*}\|l,%
\39{}$\&{node} ${}{*}\|h){}$;\par
\As27, 90\ETs107.
\U2.\fi

\M{26}Now, however, I'm ready to tackle the subroutine just named, \PB{%
\\{unique\_find}},
which is one of the most crucial in the entire program.
Given a variable~\PB{\|v}, together with node pointers \PB{\|l} and~\PB{\|h},
we often
want to see if the BDD base contains a node $(v,l,h)$---namely, a branch
on~$x_v$ with {\mc LO} pointer~\PB{\|l} and {\mc HI} pointer~\PB{\|h}.
If no such node exists, we want to create it. The subroutine should return a
pointer to that (unique) node. Furthermore,
the reference counts of \PB{\|l} and \PB{\|h} should be decreased afterwards.

To do this task, we look for $(l,h)$ in the unique table for $v$,
using the hash code
$$\hbox{\PB{$(\|l\MG\\{index}\LL\T{3})\XOR(\|h\MG\\{index}\LL\T{2})$}}.$$
(This hash code is a multiple of~4,
the size of each entry in the unique table.)

Several technicalities should be noted. First, no branch is needed
when $l=h$. Second, we consider that a
new reference is being made to the node returned, as well as to nodes
\PB{\|l} and~\PB{\|h} if a new node is created;
the \PB{\\{xref}} fields (reference counts) must be adjusted accordingly.
Third, we might discover that the node exists, but it is dead;
in other words, all prior links to it might have gone away, but we haven't
discarded it yet. In such a case we should bring it back to life.
Fourth, \PB{\|l} and \PB{\|h} will not become dead
when their reference counts decrease, because the calling routine knows them.
And finally, in the worst case we won't have room for a new node, so we'll
have to return \PB{$\NULL$}. The calling routine must be prepared to cope with
such failures (which we hope are only temporary).

The following inscrutable macros try to make my homegrown dynamic array
addressing palatable. I have to admit that I didn't get them right
the first time. Or even the second time. Or even \dots~.

\Y\B\4\D$\\{hashcode}(\|l,\|h)$ \5
((\&{addr} ${}{*})(\&{size\_t})(\\{oo},\39((\|l)\MG\\{index}\LL\T{3})\XOR((\|h)%
\MG\\{index}\LL\T{2})){}$)\par
\B\4\D$\\{hashedcode}(\|p)$ \5
$\\{hashcode}(\\{node\_}(\|p\MG\\{lo}),\39\\{node\_}(\|p\MG\\{hi}){}$)\par
\B\4\D$\\{addr\_\_}(\|x)$ \5
$({*}{}$((\&{addr} ${}{*})(\&{size\_t})(\|x)){}$)\par
\B\4\D$\\{fetchnode}(\|v,\|k)$ \5
$\\{node\_}(\\{addr\_\_}(\|v\MG\\{base}[(\|k)\GG\\{logpagesize}]+((\|k)\AND%
\\{pagemask})){}$)\par
\B\4\D$\\{storenode}(\|v,\|k,\|p)$ \5
$\|o,\39\\{addr\_\_}(\|v\MG\\{base}[(\|k)\GG\\{logpagesize}]+((\|k)\AND%
\\{pagemask}))\K\\{addr\_}{}$(\|p)\par
\Y\B\4\X7:Subroutines\X${}\mathrel+\E{}$\6
\&{node} ${}{*}{}$\\{unique\_find}(\&{var} ${}{*}\|v,\39{}$\&{node} ${}{*}\|l,%
\39{}$\&{node} ${}{*}\|h){}$\1\1\2\2\6
${}\{{}$\1\6
\&{register} \&{int} \|j${},{}$ \|k${},{}$ \\{mask}${},{}$ \\{free};\6
\&{register} \&{addr} ${}{*}\\{hash};{}$\6
\&{register} \&{node} ${}{*}\|p,{}$ ${}{*}\|r;{}$\7
\&{if} ${}(\|l\E\|h){}$\5
${}\{{}$\C{ easy case }\1\6
\&{return} \\{oo}${},\39\|l\MG\\{xref}\MM,\39\|l{}$;\C{ \PB{$\|l\MG\\{xref}$}
will still be $\ge0$ }\6
\4${}\}{}$\2\6
\4\\{restart}:\5
${}\|o,\39\\{mask}\K\|v\MG\\{mask},\39\\{free}\K\|v\MG\\{free};{}$\6
\&{for} ${}(\\{hash}\K\\{hashcode}(\|l,\39\|h);{}$  ; ${}\\{hash}\PP){}$\5
${}\{{}$\C{ ye olde linear probing }\1\6
${}\|k\K\\{addr\_}(\\{hash})\AND\\{mask};{}$\6
${}\\{oo},\39\|p\K\\{fetchnode}(\|v,\39\|k);{}$\6
\&{if} ${}(\R\|p){}$\1\5
\&{goto} \\{newnode};\2\6
\&{if} ${}(\\{node\_}(\|p\MG\\{lo})\E\|l\W\\{node\_}(\|p\MG\\{hi})\E\|h){}$\1\5
\&{break};\2\6
\4${}\}{}$\2\6
\&{if} ${}(\|o,\39\|p\MG\\{xref}<\T{0}){}$\5
${}\{{}$\1\6
${}\\{deadnodes}\MM,\39\|o,\39\|p\MG\\{xref}\K\T{0}{}$;\C{ a lucky hit; its
children are alive }\6
\&{return} \|p;\6
\4${}\}{}$\2\6
${}\\{oooo},\39\|l\MG\\{xref}\MM,\39\|h\MG\\{xref}\MM;{}$\6
\&{return} \|o${},\39\|p\MG\\{xref}\PP,\39\|p;{}$\6
\4\\{newnode}:\5
\X29:Periodically try to conserve space\X;\6
\X31:Create a new node and return it\X;\6
\4${}\}{}$\2\par
\fi

\M{27}\B\X25:Templates for subroutines\X${}\mathrel+\E{}$\6
\&{void} \\{recursively\_revive}(\&{node} ${}{*}\|p){}$;\C{ recursive
resuscitation }\6
\&{void} \\{recursively\_kill}(\&{node} ${}{*}\|p){}$;\C{ recursive
euthanization }\6
\&{void} \\{collect\_garbage}(\&{int} \\{level});\C{ invocation of the recycler
}\par
\fi

\M{28}Before we can call \PB{\\{unique\_find}}, we need a hash table to work
with.
We start small.

\Y\B\4\D$\\{storenulls}(\|k)$ \5
${*}{}$(\&{long} \&{long} ${}{*})(\&{size\_t})(\|k)\K\T{0\$L\$L}{}$;\par
\Y\B\4\X28:Create a unique table for variable \PB{\\{hv}} with size 2\X${}\E{}$%
\6
$\|o,\39\\{hv}\MG\\{free}\K\T{2},\39\\{hv}\MG\\{mask}\K\T{7};{}$\6
${}\\{storenulls}(\\{hv}\MG\\{base}[\T{0}]){}$;\C{ both slots start out \PB{$%
\NULL$} }\6
${}\\{zmems}\PP{}$;\par
\U24.\fi

\M{29}A little timer starts ticking at the beginning of this program,
and it advances whenever we reach the present point.
Whenever the timer reaches a multiple of \PB{\\{timerinterval}}, we pause to
examine the memory situation, in an attempt to keep node growth under
control.

Memory can be conserved in two ways. First, we can recycle all the dead
nodes. That's a somewhat expensive proposition; but it's worthwhile
if the number of such nodes is more than, say, 1/8 of the total
number of nodes allocated. Second, we can try to change the ordering
of the variables. The present program includes Rudell's
``sifting algorithm'' for dynamically improving the variable order; but
it invokes that algorithm only under user control. Perhaps I will have
time someday to make reordering more automatic.

\Y\B\4\D$\\{timerinterval}$ \5
\T{1024}\par
\B\4\D$\\{deadfraction}$ \5
\T{8}\par
\Y\B\4\X29:Periodically try to conserve space\X${}\E{}$\6
\&{if} ${}((\PP\\{timer}\MOD\\{timerinterval})\E\T{0}){}$\5
${}\{{}$\1\6
\&{if} ${}(\\{deadnodes}>\\{totalnodes}/\\{deadfraction}){}$\5
${}\{{}$\1\6
\\{collect\_garbage}(\T{0});\6
\&{goto} \\{restart};\C{ the hash table might now be different }\6
\4${}\}{}$\2\6
\4${}\}{}$\2\par
\U26.\fi

\M{30}\B\X4:Global variables\X${}\mathrel+\E{}$\6
\&{unsigned} \&{long} \&{long} \\{timer};\par
\fi

\M{31}Brand-new nodes enter the fray here.

\Y\B\4\X31:Create a new node and return it\X${}\E{}$\6
$\|p\K\\{reserve\_node}(\,);{}$\6
\&{if} ${}(\R\|p){}$\1\5
\&{goto} \\{cramped};\C{ sorry, there ain't no more room }\2\6
\&{if} ${}(\MM\\{free}\Z\\{mask}\GG\T{4}){}$\5
${}\{{}$\1\6
\\{free\_node}(\|p);\6
\X32:Double the table size and \PB{\&{goto} \\{restart}}\X;\6
\4${}\}{}$\2\6
${}\\{storenode}(\|v,\39\|k,\39\|p){}$;\5
${}\|o,\39\|v\MG\\{free}\K\\{free};{}$\6
${}\\{initnewnode}(\|p,\39\|v-\\{varhead},\39\|l,\39\|h);{}$\6
\&{return} \|p;\6
\4\\{cramped}:\C{ after failure, we need to keep the xrefs tidy }\6
\\{deref}(\|l);\C{ decrease \PB{$\|l\MG\\{xref}$}, and recurse if it becomes
dead }\6
\\{deref}(\|h);\C{ ditto for \PB{\|h} }\6
\&{return} ${}\NULL{}$;\par
\U26.\fi

\M{32}We get to this part of the code when the table has become too dense.
The density will now decrease from 3/4 to 3/8.

\Y\B\4\X32:Double the table size and \PB{\&{goto} \\{restart}}\X${}\E{}$\6
${}\{{}$\1\6
\&{register} \&{int} \\{newmask}${}\K\\{mask}+\\{mask}+\T{1},{}$ \\{kk}${}\K%
\\{newmask}\GG\\{logpagesize};{}$\7
\&{if} ${}(\\{verbose}\AND\T{256}){}$\1\5
${}\\{printf}(\.{"doubling\ the\ hash\ t}\)\.{able\ for\ level\ \%d(x\%}\)\.{d)%
\ (\%d\ slots)\\n"},\39\|v-\\{varhead},\39\|v\MG\\{name},\39(\\{newmask}+%
\T{1})/\&{sizeof}(\&{addr}));{}$\2\6
\&{if} (\\{kk})\1\5
\X33:Reserve new all-\PB{$\NULL$} pages for the bigger table\X\2\6
\&{else}\5
${}\{{}$\1\6
\&{for} ${}(\|k\K\|v\MG\\{base}[\T{0}]+\\{mask}+\T{1};{}$ ${}\|k<\|v\MG%
\\{base}[\T{0}]+\\{newmask};{}$ ${}\|k\MRL{+{\K}}{}$\&{sizeof}(\&{long} %
\&{long}))\1\5
\\{storenulls}(\|k);\2\6
${}\\{zmems}\MRL{+{\K}}(\\{newmask}-\\{mask})/{}$\&{sizeof}(\&{long} \&{long});%
\6
\4${}\}{}$\2\6
\X34:Rehash everything in the low half\X;\6
${}\|v\MG\\{mask}\K\\{newmask}{}$;\C{ mems are counted after restarting }\6
${}\|v\MG\\{free}\K\\{free}+\T{1}+(\\{newmask}-\\{mask})/\&{sizeof}(%
\&{addr});{}$\6
\&{goto} \\{restart};\6
\4${}\}{}$\2\par
\Us31, 138\ETs142.\fi

\M{33}\B\D$\\{maxmask}$ \5
$((\T{1}\LL\\{logmaxhashsize})*\&{sizeof}(\&{addr})-\T{1}{}$)\C{ the biggest
possible \PB{\\{mask}} }\par
\Y\B\4\X33:Reserve new all-\PB{$\NULL$} pages for the bigger table\X${}\E{}$\6
${}\{{}$\1\6
\&{if} ${}(\\{newmask}>\\{maxmask}){}$\5
${}\{{}$\C{ too big: can't go there }\1\6
\&{if} ${}(\\{verbose}\AND(\T{2}+\T{256}+\T{512})){}$\1\5
${}\\{printf}(\.{"profile\ limit\ reach}\)\.{ed\ for\ level\ \%d(x\%d)}\)\.{%
\\n"},\39\|v-\\{varhead},\39\|v\MG\\{name});{}$\2\6
\&{goto} \\{cramped};\6
\4${}\}{}$\2\6
\&{for} ${}(\|k\K(\\{mask}\GG\\{logpagesize})+\T{1};{}$ ${}\|k\Z\\{kk};{}$ ${}%
\|k\PP){}$\5
${}\{{}$\1\6
${}\|o,\39\|v\MG\\{base}[\|k]\K\\{addr\_}(\\{reserve\_page}(\,));{}$\6
\&{if} ${}(\R\|v\MG\\{base}[\|k]){}$\5
${}\{{}$\C{ oops, we're out of space }\1\6
\&{for} ${}(\|k\MM;{}$ ${}\|k>\\{mask}\GG\\{logpagesize};{}$ ${}\|k\MM){}$\5
${}\{{}$\1\6
${}\|o,\39\\{free\_page}(\\{page\_}(\|v\MG\\{base}[\|k]));{}$\6
\4${}\}{}$\2\6
\&{goto} \\{cramped};\6
\4${}\}{}$\2\6
\&{for} ${}(\|j\K\|v\MG\\{base}[\|k];{}$ ${}\|j<\|v\MG\\{base}[\|k]+%
\\{pagesize};{}$ ${}\|j\MRL{+{\K}}{}$\&{sizeof}(\&{long} \&{long}))\1\5
\\{storenulls}(\|j);\2\6
${}\\{zmems}\MRL{+{\K}}\\{pagesize}/{}$\&{sizeof}(\&{long} \&{long});\6
\4${}\}{}$\2\6
\4${}\}{}$\2\par
\U32.\fi

\M{34}Some subtle cases can arise at this point.
For example, consider the hash table
{\let\\=\Lambda $(a,\\,\\,b)$, with hash$(a)=3$ and hash$(b)=7$; when
doubling the size, we need to rehash $a$ twice, going from
the doubled-up table
$(a,\\,\\,b,\\,\\,\\,\\)$ to
$(\\,\\,\\,b,a,\\,\\,\\)$ to
$(\\,\\,\\,\\,a,\\,\\,b)$ to
$(\\,\\,\\,a,\\,\\,\\,b)$.}

I learned this interesting algorithm from Rick Rudell.

\Y\B\4\X34:Rehash everything in the low half\X${}\E{}$\6
\&{for} ${}(\|k\K\T{0};{}$ ${}\|k<\\{newmask};{}$ ${}\|k\MRL{+{\K}}\&{sizeof}(%
\&{addr})){}$\5
${}\{{}$\1\6
${}\\{oo},\39\|r\K\\{fetchnode}(\|v,\39\|k);{}$\6
\&{if} (\|r)\5
${}\{{}$\1\6
${}\\{storenode}(\|v,\39\|k,\39\NULL){}$;\C{ prevent propagation past this slot
}\6
\&{for} ${}(\|o,\39\\{hash}\K\\{hashedcode}(\|r);{}$  ; ${}\\{hash}\PP){}$\5
${}\{{}$\1\6
${}\|j\K\\{addr\_}(\\{hash})\AND\\{newmask};{}$\6
${}\\{oo},\39\|p\K\\{fetchnode}(\|v,\39\|j);{}$\6
\&{if} ${}(\R\|p){}$\1\5
\&{break};\2\6
\4${}\}{}$\2\6
${}\\{storenode}(\|v,\39\|j,\39\|r);{}$\6
\4${}\}{}$\5
\2\&{else} \&{if} ${}(\|k>\\{mask}){}$\1\5
\&{break};\C{ see the example above }\2\6
\4${}\}{}$\2\par
\U32.\fi

\M{35}While I've got linear probing firmly in mind, I might as well
write a subroutine that will be needed later for garbage collection.
The \PB{\\{table\_purge}} routine deletes all dead nodes that branch
on a given variable~$x_v$.

\Y\B\4\X7:Subroutines\X${}\mathrel+\E{}$\6
\&{void} \\{table\_purge}(\&{var} ${}{*}\|v){}$\1\1\2\2\6
${}\{{}$\1\6
\&{register} \&{int} \\{free}${},{}$ \|i${},{}$ \|j${},{}$ \\{jj}${},{}$ %
\|k${},{}$ \\{kk}${},{}$ \\{mask}${},{}$ \\{newmask}${},{}$ \\{oldtotal};\6
\&{register} \&{node} ${}{*}\|p,{}$ ${}{*}\|r;{}$\6
\&{register} \&{addr} ${}{*}\\{hash};{}$\7
${}\|o,\39\\{mask}\K\|v\MG\\{mask},\39\\{free}\K\|v\MG\\{free};{}$\6
\&{if} ${}(\|o,\39\|v\MG\\{proj}){}$\5
${}\{{}$\C{ \PB{$\|v\MG\\{proj}\I\T{0}$} if and only if $x_v$ exists }\1\6
${}\\{oldtotal}\K\\{totalnodes};{}$\6
\&{for} ${}(\|k\K\T{0};{}$ ${}\|k<\\{mask};{}$ ${}\|k\MRL{+{\K}}\&{sizeof}(%
\&{addr})){}$\5
${}\{{}$\1\6
${}\\{oo},\39\|p\K\\{fetchnode}(\|v,\39\|k);{}$\6
\&{if} ${}(\|p\W\|p\MG\\{xref}<\T{0}){}$\5
${}\{{}$\1\6
\\{free\_node}(\|p);\6
\X36:Remove entry \PB{\|k} from the hash table\X;\6
\4${}\}{}$\2\6
\4${}\}{}$\2\6
${}\\{deadnodes}\MRL{-{\K}}\\{oldtotal}-\\{totalnodes},\39\\{free}\MRL{+{\K}}%
\\{oldtotal}-\\{totalnodes};{}$\6
\X37:Downsize the table if only a few entries are left\X;\6
${}\|o,\39\|v\MG\\{free}\K\\{free};{}$\6
\4${}\}{}$\2\6
\4${}\}{}$\2\par
\fi

\M{36}Deletion from a linearly probed hash table is tricky, as noted in
Algorithm 6.4R of TAOCP. Here I can speed that algorithm up slightly,
because there's no need to move dead entries that will be deleted later.

Furthermore, if I do meet a dead entry, I can take a slightly tricky
shortcut and continue the removals.

\Y\B\4\X36:Remove entry \PB{\|k} from the hash table\X${}\E{}$\6
\&{do}\5
${}\{{}$\1\6
\&{for} ${}(\\{kk}\K\|k,\39\|j\K\|k+\&{sizeof}(\&{addr}),\39\|k\K\T{0};{}$  ;
${}\|j\MRL{+{\K}}\&{sizeof}(\&{addr})){}$\5
${}\{{}$\1\6
${}\\{jj}\K\|j\AND\\{mask};{}$\6
${}\\{oo},\39\|p\K\\{fetchnode}(\|v,\39\\{jj});{}$\6
\&{if} ${}(\R\|p){}$\1\5
\&{break};\2\6
\&{if} ${}(\|p\MG\\{xref}\G\T{0}){}$\5
${}\{{}$\1\6
${}\|o,\39\|i\K\\{addr\_}(\\{hashedcode}(\|p))\AND\\{mask};{}$\6
\&{if} ${}((\|i\Z\\{kk})+(\\{jj}<\|i)+(\\{kk}<\\{jj})>\T{1}){}$\1\5
${}\\{storenode}(\|v,\39\\{kk},\39\|p),\39\\{kk}\K\\{jj};{}$\2\6
\4${}\}{}$\5
\2\&{else} \&{if} ${}(\R\|k){}$\1\5
${}\|k\K\|j,\39\\{free\_node}(\|p){}$;\C{ shortcut }\2\6
\4${}\}{}$\2\6
${}\\{storenode}(\|v,\39\\{kk},\39\NULL);{}$\6
\4${}\}{}$\2\5
\&{while} (\|k);\6
${}\|k\K\|j{}$;\C{ the last run through that loop saw no dead nodes }\par
\U35.\fi

\M{37}At least one node, \PB{$\|v\MG\\{proj}$}, branches on $x_v$ at this
point.

\Y\B\4\X37:Downsize the table if only a few entries are left\X${}\E{}$\6
$\|k\K(\\{mask}\GG\T{2})+\T{1}-\\{free}{}$;\C{ this many nodes still branch on
$x_v$ }\6
\&{for} ${}(\\{newmask}\K\\{mask};{}$ ${}(\\{newmask}\GG\T{5})\G\|k;{}$ ${}%
\\{newmask}\MRL{{\GG}{\K}}\T{1}){}$\1\5
;\2\6
\&{if} ${}(\\{newmask}\I\\{mask}){}$\5
${}\{{}$\1\6
\&{if} ${}(\\{verbose}\AND\T{256}){}$\1\5
${}\\{printf}(\.{"downsizing\ the\ hash}\)\.{\ table\ for\ level\ \%d(}\)\.{x%
\%d)\ (\%d\ slots)\\n"},\39\|v-\\{varhead},\39\|v\MG\\{name},\39(\\{newmask}+%
\T{1})/\&{sizeof}(\&{addr}));{}$\2\6
${}\\{free}\MRL{-{\K}}(\\{mask}-\\{newmask})\GG\T{2};{}$\6
\X38:Rehash everything in the upper half\X;\6
\&{for} ${}(\|k\K\\{mask}\GG\\{logpagesize};{}$ ${}\|k>\\{newmask}\GG%
\\{logpagesize};{}$ ${}\|k\MM){}$\1\5
${}\|o,\39\\{free\_page}(\\{page\_}(\|v\MG\\{base}[\|k]));{}$\2\6
${}\|v\MG\\{mask}\K\\{newmask};{}$\6
\4${}\}{}$\2\par
\U35.\fi

\M{38}Finally, another algorithm learned from Rudell. To prove its correctness,
one can verify the following fact:
Any entries that wrapped around from the upper half to
the bottom in the original table will still wrap around in the smaller table.

\Y\B\4\X38:Rehash everything in the upper half\X${}\E{}$\6
\&{for} ${}(\|k\K\\{newmask}+\T{1};{}$ ${}\|k<\\{mask};{}$ ${}\|k\MRL{+{\K}}%
\&{sizeof}(\&{addr})){}$\5
${}\{{}$\1\6
${}\\{oo},\39\|r\K\\{fetchnode}(\|v,\39\|k);{}$\6
\&{if} (\|r)\5
${}\{{}$\1\6
\&{for} ${}(\|o,\39\\{hash}\K\\{hashedcode}(\|r);{}$  ; ${}\\{hash}\PP){}$\5
${}\{{}$\1\6
${}\|j\K\\{addr\_}(\\{hash})\AND\\{newmask};{}$\6
${}\\{oo},\39\|p\K\\{fetchnode}(\|v,\39\|j);{}$\6
\&{if} ${}(\R\|p){}$\1\5
\&{break};\2\6
\4${}\}{}$\2\6
${}\\{storenode}(\|v,\39\|j,\39\|r);{}$\6
\4${}\}{}$\2\6
\4${}\}{}$\2\par
\U37.\fi

\N{1}{39}The cache. The other principal data structure we need, besides the BDD
base
itself, is a software cache that helps us avoid repeating the calculations
that we've already done. If, for example, $f$ and $g$ are nodes of the BDD for
which we've already computed $h=f\land g$, the cache should contain the
information that $f\land g$ is known to be node~$h$.

But that description is only approximately correct, because
the cost of forgetting the value of $f\land g$ is less than the cost of
building a fancy data structure that is able to remember every result.
(If we forget only a few things, we need to do only a few recomputations.)
Therefore we adopt a simple scheme that is designed to be reliable most of
the time, yet not perfect: We look for $f\land g$ in only one position
within the cache, based on a hash code. If two or more results happen
to hash to the same cache slot, we remember only the most recent one.

Every entry of the cache consists of four tetrabytes, called
$f$, $g$, $h$, and~$r$. The last of these, $r$, is nonzero if and only if the
cache entry is meaningful; in that case $r$ points to a BDD node, the result
of an operation encoded by $f$, $g$, and~$h$.
This $(f,g,h)$ encoding has several variants:

\smallskip\textindent{$\bullet$} If $h$ is 0, then $g$ is
a ``time stamp,'' and $f$ points to a BDD node. This case is used for
functional composition, when we want to
invalidate a block of cache entries quickly by simply changing an
external time stamp; items with a stale time stamp won't match any further
cache lookups.

\smallskip\textindent{$\bullet$} If $0<h\le\PB{\\{maxbinop}}$, then $h$
denotes a binary operation on the BDD nodes $f$ and~$g$.
For example, $h=1$ stands for $\land$. The binary operations currently
implemented are:
and~(1),
but-not~(2),
not-but~(4),
xor~(6),
or~(7),
constrain~(8),
all-quantifier~(9),
no-quantifier~(10),
yes-quantifier~(12),
diff-quantifier~(14),
exists-quantifier~(15).

\smallskip\textindent{$\bullet$} Otherwise $(f,g,h)$ encodes a ternary
operation on the three BDD nodes $f$, $g$, \PB{$\|h\AND{-}\T{16}$}. The four
least-significant
bits of~$h$ are used to identify the ternary operation involved:
if-then-else~(0),
median~(1),
and-and~(2),
and-exist~(3),
not-yet-implemented~(4--15).

\Y\B\4\D$\\{memo\_}(\|a)$ \5
((\&{memo} ${}{*})(\&{size\_t})(\|a){}$)\par
\Y\B\4\X10:Type definitions\X${}\mathrel+\E{}$\6
\&{typedef} \&{struct} \&{memo\_struct} ${}\{{}$\1\6
\&{addr} \|f;\C{ first operand }\6
\&{addr} \|g;\C{ second operand, or time stamp }\6
\&{addr} \|h;\C{ third operand and/or operation code }\6
\&{addr} \|r;\C{ result }\2\6
${}\}{}$ \&{memo};\par
\fi

\M{40}The cache always occupies $2^e$ pages of the dynamic memory,
for some integer $e\ge0$. If we have leisure to choose this size, we pick
the smallest $e\ge0$ such that the cache has at least $\max(4m,n/4)$ slots,
where $m$ is the number of nonempty items in the cache and $n$ is
the number of live nodes in the BDD. Furthermore, the cache size
will double whenever the number of cache insertions reaches a
given threshold.

\Y\B\4\D$\\{logmaxcachepages}$ \5
\T{15}\C{ shouldn't be large if \PB{\\{logvarsize}} is large }\par
\B\4\D$\\{maxcachepages}$ \5
$(\T{1}\LL\\{logmaxcachepages}{}$)\par
\B\4\D$\\{cacheslotsperpage}$ \5
$(\\{pagesize}/\&{sizeof}(\&{memo}){}$)\par
\B\4\D$\\{maxbinop}$ \5
\T{15}\par
\Y\B\4\X4:Global variables\X${}\mathrel+\E{}$\6
\&{addr} \\{cachepage}[\\{maxcachepages}];\C{ base addresses for the cache }\6
\&{int} \\{cachepages};\C{ the current number of pages in the cache }\6
\&{int} \\{cacheinserts};\C{ the number of times we've inserted a memo }\6
\&{int} \\{threshold};\C{ the number of inserts that trigger cache doubling }\6
\&{int} \\{cachemask};\C{ index of the first slot following the cache, minus 1
}\par
\fi

\M{41}The following subroutines, useful for debugging, print out the
cache contents in symbolic form.

If \PB{\|p} points to a node, \PB{\\{id}(\|p)} is \PB{$\|p-\\{botsink}$}.

\Y\B\4\D$\\{id}(\|a)$ \5
$(((\&{size\_t})(\|a)-{}$(\&{size\_t}) \\{mem})${}/\&{sizeof}(\&{node}){}$)\C{
node number in \PB{\\{mem}} }\par
\Y\B\4\X7:Subroutines\X${}\mathrel+\E{}$\6
\&{void} \\{print\_memo}(\&{memo} ${}{*}\|m){}$\1\1\2\2\6
${}\{{}$\1\6
${}\\{printf}(\.{"\%x"},\39\\{id}(\|m\MG\|f));{}$\6
\&{if} ${}(\|m\MG\|h\E\T{0}){}$\1\5
${}\\{printf}(\.{"[\%d]"},\39\|m\MG\|g){}$;\C{ time stamp }\2\6
\&{else} \&{if} ${}(\|m\MG\|h\Z\\{maxbinop}){}$\1\5
${}\\{printf}(\.{"\%s\%x"},\39\\{binopname}[\|m\MG\|h],\39\\{id}(\|m\MG%
\|g));{}$\2\6
\&{else}\1\5
${}\\{printf}(\.{"\%s\%x\%s\%x"},\39\\{ternopname1}[\|m\MG\|h\AND\T{\^f}],\39%
\\{id}(\|m\MG\|g),\39\\{ternopname2}[\|m\MG\|h\AND\T{\^f}],\39\\{id}(\|m\MG%
\|h));{}$\2\6
${}\\{printf}(\.{"=\%x\\n"},\39\\{id}(\|m\MG\|r));{}$\6
\4${}\}{}$\2\7
\&{void} \\{print\_cache}(\&{void})\1\1\2\2\6
${}\{{}$\1\6
\&{register} \&{int} \|k;\6
\&{register} \&{memo} ${}{*}\|m;{}$\7
\&{for} ${}(\|k\K\T{0};{}$ ${}\|k<\\{cachepages};{}$ ${}\|k\PP){}$\1\6
\&{for} ${}(\|m\K\\{memo\_}(\\{cachepage}[\|k]);{}$ ${}\|m<\\{memo\_}(%
\\{cachepage}[\|k])+\\{cacheslotsperpage};{}$ ${}\|m\PP){}$\1\6
\&{if} ${}(\|m\MG\|r){}$\1\5
\\{print\_memo}(\|m);\2\2\2\6
\4${}\}{}$\2\par
\fi

\M{42}Many of the symbolic names here are presently unused. I've filled them
in just to facilitate extensions to this program.

\Y\B\4\X4:Global variables\X${}\mathrel+\E{}$\6
\&{char} ${}{*}\\{binopname}[\,]\K\{\.{""},\39\.{"\&"},\39\.{">"},\39\.{"!"},%
\39\.{"<"},\39\.{"@"},\39\.{"\^"},\39\.{"|"},\39\.{"\_"},\39\.{"A"},\39\.{"N"},%
\39\.{"\#"},\39\.{"Y"},\39\.{"\$"},\39\.{"D"},\39\.{"E"}\};{}$\6
\&{char} ${}{*}\\{ternopname1}[\,]\K\{\.{"?"},\39\.{"."},\39\.{"\&"},\39\.{"%
\&"},\39\.{"@"},\39\.{"\#"},\39\.{"\$"},\39\.{"\%"},\39\.{"*"},\39\.{"<"},\39%
\.{"-"},\39\.{"+"},\39\.{"|"},\39\.{"/"},\39\.{"\\\\"},\39\.{"\~"}\};{}$\6
\&{char} ${}{*}\\{ternopname2}[\,]\K\{\.{":"},\39\.{"."},\39\.{"\&"},\39%
\.{"E"},\39\.{"@"},\39\.{"\#"},\39\.{"\$"},\39\.{"\%"},\39\.{"*"},\39\.{"<"},%
\39\.{"-"},\39\.{"+"},\39\.{"|"},\39\.{"/"},\39\.{"\\\\"},\39\.{"\~"}\}{}$;\par
\fi

\M{43}The threshold is set to half the total number of cache slots,
because this many random insertions will keep about $e^{-1/2}\approx
61$\% of the cache slots unclobbered. (If $p$ denotes this probability,
a random large binary tree will need about $E$ steps to recalculate a
lost result, where $E=p\cdot1+(1-p)\cdot(1+2E)$; hence we want
$p>1/2$ to avoid blowup, and $E=1/(2p-1)$.)

\Y\B\4\X7:Subroutines\X${}\mathrel+\E{}$\6
\&{int} \\{choose\_cache\_size}(\&{int} \\{items})\1\1\2\2\6
${}\{{}$\1\6
\&{register} \&{int} \|k${},{}$ \\{slots};\7
${}\|k\K\T{1},\39\\{slots}\K\\{cacheslotsperpage};{}$\6
\&{while} ${}(\T{4}*\\{slots}<\\{totalnodes}-\\{deadnodes}\W\|k<%
\\{maxcachepages}){}$\1\5
${}\|k\MRL{{\LL}{\K}}\T{1},\39\\{slots}\MRL{{\LL}{\K}}\T{1};{}$\2\6
\&{while} ${}(\\{slots}<\T{4}*\\{items}\W\|k<\\{maxcachepages}){}$\1\5
${}\|k\MRL{{\LL}{\K}}\T{1},\39\\{slots}\MRL{{\LL}{\K}}\T{1};{}$\2\6
\&{return} \|k;\6
\4${}\}{}$\2\7
\&{void} \\{cache\_init}(\&{void})\1\1\2\2\6
${}\{{}$\1\6
\&{register} \&{int} \|k;\6
\&{register} \&{memo} ${}{*}\|m;{}$\7
${}\\{cachepages}\K\\{choose\_cache\_size}(\T{0});{}$\6
\&{if} ${}(\\{verbose}\AND(\T{8}+\T{16}+\T{32}+\T{512})){}$\1\5
${}\\{printf}(\.{"initializing\ the\ ca}\)\.{che\ (\%d\ page\%s)\\n"},\39%
\\{cachepages},\39\\{cachepages}\E\T{1}\?\.{""}:\.{"s"});{}$\2\6
\&{for} ${}(\|k\K\T{0};{}$ ${}\|k<\\{cachepages};{}$ ${}\|k\PP){}$\5
${}\{{}$\1\6
${}\|o,\39\\{cachepage}[\|k]\K\\{addr\_}(\\{reserve\_page}(\,));{}$\6
\&{if} ${}(\R\\{cachepage}[\|k]){}$\5
${}\{{}$\1\6
${}\\{fprintf}(\\{stderr},\39\.{"(trouble\ allocating}\)\.{\ cache\ pages!)%
\\n"});{}$\6
\&{for} ${}(\|k\MM;{}$ ${}(\|k+\T{1})\AND\|k;{}$ ${}\|k\MM){}$\1\5
${}\|o,\39\\{free\_page}(\\{page\_}(\\{cachepage}[\|k]));{}$\2\6
${}\\{cachepages}\K\|k+\T{1};{}$\6
\&{break};\6
\4${}\}{}$\2\6
\&{for} ${}(\|m\K\\{memo\_}(\\{cachepage}[\|k]);{}$ ${}\|m<\\{memo\_}(%
\\{cachepage}[\|k])+\\{cacheslotsperpage};{}$ ${}\|m\PP){}$\1\5
${}\|m\MG\|r\K\T{0};{}$\2\6
${}\\{zmems}\MRL{+{\K}}\\{cacheslotsperpage};{}$\6
\4${}\}{}$\2\6
${}\\{cachemask}\K(\\{cachepages}\LL\\{logpagesize})-\T{1};{}$\6
${}\\{cacheinserts}\K\T{0};{}$\6
${}\\{threshold}\K\T{1}+(\\{cachepages}*\\{cacheslotsperpage})/\T{2};{}$\6
\4${}\}{}$\2\par
\fi

\M{44}\B\X5:Initialize everything\X${}\mathrel+\E{}$\6
\\{cache\_init}(\,);\par
\fi

\M{45}Here's how we look for a memo in the cache. Memos might point to dead
nodes, as long as those nodes still exist.

A simple hash function is adequate for caching, because no clustering
can occur.

No mems are charged for computing \PB{\\{cachehash}}, because we assume that
the calling routine has taken responsibility for accessing \PB{$\|f\MG%
\\{index}$}
and \PB{$\|g\MG\\{index}$}.

\Y\B\4\D$\\{cachehash}(\|f,\|g,\|h)$ \5
$((\|f)\MG\\{index}\LL\T{4})\XOR(((\|h)\?(\|g)\MG\\{index}:\\{addr\_}(\|g))\LL%
\T{5})\XOR(\\{addr\_}(\|h)\LL\T{6}{}$)\par
\B\4\D$\\{thememo}(\|s)$ \5
$\\{memo\_}(\\{cachepage}[((\|s)\AND\\{cachemask})\GG\\{logpagesize}]+((\|s)%
\AND\\{pagemask}){}$)\par
\Y\B\4\X7:Subroutines\X${}\mathrel+\E{}$\6
\&{node} ${}{*}{}$\\{cache\_lookup}(\&{node} ${}{*}\|f,\39{}$\&{node} ${}{*}%
\|g,\39{}$\&{node} ${}{*}\|h){}$\1\1\2\2\6
${}\{{}$\1\6
\&{register} \&{node} ${}{*}\|r;{}$\6
\&{register} \&{memo} ${}{*}\|m;{}$\6
\&{register} \&{addr} \\{slot}${}\K\\{cachehash}(\|f,\39\|g,\39\|h);{}$\7
${}\|o,\39\|m\K\\{thememo}(\\{slot});{}$\6
${}\|o,\39\|r\K\\{node\_}(\|m\MG\|r);{}$\6
\&{if} ${}(\R\|r){}$\1\5
\&{return} ${}\NULL;{}$\2\6
\&{if} ${}(\|o,\39\\{node\_}(\|m\MG\|f)\E\|f\W\\{node\_}(\|m\MG\|g)\E\|g\W%
\\{node\_}(\|m\MG\|h)\E\|h){}$\5
${}\{{}$\1\6
\&{if} ${}(\\{verbose}\AND\T{8}){}$\5
${}\{{}$\1\6
${}\\{printf}(\.{"hit\ \%x:\ "},\39(\\{slot}\AND\\{cachemask})/\&{sizeof}(%
\&{memo}));{}$\6
\\{print\_memo}(\|m);\6
\4${}\}{}$\2\6
\&{if} ${}(\|o,\39\|r\MG\\{xref}<\T{0}){}$\5
${}\{{}$\1\6
\\{recursively\_revive}(\|r);\6
\&{return} \|r;\6
\4${}\}{}$\2\6
\&{return} \|o${},\39\|r\MG\\{xref}\PP,\39\|r;{}$\6
\4${}\}{}$\2\6
\&{return} ${}\NULL;{}$\6
\4${}\}{}$\2\par
\fi

\M{46}Insertion into the cache is even easier, except that we might
want to double the cache size while we're at it.

\Y\B\4\X7:Subroutines\X${}\mathrel+\E{}$\6
\&{void} \\{cache\_insert}(\&{node} ${}{*}\|f,\39{}$\&{node} ${}{*}\|g,\39{}$%
\&{node} ${}{*}\|h,\39{}$\&{node} ${}{*}\|r){}$\1\1\2\2\6
${}\{{}$\1\6
\&{register} \&{memo} ${}{*}\|m,{}$ ${}{*}\\{mm};{}$\6
\&{register} \&{int} \|k;\6
\&{register} \&{int} \\{slot}${}\K\\{cachehash}(\|f,\39\|g,\39\|h);{}$\7
\&{if} (\|h)\1\5
\\{oo};\5
\2\&{else}\1\5
\|o;\C{ mems for computing \PB{\\{cachehash}} }\2\6
\&{if} ${}(\PP\\{cacheinserts}\G\\{threshold}){}$\1\5
\X47:Double the cache size\X;\2\6
${}\|o,\39\|m\K\\{thememo}(\\{slot});{}$\6
\&{if} ${}((\\{verbose}\AND\T{16})\W\|m\MG\|r){}$\5
${}\{{}$\1\6
${}\\{printf}(\.{"lose\ \%x:\ "},\39(\\{slot}\AND\\{cachemask})/\&{sizeof}(%
\&{memo}));{}$\6
\\{print\_memo}(\|m);\6
\4${}\}{}$\2\6
${}\\{oo},\39\|m\MG\|f\K\\{addr\_}(\|f),\39\|m\MG\|g\K\\{addr\_}(\|g),\39\|m\MG%
\|h\K\\{addr\_}(\|h),\39\|m\MG\|r\K\\{addr\_}(\|r);{}$\6
\&{if} ${}(\\{verbose}\AND\T{32}){}$\5
${}\{{}$\1\6
${}\\{printf}(\.{"set\ \%x:\ "},\39(\\{slot}\AND\\{cachemask})/\&{sizeof}(%
\&{memo}));{}$\6
\\{print\_memo}(\|m);\6
\4${}\}{}$\2\6
\4${}\}{}$\2\par
\fi

\M{47}\B\X47:Double the cache size\X${}\E{}$\6
\&{if} ${}(\\{cachepages}<\\{maxcachepages}){}$\5
${}\{{}$\1\6
\&{if} ${}(\\{verbose}\AND(\T{8}+\T{16}+\T{32}+\T{512})){}$\1\5
${}\\{printf}(\.{"doubling\ the\ cache\ }\)\.{(\%d\ pages)\\n"},\39%
\\{cachepages}\LL\T{1});{}$\2\6
\&{for} ${}(\|k\K\\{cachepages};{}$ ${}\|k<\\{cachepages}+\\{cachepages};{}$
${}\|k\PP){}$\5
${}\{{}$\1\6
${}\|o,\39\\{cachepage}[\|k]\K\\{addr\_}(\\{reserve\_page}(\,));{}$\6
\&{if} ${}(\R\\{cachepage}[\|k]){}$\5
${}\{{}$\C{ sorry, we can't double the cache after all }\1\6
${}\\{fprintf}(\\{stderr},\39\.{"(trouble\ doubling\ c}\)\.{ache\ pages!)%
\\n"});{}$\6
\&{for} ${}(\|k\MM;{}$ ${}\|k\G\\{cachepages};{}$ ${}\|k\MM){}$\1\5
${}\|o,\39\\{free\_page}(\\{page\_}(\\{cachepage}[\|k]));{}$\2\6
\&{goto} \\{done};\6
\4${}\}{}$\2\6
\&{for} ${}(\|m\K\\{memo\_}(\\{cachepage}[\|k]);{}$ ${}\|m<\\{memo\_}(%
\\{cachepage}[\|k])+\\{cacheslotsperpage};{}$ ${}\|m\PP){}$\1\5
${}\|m\MG\|r\K\T{0};{}$\2\6
${}\\{zmems}\MRL{+{\K}}\\{cacheslotsperpage};{}$\6
\4${}\}{}$\2\6
${}\\{cachepages}\MRL{{\LL}{\K}}\T{1};{}$\6
${}\\{cachemask}\MRL{+{\K}}\\{cachemask}+\T{1};{}$\6
${}\\{threshold}\K\T{1}+(\\{cachepages}*\\{cacheslotsperpage})/\T{2};{}$\6
\X48:Recache the items in the bottom half\X;\6
\4${}\}{}$\2\6
\4\\{done}:\par
\U46.\fi

\M{48}\B\X48:Recache the items in the bottom half\X${}\E{}$\6
\&{for} ${}(\|k\K\\{cachepages}\GG\T{1};{}$ ${}\|k<\\{cachepages};{}$ ${}\|k%
\PP){}$\5
${}\{{}$\1\6
\&{for} ${}(\|o,\39\|m\K\\{memo\_}(\\{cachepage}[\|k]);{}$ ${}\|m<\\{memo\_}(%
\\{cachepage}[\|k])+\\{cacheslotsperpage};{}$ ${}\|m\PP){}$\1\6
\&{if} ${}(\|o,\39\|m\MG\|r){}$\5
${}\{{}$\1\6
\&{if} ${}(\|m\MG\|h){}$\1\5
\\{oo};\5
\2\&{else}\1\5
\|o;\C{ mems for computing \PB{\\{cachehash}} }\2\6
${}\\{oo},\39\\{mm}\K\\{thememo}(\\{cachehash}(\\{node\_}(\|m\MG\|f),\39\\{node%
\_}(\|m\MG\|g),\39\\{node\_}(\|m\MG\|h)));{}$\6
\&{if} ${}(\|m\I\\{mm}){}$\5
${}\{{}$\1\6
${}\\{oo},\39{*}\\{mm}\K{*}\|m;{}$\6
${}\|o,\39\|m\MG\|r\K\T{0};{}$\6
\4${}\}{}$\2\6
\4${}\}{}$\2\2\6
\4${}\}{}$\2\par
\U47.\fi

\M{49}Before we purge elements from the unique tables, we need to purge all
references to dead nodes from the cache. At the same time, we might as
well purge items whose time stamp has expired.

\Y\B\4\X7:Subroutines\X${}\mathrel+\E{}$\6
\&{void} \\{cache\_purge}(\&{void})\1\1\2\2\6
${}\{{}$\1\6
\&{register} \&{int} \|k${},{}$ \\{items}${},{}$ \\{newcachepages};\6
\&{register} \&{memo} ${}{*}\|m,{}$ ${}{*}\\{mm};{}$\7
\&{for} ${}(\|k\K\\{items}\K\T{0};{}$ ${}\|k<\\{cachepages};{}$ ${}\|k\PP){}$\5
${}\{{}$\1\6
\&{for} ${}(\|m\K\\{memo\_}(\\{cachepage}[\|k]);{}$ ${}\|m<\\{memo\_}(%
\\{cachepage}[\|k])+\\{cacheslotsperpage};{}$ ${}\|m\PP){}$\1\6
\&{if} ${}(\|o,\39\|m\MG\|r){}$\5
${}\{{}$\1\6
\&{if} ${}((\|o,\39\\{node\_}(\|m\MG\|r)\MG\\{xref}<\T{0})\V(\\{oo},\39\\{node%
\_}(\|m\MG\|f)\MG\\{xref}<\T{0})){}$\1\5
\&{goto} \\{purge};\2\6
\&{if} ${}(\|m\MG\|h\E\T{0}){}$\5
${}\{{}$\1\6
\&{if} ${}(\|m\MG\|g\I\\{thevar}(\\{node\_}(\|m\MG\|f))\MG\\{timestamp}){}$\1\5
\&{goto} \\{purge};\2\6
\4${}\}{}$\5
\2\&{else}\5
${}\{{}$\1\6
\&{if} ${}(\|o,\39\\{node\_}(\|m\MG\|g)\MG\\{xref}<\T{0}){}$\1\5
\&{goto} \\{purge};\2\6
\&{if} ${}(\|m\MG\|h>\\{maxbinop}\W(\|o,\39\\{node\_}(\|m\MG\|h\AND{-}\T{\^10})%
\MG\\{xref}<\T{0})){}$\1\5
\&{goto} \\{purge};\2\6
\4${}\}{}$\2\6
${}\\{items}\PP{}$;\5
\&{continue};\6
\4\\{purge}:\5
${}\|o,\39\|m\MG\|r\K\T{0};{}$\6
\4${}\}{}$\2\2\6
\4${}\}{}$\2\6
\&{if} ${}(\\{verbose}\AND(\T{8}+\T{16}+\T{32}+\T{512})){}$\1\5
${}\\{printf}(\.{"purging\ the\ cache\ (}\)\.{\%d\ items\ left)\\n"},\39%
\\{items});{}$\2\6
\X50:Downsize the cache if it has now become too sparse\X;\6
${}\\{cacheinserts}\K\\{items};{}$\6
\4${}\}{}$\2\par
\fi

\M{50}\B\X50:Downsize the cache if it has now become too sparse\X${}\E{}$\6
$\\{newcachepages}\K\\{choose\_cache\_size}(\\{items});{}$\6
\&{if} ${}(\\{newcachepages}<\\{cachepages}){}$\5
${}\{{}$\1\6
\&{if} ${}(\\{verbose}\AND(\T{8}+\T{16}+\T{32}+\T{512})){}$\1\5
${}\\{printf}(\.{"downsizing\ the\ cach}\)\.{e\ (\%d\ page\%s)\\n"},\39%
\\{newcachepages},\39\\{newcachepages}\E\T{1}\?\.{""}:\.{"s"});{}$\2\6
${}\\{cachemask}\K(\\{newcachepages}\LL\\{logpagesize})-\T{1};{}$\6
\&{for} ${}(\|k\K\\{newcachepages};{}$ ${}\|k<\\{cachepages};{}$ ${}\|k\PP){}$\5
${}\{{}$\1\6
\&{for} ${}(\|o,\39\|m\K\\{memo\_}(\\{cachepage}[\|k]);{}$ ${}\|m<\\{memo\_}(%
\\{cachepage}[\|k])+\\{cacheslotsperpage};{}$ ${}\|m\PP){}$\1\6
\&{if} ${}(\|o,\39\|m\MG\|r){}$\5
${}\{{}$\1\6
\&{if} ${}(\|m\MG\|h){}$\1\5
\\{oo};\5
\2\&{else}\1\5
\|o;\C{ mems for computing \PB{\\{cachehash}} }\2\6
${}\\{oo},\39\\{mm}\K\\{thememo}(\\{cachehash}(\\{node\_}(\|m\MG\|f),\39\\{node%
\_}(\|m\MG\|g),\39\\{node\_}(\|m\MG\|h)));{}$\6
\&{if} ${}(\|m\I\\{mm}){}$\5
${}\{{}$\1\6
${}\\{oo},\39{*}\\{mm}\K{*}\|m;{}$\6
\4${}\}{}$\2\6
\4${}\}{}$\2\2\6
\\{free\_page}(\\{page\_}(\\{cachepage}[\|k]));\6
\4${}\}{}$\2\6
${}\\{cachepages}\K\\{newcachepages};{}$\6
${}\\{threshold}\K\T{1}+(\\{cachepages}*\\{cacheslotsperpage})/\T{2};{}$\6
\4${}\}{}$\2\par
\U49.\fi

\N{1}{51}BDD structure. The reader of this program ought to be familiar with
the basics of BDDs, namely the facts that a BDD base consists of
two sink nodes together with an unlimited number of branch nodes,
where each branch node $(v,l,h)$ names a variable $x_v$ and points
to other nodes $l\ne h$ that correspond to the cases where $x_v=0$
and $x_v=1$. The variables on every path have increasing rank~$v$, and no
two nodes have the same $(v,l,h)$.

Besides the nodes of the BDD, this program deals with external pointers $f_j$
for $0\le j<\PB{\\{extsize}}$. Each $f_j$ is either \PB{$\NULL$} or points to a
BDD node.

\Y\B\4\D$\\{extsize}$ \5
\T{1000}\par
\Y\B\4\X4:Global variables\X${}\mathrel+\E{}$\6
\&{node} ${}{*}\|f[\\{extsize}]{}$;\C{ external pointers to functions in the
BDD base }\par
\fi

\M{52}Sometimes we want to mark the nodes of a subfunction temporarily.
The following routine sets the leading bit of the \PB{\\{xref}} field
in all nodes reachable from~\PB{\|p}.

\Y\B\4\X7:Subroutines\X${}\mathrel+\E{}$\6
\&{void} \\{mark}(\&{node} ${}{*}\|p){}$\1\1\2\2\6
${}\{{}$\1\6
${}\\{rmems}\PP{}$;\C{ track recursion overhead }\6
\4\\{restart}:\5
\&{if} ${}(\|o,\39\|p\MG\\{xref}\G\T{0}){}$\5
${}\{{}$\1\6
${}\|o,\39\|p\MG\\{xref}\MRL{{\XOR}{\K}}\T{\^80000000};{}$\6
${}\\{ooo},\39\\{mark}(\\{node\_}(\|p\MG\\{lo})){}$;\C{ two extra mems to save
and restore \PB{\|p} }\6
${}\|o,\39\|p\K\\{node\_}(\|p\MG\\{hi});{}$\6
\&{goto} \\{restart};\C{ tail recursion }\6
\4${}\}{}$\2\6
\4${}\}{}$\2\par
\fi

\M{53}We need to remove those marks soon after \PB{\\{mark}} has been called,
because the \PB{\\{xref}} field is really supposed to count references.

\Y\B\4\X7:Subroutines\X${}\mathrel+\E{}$\6
\&{void} \\{unmark}(\&{node} ${}{*}\|p){}$\1\1\2\2\6
${}\{{}$\1\6
${}\\{rmems}\PP{}$;\C{ track recursion overhead }\6
\4\\{restart}:\5
\&{if} ${}(\|o,\39\|p\MG\\{xref}<\T{0}){}$\5
${}\{{}$\1\6
${}\|o,\39\|p\MG\\{xref}\MRL{{\XOR}{\K}}\T{\^80000000};{}$\6
${}\\{ooo},\39\\{unmark}(\\{node\_}(\|p\MG\\{lo})){}$;\C{ two extra mems to
save and restore \PB{\|p} }\6
${}\|o,\39\|p\K\\{node\_}(\|p\MG\\{hi});{}$\6
\&{goto} \\{restart};\C{ tail recursion }\6
\4${}\}{}$\2\6
\4${}\}{}$\2\par
\fi

\M{54}Here's a simple routine that prints out the current BDDs, in order
of the variables in branch nodes. If the \PB{\\{marked}} parameter is nonzero,
the output is restricted to branch nodes whose \PB{\\{xref}} field is
marked. Otherwise all nodes are shown, with nonzero \PB{\\{xref}}s in
parentheses.

\Y\B\4\D$\\{thevar}(\|p)$ \5
$({\AND}\\{varhead}[\\{varpart}((\|p)\MG\\{index})]{}$)\par
\B\4\D$\\{print\_node}(\|p)$ \5
$\\{printf}(\.{"\%x:\ (\~\%d?\%x:\%x)"},\39\\{id}(\|p),\39\\{thevar}(\|p)\MG%
\\{name},\39\\{id}((\|p)\MG\\{lo}),\39\\{id}((\|p)\MG\\{hi}){}$)\par
\B\4\D$\\{print\_node\_unmapped}(\|p)$ \5
$\\{printf}(\.{"\%x:\ (\~\%d?\%x:\%x)"},\39\\{id}(\|p),\39\\{thevar}(\|p)-%
\\{varhead},\39\\{id}((\|p)\MG\\{lo}),\39\\{id}((\|p)\MG\\{hi}){}$)\par
\Y\B\4\X7:Subroutines\X${}\mathrel+\E{}$\6
\&{void} \\{print\_base}(\&{int} \\{marked}${},\39{}$\&{int} \\{unmapped})\1\1%
\2\2\6
${}\{{}$\1\6
\&{register} \&{int} \|j${},{}$ \|k;\6
\&{register} \&{node} ${}{*}\|p;{}$\6
\&{register} \&{var} ${}{*}\|v;{}$\7
\&{for} ${}(\|v\K\\{varhead};{}$ ${}\|v<\\{topofvars};{}$ ${}\|v\PP){}$\1\6
\&{if} ${}(\|v\MG\\{proj}){}$\5
${}\{{}$\1\6
\&{for} ${}(\|k\K\T{0};{}$ ${}\|k<\|v\MG\\{mask};{}$ ${}\|k\MRL{+{\K}}%
\&{sizeof}(\&{addr})){}$\5
${}\{{}$\1\6
${}\|p\K\\{fetchnode}(\|v,\39\|k);{}$\6
\&{if} ${}(\|p\W(\R\\{marked}\V(\|p\MG\\{xref}+\T{1})<\T{0})){}$\5
${}\{{}$\1\6
\&{if} (\\{unmapped})\1\5
\\{print\_node\_unmapped}(\|p);\5
\2\&{else}\1\5
\\{print\_node}(\|p);\2\6
\&{if} ${}(\\{marked}\V\|p\MG\\{xref}\E\T{0}){}$\1\5
\\{printf}(\.{"\\n"});\2\6
\&{else}\1\5
${}\\{printf}(\.{"\ (\%d)\\n"},\39\|p\MG\\{xref});{}$\2\6
\4${}\}{}$\2\6
\4${}\}{}$\2\6
\&{if} ${}(\R\\{marked}\W\|v\MG\\{repl}){}$\1\5
${}\\{printf}(\.{"y\%d=\%x\\n"},\39\|v\MG\\{name},\39\\{id}(\|v\MG%
\\{repl}));{}$\2\6
\4${}\}{}$\2\2\6
\&{if} ${}(\R\\{marked}){}$\5
${}\{{}$\C{ we also print the external functions }\1\6
\&{for} ${}(\|j\K\T{0};{}$ ${}\|j<\\{extsize};{}$ ${}\|j\PP){}$\1\6
\&{if} (\|f[\|j])\1\5
${}\\{printf}(\.{"f\%d=\%x\\n"},\39\|j,\39\\{id}(\|f[\|j]));{}$\2\2\6
\4${}\}{}$\2\6
\4${}\}{}$\2\par
\fi

\M{55}The marking feature is useful when we want to print out only a single
BDD.

\Y\B\4\X7:Subroutines\X${}\mathrel+\E{}$\6
\&{void} \\{print\_function}(\&{node} ${}{*}\|p,\39{}$\&{int} \\{unmapped})\1\1%
\2\2\6
${}\{{}$\1\6
\&{unsigned} \&{long} \&{long} \\{savemems}${}\K\\{mems},{}$ \\{savermems}${}\K%
\\{rmems}{}$;\C{ mems aren't counted while printing }\7
\&{if} ${}(\|p\E\\{botsink}\V\|p\E\\{topsink}){}$\1\5
${}\\{printf}(\.{"\%d\\n"},\39\|p-\\{botsink});{}$\2\6
\&{else} \&{if} (\|p)\5
${}\{{}$\1\6
\\{mark}(\|p);\6
${}\\{print\_base}(\T{1},\39\\{unmapped});{}$\6
\\{unmark}(\|p);\6
\4${}\}{}$\2\6
${}\\{mems}\K\\{savemems},\39\\{rmems}\K\\{savermems};{}$\6
\4${}\}{}$\2\par
\fi

\M{56}\B\X7:Subroutines\X${}\mathrel+\E{}$\6
\&{void} \\{print\_profile}(\&{node} ${}{*}\|p){}$\1\1\2\2\6
${}\{{}$\1\6
\&{unsigned} \&{long} \&{long} \\{savemems}${}\K\\{mems},{}$ \\{savermems}${}\K%
\\{rmems};{}$\6
\&{register} \&{int} \|j${},{}$ \|k${},{}$ \\{tot};\6
\&{register} \&{var} ${}{*}\|v;{}$\7
\&{if} ${}(\R\|p){}$\1\5
\\{printf}(\.{"\ 0\\n"});\C{ vacuous }\2\6
\&{else} \&{if} ${}(\|p\Z\\{topsink}){}$\1\5
\\{printf}(\.{"\ 1\\n"});\C{ constant }\2\6
\&{else}\5
${}\{{}$\1\6
${}\\{tot}\K\T{0};{}$\6
\\{mark}(\|p);\6
\&{for} ${}(\|v\K\\{varhead};{}$ ${}\|v<\\{topofvars};{}$ ${}\|v\PP){}$\1\6
\&{if} ${}(\|v\MG\\{proj}){}$\5
${}\{{}$\1\6
\X57:Print the number of marked nodes that branch on \PB{\|v}\X;\6
\4${}\}{}$\2\2\6
\\{unmark}(\|p);\6
${}\\{printf}(\.{"\ 2\ (total\ \%d)\\n"},\39\\{tot}+\T{2}){}$;\C{ the sinks }\6
\4${}\}{}$\2\6
${}\\{mems}\K\\{savemems},\39\\{rmems}\K\\{savermems};{}$\6
\4${}\}{}$\2\par
\fi

\M{57}\B\X57:Print the number of marked nodes that branch on \PB{\|v}\X${}\E{}$%
\6
\&{for} ${}(\|j\K\|k\K\T{0};{}$ ${}\|k<\|v\MG\\{mask};{}$ ${}\|k\MRL{+{\K}}%
\&{sizeof}(\&{addr})){}$\5
${}\{{}$\1\6
\&{register} \&{node} ${}{*}\|q\K\\{fetchnode}(\|v,\39\|k);{}$\7
\&{if} ${}(\|q\W(\|q\MG\\{xref}+\T{1})<\T{0}){}$\1\5
${}\|j\PP;{}$\2\6
\4${}\}{}$\2\6
${}\\{printf}(\.{"\ \%d"},\39\|j);{}$\6
${}\\{tot}\MRL{+{\K}}\|j{}$;\par
\U56.\fi

\M{58}In order to deal efficiently with large BDDs, we've introduced highly
redundant data structures, including things like hash tables and the cache.
Furthermore, we assume that every BDD node~\PB{\|p} has a redundant field
\PB{$\|p\MG\\{xref}$}, which counts the total number of branch nodes, external
nodes,
replacement functions, and projection functions that point to~\PB{\|p},
minus~1.

Bugs in this program might easily corrupt the data structure by putting it
into an inconsistent state. Yet the inconsistency might not show up at the
time of the error; the computer might go on to execute millions of
instructions before the anomalies lead to disaster.

Therefore I've written a \PB{\\{sanity\_check}} routine, which laboriously
checks the
integrity of all the data structures. This routine should help me to pinpoint
problems readily whenever I make mistakes. And besides, the \PB{\\{sanity%
\_check}}
calculations document the structures in a way that should be especially
helpful when I reread this program a year from now.

Even today, I think that the very experience of writing \PB{\\{sanity\_check}}
has
made me much more familiar with the structures themselves. This reinforced
knowledge will surely be valuable as I write the rest of the code.

\Y\B\4\D$\\{includesanity}$ \5
\T{1}\par
\Y\B\4\X7:Subroutines\X${}\mathrel+\E{}$\6
\8\#\&{if} \\{includesanity}\6
\&{unsigned} \&{int} \\{sanitycount};\C{ how many sanity checks have been
started? }\7
\&{void} \\{sanity\_check}(\&{void})\1\1\2\2\6
${}\{{}$\1\6
\&{register} \&{node} ${}{*}\|p,{}$ ${}{*}\|q;{}$\6
\&{register} \&{int} \|j${},{}$ \|k${},{}$ \\{count}${},{}$ \\{extra};\6
\&{register} \&{var} ${}{*}\|v;{}$\6
\&{unsigned} \&{long} \&{long} \\{savemems}${}\K\\{mems};{}$\7
${}\\{sanitycount}\PP;{}$\6
\X60:Build the shadow memory\X;\6
\X66:Check the reference counts\X;\6
\X68:Check the unique tables\X;\6
\X69:Check the cache\X;\6
\X70:Check the list of free pages\X;\6
${}\\{mems}\K\\{savemems};{}$\6
\4${}\}{}$\2\6
\8\#\&{endif}\par
\fi

\M{59}Sanity checking is done with a ``shadow memory'' \PB{\\{smem}}, which is
just
as big as~\PB{\\{mem}}. If \PB{\|p} points to a node in \PB{\\{mem}}, there's a
corresponding
``ghost'' in \PB{\\{smem}}, pointed to by \PB{$\|q\K\\{ghost}(\|p)$}. The ghost
nodes have four
fields \PB{\\{lo}}, \PB{\\{hi}}, \PB{\\{xref}}, and \PB{\\{index}}, just like
ordinary nodes do; but the
meanings of those fields are quite different: \PB{$\|q\MG\\{xref}$} is $-1$ if
node~\PB{\|p}
is in the free list, otherwise \PB{$\|q\MG\\{xref}$} is a backpointer to a
field
that points to~\PB{\|p}. If \PB{$\|p\MG\\{lo}$} points to~\PB{\|r}, then \PB{$%
\|q\MG\\{lo}$} will be
a backpointer that continues the list of pointers to~\PB{\|r} that began
with the \PB{\\{xref}} field in \PB{\|r}'s ghost; and there's a similar
relationship
between \PB{$\|p\MG\\{hi}$} and \PB{$\|q\MG\\{hi}$}. (Thus we can find all
nodes that point to~\PB{\|p}.)
Finally, \PB{$\|q\MG\\{index}$} counts the number of references to~\PB{\|p}
from external
pointers, projection functions, and replacement functions.

\Y\B\4\D$\\{ghost}(\|p)$ \5
$\\{node\_}((\&{size\_t})(\|p)-{}$(\&{size\_t}) \\{mem}${}+{}$(\&{size\_t}) %
\\{smem})\par
\Y\B\4\X4:Global variables\X${}\mathrel+\E{}$\6
\8\#\&{if} \\{includesanity}\6
\&{char} \\{smem}[\\{memsize}];\C{ the shadow memory }\6
\8\#\&{endif}\par
\fi

\M{60}\B\D$\\{complain}(\\{complaint})$ \6
${}\{{}$\5
\1${}\\{printf}(\.{"!\ \%s\ in\ node\ "},\39\\{complaint}){}$;\5
\\{print\_node}(\|p);\5
\\{printf}(\.{"\\n"});\5
${}\}{}$\2\par
\B\4\D$\\{legit}(\|p)$ \5
$(((\&{size\_t})(\|p)\AND(\&{sizeof}(\&{node})-\T{1}))\E\T{0}\W(\|p)<%
\\{nodeptr}\W(\|p)\G\\{botsink}\W\\{ghost}(\|p)\MG\\{xref}\I{-}\T{1}{}$)\par
\B\4\D$\\{superlegit}(\|p)$ \5
$(((\&{size\_t})(\|p)\AND(\&{sizeof}(\&{node})-\T{1}))\E\T{0}\W(\|p)<%
\\{nodeptr}\W(\|p)>\\{topsink}\W\\{ghost}(\|p)\MG\\{xref}\I{-}\T{1}{}$)\par
\Y\B\4\X60:Build the shadow memory\X${}\E{}$\6
\&{for} ${}(\|p\K\\{botsink};{}$ ${}\|p<\\{nodeptr};{}$ ${}\|p\PP){}$\1\5
${}\\{ghost}(\|p)\MG\\{xref}\K\T{0},\39\\{ghost}(\|p)\MG\\{index}\K{-}\T{1};{}$%
\2\6
\X64:Check the list of free nodes\X;\6
\X65:Compute the ghost index fields\X;\6
\&{for} ${}(\\{count}\K\T{2},\39\|p\K\\{topsink}+\T{1};{}$ ${}\|p<%
\\{nodeptr};{}$ ${}\|p\PP){}$\1\6
\&{if} ${}(\\{ghost}(\|p)\MG\\{xref}\I{-}\T{1}){}$\5
${}\{{}$\1\6
${}\\{count}\PP;{}$\6
\&{if} ${}(\R\\{legit}(\\{node\_}(\|p\MG\\{lo}))\V\R\\{legit}(\\{node\_}(\|p\MG%
\\{hi}))){}$\1\5
\\{complain}(\.{"bad\ pointer"})\2\6
\&{else} \&{if} ${}(\\{node\_}(\\{thevar}(\|p)\MG\\{proj})\E\NULL){}$\1\5
\\{complain}(\.{"bad\ var"})\2\6
\&{else} \&{if} ${}(\|p\MG\\{lo}\E\|p\MG\\{hi}){}$\1\5
\\{complain}(\.{"lo=hi"})\2\6
\&{else}\5
${}\{{}$\1\6
\X63:Check that \PB{\|p} is findable in the unique table\X;\6
\&{if} ${}(\\{node\_}(\|p\MG\\{lo})>\\{topsink}\W\\{thevar}(\|p)\G\\{thevar}(%
\\{node\_}(\|p\MG\\{lo}))){}$\1\5
\\{complain}(\.{"bad\ lo\ rank"});\2\6
\&{if} ${}(\\{node\_}(\|p\MG\\{hi})>\\{topsink}\W\\{thevar}(\|p)\G\\{thevar}(%
\\{node\_}(\|p\MG\\{hi}))){}$\1\5
\\{complain}(\.{"bad\ hi\ rank"});\2\6
\&{if} ${}(\|p\MG\\{xref}\G\T{0}){}$\5
${}\{{}$\C{ dead nodes don't point }\1\6
${}\|q\K\\{ghost}(\|p);{}$\6
${}\|q\MG\\{lo}\K\\{ghost}(\|p\MG\\{lo})\MG\\{xref},\39\\{ghost}(\|p\MG\\{lo})%
\MG\\{xref}\K\\{addr\_}({\AND}(\|p\MG\\{lo}));{}$\6
${}\|q\MG\\{hi}\K\\{ghost}(\|p\MG\\{hi})\MG\\{xref},\39\\{ghost}(\|p\MG\\{hi})%
\MG\\{xref}\K\\{addr\_}({\AND}(\|p\MG\\{hi}));{}$\6
\4${}\}{}$\2\6
\4${}\}{}$\2\6
\4${}\}{}$\2\2\6
\&{if} ${}(\\{count}\I\\{totalnodes}){}$\1\5
${}\\{printf}(\.{"!\ totalnodes\ should}\)\.{\ be\ \%d,\ not\ \%d\\n"},\39%
\\{count},\39\\{totalnodes});{}$\2\6
\&{if} ${}(\\{extra}\I\\{totalnodes}){}$\1\5
${}\\{printf}(\.{"!\ \%d\ nodes\ have\ lea}\)\.{ked\\n"},\39\\{extra}-%
\\{totalnodes}){}$;\2\par
\U58.\fi

\M{61}The macros above and the \PB{\\{who\_points\_to}} routine below rely on
the fact that
\PB{$\&{sizeof}(\&{node})\K\T{16}$}.

\fi

\M{62}\B\X5:Initialize everything\X${}\mathrel+\E{}$\6
\&{if} ${}(\&{sizeof}(\&{node})\I\T{16}){}$\5
${}\{{}$\1\6
${}\\{fprintf}(\\{stderr},\39\.{"Sorry,\ I\ assume\ tha}\)\.{t\ sizeof(node)\
is\ 16}\)\.{!\\n"});{}$\6
${}\\{exit}({-}\T{3});{}$\6
\4${}\}{}$\2\par
\fi

\M{63}\B\X63:Check that \PB{\|p} is findable in the unique table\X${}\E{}$\6
${}\{{}$\1\6
\&{register} \&{addr} ${}{*}\\{hash};{}$\6
\&{register} \&{var} ${}{*}\|v\K\\{thevar}(\|p);{}$\7
${}\|j\K\|v\MG\\{mask};{}$\6
\&{for} ${}(\\{hash}\K\\{hashcode}(\\{node\_}(\|p\MG\\{lo}),\39\\{node\_}(\|p%
\MG\\{hi}));{}$  ; ${}\\{hash}\PP){}$\5
${}\{{}$\1\6
${}\|k\K\\{addr\_}(\\{hash})\AND\|j;{}$\6
${}\|q\K\\{fetchnode}(\|v,\39\|k);{}$\6
\&{if} ${}(\R\|q){}$\1\5
\&{break};\2\6
\&{if} ${}(\|q\MG\\{lo}\E\|p\MG\\{lo}\W\|q\MG\\{hi}\E\|p\MG\\{hi}){}$\1\5
\&{break};\2\6
\4${}\}{}$\2\6
\&{if} ${}(\|q\I\|p){}$\1\5
\\{complain}(\.{"unfindable\ (lo,hi)"});\2\6
${}\\{addr\_\_}((\&{size\_t})(\|v\MG\\{base}[\|k\GG\\{logpagesize}]+(\|k\AND%
\\{pagemask}))-{}$(\&{size\_t}) \\{mem}${}+{}$(\&{size\_t}) \\{smem})${}\K%
\\{sanitycount};{}$\6
\4${}\}{}$\2\par
\U60.\fi

\M{64}\B\X64:Check the list of free nodes\X${}\E{}$\6
$\\{extra}\K\\{nodeptr}-\\{botsink};{}$\6
\&{for} ${}(\|p\K\\{nodeavail};{}$ \|p; ${}\|p\K\\{node\_}(\|p\MG\\{xref})){}$\5
${}\{{}$\1\6
\&{if} ${}(\R\\{superlegit}(\|p)){}$\1\5
${}\\{printf}(\.{"!\ illegal\ node\ \%x\ i}\)\.{n\ the\ list\ of\ free\ n}\)%
\.{odes\\n"},\39\\{id}(\|p));{}$\2\6
\&{else}\1\5
${}\\{extra}\MM,\39\\{ghost}(\|p)\MG\\{xref}\K{-}\T{1};{}$\2\6
\4${}\}{}$\2\par
\U60.\fi

\M{65}\B\X65:Compute the ghost index fields\X${}\E{}$\6
$\\{ghost}(\\{botsink})\MG\\{index}\K\\{ghost}(\\{topsink})\MG\\{index}\K%
\T{0};{}$\6
\&{for} ${}(\|v\K\\{varhead};{}$ ${}\|v<\\{topofvars};{}$ ${}\|v\PP){}$\1\6
\&{if} ${}(\|v\MG\\{proj}){}$\5
${}\{{}$\1\6
\&{if} ${}(\R\\{superlegit}(\\{node\_}(\|v\MG\\{proj}))){}$\1\5
${}\\{printf}(\.{"!\ illegal\ projectio}\)\.{n\ function\ for\ level}\)\.{\ \%d%
\\n"},\39\|v-\\{varhead});{}$\2\6
\&{else}\1\5
${}\\{ghost}(\|v\MG\\{proj})\MG\\{index}\PP;{}$\2\6
\&{if} ${}(\|v\MG\\{repl}){}$\5
${}\{{}$\1\6
\&{if} ${}(\R\\{legit}(\\{node\_}(\|v\MG\\{repl}))){}$\1\5
${}\\{printf}(\.{"!\ illegal\ replaceme}\)\.{nt\ function\ for\ leve}\)\.{l\ %
\%d\\n"},\39\|v-\\{varhead});{}$\2\6
\&{else}\1\5
${}\\{ghost}(\|v\MG\\{repl})\MG\\{index}\PP;{}$\2\6
\4${}\}{}$\2\6
\4${}\}{}$\2\2\6
\&{for} ${}(\|j\K\T{0};{}$ ${}\|j<\\{extsize};{}$ ${}\|j\PP){}$\1\6
\&{if} (\|f[\|j])\5
${}\{{}$\1\6
\&{if} ${}(\|f[\|j]>\\{topsink}\W\R\\{superlegit}(\|f[\|j])){}$\1\5
${}\\{printf}(\.{"!\ illegal\ external\ }\)\.{pointer\ f\%d\\n"},\39\|j);{}$\2\6
\&{else}\1\5
${}\\{ghost}(\|f[\|j])\MG\\{index}\PP;{}$\2\6
\4${}\}{}$\2\2\par
\U60.\fi

\M{66}\B\X66:Check the reference counts\X${}\E{}$\6
\&{for} ${}(\|p\K\\{botsink},\39\\{count}\K\T{0};{}$ ${}\|p<\\{nodeptr};{}$ ${}%
\|p\PP){}$\5
${}\{{}$\1\6
${}\|q\K\\{ghost}(\|p);{}$\6
\&{if} ${}(\|q\MG\\{xref}\E{-}\T{1}){}$\1\5
\&{continue};\C{ \PB{\|p} is free }\2\6
\&{for} ${}(\|k\K\|q\MG\\{index},\39\|q\K\\{node\_}(\|q\MG\\{xref});{}$ \|q;
${}\|q\K\\{node\_}(\\{addr\_\_}(\\{ghost}(\|q)))){}$\1\5
${}\|k\PP;{}$\2\6
\&{if} ${}(\|p\MG\\{xref}\I\|k){}$\1\5
${}\\{printf}(\.{"!\ \%x->xref\ should\ b}\)\.{e\ \%d,\ not\ \%d\\n"},\39%
\\{id}(\|p),\39\|k,\39\|p\MG\\{xref});{}$\2\6
\&{if} ${}(\|k<\T{0}){}$\1\5
${}\\{count}\PP{}$;\C{ \PB{\|p} is dead }\2\6
\4${}\}{}$\2\6
\&{if} ${}(\\{count}\I\\{deadnodes}){}$\1\5
${}\\{printf}(\.{"!\ deadnodes\ should\ }\)\.{be\ \%d,\ not\ \%d\\n"},\39%
\\{count},\39\\{deadnodes}){}$;\2\par
\U58.\fi

\M{67}If a reference count turns out to be wrong, I'll probably want to know
why.
The following subroutine provides additional clues.

\Y\B\4\X7:Subroutines\X${}\mathrel+\E{}$\6
\8\#\&{if} \\{includesanity}\6
\&{void} \\{who\_points\_to}(\&{node} ${}{*}\|p){}$\1\1\2\2\6
${}\{{}$\1\6
\&{register} \&{addr} \|q;\C{ the address of a \PB{\\{lo}} or \PB{\\{hi}} field
in a node }\7
\&{for} ${}(\|q\K\\{addr\_}(\\{ghost}(\|p)\MG\\{xref});{}$ \|q; ${}\|q\K\\{addr%
\_\_}(\\{ghost}(\|q))){}$\5
${}\{{}$\1\6
${}\\{print\_node}(\\{node\_}(\|q\AND{-}\&{sizeof}(\&{node})));{}$\6
\\{printf}(\.{"\\n"});\6
\4${}\}{}$\2\6
\4${}\}{}$\2\6
\8\#\&{endif}\par
\fi

\M{68}We've seen that every superlegitimate node is findable in the
proper unique table. Conversely, we want to check that everything
is those tables is superlegitimate, and found.

\Y\B\4\D$\\{badpage}(\|p)$ \5
$((\|p)<\\{pageptr}\V(\|p)\G\\{topofmem}{}$)\par
\Y\B\4\X68:Check the unique tables\X${}\E{}$\6
$\\{extra}\K\\{topofmem}-\\{pageptr}{}$;\C{ this many pages allocated }\6
\&{for} ${}(\|v\K\\{varhead};{}$ ${}\|v<\\{topofvars};{}$ ${}\|v\PP){}$\1\6
\&{if} ${}(\|v\MG\\{proj}){}$\5
${}\{{}$\1\6
\&{for} ${}(\|k\K\T{0};{}$ ${}\|k\Z\|v\MG\\{mask}\GG\\{logpagesize};{}$ ${}\|k%
\PP){}$\1\6
\&{if} ${}(\\{badpage}(\\{page\_}(\|v\MG\\{base}[\|k]))){}$\1\5
${}\\{printf}(\.{"!\ bad\ page\ base\ \%x\ }\)\.{in\ unique\ table\ for\ }\)%
\.{level\ \%d\\n"},\39\\{id}(\|v\MG\\{base}[\|k]),\39\|v-\\{varhead});{}$\2\2\6
${}\\{extra}\MRL{-{\K}}\T{1}+(\|v\MG\\{mask}\GG\\{logpagesize});{}$\6
\&{for} ${}(\|k\K\\{count}\K\T{0};{}$ ${}\|k<\|v\MG\\{mask};{}$ ${}\|k\MRL{+{%
\K}}\&{sizeof}(\&{addr})){}$\5
${}\{{}$\1\6
${}\|p\K\\{fetchnode}(\|v,\39\|k);{}$\6
\&{if} ${}(\R\|p){}$\1\5
${}\\{count}\PP;{}$\2\6
\&{else}\5
${}\{{}$\1\6
\&{if} ${}(\\{addr\_\_}((\&{size\_t})(\|v\MG\\{base}[\|k\GG\\{logpagesize}]+(%
\|k\AND\\{pagemask}))-{}$(\&{size\_t}) \\{mem}${}+{}$(\&{size\_t}) \\{smem})${}%
\I\\{sanitycount}){}$\1\5
${}\\{printf}(\.{"!\ extra\ node\ \%x\ in\ }\)\.{unique\ table\ for\ lev}\)%
\.{el\ \%d\\n"},\39\\{id}(\|p),\39\|v-\\{varhead});{}$\2\6
\&{if} ${}(\R\\{superlegit}(\|p)){}$\1\5
${}\\{printf}(\.{"!\ illegal\ node\ \%x\ i}\)\.{n\ unique\ table\ for\ l}\)%
\.{evel\ \%d\\n"},\39\\{id}(\|p),\39\|v-\\{varhead});{}$\2\6
\&{else} \&{if} ${}(\\{varpart}(\|p\MG\\{index})\I\|v-\\{varhead}){}$\1\5
\\{complain}(\.{"wrong\ var"});\2\6
\4${}\}{}$\2\6
\4${}\}{}$\2\6
\&{if} ${}(\\{count}\I\|v\MG\\{free}){}$\1\5
${}\\{printf}(\.{"!\ unique\ table\ \%d\ h}\)\.{as\ \%d\ free\ slots,\ no}\)%
\.{t\ \%d\\n"},\39\|v-\\{varhead},\39\\{count},\39\|v\MG\\{free});{}$\2\6
\4${}\}{}$\2\2\par
\U58.\fi

\M{69}The fields in cache memos that refer to nodes should refer to
legitimate nodes.

\Y\B\4\X69:Check the cache\X${}\E{}$\6
${}\{{}$\1\6
\&{register} \&{memo} ${}{*}\|m;{}$\7
${}\\{extra}\MRL{-{\K}}\T{1}+(\\{cachemask}\GG\\{logpagesize});{}$\6
\&{for} ${}(\|k\K\T{0};{}$ ${}\|k<\\{cachepages};{}$ ${}\|k\PP){}$\5
${}\{{}$\1\6
\&{if} (\\{badpage}(\\{page\_}(\\{cachepage}[\|k])))\1\5
${}\\{printf}(\.{"!\ bad\ page\ base\ \%x\ }\)\.{in\ the\ cache\\n"},\39\\{id}(%
\\{cachepage}[\|k]));{}$\2\6
\&{for} ${}(\|m\K\\{memo\_}(\\{cachepage}[\|k]);{}$ ${}\|m<\\{memo\_}(%
\\{cachepage}[\|k])+\\{cacheslotsperpage};{}$ ${}\|m\PP){}$\1\6
\&{if} ${}(\|m\MG\|r){}$\5
${}\{{}$\1\6
\&{if} ${}(\R\\{legit}(\\{node\_}(\|m\MG\|r))){}$\1\5
\&{goto} \\{nogood};\2\6
\&{if} ${}(\R\\{legit}(\\{node\_}(\|m\MG\|f))){}$\1\5
\&{goto} \\{nogood};\2\6
\&{if} ${}(\|m\MG\|h>\T{0}){}$\5
${}\{{}$\1\6
\&{if} ${}(\R\\{legit}(\\{node\_}(\|m\MG\|g))){}$\1\5
\&{goto} \\{nogood};\2\6
\&{if} ${}(\|m\MG\|h>\\{maxbinop}\W\R\\{legit}(\\{node\_}(\|m\MG\|h\AND{-}\T{%
\^10}))){}$\1\5
\&{goto} \\{nogood};\2\6
\4${}\}{}$\2\6
\4${}\}{}$\2\2\6
\&{continue};\6
\4\\{nogood}:\5
\\{printf}(\.{"!\ bad\ node\ in\ cache}\)\.{\ entry\ "});\5
\\{print\_memo}(\|m);\6
\4${}\}{}$\2\6
\4${}\}{}$\2\par
\U58.\fi

\M{70}Finally, \PB{\\{sanity\_check}} ensures that we haven't forgotten to free
unused
pages, nor have we freed a page that was already free.

\Y\B\4\X70:Check the list of free pages\X${}\E{}$\6
${}\{{}$\1\6
\&{register} \&{page} ${}{*}\|p\K\\{pageavail};{}$\7
\&{while} ${}(\|p\W\\{extra}>\T{0}){}$\5
${}\{{}$\1\6
\&{if} (\\{badpage}(\|p))\1\5
${}\\{printf}(\.{"!\ bad\ free\ page\ \%x\\}\)\.{n"},\39\\{id}(\|p));{}$\2\6
${}\|p\K\\{page\_}(\|p\MG\\{dat}[\T{0}]),\39\\{extra}\MM;{}$\6
\4${}\}{}$\2\6
\&{if} ${}(\\{extra}>\T{0}){}$\1\5
${}\\{printf}(\.{"!\ \%d\ pages\ have\ lea}\)\.{ked\\n"},\39\\{extra});{}$\2\6
\&{else} \&{if} (\|p)\1\5
\\{printf}(\.{"!\ the\ free\ pages\ fo}\)\.{rm\ a\ loop\\n"});\2\6
\4${}\}{}$\2\par
\U58.\fi

\M{71}The following routine brings a dead node back to life.
It also increases the reference counts of the node's children,
and resuscitates them if they were dead.

\Y\B\4\X7:Subroutines\X${}\mathrel+\E{}$\6
\&{void} \\{recursively\_revive}(\&{node} ${}{*}\|p){}$\1\1\2\2\6
${}\{{}$\1\6
\&{register} \&{node} ${}{*}\|q;{}$\7
${}\\{rmems}\PP{}$;\C{ track recursion overhead }\6
\4\\{restart}:\5
\&{if} ${}(\\{verbose}\AND\T{4}){}$\1\5
${}\\{printf}(\.{"reviving\ \%x\\n"},\39\\{id}(\|p));{}$\2\6
${}\|o,\39\|p\MG\\{xref}\K\T{0};{}$\6
${}\\{deadnodes}\MM;{}$\6
${}\|q\K\\{node\_}(\|p\MG\\{lo});{}$\6
\&{if} ${}(\|o,\39\|q\MG\\{xref}<\T{0}){}$\1\5
${}\\{oooo},\39\\{recursively\_revive}(\|q);{}$\2\6
\&{else}\1\5
${}\|o,\39\|q\MG\\{xref}\PP;{}$\2\6
${}\|p\K\\{node\_}(\|p\MG\\{hi});{}$\6
\&{if} ${}(\|o,\39\|p\MG\\{xref}<\T{0}){}$\1\5
\&{goto} \\{restart};\C{ tail recursion }\2\6
\&{else}\1\5
${}\|o,\39\|p\MG\\{xref}\PP;{}$\2\6
\4${}\}{}$\2\par
\fi

\M{72}Conversely, we sometimes must go the other way, with as much dignity
as we can muster.

\Y\B\4\D$\\{deref}(\|p)$ \6
\&{if} ${}(\|o,\39(\|p)\MG\\{xref}\E\T{0}){}$\1\5
\\{recursively\_kill}(\|p);\5
\2\&{else} $\|o,\39(\|p)\MG\\{xref}\MM{}$\par
\Y\B\4\X7:Subroutines\X${}\mathrel+\E{}$\6
\&{void} \\{recursively\_kill}(\&{node} ${}{*}\|p){}$\1\1\2\2\6
${}\{{}$\1\6
\&{register} \&{node} ${}{*}\|q;{}$\7
${}\\{rmems}\PP{}$;\C{ track recursion overhead }\6
\4\\{restart}:\5
\&{if} ${}(\\{verbose}\AND\T{4}){}$\1\5
${}\\{printf}(\.{"burying\ \%x\\n"},\39\\{id}(\|p));{}$\2\6
${}\|o,\39\|p\MG\\{xref}\K{-}\T{1};{}$\6
${}\\{deadnodes}\PP;{}$\6
${}\|q\K\\{node\_}(\|p\MG\\{lo});{}$\6
\&{if} ${}(\|o,\39\|q\MG\\{xref}\E\T{0}){}$\1\5
${}\\{oooo},\39\\{recursively\_kill}(\|q);{}$\2\6
\&{else}\1\5
${}\|o,\39\|q\MG\\{xref}\MM;{}$\2\6
${}\|p\K\\{node\_}(\|p\MG\\{hi});{}$\6
\&{if} ${}(\|o,\39\|p\MG\\{xref}\E\T{0}){}$\1\5
\&{goto} \\{restart};\C{ tail recursion }\2\6
\&{else}\1\5
${}\|o,\39\|p\MG\\{xref}\MM;{}$\2\6
\4${}\}{}$\2\par
\fi

\N{1}{73}Binary operations. OK, now we've got a bunch of powerful routines for
making
and maintaining BDDs, and it's time to have fun. Let's start with a typical
synthesis routine, which constructs the BDD for $f\land g$ from the BDDs for
$f$ and~$g$.

The general pattern is to have a top-level subroutine and a recursive
subroutine. The top-level one updates overall status variables and
invokes the recursive one; and it keeps trying, if temporary setbacks arise.

The recursive routine exits quickly if given a simple case.
Otherwise it checks the cache, and calls itself if necessary.
I write the recursive routine first, since it embodies the guts
of the computation. % and since C wants me to

The top-level routines are rather boring, so I'll defer them till later.

Incidentally, I learned the \CEE/ language long ago, and didn't know
until recently that it's now legal to modify the formal parameters
to a function. (Wow!)

\Y\B\4\X7:Subroutines\X${}\mathrel+\E{}$\6
\&{node} ${}{*}{}$\\{and\_rec}(\&{node} ${}{*}\|f,\39{}$\&{node} ${}{*}\|g){}$%
\1\1\2\2\6
${}\{{}$\1\6
\&{var} ${}{*}\|v,{}$ ${}{*}\\{vf},{}$ ${}{*}\\{vg};{}$\6
\&{node} ${}{*}\|r,{}$ ${}{*}\\{r0},{}$ ${}{*}\\{r1};{}$\7
\&{if} ${}(\|f\E\|g){}$\1\5
\&{return} \\{oo}${},\39\|f\MG\\{xref}\PP,\39\|f{}$;\C{ $f\land f=f$ }\2\6
\&{if} ${}(\|f>\|g){}$\1\5
${}\|r\K\|f,\39\|f\K\|g,\39\|g\K\|r{}$;\C{ wow }\2\6
\&{if} ${}(\|f\Z\\{topsink}){}$\5
${}\{{}$\1\6
\&{if} ${}(\|f\E\\{topsink}){}$\1\5
\&{return} \\{oo}${},\39\|g\MG\\{xref}\PP,\39\|g{}$;\C{ $1\land g=g$ }\2\6
\&{return} \\{oo}${},\39\\{botsink}\MG\\{xref}\PP,\39\\{botsink}{}$;\C{ $0\land
g=0$ }\6
\4${}\}{}$\2\6
${}\\{oo},\39\|r\K\\{cache\_lookup}(\|f,\39\|g,\39\\{node\_}(\T{1})){}$;\C{ two
mems for \PB{$\|f\MG\\{index}$}, \PB{$\|g\MG\\{index}$} }\6
\&{if} (\|r)\1\5
\&{return} \|r;\2\6
\X74:Find $f\land g$ recursively\X;\6
\4${}\}{}$\2\par
\fi

\M{74}I assume that \PB{$\|f\MG\\{lo}$} and \PB{$\|f\MG\\{hi}$} belong to the
same octabyte.

The \PB{\\{rmems}} counter is incremented only after we've checked for
special terminal cases. When none of the simplifications apply,
we must prepare to plunge in to deeper waters.

\Y\B\4\X74:Find $f\land g$ recursively\X${}\E{}$\6
$\\{rmems}\PP{}$;\C{ track recursion overhead }\6
${}\\{vf}\K\\{thevar}(\|f){}$;\C{ \PB{$\|f\MG\\{index}$} has already been
fetched }\6
${}\\{vg}\K\\{thevar}(\|g){}$;\C{ \PB{$\|g\MG\\{index}$} has already been
fetched }\6
\&{if} ${}(\\{vf}<\\{vg}){}$\1\5
${}\|v\K\\{vf};{}$\2\6
\&{else}\1\5
${}\|v\K\\{vg}{}$;\C{ choose the top variable, \PB{\|v} }\2\6
${}\\{r0}\K\\{and\_rec}((\\{vf}\E\|v\?\|o,\39\\{node\_}(\|f\MG\\{lo}):\|f),\39(%
\\{vg}\E\|v\?\|o,\39\\{node\_}(\|g\MG\\{lo}):\|g));{}$\6
\&{if} ${}(\R\\{r0}){}$\1\5
\&{return} ${}\NULL{}$;\C{ oops, trouble }\2\6
${}\\{r1}\K\\{and\_rec}((\\{vf}\E\|v\?\\{node\_}(\|f\MG\\{hi}):\|f),\39(\\{vg}%
\E\|v\?\\{node\_}(\|g\MG\\{hi}):\|g));{}$\6
\&{if} ${}(\R\\{r1}){}$\5
${}\{{}$\1\6
\\{deref}(\\{r0});\C{ too bad, but we have to abort in midstream }\6
\&{return} ${}\NULL;{}$\6
\4${}\}{}$\2\6
${}\|r\K\\{unique\_find}(\|v,\39\\{r0},\39\\{r1});{}$\6
\&{if} (\|r)\5
${}\{{}$\1\6
\&{if} ${}((\\{verbose}\AND\T{128})\W(\|v<\\{tvar})){}$\1\5
${}\\{printf}(\.{"\ \ \ \%x=\%x\&\%x\ (level\ }\)\.{\%d)\\n"},\39\\{id}(\|r),%
\39\\{id}(\|f),\39\\{id}(\|g),\39\|v-\\{varhead});{}$\2\6
${}\\{cache\_insert}(\|f,\39\|g,\39\\{node\_}(\T{1}),\39\|r);{}$\6
\4${}\}{}$\2\6
\&{return} \|r;\par
\U73.\fi

\M{75}Of course $f\lor g$ is dual to $f\land g$. I could have combined
the two routines into one, but what the heck; a long program is more
impressive.

\Y\B\4\X7:Subroutines\X${}\mathrel+\E{}$\6
\&{node} ${}{*}{}$\\{or\_rec}(\&{node} ${}{*}\|f,\39{}$\&{node} ${}{*}\|g){}$\1%
\1\2\2\6
${}\{{}$\1\6
\&{var} ${}{*}\|v,{}$ ${}{*}\\{vf},{}$ ${}{*}\\{vg};{}$\6
\&{node} ${}{*}\|r,{}$ ${}{*}\\{r0},{}$ ${}{*}\\{r1};{}$\7
\&{if} ${}(\|f\E\|g){}$\1\5
\&{return} \\{oo}${},\39\|f\MG\\{xref}\PP,\39\|f{}$;\C{ $f\lor f=f$ }\2\6
\&{if} ${}(\|f>\|g){}$\1\5
${}\|r\K\|f,\39\|f\K\|g,\39\|g\K\|r{}$;\C{ wow }\2\6
\&{if} ${}(\|f\Z\\{topsink}){}$\5
${}\{{}$\1\6
\&{if} ${}(\|f\E\\{topsink}){}$\1\5
\&{return} \\{oo}${},\39\\{topsink}\MG\\{xref}\PP,\39\\{topsink}{}$;\C{ $1\lor
g=1$ }\2\6
\&{return} \\{oo}${},\39\|g\MG\\{xref}\PP,\39\|g{}$;\C{ $0\lor g=g$ }\6
\4${}\}{}$\2\6
${}\\{oo},\39\|r\K\\{cache\_lookup}(\|f,\39\|g,\39\\{node\_}(\T{7}));{}$\6
\&{if} (\|r)\1\5
\&{return} \|r;\2\6
\X76:Find $f\lor g$ recursively\X;\6
\4${}\}{}$\2\par
\fi

\M{76}\B\X76:Find $f\lor g$ recursively\X${}\E{}$\6
$\\{rmems}\PP{}$;\C{ track recursion overhead }\6
${}\\{vf}\K\\{thevar}(\|f);{}$\6
${}\\{vg}\K\\{thevar}(\|g);{}$\6
\&{if} ${}(\\{vf}<\\{vg}){}$\1\5
${}\|v\K\\{vf};{}$\2\6
\&{else}\1\5
${}\|v\K\\{vg}{}$;\C{ choose the top variable, \PB{\|v} }\2\6
${}\\{r0}\K\\{or\_rec}((\\{vf}\E\|v\?\|o,\39\\{node\_}(\|f\MG\\{lo}):\|f),\39(%
\\{vg}\E\|v\?\|o,\39\\{node\_}(\|g\MG\\{lo}):\|g));{}$\6
\&{if} ${}(\R\\{r0}){}$\1\5
\&{return} ${}\NULL{}$;\C{ oops, trouble }\2\6
${}\\{r1}\K\\{or\_rec}((\\{vf}\E\|v\?\\{node\_}(\|f\MG\\{hi}):\|f),\39(\\{vg}\E%
\|v\?\\{node\_}(\|g\MG\\{hi}):\|g));{}$\6
\&{if} ${}(\R\\{r1}){}$\5
${}\{{}$\1\6
\\{deref}(\\{r0});\C{ too bad, but we have to abort in midstream }\6
\&{return} ${}\NULL;{}$\6
\4${}\}{}$\2\6
${}\|r\K\\{unique\_find}(\|v,\39\\{r0},\39\\{r1});{}$\6
\&{if} (\|r)\5
${}\{{}$\1\6
\&{if} ${}((\\{verbose}\AND\T{128})\W(\|v<\\{tvar})){}$\1\5
${}\\{printf}(\.{"\ \ \ \%x=\%x|\%x\ (level\ }\)\.{\%d)\\n"},\39\\{id}(\|r),\39%
\\{id}(\|f),\39\\{id}(\|g),\39\|v-\\{varhead});{}$\2\6
${}\\{cache\_insert}(\|f,\39\|g,\39\\{node\_}(\T{7}),\39\|r);{}$\6
\4${}\}{}$\2\6
\&{return} \|r;\par
\U75.\fi

\M{77}Exclusive or is much the same.

\Y\B\4\X7:Subroutines\X${}\mathrel+\E{}$\6
\&{node} ${}{*}{}$\\{xor\_rec}(\&{node} ${}{*}\|f,\39{}$\&{node} ${}{*}\|g){}$%
\1\1\2\2\6
${}\{{}$\1\6
\&{var} ${}{*}\|v,{}$ ${}{*}\\{vf},{}$ ${}{*}\\{vg};{}$\6
\&{node} ${}{*}\|r,{}$ ${}{*}\\{r0},{}$ ${}{*}\\{r1};{}$\7
\&{if} ${}(\|f\E\|g){}$\1\5
\&{return} \\{oo}${},\39\\{botsink}\MG\\{xref}\PP,\39\\{botsink}{}$;\C{ $f%
\oplus f=0$ }\2\6
\&{if} ${}(\|f>\|g){}$\1\5
${}\|r\K\|f,\39\|f\K\|g,\39\|g\K\|r{}$;\C{ wow }\2\6
\&{if} ${}(\|f\E\\{botsink}){}$\1\5
\&{return} \\{oo}${},\39\|g\MG\\{xref}\PP,\39\|g{}$;\C{ $0\oplus g=g$ }\2\6
${}\\{oo},\39\|r\K\\{cache\_lookup}(\|f,\39\|g,\39\\{node\_}(\T{6}));{}$\6
\&{if} (\|r)\1\5
\&{return} \|r;\2\6
\X78:Find $f\oplus g$ recursively\X;\6
\4${}\}{}$\2\par
\fi

\M{78}After discovering that $f\oplus g=r$, we could also put $f\oplus r=g$ and
$g\oplus r=f$ into the cache. I tried that, in the first draft of this code.
Unfortunately it cost more than it saved.

\Y\B\4\X78:Find $f\oplus g$ recursively\X${}\E{}$\6
$\\{rmems}\PP{}$;\C{ track recursion overhead }\6
\&{if} ${}(\|f\E\\{topsink}){}$\1\5
${}\\{vf}\K\T{0},\39\|v\K\\{vg}\K\\{thevar}(\|g);{}$\2\6
\&{else}\5
${}\{{}$\1\6
${}\\{vf}\K\\{thevar}(\|f),\39\\{vg}\K\\{thevar}(\|g);{}$\6
\&{if} ${}(\\{vf}<\\{vg}){}$\1\5
${}\|v\K\\{vf};{}$\2\6
\&{else}\1\5
${}\|v\K\\{vg}{}$;\C{ choose the top variable, \PB{\|v} }\2\6
\4${}\}{}$\2\6
${}\\{r0}\K\\{xor\_rec}((\\{vf}\E\|v\?\|o,\39\\{node\_}(\|f\MG\\{lo}):\|f),\39(%
\\{vg}\E\|v\?\|o,\39\\{node\_}(\|g\MG\\{lo}):\|g));{}$\6
\&{if} ${}(\R\\{r0}){}$\1\5
\&{return} ${}\NULL{}$;\C{ oops, trouble }\2\6
${}\\{r1}\K\\{xor\_rec}((\\{vf}\E\|v\?\\{node\_}(\|f\MG\\{hi}):\|f),\39(\\{vg}%
\E\|v\?\\{node\_}(\|g\MG\\{hi}):\|g));{}$\6
\&{if} ${}(\R\\{r1}){}$\5
${}\{{}$\1\6
\\{deref}(\\{r0});\C{ too bad, but we have to abort in midstream }\6
\&{return} ${}\NULL;{}$\6
\4${}\}{}$\2\6
${}\|r\K\\{unique\_find}(\|v,\39\\{r0},\39\\{r1});{}$\6
\&{if} (\|r)\5
${}\{{}$\1\6
\&{if} ${}((\\{verbose}\AND\T{128})\W(\|v<\\{tvar})){}$\1\5
${}\\{printf}(\.{"\ \ \ \%x=\%x\^\%x\ (level\ }\)\.{\%d)\\n"},\39\\{id}(\|r),%
\39\\{id}(\|f),\39\\{id}(\|g),\39\|v-\\{varhead});{}$\2\6
${}\\{cache\_insert}(\|f,\39\|g,\39\\{node\_}(\T{6}),\39\|r);{}$\6
\4${}\}{}$\2\6
\&{return} \|r;\par
\U77.\fi

\M{79}Now for variety, let's do the ``constrain'' operator of Coudert and
Madre.
The function $f\downarrow g$ returned by this routine depends on the
current variable ordering. In brief, $f\downarrow g=0$ if $g$ is
identically zero; otherwise $(f\downarrow g)(x)=f(y)$ when $y$ is the
first element of the sequence $x$, $x\oplus1$, $x\oplus2$, \dots\
such that $g(y)=1$. (This definition treats $x=(x_0\ldots x_n)_2$ and
$y=(y_0\ldots y_n)_2$ as binary numbers.)

\Y\B\4\X7:Subroutines\X${}\mathrel+\E{}$\6
\&{node} ${}{*}{}$\\{constrain\_rec}(\&{node} ${}{*}\|f,\39{}$\&{node} ${}{*}%
\|g){}$\1\1\2\2\6
${}\{{}$\1\6
\&{var} ${}{*}\|v,{}$ ${}{*}\\{vf},{}$ ${}{*}\\{vg};{}$\6
\&{node} ${}{*}\|r,{}$ ${}{*}\\{r0},{}$ ${}{*}\\{r1};{}$\7
\&{if} ${}(\|g\E\\{botsink}){}$\1\5
\&{return} \\{oo}${},\39\\{botsink}\MG\\{xref}\PP,\39\\{botsink}{}$;\C{ $f%
\downarrow 0=0$ }\2\6
\&{if} ${}(\|g\E\\{topsink}\V\|f\Z\\{topsink}){}$\1\5
\&{return} \\{oo}${},\39\|f\MG\\{xref}\PP,\39\|f{}$;\C{ $f\downarrow1=f$, $0%
\downarrow g=0$; also $1\downarrow g=1$ if $g\ne0$ }\2\6
\&{if} ${}(\|f\E\|g){}$\1\5
\&{return} \\{oo}${},\39\\{topsink}\MG\\{xref}\PP,\39\\{topsink}{}$;\C{ $f%
\downarrow f=1$ }\2\6
${}\\{oo},\39\|r\K\\{cache\_lookup}(\|f,\39\|g,\39\\{node\_}(\T{8}));{}$\6
\&{if} (\|r)\1\5
\&{return} \|r;\2\6
\X80:Find $f\downarrow g$ recursively\X;\6
\4${}\}{}$\2\par
\fi

\M{80}\B\X80:Find $f\downarrow g$ recursively\X${}\E{}$\6
$\\{rmems}\PP{}$;\C{ track recursion overhead }\6
${}\\{vf}\K\\{thevar}(\|f);{}$\6
${}\\{vg}\K\\{thevar}(\|g);{}$\6
\&{if} ${}(\\{vf}<\\{vg}){}$\1\5
${}\|v\K\\{vf};{}$\2\6
\&{else}\5
${}\{{}$\1\6
${}\|v\K\\{vg}{}$;\C{ choose the top variable, \PB{\|v} }\6
\&{if} ${}(\|o,\39\\{node\_}(\|g\MG\\{lo})\E\\{botsink}){}$\5
${}\{{}$\1\6
${}\\{oo},\39\|r\K\\{constrain\_rec}((\\{vf}\E\|v\?\|o,\39\\{node\_}(\|f\MG%
\\{hi}):\|f),\39\\{node\_}(\|g\MG\\{hi}));{}$\6
\&{goto} \\{shortcut};\6
\4${}\}{}$\2\6
\&{if} ${}(\|o,\39\\{node\_}(\|g\MG\\{hi})\E\\{botsink}){}$\5
${}\{{}$\1\6
${}\\{oo},\39\|r\K\\{constrain\_rec}((\\{vf}\E\|v\?\|o,\39\\{node\_}(\|f\MG%
\\{lo}):\|f),\39\\{node\_}(\|g\MG\\{lo}));{}$\6
\&{goto} \\{shortcut};\6
\4${}\}{}$\2\6
\4${}\}{}$\2\6
${}\\{r0}\K\\{constrain\_rec}((\\{vf}\E\|v\?\|o,\39\\{node\_}(\|f\MG\\{lo}):%
\|f),\39(\\{vg}\E\|v\?\|o,\39\\{node\_}(\|g\MG\\{lo}):\|g));{}$\6
\&{if} ${}(\R\\{r0}){}$\1\5
\&{return} ${}\NULL{}$;\C{ oops, trouble }\2\6
${}\\{r1}\K\\{constrain\_rec}((\\{vf}\E\|v\?\\{node\_}(\|f\MG\\{hi}):\|f),\39(%
\\{vg}\E\|v\?\\{node\_}(\|g\MG\\{hi}):\|g));{}$\6
\&{if} ${}(\R\\{r1}){}$\5
${}\{{}$\1\6
\\{deref}(\\{r0});\C{ too bad, but we have to abort in midstream }\6
\&{return} ${}\NULL;{}$\6
\4${}\}{}$\2\6
${}\|r\K\\{unique\_find}(\|v,\39\\{r0},\39\\{r1});{}$\6
\4\\{shortcut}:\5
\&{if} (\|r)\5
${}\{{}$\1\6
\&{if} ${}((\\{verbose}\AND\T{128})\W(\|v<\\{tvar})){}$\1\5
${}\\{printf}(\.{"\ \ \ \%x=\%x\_\%x\ (level\ }\)\.{\%d)\\n"},\39\\{id}(\|r),%
\39\\{id}(\|f),\39\\{id}(\|g),\39\|v-\\{varhead});{}$\2\6
${}\\{cache\_insert}(\|f,\39\|g,\39\\{node\_}(\T{8}),\39\|r);{}$\6
\4${}\}{}$\2\6
\&{return} \|r;\par
\U79.\fi

\N{1}{81}Quantifiers. Are we having fun yet? Sure, and quantifiers are even
funner.

In each of the following routines, the second operand \PB{\|g} is supposed
to be a conjunction of positive literals, like $x_2\land x_4\land x_5$. Then,
\def\(#1){\mathbin{\rm#1}}
for example, $f\(A)g$ stands for
$\forall x_2\,\forall x_4\,\forall x_5\,f(x_1,\ldots,x_n)$.

The program doesn't actually bother to check that \PB{\|g} has this form.
But if \PB{\|g} is a general function, the meaning is dependent on the ordering
of variables; consider, for instance, the case $g=\bar x_1\lor x_2$.
The user who tries general functions had better be aware (or beware)
of this fact.

If $g$ is a conjunction of $k$ literals, and if they all have the
highest possible rank (so that they occur at the bottom of the BDDs),
quantification takes linear time in the size of the BDD for~\PB{\|f}.
But if the literals have low rank and occur near the top, the running
time can be as bad as the $2^k$th power of that BDD size(!).

\fi

\M{82}The first case, existential quantification ($f\(E)g$), is typical.
Notice that we don't use the cache unless there are multiple references
to~\PB{\|f}. The number of references to~\PB{\|g} is immaterial here, since
we're
treating \PB{\|g} as a one-dimensional list of literals.

\Y\B\4\X7:Subroutines\X${}\mathrel+\E{}$\6
\&{node} ${}{*}{}$\\{exist\_rec}(\&{node} ${}{*}\|f,\39{}$\&{node} ${}{*}%
\|g){}$\1\1\2\2\6
${}\{{}$\1\6
\&{var} ${}{*}\|v,{}$ ${}{*}\\{vg};{}$\6
\&{node} ${}{*}\|r,{}$ ${}{*}\\{r0},{}$ ${}{*}\\{r1};{}$\7
\4\\{restart}:\5
\&{if} ${}(\|g\Z\\{topsink}){}$\1\5
\&{return} \\{oo}${},\39\|f\MG\\{xref}\PP,\39\|f{}$;\C{ $f\(E)1=f$ }\2\6
\&{if} ${}(\|f\Z\\{topsink}){}$\1\5
\&{return} \\{oo}${},\39\|f\MG\\{xref}\PP,\39\|f{}$;\C{ $0\(E)g=0$, $1\(E)g=1$
}\2\6
${}\|v\K\\{thevar}(\|f);{}$\6
${}\\{vg}\K\\{thevar}(\|g);{}$\6
\&{if} ${}(\|v>\\{vg}){}$\5
${}\{{}$\1\6
${}\|o,\39\|g\K\\{node\_}(\|g\MG\\{hi}){}$;\5
\&{goto} \\{restart};\C{ \PB{\|f} doesn't depend on \PB{\\{vg}} }\6
\4${}\}{}$\2\6
${}\\{oo},\39\|r\K\\{cache\_lookup}(\|f,\39\|g,\39\\{node\_}(\T{15}));{}$\6
\&{if} (\|r)\1\5
\&{return} \|r;\2\6
\X83:Find $f\(E)g$ recursively\X;\6
\4${}\}{}$\2\par
\fi

\M{83}When the top variable of \PB{\|g} is the same as the top variable of~\PB{%
\|f},
we always descend to the \PB{\\{hi}} branch below node~\PB{\|g}, since \PB{\|g}
is
supposed to be a conjunction.

\Y\B\4\X83:Find $f\(E)g$ recursively\X${}\E{}$\6
$\\{rmems}\PP{}$;\C{ track recursion overhead }\6
${}\|o,\39\\{r0}\K\\{exist\_rec}(\\{node\_}(\|f\MG\\{lo}),\39(\\{vg}\E\|v\?\|o,%
\39\\{node\_}(\|g\MG\\{hi}):\|g));{}$\6
\&{if} ${}(\R\\{r0}){}$\1\5
\&{return} ${}\NULL{}$;\C{ oops, trouble }\2\6
\&{if} ${}(\\{r0}\E\\{topsink}\W\\{vg}\E\|v){}$\5
${}\{{}$\1\6
${}\|r\K\\{r0};{}$\6
\&{goto} \\{gotr};\6
\4${}\}{}$\2\6
${}\\{r1}\K\\{exist\_rec}(\\{node\_}(\|f\MG\\{hi}),\39(\\{vg}\E\|v\?\\{node\_}(%
\|g\MG\\{hi}):\|g));{}$\6
\&{if} ${}(\R\\{r1}){}$\5
${}\{{}$\1\6
\\{deref}(\\{r0});\C{ too bad, but we have to abort in midstream }\6
\&{return} ${}\NULL;{}$\6
\4${}\}{}$\2\6
\&{if} ${}(\\{vg}>\|v){}$\1\5
${}\|r\K\\{unique\_find}(\|v,\39\\{r0},\39\\{r1});{}$\2\6
\&{else}\5
${}\{{}$\1\6
${}\|r\K\\{or\_rec}(\\{r0},\39\\{r1}){}$;\C{ existential quantification happens
here }\6
\\{deref}(\\{r0});\5
\\{deref}(\\{r1});\C{ we're done with \PB{\\{r0}} and \PB{\\{r1}} }\6
\4${}\}{}$\2\6
\4\\{gotr}:\5
\&{if} (\|r)\5
${}\{{}$\1\6
\&{if} ${}((\\{verbose}\AND\T{128})\W(\|v<\\{tvar})){}$\1\5
${}\\{printf}(\.{"\ \ \ \%x=\%xE\%x\ (level\ }\)\.{\%d)\\n"},\39\\{id}(\|r),\39%
\\{id}(\|f),\39\\{id}(\|g),\39\|v-\\{varhead});{}$\2\6
${}\\{cache\_insert}(\|f,\39\|g,\39\\{node\_}(\T{15}),\39\|r);{}$\6
\4${}\}{}$\2\6
\&{return} \|r;\par
\U82.\fi

\M{84}The code for universal quantification ($\forall$) is almost line-for-line
identical to the code for existential quantification.

\Y\B\4\X7:Subroutines\X${}\mathrel+\E{}$\6
\&{node} ${}{*}{}$\\{all\_rec}(\&{node} ${}{*}\|f,\39{}$\&{node} ${}{*}\|g){}$%
\1\1\2\2\6
${}\{{}$\1\6
\&{var} ${}{*}\|v,{}$ ${}{*}\\{vg};{}$\6
\&{node} ${}{*}\|r,{}$ ${}{*}\\{r0},{}$ ${}{*}\\{r1};{}$\7
\4\\{restart}:\5
\&{if} ${}(\|g\Z\\{topsink}){}$\1\5
\&{return} \\{oo}${},\39\|f\MG\\{xref}\PP,\39\|f{}$;\C{ $f\(A)1=f$ }\2\6
\&{if} ${}(\|f\Z\\{topsink}){}$\1\5
\&{return} \\{oo}${},\39\|f\MG\\{xref}\PP,\39\|f{}$;\C{ $0\(A)g=0$, $1\(A)g=1$
}\2\6
${}\|v\K\\{thevar}(\|f);{}$\6
${}\\{vg}\K\\{thevar}(\|g);{}$\6
\&{if} ${}(\|v>\\{vg}){}$\5
${}\{{}$\1\6
${}\|o,\39\|g\K\\{node\_}(\|g\MG\\{hi}){}$;\5
\&{goto} \\{restart};\C{ \PB{\|f} doesn't depend on \PB{\\{vg}} }\6
\4${}\}{}$\2\6
${}\\{oo},\39\|r\K\\{cache\_lookup}(\|f,\39\|g,\39\\{node\_}(\T{9}));{}$\6
\&{if} (\|r)\1\5
\&{return} \|r;\2\6
\X85:Find $f\(A)g$ recursively\X;\6
\4${}\}{}$\2\par
\fi

\M{85}\B\X85:Find $f\(A)g$ recursively\X${}\E{}$\6
$\\{rmems}\PP{}$;\C{ track recursion overhead }\6
${}\|o,\39\\{r0}\K\\{all\_rec}(\\{node\_}(\|f\MG\\{lo}),\39(\\{vg}\E\|v\?\|o,%
\39\\{node\_}(\|g\MG\\{hi}):\|g));{}$\6
\&{if} ${}(\R\\{r0}){}$\1\5
\&{return} ${}\NULL{}$;\C{ oops, trouble }\2\6
\&{if} ${}(\\{r0}\E\\{botsink}\W\\{vg}\E\|v){}$\5
${}\{{}$\1\6
${}\|r\K\\{r0};{}$\6
\&{goto} \\{gotr};\6
\4${}\}{}$\2\6
${}\\{r1}\K\\{all\_rec}(\\{node\_}(\|f\MG\\{hi}),\39(\\{vg}\E\|v\?\\{node\_}(%
\|g\MG\\{hi}):\|g));{}$\6
\&{if} ${}(\R\\{r1}){}$\5
${}\{{}$\1\6
\\{deref}(\\{r0});\C{ too bad, but we have to abort in midstream }\6
\&{return} ${}\NULL;{}$\6
\4${}\}{}$\2\6
\&{if} ${}(\\{vg}>\|v){}$\1\5
${}\|r\K\\{unique\_find}(\|v,\39\\{r0},\39\\{r1});{}$\2\6
\&{else}\5
${}\{{}$\1\6
${}\|r\K\\{and\_rec}(\\{r0},\39\\{r1}){}$;\C{ universal quantification happens
here }\6
\\{deref}(\\{r0});\5
\\{deref}(\\{r1});\C{ we're done with \PB{\\{r0}} and \PB{\\{r1}} }\6
\4${}\}{}$\2\6
\4\\{gotr}:\5
\&{if} (\|r)\5
${}\{{}$\1\6
\&{if} ${}((\\{verbose}\AND\T{128})\W(\|v<\\{tvar})){}$\1\5
${}\\{printf}(\.{"\ \ \ \%x=\%xA\%x\ (level\ }\)\.{\%d)\\n"},\39\\{id}(\|r),\39%
\\{id}(\|f),\39\\{id}(\|g),\39\|v-\\{varhead});{}$\2\6
${}\\{cache\_insert}(\|f,\39\|g,\39\\{node\_}(\T{9}),\39\|r);{}$\6
\4${}\}{}$\2\6
\&{return} \|r;\par
\U84.\fi

\M{86}The Boolean difference, $f\(D)g$, is easier, because $f\oplus f=0$.

\Y\B\4\X7:Subroutines\X${}\mathrel+\E{}$\6
\&{node} ${}{*}{}$\\{diff\_rec}(\&{node} ${}{*}\|f,\39{}$\&{node} ${}{*}\|g){}$%
\1\1\2\2\6
${}\{{}$\1\6
\&{var} ${}{*}\|v,{}$ ${}{*}\\{vg};{}$\6
\&{node} ${}{*}\|r,{}$ ${}{*}\\{r0},{}$ ${}{*}\\{r1};{}$\7
\&{if} ${}(\|g\Z\\{topsink}){}$\1\5
\&{return} \\{oo}${},\39\|f\MG\\{xref}\PP,\39\|f{}$;\C{ $f\(D)1=f$ }\2\6
\&{if} ${}(\|f\Z\\{topsink}){}$\1\5
\&{return} \\{oo}${},\39\\{botsink}\MG\\{xref}\PP,\39\\{botsink}{}$;\C{ $0%
\(D)g=1\(D)g=0$ when \PB{\|g} isn't constant }\2\6
${}\|v\K\\{thevar}(\|f);{}$\6
${}\\{vg}\K\\{thevar}(\|g);{}$\6
\&{if} ${}(\|v>\\{vg}){}$\1\5
\&{return} \\{oo}${},\39\\{botsink}\MG\\{xref}\PP,\39\\{botsink}{}$;\C{ \PB{%
\|f} doesn't depend on \PB{\\{vg}} }\2\6
${}\\{oo},\39\|r\K\\{cache\_lookup}(\|f,\39\|g,\39\\{node\_}(\T{14}));{}$\6
\&{if} (\|r)\1\5
\&{return} \|r;\2\6
\X87:Find $f\(D)g$ recursively\X;\6
\4${}\}{}$\2\par
\fi

\M{87}\B\X87:Find $f\(D)g$ recursively\X${}\E{}$\6
$\\{rmems}\PP{}$;\C{ track recursion overhead }\6
${}\|o,\39\\{r0}\K\\{diff\_rec}(\\{node\_}(\|f\MG\\{lo}),\39(\\{vg}\E\|v\?\|o,%
\39\\{node\_}(\|g\MG\\{hi}):\|g));{}$\6
\&{if} ${}(\R\\{r0}){}$\1\5
\&{return} ${}\NULL{}$;\C{ oops, trouble }\2\6
${}\\{r1}\K\\{diff\_rec}(\\{node\_}(\|f\MG\\{hi}),\39(\\{vg}\E\|v\?\\{node\_}(%
\|g\MG\\{hi}):\|g));{}$\6
\&{if} ${}(\R\\{r1}){}$\5
${}\{{}$\1\6
\\{deref}(\\{r0});\C{ too bad, but we have to abort in midstream }\6
\&{return} ${}\NULL;{}$\6
\4${}\}{}$\2\6
\&{if} ${}(\\{vg}>\|v){}$\1\5
${}\|r\K\\{unique\_find}(\|v,\39\\{r0},\39\\{r1});{}$\2\6
\&{else}\5
${}\{{}$\1\6
${}\|r\K\\{xor\_rec}(\\{r0},\39\\{r1}){}$;\C{ differencing happens here }\6
\\{deref}(\\{r0});\5
\\{deref}(\\{r1});\C{ we're done with \PB{\\{r0}} and \PB{\\{r1}} }\6
\4${}\}{}$\2\6
\&{if} (\|r)\5
${}\{{}$\1\6
\&{if} ${}((\\{verbose}\AND\T{128})\W(\|v<\\{tvar})){}$\1\5
${}\\{printf}(\.{"\ \ \ \%x=\%xD\%x\ (level\ }\)\.{\%d)\\n"},\39\\{id}(\|r),\39%
\\{id}(\|f),\39\\{id}(\|g),\39\|v-\\{varhead});{}$\2\6
${}\\{cache\_insert}(\|f,\39\|g,\39\\{node\_}(\T{14}),\39\|r);{}$\6
\4${}\}{}$\2\6
\&{return} \|r;\par
\U86.\fi

\M{88}The other two quantifiers are unusual; in fact, I haven't seen
them in any books, although I haven't read all possible books.
Still, the concept is natural enough, when \PB{\|g} is a {\it single\/}
variable $x_v$. In that case, $f\(Y)x_v$ is the function of the remaining
variables such that $f(x_1,\ldots,x_n)=x_v$; and $f\(N)x_v$
is the similar function that makes $f(x_1,\ldots,x_n)=\bar x_v$.

For example, the function $f(x_1,\ldots,x_n)$ is monotonic if and only if
$f\(N)x_v=0$ for $1\le v\le n$.

On the other hand, these yes-no quantifiers make little sense when
\PB{\|g} involves more than one literal, because the result depends
on the variable ordering. It's best to forget that general case---don't
even {\it think\/} about it. Just enjoy the case that works.

\Y\B\4\X7:Subroutines\X${}\mathrel+\E{}$\6
\&{node} ${}{*}{}$\\{yes\_no\_rec}(\&{int} \\{curop}${},\39{}$\&{node} ${}{*}%
\|f,\39{}$\&{node} ${}{*}\|g){}$\1\1\2\2\6
${}\{{}$\1\6
\&{var} ${}{*}\|v,{}$ ${}{*}\\{vg};{}$\6
\&{node} ${}{*}\|r,{}$ ${}{*}\\{r0},{}$ ${}{*}\\{r1};{}$\7
\&{if} ${}(\|g\Z\\{topsink}){}$\1\5
\&{return} \\{oo}${},\39\|f\MG\\{xref}\PP,\39\|f{}$;\C{ $f\(Y)1=f\(N)1=f$ }\2\6
\&{if} ${}(\|f\Z\\{topsink}){}$\1\5
\&{return} \\{oo}${},\39\\{botsink}\MG\\{xref}\PP,\39\\{botsink}{}$;\C{ $0%
\(Y)g=1\(Y)g=0\(N)g=1\(N)g=0$ when \PB{\|g} isn't constant }\2\6
${}\|v\K\\{thevar}(\|f);{}$\6
${}\\{vg}\K\\{thevar}(\|g);{}$\6
\&{if} ${}(\|v>\\{vg}){}$\1\5
\&{return} \\{oo}${},\39\\{botsink}\MG\\{xref}\PP,\39\\{botsink}{}$;\C{ \PB{%
\|f} doesn't depend on \PB{\\{vg}} }\2\6
${}\\{oo},\39\|r\K\\{cache\_lookup}(\|f,\39\|g,\39\\{node\_}(\\{curop}));{}$\6
\&{if} (\|r)\1\5
\&{return} \|r;\2\6
\X89:Find $f\(Y)g$ or $f\(N)g$ recursively\X;\6
\4${}\}{}$\2\par
\fi

\M{89}\B\X89:Find $f\(Y)g$ or $f\(N)g$ recursively\X${}\E{}$\6
$\\{rmems}\PP{}$;\C{ track recursion overhead }\6
${}\|o,\39\\{r0}\K\\{yes\_no\_rec}(\\{curop},\39\\{node\_}(\|f\MG\\{lo}),\39(%
\\{vg}\E\|v\?\|o,\39\\{node\_}(\|g\MG\\{hi}):\|g));{}$\6
\&{if} ${}(\R\\{r0}){}$\1\5
\&{return} ${}\NULL{}$;\C{ oops, trouble }\2\6
\&{if} ${}(\\{r0}\Z\\{topsink}\W\\{vg}\E\|v){}$\5
${}\{{}$\1\6
\&{if} ${}((\\{r0}\E\\{topsink})\E(\\{curop}\E\T{12})){}$\5
${}\{{}$\1\6
${}\|r\K\\{botsink}{}$;\5
\\{deref}(\\{r0});\5
${}\\{botsink}\MG\\{xref}\PP;{}$\6
\&{goto} \\{gotr};\6
\4${}\}{}$\2\6
\4${}\}{}$\2\6
${}\\{r1}\K\\{yes\_no\_rec}(\\{curop},\39\\{node\_}(\|f\MG\\{hi}),\39(\\{vg}\E%
\|v\?\\{node\_}(\|g\MG\\{hi}):\|g));{}$\6
\&{if} ${}(\R\\{r1}){}$\5
${}\{{}$\1\6
\\{deref}(\\{r0});\C{ too bad, but we have to abort in midstream }\6
\&{return} ${}\NULL;{}$\6
\4${}\}{}$\2\6
\&{if} ${}(\\{vg}>\|v){}$\1\5
${}\|r\K\\{unique\_find}(\|v,\39\\{r0},\39\\{r1});{}$\2\6
\&{else}\5
${}\{{}$\1\6
\&{if} ${}(\\{curop}\E\T{12}){}$\1\5
${}\|r\K\\{mux\_rec}(\\{r0},\39\\{botsink},\39\\{r1}){}$;\C{ $f\(Y)g=\bar r_0%
\land r_1$ }\2\6
\&{else}\1\5
${}\|r\K\\{mux\_rec}(\\{r1},\39\\{botsink},\39\\{r0}){}$;\C{ $f\(N)g=r_0\land%
\bar r_1$ }\2\6
\\{deref}(\\{r0});\5
\\{deref}(\\{r1});\C{ we're done with \PB{\\{r0}} and \PB{\\{r1}} }\6
\4${}\}{}$\2\6
\4\\{gotr}:\5
\&{if} (\|r)\5
${}\{{}$\1\6
\&{if} ${}((\\{verbose}\AND\T{128})\W(\|v<\\{tvar})){}$\1\5
${}\\{printf}(\.{"\ \ \ \%x=\%x\%s\%x\ (level}\)\.{\ \%d)\\n"},\39\\{id}(\|r),%
\39\\{id}(\|f),\39\\{binopname}[\\{curop}],\39\\{id}(\|g),\39\|v-%
\\{varhead});{}$\2\6
${}\\{cache\_insert}(\|f,\39\|g,\39\\{node\_}(\\{curop}),\39\|r);{}$\6
\4${}\}{}$\2\6
\&{return} \|r;\par
\U88.\fi

\M{90}Stay tuned: The \PB{\\{mux\_rec}} subroutine is coming soon.

\Y\B\4\X25:Templates for subroutines\X${}\mathrel+\E{}$\6
\&{node} ${}{*}{}$\\{mux\_rec}(\&{node} ${}{*}\|f,\39{}$\&{node} ${}{*}\|g,%
\39{}$\&{node} ${}{*}\|h){}$;\par
\fi

\N{1}{91}Ternary operations. All operations can be reduced to binary
operations,
but it should be interesting to see if we get a speedup by staying ternary.

I like to call the first one ``mux,'' although many other authors have
favored ``ite'' (meaning if-then-else). The latter doesn't seem right to
me when I try to pronounce it. So I'm sticking with the well-worn,
traditional name for this function.

Two special cases are worthy of note: $h=1$ gives ``$f$ implies $g$'';
$g=0$ gives ``not $f$ but $g$.'' I could have implemented those cases as
binary operators, but I chose to let them take the slightly slower
ternary route. (I'm usually a fan of speed, but this program is
already long enough.)

\Y\B\4\X7:Subroutines\X${}\mathrel+\E{}$\6
\&{node} ${}{*}{}$\\{mux\_rec}(\&{node} ${}{*}\|f,\39{}$\&{node} ${}{*}\|g,%
\39{}$\&{node} ${}{*}\|h){}$\1\1\2\2\6
${}\{{}$\1\6
\&{var} ${}{*}\|v,{}$ ${}{*}\\{vf},{}$ ${}{*}\\{vg},{}$ ${}{*}\\{vh};{}$\6
\&{node} ${}{*}\|r,{}$ ${}{*}\\{r0},{}$ ${}{*}\\{r1};{}$\7
\&{if} ${}(\|f\Z\\{topsink}){}$\5
${}\{{}$\1\6
\&{if} ${}(\|f\E\\{topsink}){}$\1\5
\&{return} \\{oo}${},\39\|g\MG\\{xref}\PP,\39\|g{}$;\C{ $(1{?}\ g{:}\ h)=g$ }\2%
\6
\&{return} \\{oo}${},\39\|h\MG\\{xref}\PP,\39\|h{}$;\C{ $(0{?}\ g{:}\ h)=h$ }\6
\4${}\}{}$\2\6
\&{if} ${}(\|g\E\|f\V\|g\E\\{topsink}){}$\1\5
\&{return} \\{or\_rec}${}(\|f,\39\|h){}$;\C{ $(f{?}\ f{:}\ h) = (f{?}\ 1{:}\ h)
= f\lor h$ }\2\6
\&{if} ${}(\|h\E\|f\V\|h\E\\{botsink}){}$\1\5
\&{return} \\{and\_rec}${}(\|f,\39\|g){}$;\C{ $(f{?}\ g{:}\ f) = (f{?}\ g{:}\
0) = f\land g$ }\2\6
\&{if} ${}(\|g\E\|h){}$\1\5
\&{return} \\{oo}${},\39\|g\MG\\{xref}\PP,\39\|g{}$;\C{ $(f{?}\ g{:}\ g)=g$ }\2%
\6
\&{if} ${}(\|g\E\\{botsink}\W\|h\E\\{topsink}){}$\1\5
\&{return} \\{xor\_rec}${}(\\{topsink},\39\|f){}$;\C{ $(f{?}\ 0{:}\ 1) = 1%
\oplus f$ }\2\6
${}\\{oo},\39\|r\K\\{cache\_lookup}(\|f,\39\|g,\39\|h);{}$\6
\&{if} (\|r)\1\5
\&{return} \|r;\2\6
\X92:Find $(f{?}\ g{:}\ h)$ recursively\X;\6
\4${}\}{}$\2\par
\fi

\M{92}\B\X92:Find $(f{?}\ g{:}\ h)$ recursively\X${}\E{}$\6
$\\{rmems}\PP{}$;\C{ track recursion overhead }\6
${}\|v\K\\{vf}\K\\{thevar}(\|f);{}$\6
\&{if} ${}(\|g\E\\{botsink}){}$\1\5
${}\\{vg}\K\\{topofvars}{}$;\5
\2\&{else}\5
${}\{{}$\1\6
${}\\{vg}\K\\{thevar}(\|g){}$;\5
\&{if} ${}(\|v>\\{vg}){}$\1\5
${}\|v\K\\{vg};{}$\2\6
\4${}\}{}$\2\6
\&{if} ${}(\|h\E\\{topsink}){}$\1\5
${}\\{vh}\K\\{topofvars}{}$;\5
\2\&{else}\5
${}\{{}$\1\6
${}\|o,\39\\{vh}\K\\{thevar}(\|h){}$;\5
\&{if} ${}(\|v>\\{vh}){}$\1\5
${}\|v\K\\{vh};{}$\2\6
\4${}\}{}$\2\6
${}\\{r0}\K\\{mux\_rec}((\\{vf}\E\|v\?\|o,\39\\{node\_}(\|f\MG\\{lo}):\|f),\39(%
\\{vg}\E\|v\?\|o,\39\\{node\_}(\|g\MG\\{lo}):\|g),\39(\\{vh}\E\|v\?\|o,\39%
\\{node\_}(\|h\MG\\{lo}):\|h));{}$\6
\&{if} ${}(\R\\{r0}){}$\1\5
\&{return} ${}\NULL{}$;\C{ oops, trouble }\2\6
${}\\{r1}\K\\{mux\_rec}((\\{vf}\E\|v\?\|o,\39\\{node\_}(\|f\MG\\{hi}):\|f),\39(%
\\{vg}\E\|v\?\|o,\39\\{node\_}(\|g\MG\\{hi}):\|g),\39(\\{vh}\E\|v\?\|o,\39%
\\{node\_}(\|h\MG\\{hi}):\|h));{}$\6
\&{if} ${}(\R\\{r1}){}$\5
${}\{{}$\1\6
\\{deref}(\\{r0});\C{ too bad, but we have to abort in midstream }\6
\&{return} ${}\NULL;{}$\6
\4${}\}{}$\2\6
${}\|r\K\\{unique\_find}(\|v,\39\\{r0},\39\\{r1});{}$\6
\&{if} (\|r)\5
${}\{{}$\1\6
\&{if} ${}((\\{verbose}\AND\T{128})\W(\|v<\\{tvar})){}$\1\5
${}\\{printf}(\.{"\ \ \ \%x=\%x?\%x:\%x\ (lev}\)\.{el\ \%d)\\n"},\39\\{id}(%
\|r),\39\\{id}(\|f),\39\\{id}(\|g),\39\\{id}(\|h),\39\|v-\\{varhead});{}$\2\6
${}\\{cache\_insert}(\|f,\39\|g,\39\|h,\39\|r);{}$\6
\4${}\}{}$\2\6
\&{return} \|r;\par
\U91.\fi

\M{93}The median (or majority) operation $\langle fgh\rangle$ has lots of nice
symmetry.

\Y\B\4\X7:Subroutines\X${}\mathrel+\E{}$\6
\&{node} ${}{*}{}$\\{med\_rec}(\&{node} ${}{*}\|f,\39{}$\&{node} ${}{*}\|g,%
\39{}$\&{node} ${}{*}\|h){}$\1\1\2\2\6
${}\{{}$\1\6
\&{var} ${}{*}\|v,{}$ ${}{*}\\{vf},{}$ ${}{*}\\{vg},{}$ ${}{*}\\{vh};{}$\6
\&{node} ${}{*}\|r,{}$ ${}{*}\\{r0},{}$ ${}{*}\\{r1};{}$\7
\&{if} ${}(\|f>\|g){}$\5
${}\{{}$\1\6
\&{if} ${}(\|g>\|h){}$\1\5
${}\|r\K\|f,\39\|f\K\|h,\39\|h\K\|r;{}$\2\6
\&{else} \&{if} ${}(\|f>\|h){}$\1\5
${}\|r\K\|f,\39\|f\K\|g,\39\|g\K\|h,\39\|h\K\|r;{}$\2\6
\&{else}\1\5
${}\|r\K\|f,\39\|f\K\|g,\39\|g\K\|r;{}$\2\6
\4${}\}{}$\5
\2\&{else} \&{if} ${}(\|g>\|h){}$\5
${}\{{}$\1\6
\&{if} ${}(\|f>\|h){}$\1\5
${}\|r\K\|f,\39\|f\K\|h,\39\|h\K\|g,\39\|g\K\|r;{}$\2\6
\&{else}\1\5
${}\|r\K\|g,\39\|g\K\|h,\39\|h\K\|r;{}$\2\6
\4${}\}{}$\C{ now $f\le g\le h$ }\2\6
\&{if} ${}(\|f\Z\\{topsink}){}$\5
${}\{{}$\1\6
\&{if} ${}(\|f\E\\{topsink}){}$\1\5
\&{return} \\{or\_rec}${}(\|g,\39\|h){}$;\C{ $\langle 1gh\rangle=g\lor h$ }\2\6
\&{return} \\{and\_rec}${}(\|g,\39\|h){}$;\C{ $\langle 0gh\rangle=g\land h$ }\6
\4${}\}{}$\2\6
\&{if} ${}(\|f\E\|g){}$\1\5
\&{return} \\{oo}${},\39\|f\MG\\{xref}\PP,\39\|f{}$;\C{ $\langle ffh\rangle=f$
}\2\6
\&{if} ${}(\|g\E\|h){}$\1\5
\&{return} \\{oo}${},\39\|g\MG\\{xref}\PP,\39\|g{}$;\C{ $\langle fgg\rangle=g$
}\2\6
${}\\{oo},\39\|r\K\\{cache\_lookup}(\|f,\39\|g,\39\\{node\_}(\\{addr\_}(\|h)+%
\T{1}));{}$\6
\&{if} (\|r)\1\5
\&{return} \|r;\2\6
\X94:Find $\langle fgh\rangle$ recursively\X;\6
\4${}\}{}$\2\par
\fi

\M{94}\B\X94:Find $\langle fgh\rangle$ recursively\X${}\E{}$\6
$\\{rmems}\PP{}$;\C{ track recursion overhead }\6
${}\\{vf}\K\\{thevar}(\|f);{}$\6
${}\\{vg}\K\\{thevar}(\|g);{}$\6
${}\|o,\39\\{vh}\K\\{thevar}(\|h);{}$\6
\&{if} ${}(\\{vf}<\\{vg}){}$\1\5
${}\|v\K\\{vf}{}$;\5
\2\&{else}\1\5
${}\|v\K\\{vg};{}$\2\6
\&{if} ${}(\|v>\\{vh}){}$\1\5
${}\|v\K\\{vh}{}$;\C{ choose the top variable, \PB{\|v} }\2\6
${}\\{r0}\K\\{med\_rec}((\\{vf}\E\|v\?\|o,\39\\{node\_}(\|f\MG\\{lo}):\|f),\39(%
\\{vg}\E\|v\?\|o,\39\\{node\_}(\|g\MG\\{lo}):\|g),\39(\\{vh}\E\|v\?\|o,\39%
\\{node\_}(\|h\MG\\{lo}):\|h));{}$\6
\&{if} ${}(\R\\{r0}){}$\1\5
\&{return} ${}\NULL{}$;\C{ oops, trouble }\2\6
${}\\{r1}\K\\{med\_rec}((\\{vf}\E\|v\?\|o,\39\\{node\_}(\|f\MG\\{hi}):\|f),\39(%
\\{vg}\E\|v\?\|o,\39\\{node\_}(\|g\MG\\{hi}):\|g),\39(\\{vh}\E\|v\?\|o,\39%
\\{node\_}(\|h\MG\\{hi}):\|h));{}$\6
\&{if} ${}(\R\\{r1}){}$\5
${}\{{}$\1\6
\\{deref}(\\{r0});\C{ too bad, but we have to abort in midstream }\6
\&{return} ${}\NULL;{}$\6
\4${}\}{}$\2\6
${}\|r\K\\{unique\_find}(\|v,\39\\{r0},\39\\{r1});{}$\6
\&{if} (\|r)\5
${}\{{}$\1\6
\&{if} ${}((\\{verbose}\AND\T{128})\W(\|v<\\{tvar})){}$\1\5
${}\\{printf}(\.{"\ \ \ \%x=\%x.\%x.\%x\ (lev}\)\.{el\ \%d)\\n"},\39\\{id}(%
\|r),\39\\{id}(\|f),\39\\{id}(\|g),\39\\{id}(\|h),\39\|v-\\{varhead});{}$\2\6
${}\\{cache\_insert}(\|f,\39\|g,\39\\{node\_}(\\{addr\_}(\|h)+\T{1}),\39%
\|r);{}$\6
\4${}\}{}$\2\6
\&{return} \|r;\par
\U93.\fi

\M{95}We can also exploit the symmetry of $f\land g\land h$.
(If I had lots of time, I could play similarly with $f\lor g\lor h$ and
$f\oplus g\oplus h$.)

\Y\B\4\X7:Subroutines\X${}\mathrel+\E{}$\6
\&{node} ${}{*}{}$\\{and\_and\_rec}(\&{node} ${}{*}\|f,\39{}$\&{node} ${}{*}%
\|g,\39{}$\&{node} ${}{*}\|h){}$\1\1\2\2\6
${}\{{}$\1\6
\&{var} ${}{*}\|v,{}$ ${}{*}\\{vf},{}$ ${}{*}\\{vg},{}$ ${}{*}\\{vh};{}$\6
\&{node} ${}{*}\|r,{}$ ${}{*}\\{r0},{}$ ${}{*}\\{r1};{}$\7
\&{if} ${}(\|f>\|g){}$\5
${}\{{}$\1\6
\&{if} ${}(\|g>\|h){}$\1\5
${}\|r\K\|f,\39\|f\K\|h,\39\|h\K\|r;{}$\2\6
\&{else} \&{if} ${}(\|f>\|h){}$\1\5
${}\|r\K\|f,\39\|f\K\|g,\39\|g\K\|h,\39\|h\K\|r;{}$\2\6
\&{else}\1\5
${}\|r\K\|f,\39\|f\K\|g,\39\|g\K\|r;{}$\2\6
\4${}\}{}$\5
\2\&{else} \&{if} ${}(\|g>\|h){}$\5
${}\{{}$\1\6
\&{if} ${}(\|f>\|h){}$\1\5
${}\|r\K\|f,\39\|f\K\|h,\39\|h\K\|g,\39\|g\K\|r;{}$\2\6
\&{else}\1\5
${}\|r\K\|g,\39\|g\K\|h,\39\|h\K\|r;{}$\2\6
\4${}\}{}$\C{ now $f\le g\le h$ }\2\6
\&{if} ${}(\|f\Z\\{topsink}){}$\5
${}\{{}$\1\6
\&{if} ${}(\|f\E\\{topsink}){}$\1\5
\&{return} \\{and\_rec}${}(\|g,\39\|h){}$;\C{ $1\land g\land h=g\land h$ }\2\6
\&{return} \\{oo}${},\39\\{botsink}\MG\\{xref}\PP,\39\\{botsink}{}$;\C{ $0\land
g\land h=0$ }\6
\4${}\}{}$\2\6
\&{if} ${}(\|f\E\|g){}$\1\5
\&{return} \\{and\_rec}${}(\|g,\39\|h){}$;\C{ $f\land f\land h=f\land h$ }\2\6
\&{if} ${}(\|g\E\|h){}$\1\5
\&{return} \\{and\_rec}${}(\|f,\39\|g){}$;\C{ $f\land g\land g=f\land g$ }\2\6
${}\\{oo},\39\|r\K\\{cache\_lookup}(\|f,\39\|g,\39\\{node\_}(\\{addr\_}(\|h)+%
\T{2}));{}$\6
\&{if} (\|r)\1\5
\&{return} \|r;\2\6
\X96:Find $f\land g\land h$ recursively\X;\6
\4${}\}{}$\2\par
\fi

\M{96}\B\X96:Find $f\land g\land h$ recursively\X${}\E{}$\6
$\\{rmems}\PP{}$;\C{ track recursion overhead }\6
${}\\{vf}\K\\{thevar}(\|f);{}$\6
${}\\{vg}\K\\{thevar}(\|g);{}$\6
${}\|o,\39\\{vh}\K\\{thevar}(\|h);{}$\6
\&{if} ${}(\\{vf}<\\{vg}){}$\1\5
${}\|v\K\\{vf}{}$;\5
\2\&{else}\1\5
${}\|v\K\\{vg};{}$\2\6
\&{if} ${}(\|v>\\{vh}){}$\1\5
${}\|v\K\\{vh}{}$;\C{ choose the top variable, \PB{\|v} }\2\6
${}\\{r0}\K\\{and\_and\_rec}((\\{vf}\E\|v\?\|o,\39\\{node\_}(\|f\MG\\{lo}):%
\|f),\39(\\{vg}\E\|v\?\|o,\39\\{node\_}(\|g\MG\\{lo}):\|g),\39(\\{vh}\E\|v\?%
\|o,\39\\{node\_}(\|h\MG\\{lo}):\|h));{}$\6
\&{if} ${}(\R\\{r0}){}$\1\5
\&{return} ${}\NULL{}$;\C{ oops, trouble }\2\6
${}\\{r1}\K\\{and\_and\_rec}((\\{vf}\E\|v\?\|o,\39\\{node\_}(\|f\MG\\{hi}):%
\|f),\39(\\{vg}\E\|v\?\|o,\39\\{node\_}(\|g\MG\\{hi}):\|g),\39(\\{vh}\E\|v\?%
\|o,\39\\{node\_}(\|h\MG\\{hi}):\|h));{}$\6
\&{if} ${}(\R\\{r1}){}$\5
${}\{{}$\1\6
\\{deref}(\\{r0});\C{ too bad, but we have to abort in midstream }\6
\&{return} ${}\NULL;{}$\6
\4${}\}{}$\2\6
${}\|r\K\\{unique\_find}(\|v,\39\\{r0},\39\\{r1});{}$\6
\&{if} (\|r)\5
${}\{{}$\1\6
\&{if} ${}((\\{verbose}\AND\T{128})\W(\|v<\\{tvar})){}$\1\5
${}\\{printf}(\.{"\ \ \ \%x=\%x\&\%x\&\%x\ (lev}\)\.{el\ \%d)\\n"},\39\\{id}(%
\|r),\39\\{id}(\|f),\39\\{id}(\|g),\39\\{id}(\|h),\39\|v-\\{varhead});{}$\2\6
${}\\{cache\_insert}(\|f,\39\|g,\39\\{node\_}(\\{addr\_}(\|h)+\T{2}),\39%
\|r);{}$\6
\4${}\}{}$\2\6
\&{return} \|r;\par
\U95.\fi

\M{97}The final ternary operation computes $(f\land g)\(E)h$, when \PB{\|h} is
a conjunction of positive literals as before. Ken McMillan noticed that one can
often save a lot of time computing this ternary function recursively, instead
of first forming $f\land g$. We combine the ideas of \PB{\\{and\_rec}} and \PB{%
\\{exist\_rec}}
here.

\Y\B\4\X7:Subroutines\X${}\mathrel+\E{}$\6
\&{node} ${}{*}{}$\\{and\_exist\_rec}(\&{node} ${}{*}\|f,\39{}$\&{node} ${}{*}%
\|g,\39{}$\&{node} ${}{*}\|h){}$\1\1\2\2\6
${}\{{}$\1\6
\&{var} ${}{*}\|v,{}$ ${}{*}\\{vf},{}$ ${}{*}\\{vg},{}$ ${}{*}\\{vh};{}$\6
\&{node} ${}{*}\|r,{}$ ${}{*}\\{r0},{}$ ${}{*}\\{r1};{}$\7
\4\\{restart}:\5
\&{if} ${}(\|h\Z\\{topsink}){}$\1\5
\&{return} \\{and\_rec}${}(\|f,\39\|g){}$;\C{ $(f\land g)\(E)1=f\land g$ }\2\6
\&{if} ${}(\|f\E\|g){}$\1\5
\&{return} \\{exist\_rec}${}(\|f,\39\|h){}$;\C{ $(f\land f)\(E)h=f\(E)h$ }\2\6
\&{if} ${}(\|f>\|g){}$\1\5
${}\|r\K\|f,\39\|f\K\|g,\39\|g\K\|r{}$;\C{ wow }\2\6
\&{if} ${}(\|f\Z\\{topsink}){}$\5
${}\{{}$\1\6
\&{if} ${}(\|f\E\\{topsink}){}$\1\5
\&{return} \\{exist\_rec}${}(\|g,\39\|h){}$;\C{ $(1\land g)\(E)h=g\(E)h$ }\2\6
\&{return} \\{oo}${},\39\\{botsink}\MG\\{xref}\PP,\39\\{botsink}{}$;\C{ $(0%
\land d)\(E)h=0$ }\6
\4${}\}{}$\2\6
${}\\{oo},\39\|r\K\\{cache\_lookup}(\|f,\39\|g,\39\\{node\_}(\\{addr\_}(\|h)+%
\T{3}));{}$\6
\&{if} (\|r)\1\5
\&{return} \|r;\2\6
${}\\{vf}\K\\{thevar}(\|f);{}$\6
${}\\{vg}\K\\{thevar}(\|g);{}$\6
${}\|o,\39\\{vh}\K\\{thevar}(\|h);{}$\6
\&{if} ${}(\\{vf}<\\{vg}){}$\1\5
${}\|v\K\\{vf}{}$;\5
\2\&{else}\1\5
${}\|v\K\\{vg};{}$\2\6
\&{if} ${}(\|v>\\{vh}){}$\5
${}\{{}$\1\6
${}\|o,\39\|h\K\\{node\_}(\|h\MG\\{hi}){}$;\5
\&{goto} \\{restart};\C{ \PB{\|f} and \PB{\|g} don't depend on \PB{\\{vh}} }\6
\4${}\}{}$\2\6
\X98:Find $(f\land g)\(E)h$ recursively\X;\6
\4${}\}{}$\2\par
\fi

\M{98}\B\X98:Find $(f\land g)\(E)h$ recursively\X${}\E{}$\6
$\\{rmems}\PP{}$;\C{ track recursion overhead }\6
${}\\{r0}\K\\{and\_exist\_rec}((\\{vf}\E\|v\?\|o,\39\\{node\_}(\|f\MG\\{lo}):%
\|f),\39(\\{vg}\E\|v\?\|o,\39\\{node\_}(\|g\MG\\{lo}):\|g),\39(\\{vh}\E\|v\?%
\|o,\39\\{node\_}(\|h\MG\\{hi}):\|h));{}$\6
\&{if} ${}(\R\\{r0}){}$\1\5
\&{return} ${}\NULL{}$;\C{ oops, trouble }\2\6
\&{if} ${}(\\{r0}\E\\{topsink}\W\\{vg}\E\|v){}$\5
${}\{{}$\1\6
${}\|r\K\\{r0};{}$\6
\&{goto} \\{gotr};\6
\4${}\}{}$\2\6
${}\\{r1}\K\\{and\_exist\_rec}((\\{vf}\E\|v\?\\{node\_}(\|f\MG\\{hi}):\|f),\39(%
\\{vg}\E\|v\?\\{node\_}(\|g\MG\\{hi}):\|g),\39(\\{vh}\E\|v\?\\{node\_}(\|h\MG%
\\{hi}):\|h));{}$\6
\&{if} ${}(\R\\{r1}){}$\5
${}\{{}$\1\6
\\{deref}(\\{r0});\C{ too bad, but we have to abort in midstream }\6
\&{return} ${}\NULL;{}$\6
\4${}\}{}$\2\6
\&{if} ${}(\\{vh}>\|v){}$\1\5
${}\|r\K\\{unique\_find}(\|v,\39\\{r0},\39\\{r1});{}$\2\6
\&{else}\5
${}\{{}$\1\6
${}\|r\K\\{or\_rec}(\\{r0},\39\\{r1}){}$;\C{ existential quantification happens
here }\6
\\{deref}(\\{r0});\5
\\{deref}(\\{r1});\C{ we're done with \PB{\\{r0}} and \PB{\\{r1}} }\6
\4${}\}{}$\2\6
\4\\{gotr}:\5
\&{if} (\|r)\5
${}\{{}$\1\6
\&{if} ${}((\\{verbose}\AND\T{128})\W(\|v<\\{tvar})){}$\1\5
${}\\{printf}(\.{"\ \ \ \%x=\%x\&\%xE\%x\ (lev}\)\.{el\ \%d)\\n"},\39\\{id}(%
\|r),\39\\{id}(\|f),\39\\{id}(\|g),\39\\{id}(\|h),\39\|v-\\{varhead});{}$\2\6
${}\\{cache\_insert}(\|f,\39\|g,\39\\{node\_}(\\{addr\_}(\|h)+\T{3}),\39%
\|r);{}$\6
\4${}\}{}$\2\6
\&{return} \|r;\par
\U97.\fi

\N{1}{99}Composition of functions.
Now we're ready for the biggie, the most powerful recursive command
in our arsenal: Given the BDDs for a master function $f(x_0,\ldots,x_n)$
and for replacement functions $y_0(x_0,\ldots,x_n)$,
\dots,~$y_n(x_0,\ldots,x_n)$, construct the BDD for the grand composition
$$
F(x_0,\ldots,x_n)=f\big(y_0(x_0,\ldots,x_n),\ldots,y_n(x_0,\ldots,x_n)\bigr).
$$
This daunting task can actually be accomplished with a surprisingly short
recursive program.

Of course the result might be huge, so the running time
can be humongous in bad cases. But in not-so-bad cases, the running time
is not bad at all. Let's be optimistic.

The replacements $y_k$ are specified by individual commands like
`\.{y3=f7}' or `\.{y6=x5}' or `\.{y2=c1}'; see the syntax at the
beginning of this program. One can also cancel a previous replacement by
saying, for example, `\.{y3=.}'. If no replacement $y_k$ is currently
specified, the identity function $y_k=x_k$ is assumed.

\fi

\M{100}Our main tool for efficiency in not-so-bad cases is the cache,
so that previously known results needn't be recalculated.
Each node in the BDD base represents a function of the variables
$(x_v,x_{v+1},\ldots{})$, for some variable index~$v$, and we can
let the cache remember if we've already composed this function with
the replacements $(y_v,y_{v+1},\ldots{})$.
When the user changes $y_k$, all results
in the cache for functions with $v\le k$ are potentially incorrect;
but the results for functions with $v>k$ remain usable.

Therefore we maintain a time stamp for each variable $x_k$. The time stamps
for $x_0$ through $x_k$ are increased whenever $y_k$ is updated, so that
cache entries for the composition of such functions won't match when
\PB{\\{cache\_lookup}} is called.

The time stamp for $x_v$ is zero if and only if all of $y_v$, $y_{v+1}$,
\dots\ are currently undefined.

Two global variables control this process:
The variable \PB{\\{timestamp\_changed}} is either 0 or~1;
the difference \PB{$\\{timestamp}-\\{timestamp\_changed}$} is the number of
times
we've performed a composition operation.

\fi

\M{101}\B\X4:Global variables\X${}\mathrel+\E{}$\6
\&{int} \\{timestamp};\C{ the number of distinct compositions done or prepared
}\6
\&{int} \\{timestamp\_changed};\C{ has it changed since the last composition
was done? }\par
\fi

\M{102}Here's how the individual time stamps are updated when the user gives a
new
specification for $y_k$:

\Y\B\4\X102:Assign \PB{\|q} as the new value of \PB{$\|v\MG\\{repl}$}\X${}\E{}$%
\6
$\|v\MG\\{repl}\K\\{addr\_}(\|q);{}$\6
\&{if} ${}(\|q\E\NULL){}$\5
${}\{{}$\C{ cancel a previous replacement }\1\6
\&{if} ${}(\|v+\T{1}<\\{topofvars}\W(\|o,\39(\|v+\T{1})\MG\\{timestamp}\I%
\T{0})){}$\1\5
\&{goto} \\{newstamps};\2\6
\&{while} ${}(\|v\G\\{varhead}\W(\|o,\39\|v\MG\\{repl}\E\T{0})){}$\1\5
${}\|o,\39\|v\MG\\{timestamp}\K\T{0},\39\|v\MM{}$;\C{ cancel the previous
timestamp }\2\6
\&{if} ${}(\|v\G\\{varhead}){}$\1\5
\&{goto} \\{newstamps};\2\6
${}\\{timestamp}\MRL{-{\K}}\\{timestamp\_changed},\39\\{timestamp\_changed}\K%
\T{0}{}$;\C{ all clear }\6
\4${}\}{}$\5
\2\&{else}\5
${}\{{}$\1\6
${}\\{oo},\39\|q\MG\\{xref}\PP;{}$\6
\4\\{newstamps}:\5
\&{if} ${}(\R\\{timestamp\_changed}){}$\1\5
${}\\{timestamp\_changed}\K\T{1},\39\\{timestamp}\PP{}$;\C{ N.B.: \PB{%
\\{timestamp}} won't become zero }\2\6
\&{while} ${}(\|v\G\\{varhead}\W(\|o,\39\|v\MG\\{timestamp}\I\\{timestamp})){}$%
\1\5
${}\|o,\39\|v\MG\\{timestamp}\K\\{timestamp},\39\|v\MM{}$;\C{ this is done even
when \PB{$\|v\MG\\{proj}\K\T{0}$} }\2\6
\4${}\}{}$\2\par
\U128.\fi

\M{103}A special adjustment to the time stamps is needed when we're wiping out
the whole cache.

\Y\B\4\X103:Clear out the time stamps\X${}\E{}$\6
\&{if} ${}(\\{varhead}[\T{0}].\\{timestamp}\E\T{0}){}$\1\5
${}\|o,\39\\{timestamp}\K\\{timestamp\_changed}\K\T{0};{}$\2\6
\&{else}\5
${}\{{}$\1\6
${}\\{timestamp}\K\\{timestamp\_changed}\K\T{1};{}$\6
\&{for} ${}(\|v\K\\{varhead};{}$ ${}\|v<\\{topofvars}\W(\|o,\39\|v\MG%
\\{timestamp});{}$ ${}\|v\PP){}$\1\5
${}\|o,\39\|v\MG\\{timestamp}\K\T{1};{}$\2\6
\4${}\}{}$\2\par
\U156.\fi

\M{104}Okay, we're ready for the master recursion.

\Y\B\4\X7:Subroutines\X${}\mathrel+\E{}$\6
\&{node} ${}{*}{}$\\{compose\_rec}(\&{node} ${}{*}\|f){}$\1\1\2\2\6
${}\{{}$\1\6
\&{var} ${}{*}\\{vf};{}$\6
\&{node} ${}{*}\|r,{}$ ${}{*}\\{r0},{}$ ${}{*}\\{r1};{}$\7
\&{if} ${}(\|f\Z\\{topsink}){}$\1\5
\&{return} \\{oo}${},\39\|f\MG\\{xref}\PP,\39\|f{}$;\C{ $f$ is constant }\2\6
${}\|o,\39\\{vf}\K\\{thevar}(\|f);{}$\6
\&{if} ${}(\|o,\39\\{vf}\MG\\{timestamp}\E\T{0}){}$\1\5
\&{return} \\{oo}${},\39\|f\MG\\{xref}\PP,\39\|f{}$;\C{ $f$ doesn't depend on
$y$'s }\2\6
${}\|o,\39\|r\K\\{cache\_lookup}(\|f,\39\\{node\_}(\\{vf}\MG\\{timestamp}),\39%
\\{node\_}(\T{0}));{}$\6
\&{if} (\|r)\1\5
\&{return} \|r;\2\6
\X105:Find $f(y_0,\ldots{})$ recursively\X;\6
\4${}\}{}$\2\par
\fi

\M{105}The computations here look basically the same as those we've been
seeing in previous recursions. But in fact there is a huge difference:
The functions \PB{\\{r0}} and~\PB{\\{r1}} now can involve {\it all\/}
variables,
not just the variables near the bottom of the BDDs.

\Y\B\4\X105:Find $f(y_0,\ldots{})$ recursively\X${}\E{}$\6
$\\{rmems}\PP{}$;\C{ track recursion overhead }\6
${}\|o,\39\\{r0}\K\\{compose\_rec}(\\{node\_}(\|f\MG\\{lo}));{}$\6
\&{if} ${}(\R\\{r0}){}$\1\5
\&{return} ${}\NULL{}$;\C{ oops, trouble }\2\6
${}\\{r1}\K\\{compose\_rec}(\\{node\_}(\|f\MG\\{hi}));{}$\6
\&{if} ${}(\R\\{r1}){}$\5
${}\{{}$\1\6
\\{deref}(\\{r0});\C{ too bad, but we have to abort in midstream }\6
\&{return} ${}\NULL;{}$\6
\4${}\}{}$\2\6
\&{if} ${}(\|o,\39\\{vf}\MG\\{repl}){}$\1\5
${}\|r\K\\{node\_}(\\{vf}\MG\\{repl}){}$;\5
\2\&{else}\1\5
${}\|r\K\\{node\_}(\\{vf}\MG\\{proj});{}$\2\6
${}\|r\K\\{mux\_rec}(\|r,\39\\{r1},\39\\{r0}){}$;\C{ replacement happens here }%
\6
\\{deref}(\\{r0});\5
\\{deref}(\\{r1});\C{ we're done with \PB{\\{r0}} and \PB{\\{r1}} }\6
\&{if} (\|r)\5
${}\{{}$\1\6
\&{if} ${}((\\{verbose}\AND\T{128})\W(\\{vf}<\\{tvar})){}$\1\5
${}\\{printf}(\.{"\ \ \ \%x=\%x[\%u]\ (level}\)\.{\ \%d)\\n"},\39\\{id}(\|r),%
\39\\{id}(\|f),\39\\{vf}\MG\\{timestamp},\39\\{vf}-\\{varhead});{}$\2\6
${}\\{cache\_insert}(\|f,\39\\{node\_}(\\{vf}\MG\\{timestamp}),\39\\{node\_}(%
\T{0}),\39\|r);{}$\6
\4${}\}{}$\2\6
\&{return} \|r;\par
\U104.\fi

\N{1}{106}Top-level calls. As mentioned above, there's a top-level ``wrapper''
around
each of the recursive synthesis routines, so that we can launch them properly.

Here's the top-level routine for binary operators.

\Y\B\4\X7:Subroutines\X${}\mathrel+\E{}$\6
\&{node} ${}{*}{}$\\{binary\_top}(\&{int} \\{curop}${},\39{}$\&{node} ${}{*}%
\|f,\39{}$\&{node} ${}{*}\|g){}$\1\1\2\2\6
${}\{{}$\1\6
\&{node} ${}{*}\|r;{}$\6
\&{unsigned} \&{long} \&{long} \\{oldmems}${}\K\\{mems},{}$ \\{oldrmems}${}\K%
\\{rmems},{}$ \\{oldzmems}${}\K\\{zmems};{}$\7
\&{if} ${}(\\{verbose}\AND\T{2}){}$\1\5
${}\\{printf}(\.{"beginning\ to\ comput}\)\.{e\ \%x\ \%s\ \%x:\\n"},\39\\{id}(%
\|f),\39\\{binopname}[\\{curop}],\39\\{id}(\|g));{}$\2\6
${}\\{cacheinserts}\K\T{0};{}$\6
\&{while} (\T{1})\5
${}\{{}$\1\6
\&{switch} (\\{curop})\5
${}\{{}$\1\6
\4\&{case} \T{1}:\5
${}\|r\K\\{and\_rec}(\|f,\39\|g){}$;\5
\&{break};\C{ $f\land g$ }\6
\4\&{case} \T{2}:\5
${}\|r\K\\{mux\_rec}(\|g,\39\\{botsink},\39\|f){}$;\5
\&{break};\C{ $f\land\bar g=(g{?}\ 0{:}\ f)$ }\6
\4\&{case} \T{4}:\5
${}\|r\K\\{mux\_rec}(\|f,\39\\{botsink},\39\|g){}$;\5
\&{break};\C{ $\bar f\land g=(f{?}\ 0{:}\ g)$ }\6
\4\&{case} \T{6}:\5
${}\|r\K\\{xor\_rec}(\|f,\39\|g){}$;\5
\&{break};\C{ $f\oplus g$ }\6
\4\&{case} \T{7}:\5
${}\|r\K\\{or\_rec}(\|f,\39\|g){}$;\5
\&{break};\C{ $f\lor g$ }\6
\4\&{case} \T{8}:\5
${}\|r\K\\{constrain\_rec}(\|f,\39\|g){}$;\5
\&{break};\C{ $f\downarrow g$ }\6
\4\&{case} \T{9}:\5
${}\|r\K\\{all\_rec}(\|f,\39\|g){}$;\5
\&{break};\C{ $f\(A) g$ }\6
\4\&{case} \T{10}:\5
\&{case} \T{12}:\5
${}\|r\K\\{yes\_no\_rec}(\\{curop},\39\|f,\39\|g){}$;\5
\&{break};\C{ $f\(N)g$ or $f\(Y)g$ }\6
\4\&{case} \T{14}:\5
${}\|r\K\\{diff\_rec}(\|f,\39\|g){}$;\5
\&{break};\C{ $f\(D) g$ }\6
\4\&{case} \T{15}:\5
${}\|r\K\\{exist\_rec}(\|f,\39\|g){}$;\5
\&{break};\6
\4\&{default}:\5
${}\\{fprintf}(\\{stderr},\39\.{"This\ can't\ happen!\\}\)\.{n"}){}$;\5
${}\\{exit}({-}\T{69});{}$\6
\4${}\}{}$\2\6
\&{if} (\|r)\1\5
\&{break};\2\6
\\{attempt\_repairs}(\,);\C{ try to carry on }\6
\4${}\}{}$\2\6
\&{if} ${}(\\{verbose}\AND(\T{1}+\T{2})){}$\1\5
${}\\{printf}(\.{"\ \%x=\%x\%s\%x\ (\%llu\ me}\)\.{ms,\ \%llu\ rmems,\ \%llu}\)%
\.{\ zmems)\\n"},\3{-1}\39\\{id}(\|r),\39\\{id}(\|f),\39\\{binopname}[%
\\{curop}],\39\\{id}(\|g),\39\\{mems}-\\{oldmems},\39\\{rmems}-\\{oldrmems},\39%
\\{zmems}-\\{oldzmems});{}$\2\6
\&{return} \|r;\6
\4${}\}{}$\2\par
\fi

\M{107}\B\X25:Templates for subroutines\X${}\mathrel+\E{}$\6
\&{void} \\{attempt\_repairs}(\&{void});\C{ collect garbage or something if
there's hope }\par
\fi

\M{108}Have you any wool? Yes sir, yes sir.

\Y\B\4\X7:Subroutines\X${}\mathrel+\E{}$\6
\&{node} ${}{*}{}$\\{ternary\_top}(\&{int} \\{curop}${},\39{}$\&{node} ${}{*}%
\|f,\39{}$\&{node} ${}{*}\|g,\39{}$\&{node} ${}{*}\|h){}$\1\1\2\2\6
${}\{{}$\1\6
\&{node} ${}{*}\|r;{}$\6
\&{unsigned} \&{long} \&{long} \\{oldmems}${}\K\\{mems},{}$ \\{oldrmems}${}\K%
\\{rmems},{}$ \\{oldzmems}${}\K\\{zmems};{}$\7
\&{if} ${}(\\{verbose}\AND\T{2}){}$\1\5
${}\\{printf}(\.{"beginning\ to\ comput}\)\.{e\ \%x\ \%s\ \%x\ \%s\ \%x:\\n"},%
\39\\{id}(\|f),\39\\{ternopname1}[\\{curop}-\T{16}],\39\\{id}(\|g),\39%
\\{ternopname2}[\\{curop}-\T{16}],\39\\{id}(\|h));{}$\2\6
${}\\{cacheinserts}\K\T{0};{}$\6
\&{while} (\T{1})\5
${}\{{}$\1\6
\&{switch} (\\{curop})\5
${}\{{}$\1\6
\4\&{case} \T{16}:\5
${}\|r\K\\{mux\_rec}(\|f,\39\|g,\39\|h){}$;\5
\&{break};\C{ $f{?}\ g{:}\ h$ }\6
\4\&{case} \T{17}:\5
${}\|r\K\\{med\_rec}(\|f,\39\|g,\39\|h){}$;\5
\&{break};\C{ $\langle fgh\rangle$ }\6
\4\&{case} \T{18}:\5
${}\|r\K\\{and\_and\_rec}(\|f,\39\|g,\39\|h){}$;\5
\&{break};\C{ $f\land g\land h$ }\6
\4\&{case} \T{19}:\5
${}\|r\K\\{and\_exist\_rec}(\|f,\39\|g,\39\|h){}$;\5
\&{break};\C{ $(f\land g)\(E)h$ }\6
\4\&{default}:\5
${}\\{fprintf}(\\{stderr},\39\.{"This\ can't\ happen!\\}\)\.{n"}){}$;\5
${}\\{exit}({-}\T{69});{}$\6
\4${}\}{}$\2\6
\&{if} (\|r)\1\5
\&{break};\2\6
\\{attempt\_repairs}(\,);\C{ try to carry on }\6
\4${}\}{}$\2\6
\&{if} ${}(\\{verbose}\AND(\T{1}+\T{2})){}$\1\5
${}\\{printf}(\.{"\ \%x=\%x\%s\%x\%s\%x\ (\%ll}\)\.{u\ mems,\ \%llu\ rmems,\ }%
\)\.{\%llu\ zmems)\\n"},\3{-1}\39\\{id}(\|r),\39\\{id}(\|f),\39\\{ternopname1}[%
\\{curop}-\T{16}],\39\\{id}(\|g),\39\\{ternopname2}[\\{curop}-\T{16}],\39%
\\{id}(\|h),\39\\{mems}-\\{oldmems},\39\\{rmems}-\\{oldrmems},\39\\{zmems}-%
\\{oldzmems});{}$\2\6
\&{return} \|r;\6
\4${}\}{}$\2\par
\fi

\M{109}\B\X7:Subroutines\X${}\mathrel+\E{}$\6
\&{node} ${}{*}{}$\\{compose\_top}(\&{node} ${}{*}\|f){}$\1\1\2\2\6
${}\{{}$\1\6
\&{node} ${}{*}\|r;{}$\6
\&{unsigned} \&{long} \&{long} \\{oldmems}${}\K\\{mems},{}$ \\{oldrmems}${}\K%
\\{rmems},{}$ \\{oldzmems}${}\K\\{zmems};{}$\7
\&{if} ${}(\|f\Z\\{topsink}){}$\1\5
\&{return} \|f;\2\6
\&{if} ${}(\\{verbose}\AND\T{2}){}$\1\5
${}\\{printf}(\.{"beginning\ to\ comput}\)\.{e\ \%x[\%u]:\\n"},\39\\{id}(\|f),%
\39\\{thevar}(\|f)\MG\\{timestamp});{}$\2\6
${}\\{cacheinserts}\K\T{0};{}$\6
${}\\{timestamp\_changed}\K\T{0};{}$\6
\&{while} (\T{1})\5
${}\{{}$\1\6
${}\|r\K\\{compose\_rec}(\|f);{}$\6
\&{if} (\|r)\1\5
\&{break};\2\6
\\{attempt\_repairs}(\,);\C{ try to carry on }\6
\4${}\}{}$\2\6
\&{if} ${}(\\{verbose}\AND(\T{1}+\T{2})){}$\1\5
${}\\{printf}(\.{"\ \%x=\%x[\%u]\ (\%llu\ me}\)\.{ms,\ \%llu\ rmems,\ \%llu}\)%
\.{\ zmems)\\n"},\3{-1}\39\\{id}(\|r),\39\\{id}(\|f),\39\\{thevar}(\|f)\MG%
\\{timestamp},\39\\{mems}-\\{oldmems},\39\\{rmems}-\\{oldrmems},\39\\{zmems}-%
\\{oldzmems});{}$\2\6
\&{return} \|r;\6
\4${}\}{}$\2\par
\fi

\N{1}{110}Parsing the commands.
We're almost done, but we need to control the overall process by
obeying the user's instructions.
The syntax for elementary user commands appeared at the beginning of this
program; now we want to flesh it out and implement it.

\Y\B\4\X110:Read a command and obey it; \PB{\&{goto} \\{alldone}} if done\X${}%
\E{}$\6
${}\{{}$\1\6
\X111:Make sure the coast is clear\X;\6
\X113:Fill \PB{\\{buf}} with the next command, or \PB{\&{goto} \\{alldone}}\X;\6
\X114:Parse the command and execute it\X;\6
\4${}\}{}$\2\par
\U2.\fi

\M{111}Before we do any commands, it's helpful to ensure that no embarrassing
anomalies will arise. For example, a command like `\.{f0=x1{\tta}x2{\tta}x3}'
might necessitate making space for up to three new variables;
it would be a nuisance if those attempts failed. (See the
\PB{\\{projection}} routine.) We also want to check that \PB{\\{timestamp}}
hasn't
reached its maximum possible value.

\Y\B\4\D$\\{debugging}$ \5
\T{1}\par
\Y\B\4\X111:Make sure the coast is clear\X${}\E{}$\6
\8\#\&{if} ${}\\{debugging}\AND\\{includesanity}{}$\6
\&{if} ${}(\\{verbose}\AND\T{8192}){}$\1\5
\\{sanity\_check}(\,);\2\6
\8\#\&{endif}\6
\&{if} ${}(\\{totalnodes}\G\\{toobig}){}$\1\5
\X153:Invoke autosifting\X;\2\6
\&{if} ${}(\\{verbose}\AND\T{1024}){}$\1\5
\\{show\_stats}(\,);\2\6
\&{while} (\T{1})\5
${}\{{}$\1\6
\&{if} ${}(\\{timestamp}\I{-}\T{1}){}$\5
${}\{{}$\1\6
\X17:If there are at least three free pages and at least three free nodes, \PB{%
\&{break}}\X;\6
\4${}\}{}$\2\6
\\{attempt\_repairs}(\,);\6
\4${}\}{}$\2\par
\U110.\fi

\M{112}
\Y\B\4\D$\\{bufsize}$ \5
\T{100}\C{ all commands are very short, but comments might be long }\par
\Y\B\4\X4:Global variables\X${}\mathrel+\E{}$\6
\&{char} \\{buf}[\\{bufsize}];\C{ our master's voice }\par
\fi

\M{113}\B\X113:Fill \PB{\\{buf}} with the next command, or \PB{\&{goto} %
\\{alldone}}\X${}\E{}$\6
\&{if} (\\{infile})\5
${}\{{}$\1\6
\&{if} ${}(\R\\{fgets}(\\{buf},\39\\{bufsize},\39\\{infile})){}$\5
${}\{{}$\C{ assume end of file }\1\6
\&{if} (\\{file\_given})\1\5
\&{goto} \\{alldone};\C{ quit the program if the file was \PB{\\{argv}[\T{1}]}
}\2\6
\\{fclose}(\\{infile});\6
${}\\{infile}\K\NULL;{}$\6
\&{continue};\6
\4${}\}{}$\2\6
\&{if} ${}(\\{verbose}\AND\T{64}){}$\1\5
${}\\{printf}(\.{">\ \%s"},\39\\{buf});{}$\2\6
\4${}\}{}$\5
\2\&{else}\5
\1\&{while} (\T{1})\5
${}\{{}$\1\6
\\{printf}(\.{">\ "});\5
\\{fflush}(\\{stdout});\C{ prompt the user }\6
\&{if} ${}(\\{fgets}(\\{buf},\39\\{bufsize},\39\\{stdin})){}$\1\5
\&{break};\2\6
${}\\{freopen}(\.{"/dev/tty"},\39\.{"r"},\39\\{stdin}){}$;\C{ end of
command-line \PB{\\{stdin}} }\6
\4${}\}{}$\2\2\par
\U110.\fi

\M{114}The first nonblank character of each line identifies the type of
command.
All-blank lines are ignored; so are lines that begin with `\.\#'.

I haven't attempted to make this interface the slightest bit fancy.
Nor have I had time to write a detailed explanation of how to use
this program---sorry. Hopefully someone like David Pogue will be
motivated to write the missing manual.

\Y\B\4\D$\\{getk}$ \5
\&{for} ${}(\|k\K\T{0};{}$ ${}\\{isdigit}({*}\|c);{}$ ${}\|c\PP)$ $\|k\K\T{10}*%
\|k+{*}\|c-{}$\.{'0'}\C{ scan a number }\par
\B\4\D$\\{reporterror}$ \6
${}\{{}$\5
\1${}\\{printf}(\.{"Sorry;\ `\%c'\ confuse}\)\.{s\ me\ \%s\%s"},\3{-1}\39{*}(%
\|c-\T{1}),\39\\{infile}\?\.{"in\ this\ command:\ "}:\.{"in\ that\ command."},%
\39\\{infile}\?\\{buf}:\.{"\\n"});{}$\6
\&{goto} \\{nextcommand};\5
${}\}{}$\2\par
\Y\B\4\X114:Parse the command and execute it\X${}\E{}$\6
\4\\{rescan}:\5
\&{for} ${}(\|c\K\\{buf};{}$ ${}{*}\|c\E\.{'\ '};{}$ ${}\|c\PP){}$\1\5
;\C{ pass over initial blanks }\2\6
\&{switch} ${}({*}\|c\PP){}$\5
${}\{{}$\1\6
\4\&{case} \.{'\\n'}:\5
\&{if} ${}(\R\\{infile}){}$\1\5
\\{printf}(\.{"(Type\ `quit'\ to\ exi}\)\.{t\ the\ program.)\\n"});\2\6
\4\&{case} \.{'\#'}:\5
\&{continue};\6
\4\&{case} \.{'!'}:\5
${}\\{printf}(\\{buf}+\T{1}){}$;\5
\&{continue};\C{ echo the input line on \PB{\\{stdout}} }\6
\4\&{case} \.{'b'}:\5
\X146:Bubble sort to reestablish the natural variable order\X;\5
\&{continue};\6
\4\&{case} \.{'C'}:\5
\\{print\_cache}(\,);\5
\&{continue};\6
\4\&{case} \.{'f'}:\5
\X120:Parse and execute an assignment to $f_k$\X;\5
\&{continue};\6
\4\&{case} \.{'i'}:\5
\X116:Get ready to read a new input file\X;\5
\&{continue};\6
\4\&{case} \.{'l'}:\5
\\{getk};\5
${}\\{leasesonlife}\K\|k{}$;\5
\&{continue};\6
\4\&{case} \.{'m'}:\5
\X160:Print a Mathematica program for a generating function\X;\5
\&{continue};\6
\4\&{case} \.{'o'}:\5
\X118:Output a function\X;\5
\&{continue};\6
\4\&{case} \.{'O'}:\5
\X119:Print the current variable ordering\X;\5
\&{continue};\6
\4\&{case} \.{'p'}:\5
\X117:Print a function or its profile\X;\5
\&{continue};\6
\4\&{case} \.{'P'}:\5
${}\\{print\_base}(\T{0},\39\T{0}){}$;\5
\&{continue};\C{ \.P means ``print all'' }\6
\4\&{case} \.{'q'}:\5
\&{goto} \\{alldone};\C{ this will exit the program }\6
\4\&{case} \.{'r'}:\5
\X152:Reset the reorder trigger\X;\5
\&{continue};\6
\4\&{case} \.{'s'}:\5
\X130:Swap variable $x_k$ with its predecessor\X;\5
\&{continue};\6
\4\&{case} \.{'S'}:\6
\&{if} ${}(\\{isdigit}({*}\|c)){}$\1\5
\X147:Sift on variable $x_k$\X\2\6
\&{else}\1\5
\\{siftall}(\,);\5
\2\&{continue};\6
\4\&{case} \.{'t'}:\5
\X129:Reset \PB{\\{tvar}}\X;\5
\&{continue};\6
\4\&{case} \.{'v'}:\5
\\{getk};\5
${}\\{verbose}\K\|k{}$;\5
\&{continue};\6
\4\&{case} \.{'V'}:\5
${}\\{verbose}\K{-}\T{1}{}$;\5
\&{continue};\6
\4\&{case} \.{'y'}:\5
\X128:Parse and execute an assignment to $y_k$\X;\5
\&{continue};\6
\4\&{case} \.{'\$'}:\5
\\{show\_stats}(\,);\5
\&{continue};\6
\4\&{default}:\5
\\{reporterror};\6
\4${}\}{}$\2\6
\4\\{nextcommand}:\5
\&{continue};\par
\U110.\fi

\M{115}\B\X19:Local variables\X${}\mathrel+\E{}$\6
\&{char} ${}{*}\|c,{}$ ${}{*}\\{cc}{}$;\C{ characters being scanned }\6
\&{node} ${}{*}\|p,{}$ ${}{*}\|q,{}$ ${}{*}\|r{}$;\C{ operands }\6
\&{var} ${}{*}\|v{}$;\C{ a variable }\6
\&{int} \\{lhs};\C{ index on left side of equation }\6
\&{int} \\{curop};\C{ current operator }\par
\fi

\M{116}The \<special> command \.{include} \<filename> starts up a new infile.
(Instead of \.{include}, you could also say \.{input} or \.i, or
even \.{ignore}.)

\Y\B\4\D$\\{passblanks}$ \5
\&{for} ( ; ${}{*}\|c\E\.{'\ '};{}$ ${}\|c\PP{}$)\par
\Y\B\4\X116:Get ready to read a new input file\X${}\E{}$\6
\&{if} (\\{infile})\1\5
\\{printf}(\.{"Sorry\ ---\ you\ can't}\)\.{\ include\ one\ file\ in}\)\.{side\
of\ another.\\n"});\2\6
\&{else}\5
${}\{{}$\1\6
\&{for} ( ; ${}\\{isgraph}({*}\|c);{}$ ${}\|c\PP){}$\1\5
;\C{ pass nonblanks }\2\6
\\{passblanks};\6
\&{for} ${}(\\{cc}\K\|c;{}$ ${}\\{isgraph}({*}\|c);{}$ ${}\|c\PP){}$\1\5
;\C{ pass nonblanks }\2\6
${}{*}\|c\K\.{'\\0'};{}$\6
\&{if} ${}(\R(\\{infile}\K\\{fopen}(\\{cc},\39\.{"r"}))){}$\1\5
${}\\{printf}(\.{"Sorry\ ---\ I\ couldn'}\)\.{t\ open\ file\ `\%s'!\\n"},\39%
\\{cc});{}$\2\6
\4${}\}{}$\2\par
\U114.\fi

\M{117}The command `\.{p3}' prints out the BDD for $f_3$; the command
`\.{pp3}' prints just the profile.

\Y\B\4\D$\\{getkf}$ \5
\\{getk};\5
\&{if} ${}(\|k\G\\{extsize}){}$\5
${}\{{}$\5
\1${}\\{printf}(\.{"f\%d\ is\ out\ of\ range}\)\.{.\\n"},\39\|k){}$;\5
\&{continue};\5
${}\}{}$\2\par
\B\4\D$\\{getkv}$ \5
\\{getk};\5
\&{if} ${}(\|k\G\\{varsize}){}$\5
${}\{{}$\5
\1${}\\{printf}(\.{"x\%d\ is\ out\ of\ range}\)\.{.\\n"},\39\|k){}$;\5
\&{continue};\5
${}\}{}$\2\par
\Y\B\4\X117:Print a function or its profile\X${}\E{}$\6
\&{if} ${}({*}\|c\E\.{'p'}){}$\5
${}\{{}$\C{ \.{pp} means ``print a profile'' }\1\6
${}\|c\PP{}$;\5
\\{getkf};\6
${}\\{printf}(\.{"p\%d:"},\39\|k);{}$\6
\\{print\_profile}(\|f[\|k]);\6
\4${}\}{}$\5
\2\&{else}\5
${}\{{}$\1\6
\\{getkf};\6
${}\\{printf}(\.{"f\%d="},\39\|k);{}$\6
${}\\{print\_function}(\|f[\|k],\39\T{0});{}$\6
\4${}\}{}$\2\par
\U114.\fi

\M{118}The command `\.{o3}' outputs the BDD for $f_3$; the command
`\.{ou3}' outputs it with variable names ``unmapped'',
so that all branches go to variables with higher numbers (in spite of any
reordering that has been done). Unmapped
BDD output is important for programs such as {\mc BDDREAD-COUNT}, which
count the number of solutions, because those programs need to know
how many levels are being crossed at every downward branch.

\Y\B\4\X118:Output a function\X${}\E{}$\6
${}\{{}$\1\6
\&{int} \\{unmapped}${}\K\T{0};{}$\7
\&{if} ${}({*}\|c\E\.{'u'}){}$\1\5
${}\|c\PP,\39\\{unmapped}\K\T{1}{}$;\C{ \.{ou} means ``output unmapped'' }\2\6
\\{getkf};\6
${}\\{sprintf}(\\{buf},\39\.{"/tmp/f\%d.bdd"},\39\|k);{}$\6
${}\\{freopen}(\\{buf},\39\.{"w"},\39\\{stdout}){}$;\C{ redirect \PB{%
\\{stdout}} to a file }\6
${}\\{print\_function}(\|f[\|k],\39\\{unmapped});{}$\6
${}\\{freopen}(\.{"/dev/tty"},\39\.{"w"},\39\\{stdout}){}$;\C{ restore normal %
\PB{\\{stdout}} }\6
\4${}\}{}$\2\par
\U114.\fi

\M{119}\B\X119:Print the current variable ordering\X${}\E{}$\6
\&{for} ${}(\|v\K\\{varhead};{}$ ${}\|v<\\{topofvars};{}$ ${}\|v\PP){}$\1\6
\&{if} ${}(\|v\MG\\{proj}){}$\1\5
${}\\{printf}(\.{"\ x\%d"},\39\|v\MG\\{name});{}$\2\2\6
\\{printf}(\.{"\\n"});\par
\U114.\fi

\M{120}My little finite-state automaton.

\Y\B\4\X120:Parse and execute an assignment to $f_k$\X${}\E{}$\6
\\{getkf};\5
${}\\{lhs}\K\|k;{}$\6
\\{passblanks};\6
\&{if} ${}({*}\|c\PP\I\.{'='}){}$\1\5
\\{reporterror};\2\6
\X121:Get the first operand, \PB{\|p}\X;\6
\X122:Get the operator, \PB{\\{curop}}\X;\6
\4\\{second}:\5
\X123:Get the second operand, \PB{\|q}\X;\6
\4\\{third}:\5
\X124:If the operator is ternary, get the third operand, \PB{\|r}\X;\6
\4\\{fourth}:\5
\X125:Evaluate the right-hand side and put the answer in \PB{\|r}\X;\6
\X126:Assign \PB{\|r} to $f_k$, where \PB{$\|k\K\\{lhs}$}\X;\par
\U114.\fi

\M{121}\B\D$\\{checknull}(\|p)$ \6
\&{if} ${}(\R\|p){}$\5
${}\{{}$\5
\1${}\\{printf}(\.{"f\%d\ is\ null!\\n"},\39\|k){}$;\5
\&{continue};\5
${}\}{}$\2\par
\Y\B\4\X121:Get the first operand, \PB{\|p}\X${}\E{}$\6
\\{passblanks};\6
\&{switch} ${}({*}\|c\PP){}$\5
${}\{{}$\1\6
\4\&{case} \.{'x'}:\5
\\{getkv};\5
${}\|p\K\\{projection}(\\{varmap}[\|k]){}$;\5
\&{break};\6
\4\&{case} \.{'f'}:\5
\\{getkf};\5
${}\|p\K\|f[\|k]{}$;\5
\\{checknull}(\|p);\5
\&{break};\6
\4\&{case} \.{'c'}:\5
${}\|k\K{*}\|c\PP-\.{'0'}{}$;\5
\&{if} ${}((\|k\AND{-}\T{2})\E\T{0}){}$\1\5
${}\|p\K\\{botsink}+\|k{}$;\5
\2\&{else}\1\5
\\{reporterror};\5
\2\&{break};\6
\4\&{case} \.{'\~'}:\5
${}\|p\K\\{topsink}{}$;\5
${}\\{curop}\K\T{6}{}$;\5
\&{goto} \\{second};\C{ reduce $\lnot f$ to $1\oplus f$ }\6
\4\&{case} \.{'.'}:\5
\X127:Dereference the left-hand side\X;\5
\&{continue};\6
\4\&{default}:\5
\\{reporterror};\6
\4${}\}{}$\2\par
\U120.\fi

\M{122}Quantification uses the following conventions:
\smallskip\noindent{$\bullet$}
A command like `\.{f1=f2 A x2}' sets $f_1\gets\forall x_2\,f_2$.
\smallskip\noindent{$\bullet$}
A command like `\.{f1=f2 E f3}' where $f_3=x_4\land x_5$ sets
$f_1\gets\exists x_4\,\exists x_5\,f_2$.
\smallskip\noindent{$\bullet$}
A command like `\.{f1=f2\tta f4 E f3}' with that $f_3$ sets
$f_1\gets\exists x_4\,\exists x_5\,(f_2\land f_4)$.

\Y\B\4\X122:Get the operator, \PB{\\{curop}}\X${}\E{}$\6
\\{passblanks};\6
\&{switch} ${}({*}\|c\PP){}$\5
${}\{{}$\1\6
\4\&{case} \.{'\&'}:\5
${}\\{curop}\K\T{1}{}$;\5
\&{break};\C{ and }\6
\4\&{case} \.{'>'}:\5
${}\\{curop}\K\T{2}{}$;\5
\&{break};\C{ butnot }\6
\4\&{case} \.{'<'}:\5
${}\\{curop}\K\T{4}{}$;\5
\&{break};\C{ notbut }\6
\4\&{case} \.{'\^'}:\5
${}\\{curop}\K\T{6}{}$;\5
\&{break};\C{ xor }\6
\4\&{case} \.{'|'}:\5
${}\\{curop}\K\T{7}{}$;\5
\&{break};\C{ or }\6
\4\&{case} \.{'\_'}:\5
${}\\{curop}\K\T{8}{}$;\5
\&{break};\C{ constrain }\6
\4\&{case} \.{'A'}:\5
${}\\{curop}\K\T{9}{}$;\5
\&{break};\C{ forall }\6
\4\&{case} \.{'N'}:\5
${}\\{curop}\K\T{10}{}$;\5
\&{break};\C{ no }\6
\4\&{case} \.{'Y'}:\5
${}\\{curop}\K\T{12}{}$;\5
\&{break};\C{ yes }\6
\4\&{case} \.{'D'}:\5
${}\\{curop}\K\T{14}{}$;\5
\&{break};\C{ diff }\6
\4\&{case} \.{'E'}:\5
${}\\{curop}\K\T{15}{}$;\5
\&{break};\C{ exists }\6
\4\&{case} \.{'?'}:\5
${}\\{curop}\K\T{16}{}$;\5
\&{break};\C{ if-then-else }\6
\4\&{case} \.{'.'}:\5
${}\\{curop}\K\T{17}{}$;\5
\&{break};\C{ median }\6
\4\&{case} \.{'['}:\5
${}\\{curop}\K\T{0}{}$;\C{ functional composition }\6
\&{if} ${}({*}\|c\PP\I\.{'y'}){}$\1\5
\\{reporterror};\2\6
\&{if} ${}({*}\|c\PP\I\.{']'}){}$\1\5
\\{reporterror};\2\6
\&{goto} \\{fourth};\6
\4\&{case} \.{'\\n'}:\5
${}\\{curop}\K\T{7},\39\|q\K\|p,\39\|c\MM{}$;\5
\&{goto} \\{fourth};\C{ change unary \PB{\|p} to $p\lor p$ }\6
\4\&{default}:\5
\\{reporterror};\6
\4${}\}{}$\2\par
\U120.\fi

\M{123}\B\X123:Get the second operand, \PB{\|q}\X${}\E{}$\6
\\{passblanks};\6
\&{switch} ${}({*}\|c\PP){}$\5
${}\{{}$\1\6
\4\&{case} \.{'x'}:\5
\\{getkv};\5
${}\|q\K\\{projection}(\\{varmap}[\|k]){}$;\5
\&{break};\6
\4\&{case} \.{'f'}:\5
\\{getkf};\5
${}\|q\K\|f[\|k]{}$;\5
\\{checknull}(\|q);\5
\&{break};\6
\4\&{case} \.{'c'}:\5
${}\|k\K{*}\|c\PP-\.{'0'}{}$;\5
\&{if} ${}((\|k\AND{-}\T{2})\E\T{0}){}$\1\5
${}\|q\K\\{botsink}+\|k{}$;\5
\2\&{else}\1\5
\\{reporterror};\5
\2\&{break};\6
\4\&{default}:\5
\\{reporterror};\6
\4${}\}{}$\2\par
\Us120\ET128.\fi

\M{124}\B\X124:If the operator is ternary, get the third operand, \PB{\|r}\X${}%
\E{}$\6
\\{passblanks};\6
\&{if} ${}(\\{curop}\E\T{1}\W{*}\|c\E\.{'\&'}){}$\1\5
${}\\{curop}\K\T{18}{}$;\C{ and-and }\2\6
\&{if} ${}(\\{curop}\E\T{1}\W{*}\|c\E\.{'E'}){}$\1\5
${}\\{curop}\K\T{19}{}$;\C{ and-exists }\2\6
\&{if} ${}(\\{curop}\Z\\{maxbinop}){}$\1\5
${}\|r\K\NULL;{}$\2\6
\&{else}\5
${}\{{}$\1\6
\&{if} ${}({*}\|c\PP\I\\{ternopname2}[\\{curop}-\T{16}][\T{0}]){}$\1\5
\\{reporterror};\2\6
\\{passblanks};\6
\&{switch} ${}({*}\|c\PP){}$\5
${}\{{}$\1\6
\4\&{case} \.{'x'}:\5
\\{getkv};\5
${}\|r\K\\{projection}(\\{varmap}[\|k]){}$;\5
\&{break};\6
\4\&{case} \.{'f'}:\5
\\{getkf};\5
${}\|r\K\|f[\|k]{}$;\5
\\{checknull}(\|r);\5
\&{break};\6
\4\&{case} \.{'c'}:\5
${}\|k\K{*}\|c\PP-\.{'0'}{}$;\5
\&{if} ${}((\|k\AND{-}\T{2})\E\T{0}){}$\1\5
${}\|r\K\\{botsink}+\|k{}$;\5
\2\&{else}\1\5
\\{reporterror};\5
\2\&{break};\6
\4\&{default}:\5
\\{reporterror};\6
\4${}\}{}$\2\6
\4${}\}{}$\2\par
\U120.\fi

\M{125}We have made sure that all the necessary operands are non-\PB{$\NULL$}.

\Y\B\4\X125:Evaluate the right-hand side and put the answer in \PB{\|r}\X${}%
\E{}$\6
\\{passblanks};\6
\&{if} ${}({*}\|c\I\.{'\\n'}\W{*}\|c\I\.{'\#'}){}$\5
${}\{{}$\C{ comments may follow `\.\#' }\1\6
\4\\{reportjunk}:\5
${}\|c\PP;{}$\6
\\{reporterror};\6
\4${}\}{}$\2\6
\&{if} ${}(\\{curop}\E\T{0}){}$\1\5
${}\|r\K\\{compose\_top}(\|p);{}$\2\6
\&{else} \&{if} ${}(\\{curop}\Z\\{maxbinop}){}$\1\5
${}\|r\K\\{binary\_top}(\\{curop},\39\|p,\39\|q);{}$\2\6
\&{else}\1\5
${}\|r\K\\{ternary\_top}(\\{curop},\39\|p,\39\|q,\39\|r){}$;\2\par
\U120.\fi

\M{126}The \PB{\\{sanity\_check}} routine tells me that
I don't need to increase \PB{$\|r\MG\\{xref}$} here (although I'm not sure that
I
totally understand why).

\Y\B\4\X126:Assign \PB{\|r} to $f_k$, where \PB{$\|k\K\\{lhs}$}\X${}\E{}$\6
\&{if} ${}(\|o,\39\|f[\\{lhs}]){}$\1\5
\\{deref}(\|f[\\{lhs}]);\2\6
${}\|o,\39\|f[\\{lhs}]\K\|r{}$;\par
\U120.\fi

\M{127}\B\X127:Dereference the left-hand side\X${}\E{}$\6
\&{if} ${}(\|o,\39\|f[\\{lhs}]){}$\5
${}\{{}$\1\6
\\{deref}(\|f[\\{lhs}]);\6
${}\|o,\39\|f[\\{lhs}]\K\NULL;{}$\6
\4${}\}{}$\2\par
\U121.\fi

\M{128}\B\X128:Parse and execute an assignment to $y_k$\X${}\E{}$\6
\\{getkv};\5
${}\|v\K{\AND}\\{varhead}[\\{varmap}[\|k]];{}$\6
\&{if} ${}(\|o,\39\R\|v\MG\\{proj}){}$\1\5
\\{projection}(\|k);\C{ ensure that $x_k$ exists }\2\6
\\{passblanks};\6
\&{if} ${}({*}\|c\PP\I\.{'='}){}$\1\5
\\{reporterror};\2\6
\\{passblanks};\6
\&{if} ${}({*}\|c\E\.{'.'}){}$\1\5
${}\|c\PP,\39\|q\K\NULL;{}$\2\6
\&{else}\5
${}\{{}$\1\6
\X123:Get the second operand, \PB{\|q}\X;\6
\&{if} ${}(\|o,\39\|q\E\\{node\_}(\|v\MG\\{proj})){}$\1\5
${}\|q\K\NULL;{}$\2\6
\4${}\}{}$\2\6
\\{passblanks};\6
\&{if} ${}({*}\|c\I\.{'\\n'}\W{*}\|c\I\.{'\#'}){}$\1\5
\&{goto} \\{reportjunk};\2\6
\&{if} ${}(\|o,\39\|v\MG\\{repl}\I\\{addr\_}(\|q)){}$\5
${}\{{}$\1\6
${}\|p\K\\{node\_}(\|v\MG\\{repl});{}$\6
\&{if} (\|p)\1\5
\\{deref}(\|p);\2\6
\X102:Assign \PB{\|q} as the new value of \PB{$\|v\MG\\{repl}$}\X;\6
\4${}\}{}$\2\par
\U114.\fi

\M{129}In a long calculation, it's nice to get progress reports by setting
bit 128 of the \PB{\\{verbose}} switch. But we want to see such reports only
near the top of the BDDs. (Note that \PB{\\{varmap}} is not relevant here.)

\Y\B\4\X129:Reset \PB{\\{tvar}}\X${}\E{}$\6
\\{getkv};\6
${}\\{tvar}\K{\AND}\\{varhead}[\|k+\T{1}]{}$;\par
\U114.\fi

\N{1}{130}Reordering. Now comes the new stuff, where {\mc BDD14} enters the
territory into which {\mc BDD11} was afraid to tread. Everything is
based on a primitive swap-in-place operation, which is made available
to the user as an `\.s' command for online experimentation.

The swap-in-place algorithm interchanges $x_u\leftrightarrow x_v$
in the ordering, where $x_u$ immediately precedes~$x_v$. No new dead nodes are
introduced during this process, although some nodes will disappear
and others will be created. Furthermore, no pointers will change
except within nodes that branch on $x_u$ or~$x_v$; every node on
level $u$ or level~$v$ that is accessible either externally or from above
will therefore continue to represent the same subfunction, but in a
different way.

\Y\B\4\X130:Swap variable $x_k$ with its predecessor\X${}\E{}$\6
\\{getkv};\5
${}\|v\K{\AND}\\{varhead}[\\{varmap}[\|k]];{}$\6
\&{if} ${}(\|o,\39\R\|v\MG\\{proj}){}$\1\5
\\{projection}(\|k);\C{ ensure that $x_k$ exists }\2\6
\\{reorder\_init}(\,);\C{ prepare for reordering }\6
\&{if} ${}(\|v\MG\\{up}){}$\1\5
${}\\{swap}(\|v\MG\\{up},\39\|v);{}$\2\6
\\{reorder\_fin}(\,);\C{ go back to normal processing }\par
\U114.\fi

\M{131}Before we diddle with such a sensitive thing as the order of branching,
we must clear the cache. We also remove all dead nodes, which otherwise
get in the way. Furthermore, we set the \PB{\\{up}} and \PB{\\{down}} links
inside \PB{\&{var}} nodes.

By setting \PB{$\\{leasesonlife}\K\T{1}$} here, I'm taking a rather cowardly
approach
to the problem of memory overflow: This program will simply give up,
when it runs out of elbow room. No doubt there are much better ways
to flail about and possibly recover, when memory gets tight, but I
don't have the time or motivation to think about them today.

\Y\B\4\X7:Subroutines\X${}\mathrel+\E{}$\6
\&{void} \\{reorder\_init}(\&{void})\1\1\2\2\6
${}\{{}$\1\6
\&{var} ${}{*}\|v,{}$ ${}{*}\\{vup};{}$\7
\\{collect\_garbage}(\T{1});\6
${}\\{totalvars}\K\T{0};{}$\6
\&{for} ${}(\|v\K\\{varhead},\39\\{vup}\K\NULL;{}$ ${}\|v<\\{topofvars};{}$ ${}%
\|v\PP){}$\1\6
\&{if} ${}(\|v\MG\\{proj}){}$\5
${}\{{}$\1\6
${}\|v\MG\\{aux}\K\PP\\{totalvars};{}$\6
${}\|v\MG\\{up}\K\\{vup};{}$\6
\&{if} (\\{vup})\1\5
${}\\{vup}\MG\\{down}\K\|v{}$;\5
\2\&{else}\1\5
${}\\{firstvar}\K\|v;{}$\2\6
${}\\{vup}\K\|v;{}$\6
\4${}\}{}$\2\2\6
\&{if} (\\{vup})\1\5
${}\\{vup}\MG\\{down}\K\NULL{}$;\5
\2\&{else}\1\5
${}\\{firstvar}\K\NULL;{}$\2\6
${}\\{oldleases}\K\\{leasesonlife};{}$\6
${}\\{leasesonlife}\K\T{1}{}$;\C{ disallow reservations that fail }\6
\4${}\}{}$\2\7
\&{void} \\{reorder\_fin}(\&{void})\1\1\2\2\6
${}\{{}$\1\6
\\{cache\_init}(\,);\6
${}\\{leasesonlife}\K\\{oldleases};{}$\6
\4${}\}{}$\2\par
\fi

\M{132}\B\X4:Global variables\X${}\mathrel+\E{}$\6
\&{int} \\{totalvars};\C{ this many \PB{\&{var}} records are in use }\6
\&{var} ${}{*}\\{firstvar}{}$;\C{ and this one is the smallest in use }\6
\&{int} \\{oldleases};\C{ this many ``leases on life'' have been held over }\par
\fi

\M{133}We classify the nodes on levels $u$ and $v$ into four categories:
Level-$u$ nodes that branch to at least one level-$v$ node are called
``tangled''; the others are ``solitary.'' Level-$v$ nodes that are
reachable from levels above~$u$ or from external pointers ($f_j$ or
$x_j$ or $y_j$) are called ``remote''; the others, which are reachable
only from level~$u$, are ``hidden.''

After the swap, the tangled nodes will remain on level~$u$; but they
will now branch on the former~$x_v$, and their \PB{\\{lo}} and \PB{\\{hi}}
pointers
will probably change. The solitary nodes will move to
level~$v$, where they will become remote; they'll still branch
on the former~$x_u$ as before.
The remote nodes will move to level~$u$, where they will become
solitary---still branching as before on the former~$x_v$.
The hidden nodes will
disappear and be recycled. In their place we might create ``newbies,''
which are new nodes on level~$v$ that branch on the old~$x_u$.
The newbies are accessible only from tangled nodes that have been
transmogrified; hence they will be
the hidden nodes, if we decide to swap the levels back again immediately.

Notice that if there are $m$ tangled nodes, there are at most $2m$
hidden nodes, and at most $2m$ newbies. The swap is beneficial if and
only if the hidden nodes outnumber the newbies.

Notice also that
the projection function $x_u$ is always solitary; the projection
function $x_v$ is always remote. But the present implementation
is based on the assumptions that almost all nodes on level~$u$
are tangled and almost all nodes on level~$v$ are hidden.
Therefore, instead of retaining solitary and remote nodes in their unique
tables, deleting the other nodes, swapping unique tables, and then inserting
tangled/newbies, we use a different strategy by which both unique tables
are essentially trashed and rebuilt from scratch. (In other words,
we assume that the deletion of
tangled nodes and hidden nodes will cost more than the insertion of
solitary nodes and remote nodes.)

We need some way to form temporary lists of all the solitary, tangled, and
remote nodes. No link fields are readily available in the nodes themselves,
unless we resort to the shadow memory.
The present implementation solves the problem by reconfiguring
the unique table for level~$u$ before destroying it: We move
all solitary nodes to the beginning of that table, and all tangled
nodes to the end. This approach is consistent with our preference for
cache-friendly methods like linear probing.

\Y\B\4\X133:Declare the \PB{\\{swap}} subroutine\X${}\E{}$\6
\&{void} \\{swap}(\&{var} ${}{*}\|u,\39{}$\&{var} ${}{*}\|v){}$\1\1\2\2\6
${}\{{}$\1\6
\&{register} \&{int} \|j${},{}$ \|k${},{}$ \\{solptr}${},{}$ \\{tangptr}${},{}$
\\{umask}${},{}$ \\{vmask}${},{}$ \\{del};\6
\&{register} \&{int} \\{hcount}${}\K\T{0},{}$ \\{rcount}${}\K\T{0},{}$ %
\\{scount}${}\K\T{0},{}$ \\{tcount}${}\K\T{0},{}$ \\{icount}${}\K%
\\{totalnodes};{}$\6
\&{register} \&{node} ${}{*}\|f,{}$ ${}{*}\|g,{}$ ${}{*}\|h,{}$ ${}{*}%
\\{gg},{}$ ${}{*}\\{hh},{}$ ${}{*}\|p,{}$ ${}{*}\\{pl},{}$ ${}{*}\\{ph},{}$
${}{*}\|q,{}$ ${}{*}\\{ql},{}$ ${}{*}\\{qh},{}$ ${}{*}\\{firsthidden},{}$
${}{*}\\{lasthidden};{}$\6
\&{register} \&{var} ${}{*}\\{vg},{}$ ${}{*}\\{vh};{}$\6
\&{unsigned} \&{long} \&{long} \\{omems}${}\K\\{mems},{}$ \\{ozmems}${}\K%
\\{zmems};{}$\7
${}\\{oo},\39\\{umask}\K\|u\MG\\{mask},\39\\{vmask}\K\|v\MG\\{mask};{}$\6
${}\\{del}\K((\|u-\\{varhead})\XOR(\|v-\\{varhead}))\LL(\T{32}-%
\\{logvarsize});{}$\6
\X134:Separate the solitary nodes from the tangled nodes\X;\6
\X135:Create a new unique table for $x_u$ and move the remote nodes to it\X;\6
\&{if} ${}(\\{verbose}\AND\T{2048}){}$\1\5
${}\\{printf}(\.{"swapping\ \%d(x\%d)<->}\)\.{\%d(x\%d):\ solitary\ \%d}\)\.{,\
tangled\ \%d,\ remote}\)\.{\ \%d,\ hidden\ \%d\\n"},\3{-1}\39\|u-\\{varhead},%
\39\|u\MG\\{name},\39\|v-\\{varhead},\39\|v\MG\\{name},\39\\{scount},\39%
\\{tcount},\39\\{rcount},\39\\{hcount});{}$\2\6
\X139:Create a new unique table for $x_v$ and move the solitary nodes to it\X;\6
\X140:Transmogrify the tangled nodes and insert them in their new guise\X;\6
\X143:Delete the lists of solitary, tangled, and hidden nodes\X;\6
\&{if} ${}(\\{verbose}\AND\T{2048}){}$\1\5
${}\\{printf}(\.{"\ newbies\ \%d,\ change}\)\.{\ \%d,\ mems\ (\%llu,0,\%l}\)%
\.{lu)\\n"},\39\\{totalnodes}-\\{icount}+\\{hcount},\39\\{totalnodes}-%
\\{icount},\39\\{mems}-\\{omems},\39\\{zmems}-\\{ozmems});{}$\2\6
\X144:Swap names, projections, and replacement functions\X;\6
\4${}\}{}$\2\par
\U145.\fi

\M{134}Here's a cute algorithm something like the inner loop of quicksort.
By decreasing the reference counts of the tangled nodes' children, we will
be able to distinguish remote nodes from hidden nodes in the next step.

\Y\B\4\X134:Separate the solitary nodes from the tangled nodes\X${}\E{}$\6
$\\{solptr}\K\|j\K\T{0}{}$;\5
${}\\{tangptr}\K\|k\K\\{umask}+\T{1};{}$\6
\&{while} (\T{1})\5
${}\{{}$\1\6
\&{for} ( ; ${}\|j<\|k;{}$ ${}\|j\MRL{+{\K}}\&{sizeof}(\&{addr})){}$\5
${}\{{}$\1\6
${}\\{oo},\39\|p\K\\{fetchnode}(\|u,\39\|j);{}$\6
\&{if} ${}(\|p\E\T{0}){}$\1\5
\&{continue};\2\6
${}\|o,\39\\{pl}\K\\{node\_}(\|p\MG\\{lo}),\39\\{ph}\K\\{node\_}(\|p\MG%
\\{hi});{}$\6
\&{if} ${}((\\{pl}>\\{topsink}\W(\|o,\39\\{thevar}(\\{pl})\E\|v))\V(\\{ph}>%
\\{topsink}\W(\|o,\39\\{thevar}(\\{ph})\E\|v))){}$\5
${}\{{}$\1\6
${}\\{oooo},\39\\{pl}\MG\\{xref}\MM,\39\\{ph}\MG\\{xref}\MM;{}$\6
\&{break};\6
\4${}\}{}$\2\6
${}\\{storenode}(\|u,\39\\{solptr},\39\|p);{}$\6
${}\\{solptr}\MRL{+{\K}}\&{sizeof}(\&{addr}),\39\\{scount}\PP;{}$\6
\4${}\}{}$\2\6
\&{if} ${}(\|j\G\|k){}$\1\5
\&{break};\2\6
\&{for} ${}(\|k\MRL{-{\K}}\&{sizeof}(\&{addr});{}$ ${}\|j<\|k;{}$ ${}\|k\MRL{-{%
\K}}\&{sizeof}(\&{addr})){}$\5
${}\{{}$\1\6
${}\\{oo},\39\|q\K\\{fetchnode}(\|u,\39\|k);{}$\6
\&{if} ${}(\|q\E\T{0}){}$\1\5
\&{continue};\2\6
${}\|o,\39\\{ql}\K\\{node\_}(\|q\MG\\{lo}),\39\\{qh}\K\\{node\_}(\|q\MG%
\\{hi});{}$\6
\&{if} ${}((\\{ql}>\\{topsink}\W(\|o,\39\\{thevar}(\\{ql})\E\|v))\V(\\{qh}>%
\\{topsink}\W(\|o,\39\\{thevar}(\\{qh})\E\|v))){}$\1\5
${}\\{oooo},\39\\{ql}\MG\\{xref}\MM,\39\\{qh}\MG\\{xref}\MM;{}$\2\6
\&{else}\1\5
\&{break};\2\6
${}\\{tangptr}\MRL{-{\K}}\&{sizeof}(\&{addr}),\39\\{tcount}\PP;{}$\6
${}\\{storenode}(\|u,\39\\{tangptr},\39\|q);{}$\6
\4${}\}{}$\2\6
${}\\{tangptr}\MRL{-{\K}}\&{sizeof}(\&{addr}),\39\\{tcount}\PP;{}$\6
${}\\{storenode}(\|u,\39\\{tangptr},\39\|p);{}$\6
\&{if} ${}(\|j\G\|k){}$\1\5
\&{break};\2\6
${}\\{storenode}(\|u,\39\\{solptr},\39\|q);{}$\6
${}\\{solptr}\MRL{+{\K}}\&{sizeof}(\&{addr}),\39\\{scount}\PP;{}$\6
${}\|j\MRL{+{\K}}\&{sizeof}(\&{addr});{}$\6
\4${}\}{}$\2\par
\U133.\fi

\M{135}We temporarily save the pages of the old unique table, since they
now contain the sequential lists of solitary and tangled nodes.

The hidden nodes are linked together by \PB{\\{xref}} fields, but not yet
recycled (because we will want to look at their \PB{\\{lo}} and \PB{\\{hi}}
fields
again).

\Y\B\4\X135:Create a new unique table for $x_u$ and move the remote nodes to it%
\X${}\E{}$\6
\&{for} ${}(\|k\K\T{0};{}$ ${}\|k\Z\\{umask}\GG\\{logpagesize};{}$ ${}\|k%
\PP){}$\1\5
${}\\{oo},\39\\{savebase}[\|k]\K\|u\MG\\{base}[\|k];{}$\2\6
${}\\{new\_unique}(\|u,\39\\{tcount}+\T{1}){}$;\C{ initialize an empty unique
table }\6
\&{for} ${}(\|k\K\\{rcount}\K\\{hcount}\K\T{0};{}$ ${}\|k<\\{vmask};{}$ ${}\|k%
\MRL{+{\K}}\&{sizeof}(\&{addr})){}$\5
${}\{{}$\1\6
${}\\{oo},\39\|p\K\\{fetchnode}(\|v,\39\|k);{}$\6
\&{if} ${}(\|p\E\T{0}){}$\1\5
\&{continue};\2\6
\&{if} ${}(\|o,\39\|p\MG\\{xref}<\T{0}){}$\5
${}\{{}$\C{ \PB{\|p} is a hidden node }\1\6
\&{if} ${}(\\{hcount}\E\T{0}){}$\1\5
${}\\{firsthidden}\K\\{lasthidden}\K\|p,\39\\{hcount}\K\T{1};{}$\2\6
\&{else}\1\5
${}\|o,\39\\{hcount}\PP,\39\|p\MG\\{xref}\K\\{addr\_}(\\{lasthidden}),\39%
\\{lasthidden}\K\|p;{}$\2\6
${}\\{oo},\39\\{node\_}(\|p\MG\\{lo})\MG\\{xref}\MM{}$;\C{ recursive
euthanization won't be needed }\6
${}\\{oo},\39\\{node\_}(\|p\MG\\{hi})\MG\\{xref}\MM{}$;\C{ recursive
euthanization won't be needed }\6
\4${}\}{}$\5
\2\&{else}\5
${}\{{}$\1\6
${}\\{rcount}\PP{}$;\C{ \PB{\|p} is a remote node }\6
${}\\{oo},\39\|p\MG\\{index}\MRL{{\XOR}{\K}}\\{del}{}$;\C{ change the level
from \PB{\|v} to \PB{\|u} }\6
${}\\{insert\_node}(\|u,\39\|p){}$;\C{ put it into the new unique table (see
below) }\6
\4${}\}{}$\2\6
\4${}\}{}$\2\par
\U133.\fi

\M{136}\B\X4:Global variables\X${}\mathrel+\E{}$\6
\&{addr} \\{savebase}[\\{maxhashpages}];\C{ pages to be discarded after
swapping }\par
\fi

\M{137}The \PB{\\{new\_unique}} routine inaugurates an empty unique table with
room for
at least \PB{\|m} nodes before its size will have to double.
Those nodes will be inserted soon, so we don't mind
that it is initially sparse.

\Y\B\4\X7:Subroutines\X${}\mathrel+\E{}$\6
\&{void} \\{new\_unique}(\&{var} ${}{*}\|v,\39{}$\&{int} \|m)\1\1\2\2\6
${}\{{}$\1\6
\&{register} \&{int} \|f${},{}$ \|j${},{}$ \|k;\7
\&{for} ${}(\|f\K\T{6};{}$ ${}(\|m\LL\T{2})>\|f;{}$ ${}\|f\MRL{{\LL}{\K}}%
\T{1}){}$\1\5
;\2\6
${}\|f\K\|f\AND({-}\|f);{}$\6
${}\|o,\39\|v\MG\\{free}\K\|f,\39\|v\MG\\{mask}\K(\|f\LL\T{2})-\T{1};{}$\6
\&{for} ${}(\|k\K\T{0};{}$ ${}\|k\Z\|v\MG\\{mask}\GG\\{logpagesize};{}$ ${}\|k%
\PP){}$\5
${}\{{}$\1\6
${}\|o,\39\|v\MG\\{base}[\|k]\K\\{addr\_}(\\{reserve\_page}(\,)){}$;\C{ it
won't be \PB{$\NULL$} }\6
\&{if} (\|k)\5
${}\{{}$\1\6
\&{for} ${}(\|j\K\|v\MG\\{base}[\|k];{}$ ${}\|j<\|v\MG\\{base}[\|k]+%
\\{pagesize};{}$ ${}\|j\MRL{+{\K}}{}$\&{sizeof}(\&{long} \&{long}))\1\5
\\{storenulls}(\|j);\2\6
${}\\{zmems}\MRL{+{\K}}\\{pagesize}/{}$\&{sizeof}(\&{long} \&{long});\6
\4${}\}{}$\2\6
\4${}\}{}$\2\6
${}\|f\K\|v\MG\\{mask}\AND\\{pagemask};{}$\6
\&{for} ${}(\|j\K\|v\MG\\{base}[\T{0}];{}$ ${}\|j<\|v\MG\\{base}[\T{0}]+\|f;{}$
${}\|j\MRL{+{\K}}{}$\&{sizeof}(\&{long} \&{long}))\1\5
\\{storenulls}(\|j);\2\6
${}\\{zmems}\MRL{+{\K}}(\|f+\T{1})/{}$\&{sizeof}(\&{long} \&{long});\6
\4${}\}{}$\2\par
\fi

\M{138}The \PB{\\{insert\_node}} subroutine is somewhat analogous to \PB{%
\\{unique\_find}}, but its
parameter~\PB{\|q} is a node that's known to
be unique and not already present. The task is simply to insert
this node into the hash table. Complications arise only if the
table thereby becomes too full, and needs to be doubled in size, etc.

\Y\B\4\X7:Subroutines\X${}\mathrel+\E{}$\6
\&{void} \\{insert\_node}(\&{var} ${}{*}\|v,\39{}$\&{node} ${}{*}\|q){}$\1\1\2%
\2\6
${}\{{}$\1\6
\&{register} \&{int} \|j${},{}$ \|k${},{}$ \\{mask}${},{}$ \\{free};\6
\&{register} \&{addr} ${}{*}\\{hash};{}$\6
\&{register} \&{node} ${}{*}\|l,{}$ ${}{*}\|h,{}$ ${}{*}\|p,{}$ ${}{*}\|r;{}$\7
${}\|o,\39\|l\K\\{node\_}(\|q\MG\\{lo}),\39\|h\K\\{node\_}(\|q\MG\\{hi});{}$\6
\4\\{restart}:\5
${}\|o,\39\\{mask}\K\|v\MG\\{mask},\39\\{free}\K\|v\MG\\{free};{}$\6
\&{for} ${}(\\{hash}\K\\{hashcode}(\|l,\39\|h);{}$  ; ${}\\{hash}\PP){}$\5
${}\{{}$\C{ ye olde linear probing }\1\6
${}\|k\K\\{addr\_}(\\{hash})\AND\\{mask};{}$\6
${}\\{oo},\39\|r\K\\{fetchnode}(\|v,\39\|k);{}$\6
\&{if} ${}(\R\|r){}$\1\5
\&{break};\2\6
\4${}\}{}$\2\6
\&{if} ${}(\MM\\{free}\Z\\{mask}\GG\T{4}){}$\1\5
\X32:Double the table size and \PB{\&{goto} \\{restart}}\X;\2\6
${}\\{storenode}(\|v,\39\|k,\39\|q){}$;\5
${}\|o,\39\|v\MG\\{free}\K\\{free};{}$\6
\&{return};\6
\4\\{cramped}:\5
\\{printf}(\.{"Uh\ oh:\ insert\_node\ }\)\.{hasn't\ enough\ memory}\)\.{\ to\
continue!\\n"});\6
\\{show\_stats}(\,);\6
${}\\{exit}({-}\T{96});{}$\6
\4${}\}{}$\2\par
\fi

\M{139}\B\X139:Create a new unique table for $x_v$ and move the solitary nodes
to it\X${}\E{}$\6
\&{for} ${}(\|k\K\T{0};{}$ ${}\|k\Z\\{vmask}\GG\\{logpagesize};{}$ ${}\|k%
\PP){}$\1\5
${}\|o,\39\\{free\_page}(\\{page\_}(\|v\MG\\{base}[\|k]));{}$\2\6
${}\\{new\_unique}(\|v,\39\\{scount});{}$\6
\&{for} ${}(\|k\K\T{0};{}$ ${}\|k<\\{solptr};{}$ ${}\|k\MRL{+{\K}}\&{sizeof}(%
\&{addr})){}$\5
${}\{{}$\1\6
${}\|o,\39\|p\K\\{node\_}(\\{addr\_\_}(\\{savebase}[\|k\GG\\{logpagesize}]+(\|k%
\AND\\{pagemask})));{}$\6
${}\\{oo},\39\|p\MG\\{index}\MRL{{\XOR}{\K}}\\{del}{}$;\C{ change the level
from \PB{\|u} to \PB{\|v} }\6
${}\\{insert\_node}(\|v,\39\|p);{}$\6
\4${}\}{}$\2\par
\U133.\fi

\M{140}The most dramatic change caused by swapping occurs in this step.
Suppose \PB{\|f} is a tangled node on level~$u$ before the swap, and suppose
\PB{$\|g\K\|f\MG\\{lo}$} and \PB{$\|h\K\|f\MG\\{hi}$} are on level~$v$ at that
time. After swapping, we want
\PB{$\|f\MG\\{lo}$} and \PB{$\|f\MG\\{hi}$} to be newbie nodes \PB{\\{gg}} and %
\PB{\\{hh}},
with \PB{$\\{gg}\MG\\{lo}\K\|g\MG\\{lo}$}, \PB{$\\{gg}\MG\\{hi}\K\|h\MG%
\\{lo}$}, \PB{$\\{hh}\MG\\{lo}\K\|g\MG\\{hi}$}, \PB{$\\{hh}\MG\\{hi}\K\|h\MG%
\\{hi}$}.
(Actually, \PB{\\{gg}} and \PB{\\{hh}} might not both be newbies, because
we might have \PB{$\|g\MG\\{lo}\K\|h\MG\\{lo}$} or \PB{$\|g\MG\\{hi}\K\|h\MG%
\\{hi}$}.)
Similar formulas apply when either \PB{\|g} or \PB{\|h} lies below level~$v$.

\Y\B\4\X140:Transmogrify the tangled nodes and insert them in their new guise%
\X${}\E{}$\6
\&{for} ${}(\|k\K\\{tangptr};{}$ ${}\|k<\\{umask};{}$ ${}\|k\MRL{+{\K}}%
\&{sizeof}(\&{addr})){}$\5
${}\{{}$\1\6
${}\|o,\39\|f\K\\{node\_}(\\{addr\_\_}(\\{savebase}[\|k\GG\\{logpagesize}]+(\|k%
\AND\\{pagemask})));{}$\6
${}\|o,\39\|g\K\\{node\_}(\|f\MG\\{lo}),\39\|h\K\\{node\_}(\|f\MG\\{hi});{}$\6
\&{if} ${}(\|g\Z\\{topsink}){}$\1\5
${}\\{vg}\K\\{topofvars}{}$;\5
\2\&{else}\1\5
${}\|o,\39\\{vg}\K\\{thevar}(\|g);{}$\2\6
\&{if} ${}(\|h\Z\\{topsink}){}$\1\5
${}\\{vh}\K\\{topofvars}{}$;\5
\2\&{else}\1\5
${}\|o,\39\\{vh}\K\\{thevar}(\|h){}$;\C{ N.B.: \PB{\\{vg}} and/or \PB{\\{vh}}
might be either \PB{\|u} or \PB{\|v} at this point }\2\6
${}\\{gg}\K\\{swap\_find}(\|v,\39\\{vg}>\|v\?\|g:(\|o,\39\\{node\_}(\|g\MG%
\\{lo})),\39\\{vh}>\|v\?\|h:(\|o,\39\\{node\_}(\|h\MG\\{lo})));{}$\6
${}\\{hh}\K\\{swap\_find}(\|v,\39\\{vg}>\|v\?\|g:\\{node\_}(\|g\MG\\{hi}),\39%
\\{vh}>\|v\?\|h:\\{node\_}(\|h\MG\\{hi}));{}$\6
${}\|o,\39\|f\MG\\{lo}\K\\{addr\_}(\\{gg}),\39\|f\MG\\{hi}\K\\{addr\_}(%
\\{hh}){}$;\C{ \PB{$(\|u,\\{gg},\\{hh})$} will be unique }\6
${}\\{insert\_node}(\|u,\39\|f);{}$\6
\4${}\}{}$\2\par
\U133.\fi

\M{141}The \PB{\\{swap\_find}} procedure in the transmogrification step is
almost identical to \PB{\\{unique\_find}}; it differs only in the treatment
of reference counts (and the knowledge that no nodes are currently dead).

\Y\B\4\X7:Subroutines\X${}\mathrel+\E{}$\6
\&{node} ${}{*}{}$\\{swap\_find}(\&{var} ${}{*}\|v,\39{}$\&{node} ${}{*}\|l,%
\39{}$\&{node} ${}{*}\|h){}$\1\1\2\2\6
${}\{{}$\1\6
\&{register} \&{int} \|j${},{}$ \|k${},{}$ \\{mask}${},{}$ \\{free};\6
\&{register} \&{addr} ${}{*}\\{hash};{}$\6
\&{register} \&{node} ${}{*}\|p,{}$ ${}{*}\|r;{}$\7
\&{if} ${}(\|l\E\|h){}$\5
${}\{{}$\C{ easy case }\1\6
\&{return} \\{oo}${},\39\|l\MG\\{xref}\PP,\39\|l;{}$\6
\4${}\}{}$\2\6
\4\\{restart}:\5
${}\|o,\39\\{mask}\K\|v\MG\\{mask},\39\\{free}\K\|v\MG\\{free};{}$\6
\&{for} ${}(\\{hash}\K\\{hashcode}(\|l,\39\|h);{}$  ; ${}\\{hash}\PP){}$\5
${}\{{}$\C{ ye olde linear probing }\1\6
${}\|k\K\\{addr\_}(\\{hash})\AND\\{mask};{}$\6
${}\\{oo},\39\|p\K\\{fetchnode}(\|v,\39\|k);{}$\6
\&{if} ${}(\R\|p){}$\1\5
\&{goto} \\{newnode};\2\6
\&{if} ${}(\\{node\_}(\|p\MG\\{lo})\E\|l\W\\{node\_}(\|p\MG\\{hi})\E\|h){}$\1\5
\&{break};\2\6
\4${}\}{}$\2\6
\&{return} \|o${},\39\|p\MG\\{xref}\PP,\39\|p;{}$\6
\4\\{newnode}:\5
\X142:Create a newbie and return it\X;\6
\4${}\}{}$\2\par
\fi

\M{142}\B\X142:Create a newbie and return it\X${}\E{}$\6
\&{if} ${}(\MM\\{free}\Z\\{mask}\GG\T{4}){}$\1\5
\X32:Double the table size and \PB{\&{goto} \\{restart}}\X;\2\6
${}\|p\K\\{reserve\_node}(\,);{}$\6
${}\\{storenode}(\|v,\39\|k,\39\|p){}$;\5
${}\|o,\39\|v\MG\\{free}\K\\{free};{}$\6
${}\\{initnewnode}(\|p,\39\|v-\\{varhead},\39\|l,\39\|h);{}$\6
${}\\{oooo},\39\|l\MG\\{xref}\PP,\39\|h\MG\\{xref}\PP;{}$\6
\&{return} \|p;\6
\4\\{cramped}:\5
\\{printf}(\.{"Uh\ oh:\ swap\_find\ ha}\)\.{sn't\ enough\ memory\ t}\)\.{o\
continue!\\n"});\6
\\{show\_stats}(\,);\6
${}\\{exit}({-}\T{95}){}$;\par
\U141.\fi

\M{143}\B\X143:Delete the lists of solitary, tangled, and hidden nodes\X${}%
\E{}$\6
\&{for} ${}(\|k\K\T{0};{}$ ${}\|k\Z\\{umask}\GG\\{logpagesize};{}$ ${}\|k%
\PP){}$\1\5
${}\|o,\39\\{free\_page}(\\{page\_}(\\{savebase}[\|k]));{}$\2\6
\&{if} (\\{hcount})\5
${}\{{}$\1\6
${}\|o,\39\\{firsthidden}\MG\\{xref}\K\\{addr\_}(\\{nodeavail});{}$\6
${}\\{nodeavail}\K\\{lasthidden};{}$\6
${}\\{totalnodes}\MRL{-{\K}}\\{hcount};{}$\6
\4${}\}{}$\2\par
\U133.\fi

\M{144}Several of the following operations are unnecessary overkill. For
example,
instead of interchanging \PB{$\|u\MG\\{proj}$} with \PB{$\|v\MG\\{proj}$}, and
using
\PB{\\{projection}(\\{varmap}[\|k])} to access the projection function for
$x_k$,
I could leave \PB{$\|u\MG\\{proj}$} and \PB{$\|v\MG\\{proj}$} unchanged and
just say
`\PB{\\{projection}(\|k)}'. However, the interaction with replacement functions
and composition gets tricky, so I've decided to play it safe (for a change):
All \PB{\\{repl}} and \PB{\\{proj}} functions are kept internally consistent as
if
no reordering has taken place. The \PB{\\{varmap}} and \PB{\\{name}} tables
provide
an interface between the internal reality and the user's conventions
for numbering the variables.

\Y\B\4\X144:Swap names, projections, and replacement functions\X${}\E{}$\6
$\\{oo},\39\|j\K\|u\MG\\{name},\39\|k\K\|v\MG\\{name};{}$\6
${}\\{oooo},\39\|u\MG\\{name}\K\|k,\39\|v\MG\\{name}\K\|j,\39\\{varmap}[\|j]\K%
\|v-\\{varhead},\39\\{varmap}[\|k]\K\|u-\\{varhead};{}$\6
${}\\{oo},\39\|j\K\|u\MG\\{aux},\39\|k\K\|v\MG\\{aux};{}$\6
\&{if} ${}(\|j*\|k<\T{0}){}$\1\5
${}\\{oo},\39\|u\MG\\{aux}\K{-}\|j,\39\|v\MG\\{aux}\K{-}\|k{}$;\C{ sign of \PB{%
\\{aux}} stays with \PB{\\{name}} }\2\6
${}\|o,\39\|j\K\|u\MG\\{proj},\39\|k\K\|u\MG\\{repl};{}$\6
${}\\{oo},\39\|u\MG\\{proj}\K\|v\MG\\{proj},\39\|u\MG\\{repl}\K\|v\MG%
\\{repl};{}$\6
${}\|o,\39\|v\MG\\{proj}\K\|j,\39\|v\MG\\{repl}\K\|k;{}$\6
\&{if} ${}(\|v\MG\\{repl}\I\T{0}\W(\|o,\39\|v\MG\\{timestamp}\E\T{0})){}$\5
${}\{{}$\1\6
\&{for} ${}(\|j\K\|v-\\{varhead};{}$ ${}\|j>\|u-\\{varhead};{}$ ${}\|j\MM){}$\1%
\5
${}\|o,\39\\{varhead}[\|j].\\{timestamp}\K\T{1};{}$\2\6
\4${}\}{}$\5
\2\&{else} \&{if} ${}(\|v\MG\\{repl}\E\T{0}\W(\|o,\39\|v\MG\\{timestamp}\I%
\T{0})\W(\|v+\T{1}\E\\{topofvars}\V(\|o,\39(\|v+\T{1})\MG\\{timestamp}\E%
\T{0}))){}$\5
${}\{{}$\1\6
\&{for} ${}(\|j\K\|v-\\{varhead};{}$ ${}\|j>\|u-\\{varhead};{}$ ${}\|j\MM){}$\1%
\5
${}\|o,\39\\{varhead}[\|j].\\{timestamp}\K\T{0};{}$\2\6
\4${}\}{}$\2\par
\U133.\fi

\M{145}The \PB{\\{swap}} subroutine is now complete. I can safely declare it,
since its sub-subroutines have already been declared.

\Y\B\4\X7:Subroutines\X${}\mathrel+\E{}$\6
\X133:Declare the \PB{\\{swap}} subroutine\X\par
\fi

\M{146}\B\X146:Bubble sort to reestablish the natural variable order\X${}\E{}$\6
\&{if} (\\{totalvars})\5
${}\{{}$\1\6
\\{reorder\_init}(\,);\C{ prepare for reordering }\6
\&{for} ${}(\|o,\39\|v\K\\{firstvar}\MG\\{down};{}$ \|v; \,)\5
${}\{{}$\1\6
\&{if} ${}(\\{ooo},\39\|v\MG\\{name}>\|v\MG\\{up}\MG\\{name}){}$\1\5
${}\|o,\39\|v\K\|v\MG\\{down};{}$\2\6
\&{else}\5
${}\{{}$\1\6
${}\\{swap}(\|v\MG\\{up},\39\|v);{}$\6
\&{if} ${}(\|v\MG\\{up}\MG\\{up}){}$\1\5
${}\|v\K\|v\MG\\{up};{}$\2\6
\&{else}\1\5
${}\|o,\39\|v\K\|v\MG\\{down};{}$\2\6
\4${}\}{}$\2\6
\4${}\}{}$\2\6
\\{reorder\_fin}(\,);\C{ go back to normal processing }\6
\4${}\}{}$\2\par
\U114.\fi

\M{147}Now we come to the \PB{\\{sift}} routine, which finds the best position
for a given variable when the relative positions of the others
are left unchanged.

\Y\B\4\X147:Sift on variable $x_k$\X${}\E{}$\6
${}\{{}$\1\6
\\{getkv};\5
${}\|v\K{\AND}\\{varhead}[\\{varmap}[\|k]];{}$\6
\&{if} ${}(\|o,\39\|v\MG\\{proj}){}$\5
${}\{{}$\1\6
\\{reorder\_init}(\,);\C{ prepare for reordering }\6
\\{sift}(\|v);\6
\\{reorder\_fin}(\,);\C{ go back to normal processing }\6
\4${}\}{}$\2\6
\4${}\}{}$\2\par
\U114.\fi

\M{148}At this point \PB{$\|v\MG\\{aux}$} is the position of \PB{\|v} among all
active variables. Thus \PB{$\|v\MG\\{aux}\K\T{1}$} if and only if \PB{$\|v\MG%
\\{up}\K\NULL$}
if and only if \PB{$\|v\K\\{firstvar}$}; \PB{$\|v\MG\\{aux}\K\\{totalvars}$} if
and only if
\PB{$\|v\MG\\{down}\K\NULL$}.

\Y\B\4\X7:Subroutines\X${}\mathrel+\E{}$\6
\&{void} \\{sift}(\&{var} ${}{*}\|v){}$\1\1\2\2\6
${}\{{}$\1\6
\&{register} \&{int} \\{pass}${},{}$ \\{bestscore}${},{}$ \\{origscore}${},{}$ %
\\{swaps};\6
\&{var} ${}{*}\|u\K\|v;{}$\6
\&{double} \\{worstratio}${},{}$ \\{saferatio};\6
\&{unsigned} \&{long} \&{long} \\{oldmems}${}\K\\{mems},{}$ \\{oldrmems}${}\K%
\\{rmems},{}$ \\{oldzmems}${}\K\\{zmems};{}$\7
${}\\{bestscore}\K\\{origscore}\K\\{totalnodes};{}$\6
${}\\{worstratio}\K\\{saferatio}\K\T{1.0};{}$\6
${}\\{swaps}\K\\{pass}\K\T{0}{}$;\C{ first we go up or down; then we go down or
up }\6
\&{if} ${}(\|o,\39\\{totalvars}-\|v\MG\\{aux}<\|v\MG\\{aux}){}$\1\5
\&{goto} \\{siftdown};\2\6
\4\\{siftup}:\5
\X149:Explore in the upward direction\X;\6
\4\\{siftdown}:\5
\X150:Explore in the downward direction\X;\6
\4\\{wrapup}:\5
\&{if} ${}(\\{verbose}\AND\T{4096}){}$\1\5
${}\\{printf}(\.{"sift\ x\%d\ (\%d->\%d),\ }\)\.{\%d\ saved,\ \%.3f\ safe,}\)%
\.{\ \%d\ swaps,\ (\%llu,0,\%}\)\.{llu)\ mems\\n"},\39\|u\MG\\{name},\39\|v-%
\\{varhead},\39\|u-\\{varhead},\39\\{origscore}-\\{bestscore},\39\\{saferatio},%
\39\\{swaps},\39\\{mems}-\\{oldmems},\39\\{zmems}-\\{oldzmems});{}$\2\6
${}\\{oo},\39\|u\MG\\{aux}\K{-}\|u\MG\\{aux}{}$;\C{ mark this level as having
been sifted }\6
\4${}\}{}$\2\par
\fi

\M{149}In a production version of this program, I would stop sifting
in a given direction when the ratio \PB{$\\{totalnodes}/\\{bestscore}$} exceeds
some threshold. Here, on the other hand, I'm sifting completely;
but I calculate the \PB{\\{saferatio}} for which a production version
would obtain results just as good as the complete sift.

\Y\B\4\X149:Explore in the upward direction\X${}\E{}$\6
\&{while} ${}(\|o,\39\|u\MG\\{up}){}$\5
${}\{{}$\1\6
${}\\{swaps}\PP,\39\\{swap}(\|u\MG\\{up},\39\|u);{}$\6
${}\|u\K\|u\MG\\{up};{}$\6
\&{if} ${}(\\{bestscore}>\\{totalnodes}){}$\5
${}\{{}$\C{ we've found an improvement }\1\6
${}\\{bestscore}\K\\{totalnodes};{}$\6
\&{if} ${}(\\{saferatio}<\\{worstratio}){}$\1\5
${}\\{saferatio}\K\\{worstratio};{}$\2\6
${}\\{worstratio}\K\T{1.0};{}$\6
\4${}\}{}$\5
\2\&{else} \&{if} ${}(\\{totalnodes}>\\{worstratio}*\\{bestscore}){}$\1\5
${}\\{worstratio}\K{}$(\&{double}) \\{totalnodes}${}/\\{bestscore};{}$\2\6
\4${}\}{}$\2\6
\&{if} ${}(\\{pass}\E\T{0}){}$\5
${}\{{}$\C{ we want to go back to the starting point, then down }\1\6
\&{while} ${}(\|u\I\|v){}$\5
${}\{{}$\1\6
${}\|o,\39\\{swaps}\PP,\39\\{swap}(\|u,\39\|u\MG\\{down});{}$\6
${}\|u\K\|u\MG\\{down};{}$\6
\4${}\}{}$\2\6
${}\\{pass}\K\T{1},\39\\{worstratio}\K\T{1.0};{}$\6
\&{goto} \\{siftdown};\6
\4${}\}{}$\2\6
\&{while} ${}(\\{totalnodes}\I\\{bestscore}){}$\5
${}\{{}$\C{ we want to go back to an optimum level }\1\6
${}\|o,\39\\{swaps}\PP,\39\\{swap}(\|u,\39\|u\MG\\{down});{}$\6
${}\|u\K\|u\MG\\{down};{}$\6
\4${}\}{}$\2\6
\&{goto} \\{wrapup};\par
\U148.\fi

\M{150}\B\X150:Explore in the downward direction\X${}\E{}$\6
\&{while} ${}(\|o,\39\|u\MG\\{down}){}$\5
${}\{{}$\1\6
${}\\{swaps}\PP,\39\\{swap}(\|u,\39\|u\MG\\{down});{}$\6
${}\|u\K\|u\MG\\{down};{}$\6
\&{if} ${}(\\{bestscore}>\\{totalnodes}){}$\5
${}\{{}$\C{ we've found an improvement }\1\6
${}\\{bestscore}\K\\{totalnodes};{}$\6
\&{if} ${}(\\{saferatio}<\\{worstratio}){}$\1\5
${}\\{saferatio}\K\\{worstratio};{}$\2\6
${}\\{worstratio}\K\T{1.0};{}$\6
\4${}\}{}$\5
\2\&{else} \&{if} ${}(\\{totalnodes}>\\{worstratio}*\\{bestscore}){}$\1\5
${}\\{worstratio}\K{}$(\&{double}) \\{totalnodes}${}/\\{bestscore};{}$\2\6
\4${}\}{}$\2\6
\&{if} ${}(\\{pass}\E\T{0}){}$\5
${}\{{}$\C{ we want to go back to the starting point, then up }\1\6
\&{while} ${}(\|u\I\|v){}$\5
${}\{{}$\1\6
${}\|o,\39\\{swaps}\PP,\39\\{swap}(\|u\MG\\{up},\39\|u);{}$\6
${}\|u\K\|u\MG\\{up};{}$\6
\4${}\}{}$\2\6
${}\\{pass}\K\T{1},\39\\{worstratio}\K\T{1.0};{}$\6
\&{goto} \\{siftup};\6
\4${}\}{}$\2\6
\&{while} ${}(\\{totalnodes}\I\\{bestscore}){}$\5
${}\{{}$\C{ we want to go back to an optimum level }\1\6
${}\|o,\39\\{swaps}\PP,\39\\{swap}(\|u\MG\\{up},\39\|u);{}$\6
${}\|u\K\|u\MG\\{up};{}$\6
\4${}\}{}$\2\6
\&{goto} \\{wrapup};\par
\U148.\fi

\M{151}The \PB{\\{siftall}} subroutine sifts until every variable has found
a local sweet spot. This is as good as it gets, unless the user elects
to sift some more.

The order of sifting obviously affects the results. We could, for instance,
sift first on a variable whose level has the most nodes. But Rudell tells me
that nobody has found an ordering strategy that really stands out
and outperforms the others. (He says, ``It's a wash.'') So I've adopted
the first ordering that I thought of.

\Y\B\4\X7:Subroutines\X${}\mathrel+\E{}$\6
\&{void} \\{siftall}(\&{void})\1\1\2\2\6
${}\{{}$\1\6
\&{register} \&{var} ${}{*}\|v;{}$\7
\\{reorder\_init}(\,);\6
\&{for} ${}(\|v\K\\{firstvar};{}$ \|v; \,)\5
${}\{{}$\1\6
\&{if} ${}(\|o,\39\|v\MG\\{aux}<\T{0}){}$\5
${}\{{}$\C{ we've already sifted this guy }\1\6
${}\|o,\39\|v\K\|v\MG\\{down};{}$\6
\&{continue};\6
\4${}\}{}$\2\6
\\{sift}(\|v);\6
\4${}\}{}$\2\6
\\{reorder\_fin}(\,);\6
\4${}\}{}$\2\par
\fi

\M{152}Sifting is invoked automatically when the number of nodes is
\PB{\\{toobig}} or more. By default, the \PB{\\{toobig}} threshold is
essentially
infinite, hence autosifting is disabled. But if a trigger of~$k$
is set, we'll set \PB{\\{toobig}} to $k/100$ times the current size,
and then to $k/100$ times the size after an autosift.

\Y\B\4\X152:Reset the reorder trigger\X${}\E{}$\6
\\{getk};\6
${}\\{trigger}\K\|k/\T{100.0};{}$\6
\&{if} ${}(\\{trigger}*\\{totalnodes}\G\\{memsize}){}$\1\5
${}\\{toobig}\K\\{memsize};{}$\2\6
\&{else}\1\5
${}\\{toobig}\K\\{trigger}*\\{totalnodes}{}$;\2\par
\U114.\fi

\M{153}\B\X153:Invoke autosifting\X${}\E{}$\6
${}\{{}$\1\6
\&{if} ${}(\\{verbose}\AND(\T{4096}+\T{8192})){}$\1\5
${}\\{printf}(\.{"autosifting\ (totaln}\)\.{odes=\%d,\ trigger=\%.2}\)\.{f,\
toobig=\%d)\\n"},\39\\{totalnodes},\39\\{trigger},\39\\{toobig});{}$\2\6
\\{siftall}(\,);\C{ hopefully \PB{\\{totalnodes}} will decrease }\6
\&{if} ${}(\\{trigger}*\\{totalnodes}\G\\{memsize}){}$\1\5
${}\\{toobig}\K\\{memsize};{}$\2\6
\&{else}\1\5
${}\\{toobig}\K\\{trigger}*\\{totalnodes};{}$\2\6
\4${}\}{}$\2\par
\U111.\fi

\M{154}\B\X4:Global variables\X${}\mathrel+\E{}$\6
\&{double} \\{trigger};\C{ multiplier that governs automatic sifting }\6
\&{int} \\{toobig}${}\K\\{memsize}{}$;\C{ threshold for automatic sifting
(initially disabled) }\par
\fi

\M{155}I should mention a surprising feature of {\mc BDD14} that is not a bug:
Sometimes a sifting operation can actually {\it increase\/} the size of a
function, even when only one $f_k$ is defined!

For example, consider $(x_1{?}\ x_3\land x_4{:}\
x_2{?}\ x_3{:}\ x_4)\land x_5$. The profile of this function is
$(1,1,2,1,1,2)$; and after swapping $x_4\leftrightarrow x_5$ it is
$(1,1,2,2,1,2)$. But {\mc BDD14} does {\it not\/} consider this to be
a change in the total number of nodes, because all of the projection
functions are also implicitly present. When we consider the
projection functions $x_1$, $x_2$, $x_3$, $x_4$, $x_5$ in addition
to the stated function, the profile is $(2,2,3,2,1,2)$ both
before and after swapping.

\fi

\N{1}{156}Triage and housekeeping.
Hmmm; we can't postpone the dirty work any longer. In emergency situations,
garbage collection is a necessity. And occasionally, as a BDD base grows,
garbage collection is a nicety, to keep our house in order.

The \PB{\\{collect\_garbage}} routine frees up all of the nodes that are
currently dead. Before it can do this, all references to those nodes
must be eliminated, from the cache and from the unique tables.
When the \PB{\\{level}} parameter is nonzero, the cache is in fact
entirely cleared.

\Y\B\4\X7:Subroutines\X${}\mathrel+\E{}$\6
\&{void} \\{collect\_garbage}(\&{int} \\{level})\1\1\2\2\6
${}\{{}$\1\6
\&{register} \&{int} \|k;\6
\&{var} ${}{*}\|v;{}$\7
${}\\{last\_ditch}\K\T{0}{}$;\C{ see below }\6
\&{if} ${}(\R\\{level}){}$\1\5
\\{cache\_purge}(\,);\2\6
\&{else}\5
${}\{{}$\1\6
\&{if} ${}(\\{verbose}\AND\T{512}){}$\1\5
\\{printf}(\.{"clearing\ the\ cache\\}\)\.{n"});\2\6
\&{for} ${}(\|k\K\T{0};{}$ ${}\|k<\\{cachepages};{}$ ${}\|k\PP){}$\1\5
\\{free\_page}(\\{page\_}(\\{cachepage}[\|k]));\2\6
${}\\{cachepages}\K\T{0};{}$\6
\X103:Clear out the time stamps\X;\6
\4${}\}{}$\2\6
\&{if} ${}(\\{verbose}\AND\T{512}){}$\1\5
${}\\{printf}(\.{"collecting\ garbage\ }\)\.{(\%d/\%d)\\n"},\39\\{deadnodes},%
\39\\{totalnodes});{}$\2\6
\&{for} ${}(\|v\K\\{varhead};{}$ ${}\|v<\\{topofvars};{}$ ${}\|v\PP){}$\1\5
\\{table\_purge}(\|v);\2\6
\4${}\}{}$\2\par
\fi

\M{157}The global variable \PB{\\{last\_ditch}} is set nonzero when we resort
to
garbage collection without a guarantee of gaining at least
\PB{$\\{totalnodes}/\\{deadfraction}$} free nodes in the process.
If a last-ditch attempt fails, there's little likelihood that
we'll get much further by eking out only a few more nodes each time;
so we give up in that case.

\fi

\M{158}\B\X4:Global variables\X${}\mathrel+\E{}$\6
\&{int} \\{last\_ditch};\C{ are we backed up against the wall? }\par
\fi

\M{159}\B\X7:Subroutines\X${}\mathrel+\E{}$\6
\&{void} \\{attempt\_repairs}(\&{void})\1\1\2\2\6
${}\{{}$\1\6
\&{register} \&{int} \|j${},{}$ \|k;\7
\&{if} (\\{last\_ditch})\5
${}\{{}$\1\6
\\{printf}(\.{"sorry\ ---\ there's\ n}\)\.{ot\ enough\ memory;\ we}\)\.{\ have\
to\ quit!\\n"});\6
\X6:Print statistics about this run\X;\6
${}\\{exit}({-}\T{99}){}$;\C{ we're outta here }\6
\4${}\}{}$\2\6
\&{if} ${}(\\{verbose}\AND\T{512}){}$\1\5
\\{printf}(\.{"(making\ a\ last\ ditc}\)\.{h\ attempt\ for\ space)}\)\.{\\n"});%
\2\6
\\{collect\_garbage}(\T{1});\C{ grab all the remaining space }\6
\\{cache\_init}(\,);\C{ initialize a bare-bones cache }\6
${}\\{last\_ditch}\K\T{1}{}$;\C{ and try one last(?) time }\6
\4${}\}{}$\2\par
\fi

\N{1}{160}Mathematica output. An afterthought: It's easy to output a
(possibly huge) file from which Mathematica will compute the
generating function.

\Y\B\4\X160:Print a Mathematica program for a generating function\X${}\E{}$\6
\\{getkf};\6
\\{math\_print}(\|f[\|k]);\6
${}\\{fprintf}(\\{stderr},\39\.{"(generating\ functio}\)\.{n\ for\ f\%d\
written\ to}\)\.{\ \%s)\\n"},\39\|k,\39\\{buf}){}$;\par
\U114.\fi

\M{161}\B\X4:Global variables\X${}\mathrel+\E{}$\6
\&{FILE} ${}{*}\\{outfile};{}$\6
\&{int} \\{outcount};\C{ the number of files output so far }\par
\fi

\M{162}\B\X7:Subroutines\X${}\mathrel+\E{}$\6
\&{void} \\{math\_print}(\&{node} ${}{*}\|p){}$\1\1\2\2\6
${}\{{}$\1\6
\&{var} ${}{*}\|v;{}$\6
\&{int} \|k${},{}$ \|s${},{}$ \\{ss}${},{}$ \|t;\6
\&{node} ${}{*}\|q,{}$ ${}{*}\|r;{}$\7
\&{if} ${}(\R\|p){}$\1\5
\&{return};\2\6
${}\\{outcount}\PP;{}$\6
${}\\{sprintf}(\\{buf},\39\.{"/tmp/bdd14-out\%d.m"},\39\\{outcount});{}$\6
${}\\{outfile}\K\\{fopen}(\\{buf},\39\.{"w"});{}$\6
\&{if} ${}(\R\\{outfile}){}$\5
${}\{{}$\1\6
${}\\{fprintf}(\\{stderr},\39\.{"I\ can't\ open\ file\ \%}\)\.{s\ for\ writing!%
\\n"},\39\\{buf});{}$\6
${}\\{exit}({-}\T{71});{}$\6
\4${}\}{}$\2\6
${}\\{fprintf}(\\{outfile},\39\.{"g0=0\\ng1=1\\n"});{}$\6
\&{if} ${}(\|p>\\{topsink}){}$\5
${}\{{}$\1\6
\\{mark}(\|p);\6
\&{for} ${}(\|s\K\T{0},\39\|v\K\\{topofvars}-\T{1};{}$ ${}\|v\G\\{varhead};{}$
${}\|v\MM){}$\1\6
\&{if} ${}(\|v\MG\\{proj}){}$\1\5
\X163:Generate Mathematica outputs for variable \PB{\|v}\X;\2\2\6
\\{unmark}(\|p);\6
\4${}\}{}$\2\6
${}\\{fprintf}(\\{outfile},\39\.{"g\%x\\n"},\39\\{id}(\|p));{}$\6
\\{fclose}(\\{outfile});\6
\4${}\}{}$\2\par
\fi

\M{163}\B\X163:Generate Mathematica outputs for variable \PB{\|v}\X${}\E{}$\6
${}\{{}$\1\6
${}\|t\K\T{0};{}$\6
\&{for} ${}(\|k\K\T{0};{}$ ${}\|k<\|v\MG\\{mask};{}$ ${}\|k\MRL{+{\K}}%
\&{sizeof}(\&{addr})){}$\5
${}\{{}$\1\6
${}\|q\K\\{fetchnode}(\|v,\39\|k);{}$\6
\&{if} ${}(\|q\W(\|q\MG\\{xref}+\T{1})<\T{0}){}$\5
${}\{{}$\1\6
${}\|t\K\T{1};{}$\6
\X164:Generate a Mathematica line for node \PB{\|q}\X;\6
\4${}\}{}$\2\6
\4${}\}{}$\2\6
\&{if} (\|t)\1\5
${}\|v\MG\\{aux}\K\PP\|s{}$;\C{ this many levels exist below \PB{\|v} }\2\6
\4${}\}{}$\2\par
\U162.\fi

\M{164}\B\X164:Generate a Mathematica line for node \PB{\|q}\X${}\E{}$\6
$\\{fprintf}(\\{outfile},\39\.{"g\%x=Expand["},\39\\{id}(\|q));{}$\6
${}\|r\K\\{node\_}(\|q\MG\\{lo});{}$\6
${}\\{ss}\K(\|r\Z\\{topsink}\?\T{0}:\\{thevar}(\|r)\MG\\{aux});{}$\6
\&{if} ${}(\\{ss}<\|s){}$\5
${}\{{}$\1\6
\&{if} ${}(\|s\E\\{ss}+\T{1}){}$\1\5
${}\\{fprintf}(\\{outfile},\39\.{"(1+z)*"});{}$\2\6
\&{else}\1\5
${}\\{fprintf}(\\{outfile},\39\.{"(1+z)\^\%d*"},\39\|s-\\{ss});{}$\2\6
\4${}\}{}$\2\6
${}\\{fprintf}(\\{outfile},\39\.{"g\%x+z*"},\39\\{id}(\|r));{}$\6
${}\|r\K\\{node\_}(\|q\MG\\{hi});{}$\6
${}\\{ss}\K(\|r\Z\\{topsink}\?\T{0}:\\{thevar}(\|r)\MG\\{aux});{}$\6
\&{if} ${}(\\{ss}<\|s){}$\5
${}\{{}$\1\6
\&{if} ${}(\|s\E\\{ss}+\T{1}){}$\1\5
${}\\{fprintf}(\\{outfile},\39\.{"(1+z)*"});{}$\2\6
\&{else}\1\5
${}\\{fprintf}(\\{outfile},\39\.{"(1+z)\^\%d*"},\39\|s-\\{ss});{}$\2\6
\4${}\}{}$\2\6
${}\\{fprintf}(\\{outfile},\39\.{"g\%x]\\n"},\39\\{id}(\|r)){}$;\par
\U163.\fi

\N{1}{165}Index.
\fi

\inx
\fin
\con
