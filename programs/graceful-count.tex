\input cwebmac

\N{1}{1}Intro. Here's an easy way to calculate the number of graceful labelings
that
have $m$ edges and $n$ nonisolated vertices, for $0\le n\le m+1$,
given~$m>1$. I subdivide into connected and nonconnected graphs.

The idea is to run through all $m$-tuples $(x_1,\ldots,x_m)$ with
$0\le x_j\le m-j$; edge $j$ will go from the vertex labeled $x_j$
to the vertex labeled $x_j+j$.

I consider only the labelings in which $x_{m-1}=1$; in other words,
I assume that edge $m-1$ runs from 1~to~$m$.
(These are in one-to-one correspondence with the labelings for which that
edge runs from 0~to~$m{-}1$.) But I multiply all the answers by~2;
hence the total over all $n$ is exactly~$m!$.

I could go through those $m$-tuples in some sort of Gray code order,
with only one $x_j$ changing at a time. But I'm not trying to
be tricky or extremely efficient. So I simply use reverse colexicographic
order.
That is, for each choice of $(x_{j+1},\ldots,x_m)$, I run through the
possibilities for~$x_j$ from $m-j$ to~0, in decreasing order.

\Y\B\4\D$\\{maxm}$ \5
\T{20}\C{ this is plenty big, because $20!$ is a 61-bit number }\par
\fi

\M{2}I do, however, want to have some fun with data structures.

Every vertex is represented by its label. Vertex~$v$, for $0\le v\le m$,
is isolated if and only if label~$v$ has not been used in any of
the edges. (In particular, vertices 0, 1, and~$m$ are never isolated,
because of the assumption above.)

It's easy to maintain, for each vertex, a linked list of all
its neighbors.
These lists are stacks, since they change in first-in-last-out fashion.

It's also easy to maintain a dynamic union-find structure, because
of the first-in-last-out behavior of this algorithm.

\fi

\M{3}OK, let's get going.

\Y\B\8\#\&{include} \.{<stdio.h>}\6
\8\#\&{include} \.{<stdlib.h>}\6
\&{int} \\{mm};\C{ command-line parameter }\7
\X15:Global variables\X;\7
\\{main}(\&{int} \\{argc}${},\39{}$\&{char} ${}{*}\\{argv}[\,]){}$\1\1\2\2\6
${}\{{}$\1\6
\&{register} \|j${},{}$ \|k${},{}$ \|l${},{}$ \|m;\7
\X4:Process the command line\X;\6
\X7:Initialize to $(m-1,\ldots,2,1,0)$\X;\6
\&{while} (\T{1})\5
${}\{{}$\1\6
\X16:Study the current graph\X;\6
\X5:Move to the next $m$-tuple, or \PB{\&{goto} \\{done}}\X;\6
\4${}\}{}$\2\6
\4\\{done}:\5
\X17:Print the stats\X;\6
\4${}\}{}$\2\par
\fi

\M{4}\B\X4:Process the command line\X${}\E{}$\6
\&{if} ${}(\\{argc}\I\T{2}\V\\{sscanf}(\\{argv}[\T{1}],\39\.{"\%d"},\39{\AND}%
\\{mm})\I\T{1}){}$\5
${}\{{}$\1\6
${}\\{fprintf}(\\{stderr},\39\.{"Usage:\ \%s\ m\\n"},\39\\{argv}[\T{0}]);{}$\6
${}\\{exit}({-}\T{1});{}$\6
\4${}\}{}$\2\6
${}\|m\K\\{mm};{}$\6
\&{if} ${}(\|m<\T{2}\V\|m>\\{maxm}){}$\5
${}\{{}$\1\6
${}\\{fprintf}(\\{stderr},\39\.{"Sorry,\ m\ must\ be\ be}\)\.{tween\ 2\ and\ %
\%d!\\n"},\39\\{maxm});{}$\6
${}\\{exit}({-}\T{2});{}$\6
\4${}\}{}$\2\par
\U3.\fi

\M{5}\B\X5:Move to the next $m$-tuple, or \PB{\&{goto} \\{done}}\X${}\E{}$\6
\&{for} ${}(\|j\K\T{1};{}$ ${}\|x[\|j]\E\T{0};{}$ ${}\|j\PP){}$\5
${}\{{}$\1\6
\X9:Delete the edge from $x[j]$ to $x[j]+j$\X;\6
\4${}\}{}$\2\6
\&{if} ${}(\|j\E\|m-\T{1}){}$\1\5
\&{goto} \\{done};\2\6
\X9:Delete the edge from $x[j]$ to $x[j]+j$\X;\6
${}\|x[\|j]\MM;{}$\6
\X8:Insert an edge from $x[j]$ to $x[j]+j$\X;\6
\&{for} ${}(\|j\MM;{}$ \|j; ${}\|j\MM){}$\5
${}\{{}$\1\6
${}\|x[\|j]\K\|m-\|j;{}$\6
\X8:Insert an edge from $x[j]$ to $x[j]+j$\X;\6
\4${}\}{}$\2\par
\U3.\fi

\N{1}{6}Graceful structures.
An unusual --- indeed, somewhat amazing --- data structure works well
with graceful graphs.

Suppose $v$ has neighbors $w_1$, \dots, $w_t$. Let $f_v(w)=w-v$,
if $w>v$; $f_v(w)=m+v-w$, if $w<v$. Then we set
$\PB{\\{arcs}}[v]=f(w_1)$, or~0 if $t=0$;
$\PB{\\{link}}[f(w_j)]=f(w_{j+1})$ for $1\le j<t$; and
$\PB{\\{link}}[f(w_t)]=0$.

\vskip1pt
(Think about it. If $0<k\le m$, we use \PB{\\{link}[\|k]} only for an arc from
$v$ to $v+k$ for some~$v$. If $m<k\le2m$, we use \PB{\\{link}[\|k]} only for
an arc from $v$ to $v-(k-m)$ for some $v$. In either case at most one
such arc is present. Thus all of the memory for link storage is
preallocated; we don't need a list of available slots.)

\fi

\M{7}We silently use the facts that \PB{\\{arcs}[\|v]} is initially~0 for all~%
\PB{\|v},
and \PB{$\\{active}\K\T{0}$}. But the \PB{\|x} and \PB{\\{link}} arrays needn't
be initialized
(I mean, everything would work fine if they were initially garbage).

\Y\B\4\X7:Initialize to $(m-1,\ldots,2,1,0)$\X${}\E{}$\6
\X11:Initialize the union/find structures\X;\6
\&{for} ${}(\|j\K\|m;{}$ \|j; ${}\|j\MM){}$\5
${}\{{}$\1\6
${}\|x[\|j]\K\|m-\|j;{}$\6
\X8:Insert an edge from $x[j]$ to $x[j]+j$\X;\6
\4${}\}{}$\2\par
\U3.\fi

\M{8}\B\X8:Insert an edge from $x[j]$ to $x[j]+j$\X${}\E{}$\6
${}\{{}$\1\6
\&{register} \&{int} \|p${},{}$ \|u${},{}$ \|v${},{}$ \\{uu}${},{}$ \\{vv};\7
${}\|u\K\|x[\|j];{}$\6
${}\|v\K\|u+\|j;{}$\6
\X12:Do a union operation $u\equiv v$\X;\6
${}\|p\K\\{arcs}[\|u];{}$\6
\&{if} ${}(\R\|p){}$\1\5
${}\\{active}\PP;{}$\2\6
${}\\{link}[\|j]\K\|p,\39\\{arcs}[\|u]\K\|j;{}$\6
${}\|p\K\\{arcs}[\|v];{}$\6
\&{if} ${}(\R\|p){}$\1\5
${}\\{active}\PP;{}$\2\6
${}\\{link}[\|m+\|j]\K\|p,\39\\{arcs}[\|v]\K\|m+\|j;{}$\6
\4${}\}{}$\2\par
\Us5\ET7.\fi

\M{9}\B\X9:Delete the edge from $x[j]$ to $x[j]+j$\X${}\E{}$\6
${}\{{}$\1\6
\&{register} \&{int} \|p${},{}$ \|u${},{}$ \|v${},{}$ \\{uu}${},{}$ \\{vv};\7
${}\|u\K\|x[\|j];{}$\6
${}\|v\K\|u+\|j;{}$\6
${}\|p\K\\{link}[\|m+\|j]{}$;\C{ at this point \PB{$\\{arcs}[\|v]\K\|m+\|j$} }\6
${}\\{arcs}[\|v]\K\|p;{}$\6
\&{if} ${}(\R\|p){}$\1\5
${}\\{active}\MM;{}$\2\6
${}\|p\K\\{link}[\|j]{}$;\C{ at this point \PB{$\\{arcs}[\|u]\K\|j$} }\6
${}\\{arcs}[\|u]\K\|p;{}$\6
\&{if} ${}(\R\|p){}$\1\5
${}\\{active}\MM;{}$\2\6
\X14:Undo the union operation $u\equiv v$\X;\6
\4${}\}{}$\2\par
\U5.\fi

\M{10}Two vertices are equivalent if they belong to the same component.
We use a classic union-find data structure to keep of equivalences:
The invariant relations state that
\PB{$\\{parent}[\|v]<\T{0}$} and \PB{$\\{size}[\|v]\K\|c$} if \PB{\|v} is the
root of an
equivalence class of size~\PB{\|c}; otherwise \PB{\\{parent}[\|v]} points to
an equivalent vertex that is nearer the root. These trees have
at most $\lg m$ levels, because we never merge a tree of size~\PB{\|c}
into a tree of size \PB{$<$ \|c}.

Variable \PB{\|l} is the current number of edges. It is also, therefore, the
number
of union operations previously done but not yet undone.

\fi

\M{11}\B\X11:Initialize the union/find structures\X${}\E{}$\6
\&{for} ${}(\|j\K\T{0};{}$ ${}\|j\Z\|m;{}$ ${}\|j\PP){}$\1\5
${}\\{parent}[\|j]\K{-}\T{1},\39\\{size}[\|j]\K\T{1}{}$;\C{ and \PB{$\|l\K%
\T{0}$} }\2\6
${}\|l\K\T{0}{}$;\par
\U7.\fi

\M{12}\B\X12:Do a union operation $u\equiv v$\X${}\E{}$\6
\&{for} ${}(\\{uu}\K\|u;{}$ ${}\\{parent}[\\{uu}]\G\T{0};{}$ ${}\\{uu}\K%
\\{parent}[\\{uu}]){}$\1\5
;\2\6
\&{for} ${}(\\{vv}\K\|v;{}$ ${}\\{parent}[\\{vv}]\G\T{0};{}$ ${}\\{vv}\K%
\\{parent}[\\{vv}]){}$\1\5
;\2\6
\&{if} ${}(\\{uu}\E\\{vv}){}$\1\5
${}\\{move}[\|l]\K{-}\T{1};{}$\2\6
\&{else} \&{if} ${}(\\{size}[\\{uu}]\Z\\{size}[\\{vv}]){}$\1\5
${}\\{parent}[\\{uu}]\K\\{vv},\39\\{move}[\|l]\K\\{uu},\39\\{size}[\\{vv}]%
\MRL{+{\K}}\\{size}[\\{uu}];{}$\2\6
\&{else}\1\5
${}\\{parent}[\\{vv}]\K\\{uu},\39\\{move}[\|l]\K\\{vv},\39\\{size}[\\{uu}]%
\MRL{+{\K}}\\{size}[\\{vv}];{}$\2\6
${}\|l\PP{}$;\par
\U8.\fi

\M{13}Dynamic union-find is ridiculously easy because, as observed above,
the operations are strictly last-in-first-out.
And we didn't clobber the \PB{\\{size}} information when merging two classes.

\fi

\M{14}\B\X14:Undo the union operation $u\equiv v$\X${}\E{}$\6
$\|l\MM;{}$\6
${}\\{uu}\K\\{move}[\|l];{}$\6
\&{if} ${}(\\{uu}\G\T{0}){}$\5
${}\{{}$\1\6
${}\\{vv}\K\\{parent}[\\{uu}]{}$;\C{ we have \PB{$\\{parent}[\\{vv}]<\T{0}$} }\6
${}\\{size}[\\{vv}]\MRL{-{\K}}\\{size}[\\{uu}];{}$\6
${}\\{parent}[\\{uu}]\K{-}\T{1};{}$\6
\4${}\}{}$\2\par
\U9.\fi

\M{15}\B\X15:Global variables\X${}\E{}$\6
\&{int} \\{active};\C{ this many vertices are currently labeled (not isolated)
}\6
\&{int} ${}\\{parent}[\\{maxm}+\T{1}],{}$ ${}\\{size}[\\{maxm}+\T{1}],{}$ %
\\{move}[\\{maxm}];\C{ the union-find structures }\6
\&{int} ${}\\{arcs}[\\{maxm}+\T{1}]{}$;\C{ the first neighbor of \PB{\|v} }\6
\&{int} ${}\\{link}[\T{2}*\\{maxm}+\T{1}]{}$;\C{ the next element in a list of
neighbors }\6
\&{int} ${}\|x[\\{maxm}+\T{1}]{}$;\C{ the governing sequence of edge choices }%
\par
\A18.
\U3.\fi

\N{1}{16}Doing it.
Now we're ready to harvest the routines we've built up.

[A puzzle for the reader: Is \PB{\\{parent}[\|m]} always negative at this
point?
Answer: Not if, say, $m=7$ and $(x_1,\ldots,x_m)=(5,4,3,2,0,1,0)$.]

\Y\B\4\X16:Study the current graph\X${}\E{}$\6
\&{for} ${}(\|k\K\\{parent}[\|m];{}$ ${}\\{parent}[\|k]\G\T{0};{}$ ${}\|k\K%
\\{parent}[\|k]){}$\1\5
;\2\6
\&{if} ${}(\\{size}[\|k]\E\\{active}){}$\1\5
${}\\{connected}[\\{active}]\PP;{}$\2\6
\&{else}\1\5
${}\\{disconnected}[\\{active}]\PP{}$;\2\par
\U3.\fi

\M{17}\B\X17:Print the stats\X${}\E{}$\6
$\\{printf}(\.{"Counts\ for\ \%d\ edges}\)\.{:\\n"},\39\|m);{}$\6
\&{for} ${}(\|k\K\T{2};{}$ ${}\|k\Z\|m+\T{1};{}$ ${}\|k\PP){}$\1\6
\&{if} ${}(\\{connected}[\|k]+\\{disconnected}[\|k]){}$\5
${}\{{}$\1\6
${}\\{printf}(\.{"on\ \%5d\ vertices,\ \%l}\)\.{ld\ are\ connected,\ \%l}\)%
\.{ld\ not\\n"},\39\|k,\39\T{2}*\\{connected}[\|k],\39\T{2}*\\{disconnected}[%
\|k]);{}$\6
${}\\{totconnected}\MRL{+{\K}}\T{2}*\\{connected}[\|k],\39\\{totdisconnected}%
\MRL{+{\K}}\T{2}*\\{disconnected}[\|k];{}$\6
\4${}\}{}$\2\2\6
${}\\{printf}(\.{"Altogether\ \%lld\ con}\)\.{nected\ and\ \%lld\ not.}\)\.{%
\\n"},\39\\{totconnected},\39\\{totdisconnected}){}$;\par
\U3.\fi

\M{18}\B\X15:Global variables\X${}\mathrel+\E{}$\6
\&{unsigned} \&{long} \&{long} ${}\\{connected}[\\{maxm}+\T{2}],{}$ ${}%
\\{disconnected}[\\{maxm}+\T{2}];{}$\6
\&{unsigned} \&{long} \&{long} \\{totconnected}${},{}$ \\{totdisconnected};\par
\fi

\N{1}{19}Index.
\fi

\inx
\fin
\con
