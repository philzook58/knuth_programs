\input cwebmac

\N{1}{1}Intro. What's the lexicographically smallest solution to the
$\infty$-queens problem? I mean, consider the sequence $q_0$, $q_1$, \dots,
where $q_n$ is the least nonnegative integer not in the sets
$\{q_k\mid 0\le k<n\}$,
$\{q_k+k-n\mid 0\le k<n\}$,
$\{q_k-k+n\mid 0\le k<n\}$.

Inspired by Eq. 7.2.2--(6), I maintain bit vectors $a$, $b$, $c$
to record the occurrences of $q_k$, $q_k+k$, and $q_k-k$, for
previously calculated values. Thus, for example, we'll have
$c_r=1$ if and only if $q_k-k=r$ for some $k<n$, when I'm calculating $q_n$.
The value of $q_n$ will be the smallest $k>0$ such that $a_k=0$ and
$b_{k+n}=0$ and $c_{k-n}=0$.

It turns out (as conjectured by Neil Sloane in 2016, and proved
by him and Jeffrey Shallit shortly thereafter) that this sequence
has lots of beautiful structure, which greatly facilitates the
computation. Let $\phi$ be the golden ratio. Then the subscripts of~$a$
will range from 0 to approximately $\phi n$; the subscripts
of~$b$ will range from 0 to approximately $\phi^2 n$; and the subscripts
of~$c$ will range from approximately $-\phi^{-2}n$ to approximately
$\phi^{-1}n$. In particular, since $c$ has $n$ bits equal to~1,
and $\phi^{-1}n+\phi^{-2}n=n$, almost all the bits of~$c$ will
be equal to~1. Furthermore, $a$ will begin with a string of 1s,
whose length is approximately $\phi^{-1}n$. Therefore we only
need to look at a bounded number of bits when we're computing $q_n$.

Nice, huh?

This implementation uses one byte per bit. Of course I could
make $n$ eight times larger by packing the bits.

Another feature is an {\it exact\/} computation of the
discrepancies between $q_n$ and $\phi n$ or $\phi^{-1}n$.
For example, this program ``knows'' that
$q_{F_{40}}=F_{41}=\phi F_{40}+\phi^{-40}$.

\Y\B\4\D$\\{slack}$ \5
\T{10}\C{ approximation when we allocate memory }\par
\B\4\D$\\{phi}$ \5
\T{1.6180339887498948482}\par
\B\4\D$\\{tickmax}$ \5
\T{25}\C{ I hope to need at most this many ticks per round }\par
\B\4\D$\\{deltamax}$ \5
\T{10}\par
\B\4\D$\|o$ \5
$\\{ticks}\PP{}$\par
\B\4\D$\\{pausethresh}$ \5
\T{999999995}\par
\Y\B\8\#\&{include} \.{<stdio.h>}\6
\8\#\&{include} \.{<stdlib.h>}\6
\8\#\&{include} \.{<math.h>}\6
\&{int} \\{goal};\C{ command-line parameter }\6
\&{char} ${}{*}\|a,{}$ ${}{*}\|b,{}$ ${}{*}\|c;{}$\6
\&{long} \&{long} \&{int} \\{maxmema}${},{}$ \\{maxmemb}${},{}$ %
\\{minmemc}${},{}$ \\{maxmemc};\6
\&{int} \\{ticks};\6
\&{int} ${}\\{tickhist}[\\{tickmax}+\T{1}]{}$;\C{ histogram of run times }\6
\&{int} \\{deltalominint}${},{}$ \\{deltalomaxint}${},{}$ %
\\{deltahiminint}${},{}$ \\{deltahimaxint};\6
\&{long} \&{long} \\{deltalominfrac}${},{}$ \\{deltalomaxfrac}${},{}$ %
\\{deltahiminfrac}${},{}$ \\{deltahimaxfrac};\6
\&{int} ${}\\{deltalomin}[\\{deltamax}+\T{1}],{}$ ${}\\{deltalomax}[%
\\{deltamax}+\T{1}],{}$ ${}\\{deltahimin}[\\{deltamax}+\T{1}],{}$ ${}%
\\{deltahimax}[\\{deltamax}+\T{1}];{}$\7
\X7:Subroutines\X;\7
\\{main}(\&{int} \\{argc}${},\39{}$\&{char} ${}{*}\\{argv}[\,]){}$\1\1\2\2\6
${}\{{}$\1\6
\&{register} \&{int} \|j;\6
\&{register} \&{long} \&{long} \|k${},{}$ \|n${},{}$ \|q${},{}$ \|r${},{}$ %
\|s${},{}$ \|t${},{}$ \\{nphiint}${},{}$ \\{nphifrac};\7
\X2:Process the command line\X;\6
\X3:Allocate the arrays\X;\6
${}\|r\K\|t\K\T{0},\39\|s\K\T{1};{}$\6
\&{for} ${}(\|n\K\\{nphiint}\K\\{nphifrac}\K\T{1};{}$ ${}\|n\Z\\{goal};{}$ ${}%
\|n\PP){}$\5
${}\{{}$\1\6
\X4:Determine $q=q_n$, or \PB{\&{goto} \\{done}} if out of memory\X;\6
${}\\{printf}(\.{"\%lld\\n"},\39\|q);{}$\6
\X8:Record statistics about $q$\X;\6
\X6:Advance \PB{\\{nphiint}} and \PB{\\{nphifrac}}\X;\6
\4${}\}{}$\2\6
\4\\{done}:\5
\X9:Print the final stats\X;\6
\4${}\}{}$\2\par
\fi

\M{2}\B\X2:Process the command line\X${}\E{}$\6
\&{if} ${}(\\{argc}\I\T{2}\V\\{sscanf}(\\{argv}[\T{1}],\39\.{"\%d"},\39{\AND}%
\\{goal})\I\T{1}){}$\5
${}\{{}$\1\6
${}\\{fprintf}(\\{stderr},\39\.{"Usage:\ \%s\ n\\n"},\39\\{argv}[\T{0}]);{}$\6
${}\\{exit}({-}\T{1});{}$\6
\4${}\}{}$\2\par
\U1.\fi

\M{3}\B\X3:Allocate the arrays\X${}\E{}$\6
$\\{maxmema}\K((\&{int})(\\{phi}*\\{goal})+\\{slack});{}$\6
${}\\{maxmemb}\K(\\{maxmema}+\\{goal});{}$\6
${}\\{maxmemc}\K(\\{maxmema}-\\{goal});{}$\6
${}\\{minmemc}\K(\\{goal}-\\{maxmemc}+\T{2}*\\{slack});{}$\6
${}\|a\K{}$(\&{char} ${}{*}){}$ \\{calloc}${}(\\{maxmema},\39\&{sizeof}(%
\&{char}));{}$\6
\&{if} ${}(\R\|a){}$\5
${}\{{}$\1\6
${}\\{fprintf}(\\{stderr},\39\.{"Can't\ allocate\ arra}\)\.{y\ a!\\n"});{}$\6
${}\\{exit}({-}\T{2});{}$\6
\4${}\}{}$\2\6
${}\|b\K{}$(\&{char} ${}{*}){}$ \\{calloc}${}(\\{maxmemb},\39\&{sizeof}(%
\&{char}));{}$\6
\&{if} ${}(\R\|b){}$\5
${}\{{}$\1\6
${}\\{fprintf}(\\{stderr},\39\.{"Can't\ allocate\ arra}\)\.{y\ b!\\n"});{}$\6
${}\\{exit}({-}\T{2});{}$\6
\4${}\}{}$\2\6
${}\|c\K{}$(\&{char} ${}{*}){}$ \\{calloc}${}(\\{minmemc}+\\{maxmemc},\39%
\&{sizeof}(\&{char}));{}$\6
\&{if} ${}(\R\|c){}$\5
${}\{{}$\1\6
${}\\{fprintf}(\\{stderr},\39\.{"Can't\ allocate\ arra}\)\.{y\ c!\\n"});{}$\6
${}\\{exit}({-}\T{2});{}$\6
\4${}\}{}$\2\par
\U1.\fi

\M{4}In this algorithm, $s$ is the least positive integer such that $a_s=0$;
$t$ is the greatest integer such that $t=0$ or $c_t=1$;
$r$ is the greatest nonnegative integer $\le t$ such that $c_{-r}=0$.

\Y\B\4\X4:Determine $q=q_n$, or \PB{\&{goto} \\{done}} if out of memory\X${}%
\E{}$\6
$\\{ticks}\K\T{0};{}$\6
\&{for} ${}(\|k\K\|s;{}$ ${}\|k\Z\|n-\|r;{}$ ${}\|k\PP){}$\5
${}\{{}$\1\6
\&{if} ${}(\|k+\|n\G\\{maxmemb}){}$\1\5
\&{goto} \\{done};\2\6
\&{if} ${}(\|o,\39\|b[\|k+\|n]\E\T{0}){}$\5
${}\{{}$\1\6
\&{if} ${}(\|k-\|n+\\{minmemc}<\T{0}){}$\1\5
\&{goto} \\{done};\2\6
\&{if} ${}(\|o,\39\|c[\|k-\|n+\\{minmemc}]\E\T{0}){}$\5
${}\{{}$\1\6
\&{if} ${}(\|o,\39\|a[\|k]\E\T{0}){}$\5
${}\{{}$\1\6
${}\|q\K\|k;{}$\6
${}\|o,\39\|a[\|k]\K\T{1};{}$\6
\&{if} ${}(\|k\E\|s){}$\1\6
\&{for} ${}(\|s\K\|k+\T{1};{}$ \|o${},\39\|a[\|s]\E\T{1};{}$ ${}\|s\PP){}$\1\5
;\2\2\6
${}\|o,\39\|b[\|k+\|n]\K\T{1};{}$\6
${}\|o,\39\|c[\|k-\|n+\\{minmemc}]\K\T{1};{}$\6
\&{if} ${}(\|k-\|n\E{-}\|r){}$\1\6
\&{for} ${}(\|r\K\|n-\|k+\T{1};{}$  ; ${}\|r\PP){}$\5
${}\{{}$\1\6
\&{if} ${}(\|r>\\{minmemc}){}$\1\5
\&{goto} \\{done};\2\6
\&{if} ${}(\|o,\39\|c[\\{minmemc}-\|r]\E\T{0}){}$\1\5
\&{break};\2\6
\4${}\}{}$\2\2\6
\&{goto} \\{got\_q};\6
\4${}\}{}$\2\6
\4${}\}{}$\2\6
\4${}\}{}$\2\6
\4${}\}{}$\2\6
${}\|t\PP;{}$\6
\&{if} ${}(\|t\G\\{maxmemc}){}$\1\5
\&{goto} \\{done};\2\6
${}\|o,\39\|c[\|t+\\{minmemc}]\K\T{1};{}$\6
${}\|q\K\|n+\|t;{}$\6
\&{if} ${}(\|q\G\\{maxmema}){}$\1\5
\&{goto} \\{done};\2\6
${}\|o,\39\|a[\|q]\K\T{1};{}$\6
\&{if} ${}(\|q+\|n\G\\{maxmemb}){}$\1\5
\&{goto} \\{done};\2\6
${}\|o,\39\|b[\|q+\|n]\K\T{1};$ \\{got\_q}:\par
\U1.\fi

\M{5}I had special fun writing the next part of this program,
which expresses the value of $n\phi$ as an integer plus
$\sum_{k\ge1}x_k\phi^{-k}$, with $x_kx_{k+1}=0$ for all $k$.
For example, $9\phi=14+\phi^{-2}+\phi^{-4}+\phi^{-7}$.
We maintain the integer part in \PB{\\{nphiint}}, and the fractional
part in \PB{\\{nphifrac}}, where the latter is the {\it binary\/} integer
$(\ldots x_3x_2x_1)_2$.

This fractional part has a nice connection with the {\it negaFibonacci
number system}, which is described in equation 7.1.3--(147) of
{\sl The Art of Computer Programming}. For example, $9=F_{-7}+F_{-4}+F_{-2}$
is the negaFibonacci representation of~9; hence we have
$9\phi=(F_6-F_4-F_2)\phi=F_7-F_5-F_3-((-\phi)^{-7}+(-\phi)^{-4}+(-\phi)^{-2}=
14+\phi^{-7}+\phi^{-4}+\phi^{-2}$.

Furthermore, equation 7.1.3--(149) shows a nice way to go from
the negaFibonacci representation of $n$ to its successor.
And exercise 7.1.3--45 shows that it's surprisingly easy to
compare the fractional parts, even if they are ordered lexicographically
from right to left instead of from left to right(!).

\fi

\M{6}\B\X6:Advance \PB{\\{nphiint}} and \PB{\\{nphifrac}}\X${}\E{}$\6
$\\{nphiint}\PP;{}$\6
\&{if} ${}(\\{nphifrac}\AND\T{\^3}){}$\1\5
${}\\{nphiint}\PP;{}$\2\6
${}\{{}$\1\6
\&{register} \&{long} \&{long} \|y${},{}$ \|z;\7
${}\|y\K\\{nphifrac}\XOR\T{\^aaaaaaaaaaaaaaaa};{}$\6
${}\|z\K\|y\XOR(\|y+\T{1});{}$\6
${}\|z\K\|z\OR(\\{nphifrac}\AND(\|z\LL\T{1}));{}$\6
${}\\{nphifrac}\MRL{{\XOR}{\K}}\|z\XOR((\|z+\T{1})\GG\T{2});{}$\6
\4${}\}{}$\2\par
\U1.\fi

\M{7}\B\X7:Subroutines\X${}\E{}$\6
\&{int} \\{compfrac}(\&{long} \&{long} \|x${},\39{}$\&{long} \&{long} \|y)\1\1%
\2\2\6
${}\{{}$\1\6
\&{register} \&{int} \&{long} \&{long} \|d${}\K(\|x-\|y)\AND(\|y-\|x){}$;\C{
Rockicki's hack }\7
\&{return} ${}((\|d\AND\|y)\I\T{0}){}$;\C{ 1 if $x^R<y^R$, 0 otherwise }\6
\4${}\}{}$\2\par
\A10.
\U1.\fi

\M{8}\B\X8:Record statistics about $q$\X${}\E{}$\6
\&{if} ${}(\|n>\\{pausethresh}){}$\1\5
\\{debug}(\.{"watch\ me\ now"});\2\6
\&{if} ${}(\\{ticks}\G\\{tickmax}){}$\1\5
${}\\{tickhist}[\\{tickmax}]\PP;{}$\2\6
\&{else}\1\5
${}\\{tickhist}[\\{ticks}]\PP;{}$\2\6
\&{if} ${}(\|q\G\|n){}$\5
${}\{{}$\1\6
\&{if} ${}(\|q>\\{nphiint}){}$\5
${}\{{}$\1\6
\&{if} ${}(\|q-\\{nphiint}>\\{deltahimaxint}\V(\|q-\\{nphiint}\E%
\\{deltahimaxint}\W\\{compfrac}(\\{nphifrac},\39\\{deltahimaxfrac}))){}$\5
${}\{{}$\1\6
${}\\{deltahimaxint}\K\|q-\\{nphiint},\39\\{deltahimaxfrac}\K\\{nphifrac};{}$\6
${}\\{fprintf}(\\{stderr},\39\.{"n=\%lld,\ deltahimax=}\)\.{\%d,\%llx\\n"},\39%
\|n,\39\\{deltahimaxint},\39\\{deltahimaxfrac});{}$\6
\4${}\}{}$\2\6
${}\|j\K\|q-\\{nphiint}-\T{1};{}$\6
\&{if} ${}(\|j\G\\{deltamax}){}$\1\5
${}\\{deltahimax}[\\{deltamax}]\PP;{}$\2\6
\&{else}\1\5
${}\\{deltahimax}[\|j]\PP;{}$\2\6
\4${}\}{}$\5
\2\&{else}\5
${}\{{}$\1\6
\&{if} ${}(\|q-\\{nphiint}<\\{deltahiminint}\V(\|q-\\{nphiint}\E%
\\{deltahiminint}\W\\{compfrac}(\\{deltahiminfrac},\39\\{nphifrac}))){}$\5
${}\{{}$\1\6
${}\\{deltahiminint}\K\|q-\\{nphiint},\39\\{deltahiminfrac}\K\\{nphifrac};{}$\6
${}\\{fprintf}(\\{stderr},\39\.{"n=\%lld,\ deltahimin=}\)\.{\%d,\%llx\\n"},\39%
\|n,\39\\{deltahiminint},\39\\{deltahiminfrac});{}$\6
\4${}\}{}$\2\6
${}\|j\K\\{nphiint}-\|q;{}$\6
\&{if} ${}(\|j\G\\{deltamax}){}$\1\5
${}\\{deltahimin}[\\{deltamax}]\PP;{}$\2\6
\&{else}\1\5
${}\\{deltahimin}[\|j]\PP;{}$\2\6
\4${}\}{}$\2\6
\4${}\}{}$\5
\2\&{else} \&{if} ${}(\|q>(\\{nphiint}-\|n)){}$\5
${}\{{}$\1\6
\&{if} ${}(\|q-(\\{nphiint}-\|n)>\\{deltalomaxint}\V(\|q-(\\{nphiint}-\|n)\E%
\\{deltalomaxint}\W\\{compfrac}(\\{nphifrac},\39\\{deltalomaxfrac}))){}$\5
${}\{{}$\1\6
${}\\{deltalomaxint}\K\|q-(\\{nphiint}-\|n),\39\\{deltalomaxfrac}\K%
\\{nphifrac};{}$\6
${}\\{fprintf}(\\{stderr},\39\.{"n=\%lld,\ deltalomax=}\)\.{\%d,\%llx\\n"},\39%
\|n,\39\\{deltalomaxint},\39\\{deltalomaxfrac});{}$\6
\4${}\}{}$\2\6
${}\|j\K\|q-(\\{nphiint}-\|n)-\T{1};{}$\6
\&{if} ${}(\|j\G\\{deltamax}){}$\1\5
${}\\{deltalomax}[\\{deltamax}]\PP;{}$\2\6
\&{else}\1\5
${}\\{deltalomax}[\|j]\PP;{}$\2\6
\4${}\}{}$\5
\2\&{else}\5
${}\{{}$\1\6
\&{if} ${}(\|q-(\\{nphiint}-\|n)<\\{deltalominint}\V(\|q-(\\{nphiint}-\|n)\E%
\\{deltalominint}\W\\{compfrac}(\\{deltalominfrac},\39\\{nphifrac}))){}$\5
${}\{{}$\1\6
${}\\{deltalominint}\K\|q-(\\{nphiint}-\|n),\39\\{deltalominfrac}\K%
\\{nphifrac};{}$\6
${}\\{fprintf}(\\{stderr},\39\.{"n=\%lld,\ deltalomin=}\)\.{\%d,\%llx\\n"},\39%
\|n,\39\\{deltalominint},\39\\{deltalominfrac});{}$\6
\4${}\}{}$\2\6
${}\|j\K(\\{nphiint}-\|n)-\|q;{}$\6
\&{if} ${}(\|j\G\\{deltamax}){}$\1\5
${}\\{deltalomin}[\\{deltamax}]\PP;{}$\2\6
\&{else}\1\5
${}\\{deltalomin}[\|j]\PP;{}$\2\6
\4${}\}{}$\2\par
\U1.\fi

\M{9}\B\X9:Print the final stats\X${}\E{}$\6
$\\{fprintf}(\\{stderr},\39\.{"OK,\ I\ computed\ \%lld}\)\.{\ elements\ of\ the%
\ seq}\)\.{uence.\\n"},\39\|n-\T{1});{}$\6
${}\\{fprintf}(\\{stderr},\39\.{"tick\ histogram:"});{}$\6
\&{for} ${}(\|j\K\T{0};{}$ ${}\|j\Z\\{tickmax};{}$ ${}\|j\PP){}$\1\5
${}\\{fprintf}(\\{stderr},\39\.{"\ \%d"},\39\\{tickhist}[\|j]);{}$\2\6
${}\\{fprintf}(\\{stderr},\39\.{"\\n"});{}$\6
${}\\{fprintf}(\\{stderr},\39\.{"deltalo\ histogram:"});{}$\6
\&{for} ${}(\|j\K\\{deltamax};{}$ ${}\|j\G\T{0};{}$ ${}\|j\MM){}$\1\5
${}\\{fprintf}(\\{stderr},\39\.{"\ \%d"},\39\\{deltalomin}[\|j]);{}$\2\6
${}\\{fprintf}(\\{stderr},\39\.{"\ |"});{}$\6
\&{for} ${}(\|j\K\T{0};{}$ ${}\|j\Z\\{deltamax};{}$ ${}\|j\PP){}$\1\5
${}\\{fprintf}(\\{stderr},\39\.{"\ \%d"},\39\\{deltalomax}[\|j]);{}$\2\6
${}\\{fprintf}(\\{stderr},\39\.{"\\n"});{}$\6
${}\\{fprintf}(\\{stderr},\39\.{"deltahi\ histogram:"});{}$\6
\&{for} ${}(\|j\K\\{deltamax};{}$ ${}\|j\G\T{0};{}$ ${}\|j\MM){}$\1\5
${}\\{fprintf}(\\{stderr},\39\.{"\ \%d"},\39\\{deltahimin}[\|j]);{}$\2\6
${}\\{fprintf}(\\{stderr},\39\.{"\ |"});{}$\6
\&{for} ${}(\|j\K\T{0};{}$ ${}\|j\Z\\{deltamax};{}$ ${}\|j\PP){}$\1\5
${}\\{fprintf}(\\{stderr},\39\.{"\ \%d"},\39\\{deltahimax}[\|j]);{}$\2\6
${}\\{fprintf}(\\{stderr},\39\.{"\\n"}){}$;\par
\U1.\fi

\M{10}\B\X7:Subroutines\X${}\mathrel+\E{}$\6
\&{void} \\{debug}(\&{char} ${}{*}\|m){}$\1\1\2\2\6
${}\{{}$\1\6
${}\\{fprintf}(\\{stderr},\39\.{"\%s!\\n"},\39\|m);{}$\6
\4${}\}{}$\2\par
\fi

\N{1}{11}Index.
\fi

\inx
\fin
\con
