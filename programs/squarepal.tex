\input cwebmac
\let\mod=\bmod


\N{1}{1}Intro. This program finds all odd $n$-bit palindromes $x$ that are
perfect
squares, using roughly $2^{n/4}$ steps of computation. Thus I hope to use it
for $n$ well over 100.
The idea is to try all $2^t$ combinations of the rightmost and leftmost
$t+3$ bits, for $t\approx n/4$, and to use number theory to rule out
the bad cases rather quickly.

(When $n=100$ I'll be using $t=22$.
This program is a big improvement over the one I wrote in 2013;
that one used $t=31$ when $n=100$, and $t\approx n/3$ in general.
Michael Coriand surprised me last week by claiming that he had a method
using only about $n/4$. At first I was mystified, baffled, stumped. But
aha, I woke up this morning with a good guess about what he'd discovered!
He asked me to doublecheck his results; and
I can't resist, even though I've got more than enough other things to do,
because it's fun to write useless code like~this.)

I haven't optimized this program for computational speed. My main goal was to
get it right, with my personal time minimized.
On the other hand I could easily have made it run a lot slower:
I didn't pass up some ``obvious'' ways to avoid redundant computations.

\Y\B\4\D$\\{maxn}$ \5
\T{180}\C{ I could go a little higher, but there won't be time }\par
\Y\B\8\#\&{include} \.{<stdio.h>}\6
\8\#\&{include} \.{<stdlib.h>}\6
\&{int} \|n;\C{ the length of palindromes sought }\6
\&{unsigned} \&{long} \&{long} ${}\|y[\\{maxn}/\T{2}],{}$ ${}\|r[\\{maxn}/%
\T{2}]{}$;\C{ table of partial square roots }\6
\&{unsigned} \&{long} \&{long} ${}\|q[\\{maxn}/\T{4}],{}$ ${}\\{qq}[\\{maxn}/%
\T{4}]{}$;\C{ table of partial modular sqrts }\6
\&{unsigned} \&{long} \&{long} \\{pretrial}[\T{2}]${},{}$ \\{trial}[%
\T{3}]${},{}$ \\{acc}[\T{6}];\C{ multiplication workspace }\7
\\{main}(\&{int} \\{argc}${},\39{}$\&{char} ${}{*}\\{argv}[\,]){}$\1\1\2\2\6
${}\{{}$\1\6
\&{register} \&{unsigned} \&{long} \&{long} \\{prod}${},{}$ \\{sqrtxl}${},{}$ %
\|a${},{}$ \\{bit};\6
\&{register} \&{int} \|j${},{}$ \|k${},{}$ \|t${},{}$ \|p${},{}$ \\{jj}${},{}$ %
\\{kk};\7
\X2:Process the command line\X;\6
${}\\{printf}(\.{"Binary\ palindromic\ }\)\.{squares\ with\ \%d\ bits}\)\.{:%
\\n"},\39\|n);{}$\6
\X12:Choose \PB{\|t} and initialize the tables\X;\6
\&{for} ${}(\|a\K\T{0};{}$ ${}\|a<\T{1\$L\$L}\LL\|t;{}$ ${}\|a\PP){}$\1\5
\X13:See if case \PB{\|a} leads to any square palindromes\X;\2\6
\4${}\}{}$\2\par
\fi

\M{2}\B\X2:Process the command line\X${}\E{}$\6
\&{if} ${}(\\{argc}\I\T{2}\V\\{sscanf}(\\{argv}[\T{1}],\39\.{"\%d"},\39{\AND}%
\|n)\I\T{1}){}$\5
${}\{{}$\1\6
${}\\{fprintf}(\\{stderr},\39\.{"Usage:\ \%s\ n\\n"},\39\\{argv}[\T{0}]);{}$\6
${}\\{exit}({-}\T{1});{}$\6
\4${}\}{}$\2\6
\&{if} ${}(\|n<\T{15}\V\|n>\\{maxn}){}$\5
${}\{{}$\1\6
${}\\{fprintf}(\\{stderr},\39\.{"Sorry:\ n\ should\ be\ }\)\.{between\ 15\ and\
\%d.\\n}\)\.{"},\39\\{maxn});{}$\6
${}\\{exit}({-}\T{2});{}$\6
\4${}\}{}$\2\par
\U1.\fi

\M{3}Here's the theory: Let $a_1a_2\ldots a_t$ be a binary string, and suppose
$$x\;=\;2^{n-1}+2^{n-4}a_1+2^{n-5}a_2+\cdots+2^{n-3-t}a_t+\cdots+
2^{t+2}a_t+\cdots+2^4a_2+2^3a_1+1\;=\;y^2.$$
Let $x_l=2^{n-1}+2^{n-4}a_1+2^{n-5}a_2+\cdots+2^{n-3-t}a_t$ and
$x_u=x_l+2^{n-3-t}$.

It's easy to prove by induction on $t$ that
there's a unique integer $q$ between 0 and $2^{t+2}$ such that
$q\mod4=1$ and $q^2\mod2^{t+3}=2^{t+2}a_t+\cdots+2^4a_2+2^3a_1+1$,
whenever $t>0$.
Hence the lower bits of the square root, $y\mod 2^{t+2}$,
must be either $q$ or $2^{t+2}-q$.

On the other hand $x_l<x<x_u$; hence $\sqrt{x_l}<y<\sqrt{x_u}$.
This tells us about the upper bits: We have $x_u=x_l(1+\delta)$,
where $\delta=2^{n-3-t}\!/x_l$; hence $\sqrt{x_u}-\sqrt{x_l}
=\sqrt{x_l}\bigl(\sqrt{1+\delta}-1\bigr)
<\sqrt{x_l}\,\delta/2=2^{n-4-t}\!/\sqrt{x_l}
\le2^{n-4-t}\!/2^{(n-1)/2}=2^{n/2-7/2-t}$. The integers between
$\sqrt{x_l}$ and $\sqrt{x_u}$ will therefore be distinct, modulo $2^{t+2}$,
if we have $n/2-7/2-t\le t+2$.

It follows that we need only check two potential values of $y$, for each
of the $2^t$ choices of $a_1a_2\ldots a_t$, when $t\ge(n-11)/4$. Furthermore,
this estimate is somewhat crude; there won't be many cases to try even
if $t$ is a bit smaller.

For example, let's consider the case $n=25$ and $t=3$. In the first
case $a_1a_2a_3=000$, we have $x_l=2^{24}$ and $x_u=2^{24}+2^{19}$;
so $y$ lies between $(1000000000000)_2$ and $(1000000111111)_2$.
Furthermore $y\mod2^5$ must be 1 or~31. So the only possible
square roots of a binary palindrome having the form $(100000\ldots000001)_2$
are
$$(1000000000001)_2,\quad
(1000000011111)_2,\quad
(1000000100001)_2,\quad
(1000000111111)_2.$$
In the last case $a_1a_2a_3=111$, we have $x_l=2^{24}+2^{22}-2^{19}$
and $x_u=2^{24}+2^{22}$;
so $y$ lies between $(1000110101001)_2$ and $(1000111100011)_2$.
Furthermore $y\mod2^5$ must be 21 or~11. Again there are only
four $y$'s to try:
$$(1000110101011)_2,\quad
(1000110110101)_2,\quad
(1000111001011)_2,\quad
(1000111010101)_2.$$
(And the first of these actually works! Its square is
$(1001110000010100000111001)_2$.)

The program below actually finds it convenient to try a
few cases that could have been ruled out by the arguments above.
For example, it will try also
$(1000111101011)_2$ and
$(1000111110101)_2$ in the previous example.

\fi

\M{4}We'll have to compute $2^t$ ``modular square roots'' $q$.
Let \PB{\|q[\|j]} be the square root of $2^{j+2}a_j+\cdots+2^3a_1+1$
(modulo~$2^{j+3}$), and let \PB{\\{qq}[\|j]} be the rightmost \PB{$\|t+\T{2}$}
bits
of its square. If $j<t$, \PB{$\|q[\|j+\T{1}]$} will be either \PB{\|q[\|j]} or
$\PB{\|q[\|j]}+2^{j+1}$, depending on the $(j+2)$nd bit of \PB{\\{qq}[\|j]}.

When $a_1\ldots a_t=0\ldots0$, we have \PB{$\|q[\|j]\K\\{qq}[\|j]\K\T{1}$} for
all $j$.
And when moving from any $a_1\ldots a_t$ to its successor, we need only
recompute a few of the entries --- \PB{\|q[\|t]} always, \PB{$\|q[\|t-\T{1}]$}
half the time,
\PB{$\|q[\|t-\T{2}]$} one-fourth of the time, etc.

\Y\B\4\X4:Initialize the \PB{\|q} and \PB{\\{qq}} tables\X${}\E{}$\6
\&{for} ${}(\|j\K\T{1};{}$ ${}\|j\Z\|t;{}$ ${}\|j\PP){}$\1\5
${}\|q[\|j]\K\\{qq}[\|j]\K\T{1}{}$;\2\par
\U12.\fi

\M{5}\B\X5:Update \PB{\|q} and \PB{\\{qq}} when $a_p$ changes from 0 to~1\X${}%
\E{}$\6
$\|q[\|p]\MRL{{\XOR}{\K}}\T{1\$L\$L}\LL(\|p+\T{1});{}$\6
${}\\{qq}[\|p]\K\|q[\|p]*\|q[\|p];{}$\6
\&{for} ${}(\|j\K\|p+\T{1};{}$ ${}\|j\Z\|t;{}$ ${}\|j\PP){}$\5
${}\{{}$\1\6
\&{if} ${}(\\{qq}[\|j-\T{1}]\AND(\T{1\$L\$L}\LL(\|j+\T{2}))){}$\1\5
${}\|q[\|j]\K\|q[\|j-\T{1}]\XOR(\T{1\$L\$L}\LL(\|j+\T{1}));{}$\2\6
\&{else}\1\5
${}\|q[\|j]\K\|q[\|j-\T{1}];{}$\2\6
${}\\{qq}[\|j]\K\|q[\|j]*\|q[\|j];{}$\6
\4${}\}{}$\2\par
\U13.\fi

\M{6}Similarly, we'll have to compute $2^t$ approximate square roots
for the leading bits of~$y$. Let \PB{\|y[\|j]} be bits $m$ through~$j$
of $\sqrt{x_l}$, where $m=\lfloor n/2\rfloor-1$ is the index
of the leading bit. The classical algorithm for square root extraction
tells us how to go from \PB{\|y[\|j]} to \PB{$\|y[\|j-\T{1}]$}: We have a
``remainder''
\PB{\|r[\|j]} representing the difference from the leading bits of~$x_l$
and $\PB{\|y[\|j]}^2$, where $\PB{\|r[\|j]}\le2\PB{\|y[\|j]}$. To preserve this
invariant when $a_1\ldots a_t=0\ldots0$, we set \PB{$\|y[\|j-\T{1}]\K\T{2}\|y[%
\|j]$} and
\PB{$\|r[\|j-\T{1}]\K\T{4}\|r[\|j]$}; if then \PB{$\|r[\|j-\T{1}]>\T{2}\|y[\|j-%
\T{1}]$} we subtract \PB{$\T{2}\|y[\|j-\T{1}]+\T{1}$}
from \PB{$\|r[\|j-\T{1}]$} and increase \PB{$\|y[\|j-\T{1}]$} by~1. To preserve
the invariant
for other values of $a_1\ldots a_t$, the same steps apply except
that $r[j-1]=4r[j]+2a_i+a_{i+1}$ for an appropriate value of~$i$.
The bits of the square root need only be computed for $j\ge t+2$;
therefore all computations fit easily into a single \PB{\&{long} \&{long}}
register.

Once again it's easy to prime the pump when $a_1\ldots a_t=0\ldots0$,
and to move to the successor by updating fewer than two
entries on average (plus roughly $n/8$ entries ``in the middle''
where $x_l$ has roughly $n/4$ zeros).

\Y\B\4\X6:Initialize the \PB{\|y} and \PB{\|r} tables\X${}\E{}$\6
\&{if} ${}(\|n\AND\T{1}){}$\5
${}\{{}$\1\6
${}\|y[(\|n-\T{3})/\T{2}]\K\T{2},\39\|r[(\|n-\T{3})/\T{2}]\K\T{0};{}$\6
\&{for} ${}(\|j\K(\|n-\T{5})/\T{2};{}$ ${}\|j\G\|t+\T{2};{}$ ${}\|j\MM){}$\1\5
${}\|y[\|j]\K\T{2}*\|y[\|j+\T{1}],\39\|r[\|j]\K\T{0};{}$\2\6
\4${}\}{}$\5
\2\&{else}\5
${}\{{}$\1\6
${}\|y[\|n/\T{2}-\T{1}]\K\T{1},\39\|r[\|n/\T{2}-\T{1}]\K\T{1};{}$\6
\&{for} ${}(\|j\K\|n/\T{2}-\T{2};{}$ ${}\|j\G\|t+\T{2};{}$ ${}\|j\MM){}$\5
${}\{{}$\1\6
${}\|y[\|j]\K\T{2}*\|y[\|j+\T{1}],\39\|r[\|j]\K\T{4}*\|r[\|j+\T{1}];{}$\6
\&{if} ${}(\|r[\|j]>\T{2}*\|y[\|j]){}$\1\5
${}\|r[\|j]\MRL{-{\K}}\T{2}*\|y[\|j]+\T{1},\39\|y[\|j]\PP;{}$\2\6
\4${}\}{}$\2\6
\4${}\}{}$\2\par
\U12.\fi

\M{7}\B\X7:Update \PB{\|y} and \PB{\|r} when $a_p$ changes from 0 to~1\X${}%
\E{}$\6
$\|j\K(\|n-\T{3}-\|p)/\T{2};{}$\6
\&{if} ${}((\|n+\|p)\AND\T{1}){}$\1\5
${}\|r[\|j]\MRL{+{\K}}\T{1};{}$\2\6
\&{else}\1\5
${}\|r[\|j]\K\T{4}*\|r[\|j+\T{1}]+\T{2},\39\|y[\|j]\K\T{2}*\|y[\|j+\T{1}];{}$\2%
\6
\&{if} ${}(\|r[\|j]>\T{2}*\|y[\|j]){}$\1\5
${}\|r[\|j]\MRL{-{\K}}\T{2}*\|y[\|j]+\T{1},\39\|y[\|j]\PP;{}$\2\6
\&{for} ${}(\|j\MM;{}$ ${}\|j\G\|t+\T{2};{}$ ${}\|j\MM){}$\5
${}\{{}$\1\6
${}\|y[\|j]\K\T{2}*\|y[\|j+\T{1}],\39\|r[\|j]\K\T{4}*\|r[\|j+\T{1}];{}$\6
\&{if} ${}(\|r[\|j]>\T{2}*\|y[\|j]){}$\1\5
${}\|r[\|j]\MRL{-{\K}}\T{2}*\|y[\|j]+\T{1},\39\|y[\|j]\PP;{}$\2\6
\4${}\}{}$\2\par
\U13.\fi

\M{8}Now comes the boring stuff. I hope I don't mess up here.
To make the final test, I'll need to square a number of up to 90 bits.
I simply treat it as three 32-bit chunks, and multiply by the
textbook method.

\Y\B\4\D$\\{m32}$ \5
\T{\^ffffffff}\C{ 32-bit mask }\par
\Y\B\4\X8:Square the contents of \PB{\\{trial}}\X${}\E{}$\6
\&{for} ${}(\|j\K\T{0};{}$ ${}\|j<\T{3};{}$ ${}\|j\PP){}$\5
${}\{{}$\1\6
${}\\{prod}\K\\{trial}[\|j]*\\{trial}[\T{0}];{}$\6
\&{if} (\|j)\1\5
${}\\{prod}\MRL{+{\K}}\\{acc}[\|j];{}$\2\6
${}\\{acc}[\|j]\K\\{prod}\AND\\{m32};{}$\6
${}\\{prod}\MRL{{\GG}{\K}}\T{32};{}$\6
${}\\{prod}\MRL{+{\K}}\\{trial}[\|j]*\\{trial}[\T{1}];{}$\6
\&{if} (\|j)\1\5
${}\\{prod}\MRL{+{\K}}\\{acc}[\|j+\T{1}];{}$\2\6
${}\\{acc}[\|j+\T{1}]\K\\{prod}\AND\\{m32};{}$\6
${}\\{prod}\MRL{{\GG}{\K}}\T{32};{}$\6
${}\\{prod}\MRL{+{\K}}\\{trial}[\|j]*\\{trial}[\T{2}];{}$\6
\&{if} (\|j)\1\5
${}\\{prod}\MRL{+{\K}}\\{acc}[\|j+\T{2}];{}$\2\6
${}\\{acc}[\|j+\T{2}]\K\\{prod}\AND\\{m32};{}$\6
${}\\{acc}[\|j+\T{3}]\K\\{prod}\GG\T{32};{}$\6
\4${}\}{}$\2\par
\U14.\fi

\M{9}To manufacture the \PB{\\{trial}}, I need to shift the leading digits
appropriately and combine them with the trailing digits.
First, I put the leading digits into \PB{\\{pretrial}} and \PB{\\{trial}}.
(This can be tricky: If $n=129$ or 130, so that $t=29$, there
are 34 leading digits; one of them will go into \PB{\\{trial}[\T{0}]},
32 into \PB{\\{trial}[\T{1}]}, and one into \PB{\\{trial}[\T{2}]}.)

\Y\B\4\X9:Shift the leading digits\X${}\E{}$\6
\&{if} ${}(\|t+\T{2}<\T{32}){}$\1\5
${}\\{pretrial}[\T{0}]\K(\\{sqrtxl}\LL(\|t+\T{2}))\AND\\{m32},\39\\{pretrial}[%
\T{1}]\K(\\{sqrtxl}\GG(\T{30}-\|t))\AND\\{m32};{}$\2\6
\&{else}\1\5
${}\\{pretrial}[\T{0}]\K\T{0},\39\\{pretrial}[\T{1}]\K(\\{sqrtxl}\LL(\|t-%
\T{30}))\AND\\{m32};{}$\2\6
${}\\{trial}[\T{2}]\K\\{sqrtxl}\GG(\T{62}-\|t){}$;\par
\U13.\fi

\M{10}\B\X10:Add \PB{\|q[\|t]} to the \PB{\\{trial}}\X${}\E{}$\6
\&{if} ${}(\|t+\T{2}\Z\T{32}){}$\1\5
${}\\{trial}[\T{0}]\K\\{pretrial}[\T{0}]+\|q[\|t],\39\\{trial}[\T{1}]\K%
\\{pretrial}[\T{1}];{}$\2\6
\&{else}\1\5
${}\\{trial}[\T{0}]\K\|q[\|t]\AND\\{m32},\39\\{trial}[\T{1}]\K\\{pretrial}[%
\T{1}]+(\|q[\|t]\GG\T{32}){}$;\2\par
\U13.\fi

\M{11}\B\D$\\{comp}(\|x)$ \5
$((\T{1\$L\$L}\LL(\|t+\T{2}))-(\|x){}$)\par
\Y\B\4\X11:Add the complement of \PB{\|q[\|t]} to the \PB{\\{trial}}\X${}\E{}$\6
\&{if} ${}(\|t+\T{2}\Z\T{32}){}$\1\5
${}\\{trial}[\T{0}]\K\\{pretrial}[\T{0}]+\\{comp}(\|q[\|t]),\39\\{trial}[\T{1}]%
\K\\{pretrial}[\T{1}];{}$\2\6
\&{else}\1\5
${}\\{trial}[\T{0}]\K\\{comp}(\|q[\|t])\AND\\{m32},\39\\{trial}[\T{1}]\K%
\\{pretrial}[\T{1}]+(\\{comp}(\|q[\|t])\GG\T{32}){}$;\2\par
\U13.\fi

\M{12}I make $t=\lfloor(n-11)/4\rfloor$. (It will be between 1 and 42.)

\Y\B\4\X12:Choose \PB{\|t} and initialize the tables\X${}\E{}$\6
$\|t\K(\|n-\T{11})/\T{4};{}$\6
\X4:Initialize the \PB{\|q} and \PB{\\{qq}} tables\X;\6
\X6:Initialize the \PB{\|y} and \PB{\|r} tables\X;\par
\U1.\fi

\M{13}And now at last the denouement, where we put everything together.

\Y\B\4\X13:See if case \PB{\|a} leads to any square palindromes\X${}\E{}$\6
${}\{{}$\1\6
${}\\{sqrtxl}\K\|y[\|t+\T{2}];{}$\6
\&{for} ${}(\|p\K\|t,\39\\{bit}\K\T{1};{}$ ${}\|a\AND\\{bit};{}$ ${}\|p\MM,\39%
\\{bit}\MRL{{\LL}{\K}}\T{1}){}$\1\5
;\2\6
\X7:Update \PB{\|y} and \PB{\|r} when $a_p$ changes from 0 to~1\X;\6
\&{if} ${}(\|y[\|t+\T{2}]\G\\{sqrtxl}+\T{4}){}$\1\5
${}\\{fprintf}(\\{stderr},\39\.{"Something's\ wrong\ i}\)\.{n\ case\ \%llx!%
\\n"},\39\|a);{}$\2\6
\&{for} ( ; ${}\\{sqrtxl}\Z\|y[\|t+\T{2}];{}$ ${}\\{sqrtxl}\PP){}$\5
${}\{{}$\1\6
\X9:Shift the leading digits\X;\6
\X10:Add \PB{\|q[\|t]} to the \PB{\\{trial}}\X;\6
\X14:Check if \PB{\\{trial}} is a solution\X;\6
\X11:Add the complement of \PB{\|q[\|t]} to the \PB{\\{trial}}\X;\6
\X14:Check if \PB{\\{trial}} is a solution\X;\6
\4${}\}{}$\2\6
\X5:Update \PB{\|q} and \PB{\\{qq}} when $a_p$ changes from 0 to~1\X\6
\4${}\}{}$\2\par
\U1.\fi

\M{14}\B\X14:Check if \PB{\\{trial}} is a solution\X${}\E{}$\6
\X8:Square the contents of \PB{\\{trial}}\X;\6
\&{for} ${}(\|j\K\T{0},\39\|k\K\|n-\T{1};{}$ ${}\|j<\|k;{}$ ${}\|j\PP,\39\|k%
\MM){}$\5
${}\{{}$\1\6
${}\\{jj}\K((\\{acc}[\|j\GG\T{5}]\AND(\T{1}\LL(\|j\AND\T{\^1f})))\I\T{0});{}$\6
${}\\{kk}\K((\\{acc}[\|k\GG\T{5}]\AND(\T{1}\LL(\|k\AND\T{\^1f})))\I\T{0});{}$\6
\&{if} ${}(\\{jj}\I\\{kk}){}$\1\5
\&{break};\2\6
\4${}\}{}$\2\6
\&{if} ${}(\|j\G\|k{}$)\C{ solution! }\1\6
${}\\{printf}(\.{"\%08llx\%08llx\%08llx\^}\)\.{2=\%08llx\%08llx\%08llx}\)\.{%
\%08llx\%08llx\%08llx\\n}\)\.{"},\39\\{trial}[\T{2}],\39\\{trial}[\T{1}],\39%
\\{trial}[\T{0}],\39\\{acc}[\T{5}],\39\\{acc}[\T{4}],\39\\{acc}[\T{3}],\39%
\\{acc}[\T{2}],\39\\{acc}[\T{1}],\39\\{acc}[\T{0}]){}$;\2\par
\U13.\fi

\N{1}{15}Index.
\fi

\inx
\fin
\con
